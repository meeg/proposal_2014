\section{Introduction}


Access to higher and higher luminosities and ever faster detection and recording  techniques enables  searches for new physics at otherwise well-explored energies. This fundamental premise of Intensity Frontier physics has already seen dramatic demonstration at the e+e- B factories, where high luminosities and impressive data handling capacities have allowed extensive exploration of CP violation in the quark sector. The same principle is being exploited in new proposals to explore neutrino masses, mixings, and CP violation by directing ever more intense neutrino beams at massive detectors to push sensitivity well beyond present limits. At the Intensity Frontier, searches for new physics often rely on the study of rare processes and the search for subtle effects which would indirectly indicate physics beyond the Standard Model. But this is not the rule. New studies of otherwise commonplace phenomena at electron machines, like trident production off heavy nuclear targets, can, with sufficient sensitivity, explore whole new worlds and directly search for hidden sector particles and forces, those with very weak couplings to our Standard Model world. The Heavy Photon Search at Jefferson Laboratory does exactly this, utilizing the high duty factor CEBAF accelerator, intense beams, fast detectors, electronics and triggering, and state of the art data acquisition to explore a very common landscape in search of a most uncommon quarry.

Heavy photons, or �dark� or �hidden sector� photons, may well be part of our universe and related to the Dark Matter. Particles of dark matter, which interact very weakly with normal matter and account for a quarter of the universal mass-energy, are of course not yet detected. The Dark Matter can be thought of as constituting, or inhabiting, a � hidden sector�, since it interacts so weakly with normal, baryonic matter. This sector could include a complex of new forces and particles with which we barely interact.  Stimulated by the observation of very high energy electrons and positrons in the cosmic rays and the difficulty of understanding their production in terms of tried and true SUSY dark matter annihilation, several authors realized that models in which massive dark matter particles annihilate to heavy photons, which in turn decay to high energy electron-positron pairs, could naturally account for the observations. These theories presume heavy photons couple to the dark matter, mediate its interactions, are produced  in its annihilation, and weakly couple to electric charge. Heavy photons in the mass range of 100 to 1000 MeV  can reasonably account for the observed cosmic ray fluxes. 

Many Beyond Standard Model theories generate extra U(1) gauge groups, and the associated gauge bosons could have masses over a very wide range. As Holdum realized in the mid 80�s, it is natural that such �heavy photons� kinetically mix with our own photon, leading to their induced coupling to electric charge. This mixing can be mediated by GUT level particles which carry both Standard Model hypercharge and its hidden sector analogue. Interestingly, the natural scale for this mixing results in heavy photons coupling to Standard model charged particles with couplings of order $10^{-3}$e.  So heavy photons naturally couple to electrons, albeit with couplings much suppressed compared to those in standard QED. It follows that electrons will radiate heavy photons, and heavy photons will decay to electron-positron pairs or pairs of other kinematically accessible charged particles, but at rates significantly below QED trident production, and with lifetimes far longer than those expected from purely electromagnetic interactions.

 HPS distinguishes heavy photons from the copious background of QED tridents by  using both invariant mass and decay length signatures. With good mass resolution, heavy photons will appear as sharp resonances above the QED continuum. For suitable values of mass and coupling, heavy photons will have long lifetimes, resulting in discernible secondary decay vertices.  The Heavy Photon Search employs a large acceptance  forward magnetic spectrometer with precise momentum measurement and vertexing  capability, followed by a highly segmented crystal Electromagnetic Calorimeter for fast triggering and electron identification.  HPS depends on  the 40 MHz readout capability of the silicon microstrip vertex tracker, 250 MHz FADC readout of the  electromagnetic calorimeter, and very high rate triggering and data acquisition systems, to fully exploit CEBAF�s essentially DC beams and  high intensities.  A muon identification system just downstream of the ECal  significantly boosts the experimental reach for heavy photon masses above the dimuon threshold and provides an independent trigger. The beam is transported in vacuum through the entire apparatus to eliminate beam gas backgrounds; and the apparatus is split top-bottom, to avoid electrons  which have multiple Coulomb scattered or radiated  in the target.

HPS  probes a unique region of the mass-coupling parameter space where the heavy photon signal would be lost in the trident background without the vertex signature, and it simultaneously accesses a region at higher coupling strength by relying on bump hunting alone.  Figure 1 shows the reach of the experiment in the mass/coupling parameter space, along with limits coming from other searches and the re-interpretation of previous  experiments. Note in particular that HPS has sensitivity to a region of parameter space favored by accounting for the discrepancy between  measured and calculated  values for the muon�s g-2 with the existence of  a heavy photon, and probes an extensive region suggested by parameters which could account for dark matter annihilations into heavy photons.  In broader  terms, HPS searches for heavy photons in a region suggested on very general of theoretical grounds. As seen above, coupling strengths of order $10^{-3}$e are theoretically natural; masses of order  $\alpha m_W$ are also expected on general grounds.  Interestingly, HPS is also sensitive to the production of �true muonium�, the QED atom comprised of $\mu^+ \mu^-$ atom, which is produced with a well-defined (and detectable) cross-section, and decays with a well-defined  (and observable) lifetime to $e^+e^-$. HPS should discover true muonium, measure some of its properties, and find it a useful calibration signal.

This proposal seeks funding for the Heavy Photon Search (HPS) Experiment at Thomas Jefferson National Accelerator Facility.  This experiment  is the second stage of a program that was initiated with the Heavy Photon Search Test Run Proposal \cite{HPS_PROP, HPS_tPROP}, which was approved by the Jefferson Laboratory Program Advisory Committee PAC37 in January, 2011, and approved and funded by DOE HEP in the late Spring of 2011. PAC37 also conditionally approved the full experiment, contingent upon the Test Run results.  During the remainder of FY2011 and the first half of FY2012, the Test Run apparatus, data acquisition system, and system software were designed, constructed, and tested. On April 19, 2012, the newly constituted HPS Collaboration installed the experiment in Jefferson Lab�s Hall B experimental area, and began commissioning the experiment parasitically, using the HDice photon beam. Although the Jefferson Lab schedule did not accommodate the electron beam running which had been requested, the apparatus was fully commissioned by running parasitically in the photon beam. The trigger and data acquisition and storage systems worked well, and all systems performed as expected. Efficient track reconstruction in the Silicon Vertex Tracker was demonstrated, measurements of shower energies and positions were made  in the Electromagnetic Calorimeter, and critical assumptions about background rates were tested. The critical test run goals were accomplished.  A status report summarizing HPS�s progress and results was submitted to PAC39 \cite{HPS_PROP_UPD} along with a request for unconditional approval for the full experiment. At its June, 2012  meeting,  PAC39 graded HPS physics with an �A�, approved a commissioning run with electrons, and granted us a so-called �C1� approval, which gave Jefferson Laboratory management the final say in granting HPS the running time needed to search comprehensively for Heavy Photons. Since that approval, it has become clear that running time will become available in Hall B in late calendar 2014 for our commissioning run, when the upgraded CEBAF 12 will have been completed, commissioned, and operational. CLAS12 is the large general purpose apparatus being constructed to exploit CEBAF 12 in Hall B. Delays in the construction of the CLAS12 magnets will delay CLAS12 installation beyond 2015, thereby providing HPS the opportunity for  a commissioning run in 2014 and the extended data collection run in 2015. To take full advantage of these scheduling windfalls, the HPS Collaboration has re-visited the original HPS design, and simplified and improved it. The resulting simplifications make it possible to construct and test HPS in time for installation in late 2014. The resulting improvements extend the reach far beyond that of the Test Run experiment, maximize the physics output during this time period, and let HPS begin searching for heavy photons in a large and hitherto unexplored region of parameter space.  

 In the following, this proposal motivates and describes the new HPS Experiment, documents the experience and performance obtained with the Test Run Apparatus, demonstrates that the backgrounds expected in electron running are understood and manageable,  reviews the performance and physics reach of the new experiment, and outlines the budget, schedule, and milestones for constructing  and deploying it. It concludes with a request for beam time.

