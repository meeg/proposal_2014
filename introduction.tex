\section{Introduction {\it John, Maurik, Stepan}}


The original Proposal to Search for Heavy Photons in experimental Hall-B at Jefferson Laboratory was submitted to PAC 37 in December, 2010 \cite{HPS_PROP}.  In the proposal we proposed a two-staged approach, the first stage, a test run, to demonstrate that the HPS technical approach was sound and to measure the actual tracker occupancies and trigger rates the full experiment would encounter. This would ensure that our reach estimates are realistic and give us valuable experience. The second stage would be the full HPS experiment.  In January, PAC 37 approved the test run experiment and conditionally approved the full HPS (C2), contingent upon the outcome of the test run experiment. 

Following the recommendation, we have submitted a detailed Heavy Photon Search Test Run proposal to DOE HEP \cite{HPS_tPROP}, which was reviewed March 1, 2011, approved in June, and funded in July. Within nine months Jefferson Laboratory and SLAC, with help from Orsay, Fermilab, and UC Santa Cruz, were able to design and construct the test apparatus and installed in Hall B on April 19, 2012. The HPS Test Run experiment was designed to be run with a dedicated electron beam, but JLAB has been unable to assign dedicated electron beam running because of a scheduling conflict with a previously approved experiment in Hall B, HDice. Accordingly, we have chosen to install the HPS Test Run experiment to run parasitically with the HDice experiment, and make use of HDice�s photon beam to commission HPS and take data from photon conversions occurring just upstream of our apparatus in the Hall B pair spectrometer. 

In May 2012, we submitted the status update on the Heavy Photon Search experiment to PAC 39, \cite{HPS_PROP_UPD}, and in June 2012 we presented the preliminary results from our test run with photon beam. PAC 39 conditionally approved our proposal, this time with C1, which does not require return to PAC. PAC also gave the highest rating, "A", to the experiment and wrote in the report that {\it "This experiment has the potential to make a revolutionary discovery if carried out in a timely manner"}. PAC 39 also urged us {\it "to convince the laboratory to schedule an early test run with electron beam. Ideally, the test run could be performed as soon as beams restart in Hall B, not only to ensure efficient progress toward the full experiment but also to provide an early, high- impact result from 12 GeV operation, thanks to the impressive anticipated reach of the test experiment".}


The physics case for heavy photons has been made extensively in our original proposal, and has recently been updated in the Intensity Frontier Workshop Proceedings. It was well received by PAC 37 and 39, and it has even garnered some attention at Workshops and in the press since then. In short, the physics case is alive and well and better known than it was at the time of the PAC37 meeting.  For the sake of completeness, we include an overview and update of the physics case below. 

With this proposal, we want to make a case for the full approval of the HPS proposal, for the funding to build a new HPS detector, and with expectations of possible early running opportunity in Hall-B in end of 2014 and the first half of 2015 ask to be scheduled as one of first experiments to run in that time frame. Our proposed run plan is to run six weeks in end of 2014 and run larger portion of our beam time, order of 3-5 months, in 2015 as schedule permits. In 2014 we will use two weeks to commission the electron beam in Hall-B and our HPS setup, than run production running at $1.1$ GeV and $2.2$ GeV beam energies, two weeks each, that will allow us to search for heavy photon in g-2 preferred region. For 2015 we expect to run at $2.2$ GeV and $6.6$ GeV beam energies. At high energy, with $3$ months running we expect to detect $>50$ true muonium events, making it a positive discovery. 

In the remainder of this document, we present our goals for the experiment in 2014-2015, describe in detail detector system we want to build, summarize the HPS Test Run (parasitic run) results, review simulation of trigger rates and tracking performance, present the run plan, and make our request for the full approval and for the funding.  
