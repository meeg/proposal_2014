\section{Description of the HPS setup {\it Stepan}}
\label{setup}

\subsection{Overview } 


HPS will utilize a setup located at the upstream end of the experimental Hall-B at Jefferson Lab. The setup will be based on a three-magnet chicane, the second dipole magnet serving as the analyzing magnet. Detector package will include silicon tracker, electromagnetic calorimeter and a muon detector. High luminosities are needed to search for heavy photons with small couplings and masses in the $20$ to $1000$ MeV range. Utilizing CEBAF�s essentially continuous duty cycle, the experiment can simultaneously maximize luminosity and minimize backgrounds by employing detectors with short live times and rapid readout. The HPS setup is designed to run with $> 200$ nA electron beam at energies from $1.1$ GeV to $6.6$ GeV.

The HPS tracker consists of six double layer planes, 40 microstrip sensors in total. Placing the planes of the tracker in close proximity to the target means that the primary beam must pass directly through the middle of the tracking detector. This has necessitated that the sensors don't encroach on a �dead zone�, where multiple Coulomb scattered beam particles and radiative secondaries are bent into a horizontal plane, the so-called "wall of flame".  However, since the energy released in the decay of a low mass A$^\prime$ is small relative to its boost, the opening angle between decay daughters can be quite small. To maximize the acceptance for low mass A$^\prime$s, the vertical extent of the dead zone must be minimized and sensors placed as close as possible to the beam, so our design incorporates precision movers that can bring the silicon detectors  to the required positions. Since interactions of the primary beam with air or even helium at atmospheric pressure gives rise to low-momentum secondaries that generate unacceptable occupancies, we have chosen to place the entire tracking and vertexing system in vacuum, in the Hall B pair spectrometer's magnet vacuum chamber. Silicon microstrip sensors are used in the tracker/vertexer because they collect ionization in $10$�s of nanoseconds and produce pulses as short as $50-100$ ns. The sensors are read out continuously at $40$ MHz using the APV25 chip, developed by the CMS experiment at the LHC. Running high speed silicon module readout in vacuum further requires a vacuum compatible cooling system, and data and power vacuum feedthroughs. All these features are incorporated  in the design of apparatus, as described below and has been tested in the May 2012 test run.


The Ecal PbWO$_4$ crystals, reconfigured from the CLAS Inner Calorimeter, are read out by the APDs and amplifiers,  and have similarly short  pulse widths, so can also run at very high rates. The Ecal data is digitized in the JLAB FADC250, a $250$ MHz flash ADC developed for the $12$ GeV Upgrade detectors.  The full analogue information from the FADCs coupled with the fine spatial information of the calorimeter is available to the trigger, which uses energy deposition, position, timing, and energy-position correlations to reduce the trigger rate to a manageable $\sim 30$ kHz. Like the tracker system, the electromagnetic calorimeter is split to avoid impinging on the �dead zone�. The beam and radiative secondaries pass between the upper and lower ECal modules, which are housed in new temperature-controlled enclosures, needed to stabilize the energy calibration. 

The muon detection system will be installed behind the ECal that absorbs most of the electromagnetic background produced in the target. The remainder will be attenuated by the first absorber layer of the muon system. The muon system will consist of four double layers of scintillator hodoscopes sandwiched between iron absorbers. Light readout from scintillator strips will be done via wave-length shifting fibers embedded inside holes in the strips. As a photodetector Photo-multiplier tubes will be used. As in case of ECal, muon system will be divided into two parts, beam up and beam down. There is a vacuum chamber between two parts to allow the beam and radiated secondaries pass through. 

In the following, the various elements of the experiment are discussed in more detail, beginning with the beamline, continuing with the tracker/vertexer, electromagnetic calorimeter, muon system, data acquisition systems, trigger, and calibration system.
 
 
