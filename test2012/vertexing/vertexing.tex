By selecting e$^{+}$e$^{-}$ pairs from the triggered events we're able to study basic distributions of pair production kinematics and in particular those related to our vertex performance. Pairs of opposite charge tracks, one in the top and one in the bottom half of the SVT, with larger than 400~MeV was selected. The pair production kinematics are relatively well reproduced given the alignment of the tracker; Fig.~\ref{fig:pair_kin} shows the invariant mass and ratio of electron momentum over the sum of electron and positron. 
\begin{figure}[ht]
%   \includegraphics[ width=0.4\textwidth]{test2012/vertexing/figures/h_invM_h_invM_dataMC_0016x0_oneclselgoodquadranttwotrksel}
   \includegraphics[scale=0.25]{test2012/vertexing/figures/h_invM_h_invM_dataMC_0016x0_twotrksel.png}
   \includegraphics[scale=0.25]{test2012/vertexing/figures/h_ratioEsum_h_ratioEsum_dataMC_0016x0_twotrksel.png}
\caption{\small{The reconstructed invariant mass (left) and ratio of electron momentum over the momentum sum for pairs (right) of opposite charge tracks selected in the top and bottom half of the tracker.}}
\label{fig:pair_kin}
\end{figure}
At this moment, we are still working on understanding the relative normalization of pair events in the Test 
Run. 


For the vertexing 
performance the foremost difference compared to electron beam running is that the target was 
located $\sim67$~cm from our nominal target position; giving almost collinear tracks in the detector. This 
degrades the vertex resolution along the 
beam line compared to that expected in an electron beam with tracks from the nominal target position. 
Furthermore, tails of the vertex distributions are hard to study with the finite sample of events from the 
Test Run. 
Nevertheless, useful information can still be 
obtained by studying the vertex distributions. Figure~\ref{fig:vtx_pos} shows the distance of closest 
approach of the momentum vectors extrapolated in the 
upstream direction from our analyzing magnet, taking into account the measured fringe field. 
 \begin{figure*}[t]
\includegraphics[ scale=0.25]{test2012/vertexing/figures/h_vtx_fr_x_h_vtx_x_dataMC_twotrksel.png}
\includegraphics[ scale=0.25]{test2012/vertexing/figures/h_vtx_fr_y_h_vtx_y_dataMC_twotrksel.png}
\includegraphics[ scale=0.25]{test2012/vertexing/figures/h_vtx_fr_z_h_vtx_z_dataMC_twotrksel.png}
\caption{\small{Vertex position represented by the distance of closest approach of the extrapolated momentum vectors upstream of the analyzing magnet. The overall shift from zero is due to a 30~mrad rotation of the SVT with respect to the beam line.}}\label{fig:vtx_pos}
\end{figure*}
While the tails of the vertex distribution expected in electron beam running is not accessible here the 
fact that the core is relatively well described provides confidence of the description of the amount of 
material and the multiple scattering description; both crucial for benchmarking the physics reach of the 
HPS detector.

