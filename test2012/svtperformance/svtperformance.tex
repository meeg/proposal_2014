


%
%   svt_calibrations.tex
%       author: Omar Moreno <omoreno1@ucsc.edu>
%               Per Hansson <phansson@slac.stanford.edu>
%
%

In order to prepare the SVT for real physics data-taking, the SVT was 
calibrated. This involved the extraction of the mean baseline (pedestal),
baseline noise and gain for each of the 12,780 SVT channels. All measurements
were made with the APV25 readout chips configured to their nominal operating
points \cite{Jones:1069892} and all sensors biased to 180 V. The APV25s were
operated in ``mulit-peak'' mode with six samples being readout per trigger.
This, in turn, allowed for the extraction of the $t_0$ and amplitude of the 
signals being read out.

Figure~\ref{fig:pedestal_noise} shows an intensity plot of the pedestals 
along with the readout noise as a function of channel number for a single
hybrid.  The noise was computed by taking the RMS of the gaussian distributed
\begin{figure}[h]
    \begin{center}
    	\includegraphics[width=0.45\textwidth]{test2012/svtperformance/baseline}
    	\includegraphics[width=0.45\textwidth]{test2012/svtperformance/gain}
        \caption{Something ... }
%    	\caption{\small{The baseline across a hybrid (left) and the measured response as a function of 
%	                    input charge (right). The overall shifts in the baseline are calibrated out where distinct edges 
%	                    are associated with the five APV25 chips on the hybrid. The gain shows good linearity up to 
%	                    about three $mip$s.} {\color{red}Should we show noise instead of baseline?}}
	\label{fig:pedestal_noise}
    \end{center}
\end{figure}
pedestals for each of the channels and was observed to be consistently within 
[Find number] ADC counts ( electrons).  One observed feature are the large
noise values for the channels lying near the chip edges.  This has also been 
reported by the CMS collaboration and the cause is still under investigation.

%  Need to rewrite this ...
Another important aspect for the characterization of the SVT is the response and the associated 
gain. Using the APV25 internal calibration circuit a known fixed charge was injected into all 
channels of the which allows for an accurate determination of the response and its 
scaling with input charge, shown in Fig.~\ref{fig:baseline_and_gain}. The gain uniformity was 
within the expected range across chips and modules and show good linearity of charge 
depositions up to about 3 $mip$s. 

All reconstructed hits in an event were used to form clusters of energy 
depositions using a nearest neighbor algorithm. Fig.~\ref{fig:cluster_pulse}
shows the mean pulse shape of each of the hits associated with a track as a 
function of time.  
\begin{figure}[h]
	\includegraphics[width=\textwidth]{test2012/svtperformance/pulseshape_and_landau}
    \caption{Something else ...}
%	\caption{\small{The distribution of cluster amplitudes (left) showing the characteristic Landau 
%	shape and the pulse shape from the six samples readout (right) {\color{red} Remove one of the pulse shapes}. }}
	\label{fig:pulseshape}
\end{figure}
% Should this be included? If so, it needs to be rewritten a little better
The figure also demostrates that  the trigger system, described below, is well 
timed in with the tracker. 
Fig.~\ref{fig:cluster_pulse} shows the MIP response to be [Find value] electrons.
Taking the MIP response, the signal to noise ratio was calculated to be 
approximately 25.5 which is well matched to the expected behavior.

\bibliography{svt_calib}
 


As described in Sec.~\ref{sec:svt} the six sample readout is crucial in order to time-correlate 
hits in the tracker to lower the confusion rate in the pattern recognition during electron running. 
The time reconstruction algorithm described in \ref{sec:svt} was used to fit a single hit to each SVT channel in each event. The fit quality $\chi^2$ was shown to agree with the expected distribution 
for four degrees of freedom for an ideal $CR-RC$ function. 
%\begin{figure}[]
%	\includegraphics[width=0.6\textwidth]{test2012/svtperformance/apvfit_chisq}
%	\caption{\small{Histogram of $\chi^2$ values for pulse fits for all channels on a representative sensor. The peak at 2 is consistent with 4 degrees of freedom (2 fit parameters), as expected. Pileup was not considered due to the very low hit rate in 
%the SVT in this photon beam test. } }
%	\label{fig:apvfit}
%\end{figure}
After clustering hits on a sensor, the hit time for the each cluster is computed as the 
amplitude-weighted average of the channel hit times. Since we have no measurement of the ``true'' hit time, we study the overall SVT hit 
timing performance using the average of all cluster times in a track as the ``track time,'' and take the
 residual of the cluster time relative to that. The observed track time, shown in Figure \ref{fig:tracktime}, has the expected amount of trigger jitter due to the readout clock and trigger 
 system jitter. After correcting for offsets for each sensor (time-of-flight, clock phase) the RMS 
 of the final residual distribution is roughly 2.4~ns for each sensor. 
\begin{figure}[h]
	\includegraphics[width=0.7\textwidth]{test2012/svtperformance/track_time_top}
	\includegraphics[width=0.7\textwidth]{test2012/svtperformance/timeres}
	\caption{\small{Track time distribution (top) and cluster time residual (bottom). The 
	track time is measured relative to the APV25 clock. The width of the distribution is due to 
	trigger jitter (24 ns jitter in tracker readout clock, plus 16 ns jitter in the trigger system). 
	The cluster time residual is for a representative sensor relative to the track time.}}
	\label{fig:tracktime}
\end{figure}
Because the track time is calculated using the individual hit times, the hit time is positively correlated 
with the track time; thus the RMS of the residual is slightly smaller than the true time resolution.
The standard deviation of this residual for $n$-hit tracks where all hits have the same time resolution 
is reduced by a factor of $\sqrt{(n-1)/n}$; since most of our tracks have 8 clusters, the true time 
resolution is 2.6 ns. 
%\begin{figure}[ht]
%	\includegraphics[width=\textwidth]{test2012/svtperformance/timeres}
%	\caption{\small{Histogram and Gaussian fit of residual of cluster times for a representative sensor, relative to the track time. Because the cluster times and track time have positive covariance, the true time resolution is slightly larger than the standard deviation shown here.} }
%	\label{fig:timeres}
%\end{figure}
%\begin{figure}[ht]
%	\includegraphics[width=\textwidth]{test2012/svtperformance/hit_dt}
%	\caption{\small{} }
%	\label{fig:hit_dt}
%\end{figure}
This is somewhat worse than the $\approx 2$ ns resolution expected (see Fig.~\ref{fig:timeres} in 
Sec.~\ref{sec:performance}, but we believe this discrepancy is due to our fit function. Our pulse 
shape fit assumes an ideal CR-RC pulse shape; since the actual pulse shape has a slower rise time, 
there is a systematic pull on the hit time when a hit comes immediately before the APV clock time. 
This is visible in Figure \ref{fig:timeres_2D} as a shift in the residual at certain values of track time.
\begin{figure}[h]
	\includegraphics[width=0.7\textwidth]{test2012/svtperformance/timeres_2D}
	\caption{\small{Plot of the time residual for a representative sensor vs. the track time. 
		The kinks in the horizontal band are caused by the fitter; without them the 1-D histogram in Figure \ref{fig:timeres} (the projection of this histogram) would be narrower.} }
	\label{fig:timeres_2D}
\end{figure}
Work is in progress to use the actual pulse shape in time reconstruction; this should improve time resolution to that expected from previous studies. 
It should be noted that the demonstrated time resolution above is still adequate to achieve a good 
enough rejection due to pile-up hits in the HPS electron run. 

%\vspace{1cm}{\bf Tracking algorithms [Matt/Omar]}


%
%   trk_performance.tex
%       author: Omar Moreno <omoreno1@ucsc.edu>
%      created: December 4, 2012
%
 
Clustered hits in Si planes within each layer are combined to form
2-dimensional ``stereo hits'' which are used by the tracking algorithm to 
form tracks.  The determination of the probability that a stereo hit is 
formed, or hit efficiency, provides insight as to the performance of each of 
the SVT layers.

In order to obtained the hit efficiencies, tracks were fitted using only 4 of 
SVT layers. The resulting track was then extrapolated to the layer omitted
\begin{figure}[h]
        % TODO: Need to update the plot so that Layer 2 on the bottom doesn't 
        % look terrible. It's probably easiest to just remove the point altogether.
    	\includegraphics[width=0.95\textwidth]{test2012/svtperformance/trk_performance/hit_efficiency_vs_layer.pdf}
    	%\includegraphics[width=0.49\textwidth]{test2012/svtperformance/trk_performance/track_reco_efficiency.pdf}
        \caption{{\small
                    Average hit efficiency, excluding known bad channels, 
                    from all dedicated photon runs as a function of layer
                    number.  
                }} 
	\label{fig:hit_track_efficiency}
\end{figure}
from the fit. If the track was found to lie within the sensitive volume
of the layer, a search for a stereo hit within the layer acceptance was 
conducted.  The hit efficiency was then determined as
\[
    \varepsilon_{\mbox{hit efficiency}} = \frac{\mbox{Tracks with hit on missing layer}}
                                            {\mbox{Tracks within layer acceptance}}.
\]
The hit efficiencies per layer were calculated using all dedicated runs. It must 
be noted that those tracks found to intersect bad channels as well as those that 
lie on the sensor edges were excluded from the calculation. As 
can be seen from Figure~\ref{fig:hit_track_efficiency}, the average hit efficiency
per layer was found to be $\approx$ 99\%. 

%95\% for the top SVT volume and greater than 92\% for the bottom volume.  The 
%larger hit inefficiencies observed for some layers was simply due to the 
%sensors composing the layer having a greater number of noisy channels.

%As mentioned above, the standard pattern recognition algorithm is designed to 
%find tracks using the reconstructed stereo hits.  A set of tracking strategies
%outline which layers should be used by the track finding algorithm along
%with their role (seeding, extend, confirm), and the $\chi^2$ cut imposed 
%on the fit. Any kinematic constraints are also specified within the strategy.
%A detailed account of the tracking algorithm can be found  here~\ref{}.

%% Track reco efficiency is no longer going to be included so remove this
%% section
%In order to determine the efficiency of the track finding algorithm, a Monte
%Carlo sample containing pair produced electrons from photons incident on 
%a 1.6\% $X_0$ gold target was used.  The energy of the pair produced electrons
%ranged from .5 GeV to 5.5 GeV. An electron falling within the detector 
%acceptance was considered findable by the tracking algorithm if it 
%traversed through at least 4 of the SVT layers. The track reconstruction
%efficiency was then determined as
%\[
%    \varepsilon_{\mbox{track reco efficiency}} = \frac{\mbox{Tracks found}}
%                                            {\mbox{Tracks found to be findable}}.
%\]
%The resulting efficiency to find an electron which passes through the detector
%acceptance is shown in Figure~\ref{fig:hit_track_efficiency}. The average track
%reconstruction efficiency was found to be \% with the bulk of the inefficiency
%coming from the $\chi^2$ cut imposed on the fit to the six samples during
%the clustering stage.

All events containing pairs of oppositely charged tracks were fit to a
common vertex using a simple vertexing algorithm which searches for the distance
of closest approach between the two tracks.  The reconstructed vertex position
along the beam axis for both data and Monte Carlo is shown on 
Figure~\ref{fig:vz_position}.
\begin{figure}[h]
    \begin{center}
    	\includegraphics[width=0.60\textwidth]{test2012/svtperformance/trk_performance/zvertex.pdf}
        \caption{  
                    The reconstructed vertex position along the beam axis for
                    both data (blue) and Monte Carlo (red).
                } 
	\label{fig:vz_position}
    \end{center}
\end{figure}
Because of the geometric setup of the SVT, observation of pairs produced by
the incident photon required both electrons to experience a hard scatter
within the target.  This results in a broadening of the Gaussian 
distributed reconstructed vertex position. 



%Pattern recognition/Stereo hit reconstruction and description of the tracking algorithm. 

%Plots: tracking efficiency vs run nr, hit efficiency vs run nr for data. Overlay MC.

%\vspace{1cm}{\bf Two track events analysis [Matt]}

%Analysis of two track events. 

%Plots: invariant mass, vertex position, 2-track event multiplicity. Compare with MC for all these 


%\vspace{1cm}{\bf SVT DAQ Performance [Pelle/Omar/Ryan]}
%General discussion on SVT rates, event size.


%
%   svt_daq.tex
%       author: Omar Moreno <omoreno1@ucsc.edu>
%               Santa Cruz Institute for Particle Physics
%               University of California, Santa Cruz
%      created: November 13, 2012
%

The expected data rates and event sizes for each of the dedicated photon runs
were estimated using a full simulation of the SVT DAQ and compared to those
observed.  As discussed in 
section [], the digitized samples from three hybrids were received by a single
DPM which, in turn, applied thresholds and other data reduction techniques to 
each of the samples.  For the test run, at least three of the six samples
were required to exceed a threshold of two times the noise level.  A 
``pile-up'' cut requiring that samples two be greater than sample one or that
sample three be greater than sample two was also applied.  This was meant
to eliminate hits arising from shaper signal tails that were expected to occur 
during electron running.  Signals from the photon run were unaffected by such
a cut. 

All hits which emerged from a DPM were encapsulated into their own container 
or bank. Each of the banks contained the following information:
\begin{itemize}
    \item 32 bit header
    \item 32 bit identifier
    \item 32 bit event number
    \item 32 bit tail bit
    \item 6 hybrid temperatures each 32 bits
    \item 16 bytes $\times$ \# number of hits 
\end{itemize} 
An additional 80 bits of overhead per bank due to ... also needs to be taken 
into account.

From simulation, it was found that a typical event contained on average 100
hits per event with approximately 3 hits being due to noise.  The typical 
event was then estimated to be [] bytes.  The expected data rates
were then estimated using the trigger rates observed during the photon
run and are listed on table [] along with the observed data rates.  
\begin{table}
    \scalebox{1.0}{
    \begin{tabular}{ c | c | c | c | c }
    \hline
    %  Are the trigger rates going to be listed anywhere else? Otherwise, they should be listed here.
    Converter Thickness (\%$X_0$) & Sim Occupancy (\%)  & Sim Event Size (kB) &   Sim Data Rate (Mb/s) \\      
    \hline 
    1.6                           & .438                & 1.22                &   2.07                 \\
    0.45                          & .293                & .93                 &   .53                  \\
    0.18                          & .118                & .56                 &   .24                  \\ 
    \end{tabular} } 
    \caption{Occupancy, event size and resulting data rate expected for each of the three 
             converter thicknesses used in the test run.}
    \label{table:sim_rates}
\end{table}

\begin{table}
    \scalebox{1.0}{ 
        \begin{tabular}{ c | c | c | c }
            \hline
            Converter Thickness (\%$X_0$)   & Obs. Occupancy (\%) & Obs. Event Size (kB) & Obs. Data Rate (Mb/s) \\
            \hline
            1.6                             & 1.03                & 2.43                 & 4.12                  \\
            0.45                            & 1.22                & 2.82                 & 1.61                  \\
            0.18                            & 1.23                & 2.84                 & 1.21                  \\
        \end{tabular}
    }
    \caption{Occupancy, event size and resulting data rate observed for each of the three 
             converter thicknesses used in the test run.}
    \label{table:observed_rates}
\end{table}    

As can be seen from the table, the data rates observed during the run were
much higher than those expected.  This can be attributed to 1) a known
noisy sensor and 2) noisy chips which would randomly appear during certain
runs.  The cause of both these issues are now well understood and will be
resolved in time for the full run.


