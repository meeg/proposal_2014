
%
%   trk_performance.tex
%       author: Omar Moreno <omoreno1@ucsc.edu>
%      created: December 4, 2012
%
 
Clustered hits in Si planes within each layer are combined to form
2-dimensional ``stereo hits'' which are used by the tracking algorithm to 
form tracks.  The determination of the probability that a stereo hit is 
formed, or hit efficiency, provides insight as to the performance of each of 
the SVT layers.

In order to obtained the hit efficiencies, tracks were fitted using only 4 of 
SVT layers. The resulting track was then extrapolated to the layer omitted
\begin{figure}[h]
        % TODO: Need to update the plot so that Layer 2 on the bottom doesn't 
        % look terrible. It's probably easiest to just remove the point altogether.
    	\includegraphics[width=0.95\textwidth]{test2012/svtperformance/trk_performance/hit_efficiency_vs_layer.pdf}
    	%\includegraphics[width=0.49\textwidth]{test2012/svtperformance/trk_performance/track_reco_efficiency.pdf}
        \caption{{\small
                    Average hit efficiency, excluding known bad channels, 
                    from all dedicated photon runs as a function of layer
                    number.  
                }} 
	\label{fig:hit_track_efficiency}
\end{figure}
from the fit. If the track was found to lie within the sensitive volume
of the layer, a search for a stereo hit within the layer acceptance was 
conducted.  The hit efficiency was then determined as
\[
    \varepsilon_{\mbox{hit efficiency}} = \frac{\mbox{Tracks with hit on missing layer}}
                                            {\mbox{Tracks within layer acceptance}}.
\]
The hit efficiencies per layer were calculated using all dedicated runs. It must 
be noted that those tracks found to intersect bad channels as well as those that 
lie on the sensor edges were excluded from the calculation. As 
can be seen from Figure~\ref{fig:hit_track_efficiency}, the average hit efficiency
per layer was found to be $\approx$ 99\%. 

%95\% for the top SVT volume and greater than 92\% for the bottom volume.  The 
%larger hit inefficiencies observed for some layers was simply due to the 
%sensors composing the layer having a greater number of noisy channels.

%As mentioned above, the standard pattern recognition algorithm is designed to 
%find tracks using the reconstructed stereo hits.  A set of tracking strategies
%outline which layers should be used by the track finding algorithm along
%with their role (seeding, extend, confirm), and the $\chi^2$ cut imposed 
%on the fit. Any kinematic constraints are also specified within the strategy.
%A detailed account of the tracking algorithm can be found  here~\ref{}.

%% Track reco efficiency is no longer going to be included so remove this
%% section
%In order to determine the efficiency of the track finding algorithm, a Monte
%Carlo sample containing pair produced electrons from photons incident on 
%a 1.6\% $X_0$ gold target was used.  The energy of the pair produced electrons
%ranged from .5 GeV to 5.5 GeV. An electron falling within the detector 
%acceptance was considered findable by the tracking algorithm if it 
%traversed through at least 4 of the SVT layers. The track reconstruction
%efficiency was then determined as
%\[
%    \varepsilon_{\mbox{track reco efficiency}} = \frac{\mbox{Tracks found}}
%                                            {\mbox{Tracks found to be findable}}.
%\]
%The resulting efficiency to find an electron which passes through the detector
%acceptance is shown in Figure~\ref{fig:hit_track_efficiency}. The average track
%reconstruction efficiency was found to be \% with the bulk of the inefficiency
%coming from the $\chi^2$ cut imposed on the fit to the six samples during
%the clustering stage.

All events containing pairs of oppositely charged tracks were fit to a
common vertex using a simple vertexing algorithm which searches for the distance
of closest approach between the two tracks.  The reconstructed vertex position
along the beam axis for both data and Monte Carlo is shown on 
Figure~\ref{fig:vz_position}.
\begin{figure}[h]
    \begin{center}
    	\includegraphics[width=0.60\textwidth]{test2012/svtperformance/trk_performance/zvertex.pdf}
        \caption{  
                    The reconstructed vertex position along the beam axis for
                    both data (blue) and Monte Carlo (red).
                } 
	\label{fig:vz_position}
    \end{center}
\end{figure}
Because of the geometric setup of the SVT, observation of pairs produced by
the incident photon required both electrons to experience a hard scatter
within the target.  This results in a broadening of the Gaussian 
distributed reconstructed vertex position. 

