
%
%   svt_daq.tex
%       author: Omar Moreno <omoreno1@ucsc.edu>
%               Santa Cruz Institute for Particle Physics
%               University of California, Santa Cruz
%      created: November 13, 2012
%

The expected data rates and event sizes for each of the dedicated photon runs
were estimated using a full simulation of the SVT DAQ and compared to observed
values. As discussed in Section~\ref{sec:testrun_daq}, the digitized samples
from three hybrids were received by a single DPM.  The DPM then required that
at least three of the six samples exceeeded a threshold of two times the noise
level for that channel.  An additional ``pile-up'' cut requiring that 
(sample 2 $>$ sample 1) or (sample 3 $>$ sample 2) was also applied. This was
meant to eliminate hits arising from shaper signal tails that were expected
to occur when running in a high occupancy environment i.e. electron run.
Signals from the photon run were unaffected by such a cut. 

All samples were placed into their own container along with the 
channel number, hybrid identifier, chip address and DPM identifier. An 
additional layer of encapsulation or bank was used to store all samples 
emerging from a single DPM along with the DPM identifier, the event number
an error bit and hybrid temperatures. A diagram of the container along with
the sizes of each of the elements is shown on Figure~\ref{fig:data_format}.
\begin{figure}[h]
    \begin{center}
%    	\includegraphics[width=0.49\textwidth]{test2012/svtperformance/svt_calib/baseline_v_ch_fpga0_hybrid0.pdf}
        \caption{
                 } 
	\label{fig:data_format}
    \end{center}
\end{figure}
Overall, the container overhead will contribute a total of 326 bytes to an event
with an additional 16 bytes per hit.

The occupancy expected for each of the converter thicknesses along with the 
corresponding event size and data rate are shown on Table~\ref{table:sim_rates}.
\begin{table}[h]
    \scalebox{0.9}{
    \begin{tabular}{ c | c | c | c }
    \hline
    %  Are the trigger rates going to be listed anywhere else? Otherwise, they should be listed here.
    Converter Thickness (\%$X_0$) & Sim Occupancy (\%)  & Sim Event Size (kB) &   Sim Data Rate (Mb/s) \\      
    \hline 
    1.6                           & .438                & 1.22                &   2.07                 \\
    0.45                          & .293                & .93                 &   .53                  \\
    0.18                          & .118                & .56                 &   .24                  \\ 
    \end{tabular} } 
    \caption{Occupancy, event size and resulting data rate expected for each of the three 
             converter thicknesses used in the test run.}
    \label{table:sim_rates}
\end{table}
The occupancies shown include an contribution of .02\% (3 hits) due to noise.  
The data rates were estimated using the trigger rates observed during each
of the dedicated photon runs. From the table, it can be seen that the 
occupancies are as expected i.e. thicker targets correspond to higher 
occupancies. 

Table~\ref{table:observed_rates} list the observed occupancy, event sizes and 
\begin{table}[h]
    \scalebox{0.9}{ 
        \begin{tabular}{ c | c | c | c }
            \hline
            Converter Thickness (\%$X_0$)   & Obs. Occupancy (\%) & Obs. Event Size (kB) & Obs. Data Rate (Mb/s) \\
            \hline
            1.6                             & 1.03                & 2.43                 & 4.12                  \\
            0.45                            & 1.22                & 2.82                 & 1.61                  \\
            0.18                            & 1.23                & 2.84                 & 1.21                  \\
        \end{tabular}
    }
    \caption{Occupancy, event size and resulting data rate observed for each of the three 
             converter thicknesses used in the test run.}
    \label{table:observed_rates}
\end{table}    
rates for each of the targets.  The data rates observed during the run were
much higher than those expected.  This can be attributed to 1) a known
noisy sensor and 2) noisy chips which appeared during certain runs.  The cause
of both these issues are now well understood and will be resolved in time 
for the upcoming run.

\begin{center}
    \textbf{Need to add discussion of rates after all bad channels are masked
            in order to get a better comparison to MC rates}
\end{center}
