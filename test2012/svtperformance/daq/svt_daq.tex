
%
%   svt_daq.tex
%       author: Omar Moreno <omoreno1@ucsc.edu>
%               Santa Cruz Institute for Particle Physics
%               University of California, Santa Cruz
%      created: November 13, 2012
%

The expected data rates and event sizes for each of the dedicated photon runs
were estimated using a full simulation of the SVT DAQ and compared to those
observed.  As discussed in 
section [], the digitized samples from three hybrids were received by a single
DPM which, in turn, applied thresholds and other data reduction techniques to 
each of the samples.  For the test run, at least three of the six samples
were required to exceed a threshold of two times the noise level.  A 
``pile-up'' cut requiring that samples two be greater than sample one or that
sample three be greater than sample two was also applied.  This was meant
to eliminate hits arising from shaper signal tails that were expected to occur 
during electron running.  Signals from the photon run were unaffected by such
a cut. 

All hits which emerged from a DPM were encapsulated into their own container 
or bank. Each of the banks contained the following information:
\begin{itemize}
    \item 32 bit header
    \item 32 bit identifier
    \item 32 bit event number
    \item 32 bit tail bit
    \item 6 hybrid temperatures each 32 bits
    \item 16 bytes $\times$ \# number of hits 
\end{itemize} 
An additional 80 bits of overhead per bank due to ... also needs to be taken 
into account.

From simulation, it was found that a typical event contained on average 100
hits per event with approximately 3 hits being due to noise.  The typical 
event was then estimated to be [] bytes.  The expected data rates
were then estimated using the trigger rates observed during the photon
run and are listed on table [] along with the observed data rates.  
\begin{table}
    \scalebox{1.0}{
    \begin{tabular}{ c | c | c | c | c }
    \hline
    %  Are the trigger rates going to be listed anywhere else? Otherwise, they should be listed here.
    Converter Thickness (\%$X_0$) & Sim Occupancy (\%)  & Sim Event Size (kB) &   Sim Data Rate (Mb/s) \\      
    \hline 
    1.6                           & .438                & 1.22                &   2.07                 \\
    0.45                          & .293                & .93                 &   .53                  \\
    0.18                          & .118                & .56                 &   .24                  \\ 
    \end{tabular} } 
    \caption{Occupancy, event size and resulting data rate expected for each of the three 
             converter thicknesses used in the test run.}
    \label{table:sim_rates}
\end{table}

\begin{table}
    \scalebox{1.0}{ 
        \begin{tabular}{ c | c | c | c }
            \hline
            Converter Thickness (\%$X_0$)   & Obs. Occupancy (\%) & Obs. Event Size (kB) & Obs. Data Rate (Mb/s) \\
            \hline
            1.6                             & 1.03                & 2.43                 & 4.12                  \\
            0.45                            & 1.22                & 2.82                 & 1.61                  \\
            0.18                            & 1.23                & 2.84                 & 1.21                  \\
        \end{tabular}
    }
    \caption{Occupancy, event size and resulting data rate observed for each of the three 
             converter thicknesses used in the test run.}
    \label{table:observed_rates}
\end{table}    

As can be seen from the table, the data rates observed during the run were
much higher than those expected.  This can be attributed to 1) a known
noisy sensor and 2) noisy chips which would randomly appear during certain
runs.  The cause of both these issues are now well understood and will be
resolved in time for the full run.

