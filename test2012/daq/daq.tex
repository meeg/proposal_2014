
\subsubsection{Test Run Data Acquisition}
\label{sec:testrun_daq}
The test run served as a proof of principle for the proposed DAQ and consequently was very 
close to that proposed for HPS. In consideration of the similarities to the system proposed in Sec.~\ref{sec:daq} 
only the main differences will be highlighted here with results and experiences discussed in 
Sec.~\ref{sec:testrun_performance}.  In general, the contrasts with HPS are mainly due to:
\begin{enumerate}
\item a lower ECal cluster resolution and no calibration available at the trigger level together with simpler trigger logic, 
\item a smaller SVT without the need for power and signal aggregation inside the vacuum chamber 
and optical readout, and
\item lower bandwidth links.
\end{enumerate}
The two front-end electronics systems for the ECal and SVT are essentially unchanged compared to the HPS.
The ECal provides 
input to the Level~1 trigger system after which an accepted event is acquired from the two sub-systems 
and are processed in the data acquisition and processing system. The Readout Crate Controllers (ROCs)
 described for HPS are unchanged and installed in every VME, VME64X, VXS crates running 
 mvme6100 controllers with a prpmc880 or pmc280 co-processor modules. A hybrid approach was 
 used for the SVT DAQ in the test run where the ROC ran on a external PC connected to the ATCA crate. 
 Similar to HPS, a Foundry Router was used as the backbone of the DAQ system, providing 1Gbps and 
 10Gbps connections between components and to the JLAB Computer Center. The Event Builder, Event 
 Recorder, and other critical DAQ components ran on 
4-CPU Opteron-based servers, which was sufficient for the test run. The RAID5 test run storage system 
had a 100~MByte/sec capability, high enough for the data rates attained in the test run as 
described below. 

%\subsubsection{SVT DAQ}



The SVT DAQ was a rapidly built DAQ  designed to readout data continuously at 40 MHz from the silicon detector modules, and transfer data to the JLab DAQ once a trigger signal is received. It is built using the same 
basic architecture and layout as the HPS DAQ but without the optical readout components and 
power distribution system inside the vacuum described in Sec.~\ref{sec:svt_daq}. 

The test run had a total of 20 silicon strip sensors, each one connected to an onboard hybrid readout card 
which is similar to the HPS hybrid, each one holding five 128-channel APV25 integrated circuits. The test 
run hybrid readout card is shown in Fig.~\ref{fig:hybrid_and_apv25_testrun} 
 \begin{figure*}[t]
\includegraphics[ scale=0.3]{test2012/daq/hybrid.jpg}
\includegraphics[ scale=0.3]{test2012/daq/apvs-on-hybrid.jpg}
\caption{\small{Picture of a test run hybrid readout board holding five APV25 ASICs. The wire bonds to the 
silicon sensors can be seen as well.}}
\label{fig:hybrid_and_apv25_testrun}
\end{figure*}

Contrary to the HPS DAQ where opto-boards digitizes and converts the APV25 analog output signals to 
optical signals inside the vacuum chamber, the hybrids here carry analog signal directly to the 
Rear Transistion Module (RTM) via a multi-twisted-pair cable. The amplification and digitization of the 
analog differential voltage output of the APV25 output are therefore carried out on the RTM board 
which was designed specifically for the HPS test run. Figure~\ref{fig:svtdaq} shows an overall layout of 
the SVT test run DAQ system (compare to Fig.~\ref{fig:svt_daq_overview}).
 \begin{figure}[t]
\includegraphics[scale=0.9]{test2012/daq/svt_daq_diagram.png}
\caption{\small{Schematic of the SVT DAQ showing input from the hybrids mounted on the silicon detector to the RTM, its connection to the COB, and the Ethernet switch used to transfer data at 1 Gbps to the 
DAQ PC and ultimately to the JLAB DAQ.}}
\label{fig:svtdaq}
\end{figure}
On the RTM, a pre-amplifier converts the APV25 differential current output to a different voltage output 
scaled to the sensitive range of a 14-bit ADC. The RTM is organized into four sections with each section 
supporting 3 hybrids (15 channels). 
The ADC is operated at the system clock of 41.667 Mhz. 
%The RTM also includes a 4-channel Fiber Optic module and supporting logic which can be used to interface to the JLAB trigger supervisor card.
The ATCA main board (the Cluster On Board or COB) is similar to the HPS DAQ with the important exception that one of the DPM's functions as the trigger interface only and there is no RCE module. 
Instead, the DPMs package and send the data from the hybrids to an external PC through a 1Gbps 
ethernet connection which serve the same purpose as the RCE module in the HPS DAQ. 
The ATCA crate hosts two COB cards, one supporting four data processing DPMs and the other supporting three data processing DPMs and one trigger DPM to support a total of 21 hybrids, one more than required. 
The test run RTM and COB can be seen in Fig.~\ref{fig:rtm_testrun}. 
\begin{figure*}[t]
\includegraphics[ scale=0.25]{test2012/daq/rtm.png}
\includegraphics[ scale=0.4]{test2012/daq/svt_daq_module_noted.png}
\caption{\small{Picture of a RTM (top) and COB board (bottom) used in the HPS test run 2012.}}
\label{fig:rtm_testrun}
\end{figure*}
The external PC supports three network interfaces, 2 standard 1G-bit Ethernet and one custom low latency data reception card. The first Ethernet interface is used for slow control and monitoring of the 8 DPM modules. The second Ethernet interface serves as the interface to the JLAB data acquisition system. The third custom low latency network interface is used to receive data from the SVT ATCA crate and supports a low latency, reliable TTL trigger acknowledge interface to the trigger DPM. This PC hosts the SVT control and monitoring software as well as the JLAB ROC (Read Out Controller) application.


The ECal DAQ system used in the test run is very similar to that described for HPS in Sec.~\ref{sec:fadc_daq}. 
The only significant difference is that in the test run, the signals from the ECal modules were sent to a signal splitter. One of the outputs of the splitter is fed to a 
discriminator that also has an internal scaler, and then to a TDC channel. The other output is sent to the 
JLab FADC250 VXS module, shown in Fig.~\ref{fig:fadc}.
%, is based on information from the FADC boards and includes a cluster 
%finding algorithm using FPGA modules. With the FADC-based system, the energy of clusters used to 
%make a trigger decision is determined at the crate level. 



