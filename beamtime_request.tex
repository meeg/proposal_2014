Jefferson Labs PAC39 graded HPS physics with an �A�, approved a commissioning run with electrons, and granted so-called �C1� approval for the full HPS experiment. The total requested beam time for the experiment is 180 days. For the commissioning run we requested 3 weeks. With anticipation of an early running in Hall-B, we propose to conduct the experiment in two phases. The first phase, expected to run in 2014-2015, will consume the full beam time for the commissioning, and at least 1/3 of the production running time at 2.2 GeV and 6.6 GeV beam energies. The second phase that will use remaining beam time can be completed any time after that.

We plan to execute the first phase of the experiment in two run periods using the apparatus described above. First, we will perform a commissioning run with modest physics output, then, after a month or two down time, continue with a longer run at multiple beam energies to cover as much parameter space as possible. The experimental apparatus, if funded on time, will be ready to be commissioned and take physics data in late summer of 2014. The HPS experiment will be ready to use the first physics quality beams available in Hall-B as early as fall of 2014. The proposed run plan for this early running will be the following: 

\begin{itemize}
\item {\bf Commissioning run in 2014, total of 3 weeks of beam time ($6$ weeks on the floor assuming 50\% for combined efficiency of the accelerator and the detector):}
\begin{itemize}
\item $1$ week of detector commissioning
\item $1$ week of physics run at $2.2$ GeV
\item $1$ week of physics run at $1.1$ GeV
\end{itemize}
\item{\bf Physics run in 2015, total of $5$ weeks of time beam ($10$ weeks on the floor assuming 50\% for combined efficiency of the accelerator and the detector):}
\begin{itemize}
\item $1$ weeks of detector commissioning
\item $2$ weeks of physics run at $2.2$ GeV
\item $2$ weeks of physics run at $6.6$ GeV
\end{itemize}
\end{itemize}
If more beam time is available in 2015, HPS will continue data taking at above two beam energies.

The proposed run plan will cover remaining region of parameter space favored by muon \mbox{g-2} anomaly, and will buy in significant territory not only at large couplings ($\alpha^\prime/\alpha>10^{-7}$) but also in the regions of small couplings, down to $\alpha^\prime/\alpha\sim 10^{-10}$. These small coupling region is not accessible to any other proposed experiments. The HPS is the only experiment that can cover small coupling region using excellent vertexing capability of the Si-tracker. 

In the proposed run plan we have running at a non-standard energy for 12 GeV CEBAF, $E=1.1$ GeV, in 2014. In case this will not be possible due to schedule conflicts, we will continue to run at $2.2$ GeV instead. We then expect to complete $1.1$ GeV running at some time in 2015, reducing the time on $2.2$ GeV running in 2015. The gap between the run periods in 2014 and 2015, on the order of two months, will be used to improve, correct or fix apparatus if necessary.

In summary, we request for the first phase of HPS experiment to run in two run periods, first in the fall of 2014 for a total of 3 weeks of beam time (6 weeks on the floor), with beam energies $1.1$ GeV, two weeks, and $2.2$ GeV, two weeks. For the second part we request 5 weeks of beam (10 weeks on the floor) that will be equally shared between beam energies of $2.2$ GeV and $6.6$ GeV. We expect that the second part of the run will be In 2015.   

