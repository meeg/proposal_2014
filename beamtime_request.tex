We plan to execute the experiment in two run periods using the same apparatus. First, we will perform a commissioning run with modest physics output, then, after a month or two of down time, continue with a longer run at multiple energies to cover as much parameter space as possible. The experimental apparatus will be ready to be commissioned and take physics data in late summer of 2014. The HPS experiment will be ready to use the first physics quality beams available in Hall-B as early as fall of 2014. With the assumption that early running in Hall-B is possible, the proposed run plan for the HPS experiment will be the following: 

\begin{itemize}
\item {\bf Commissioning run in 2014: total of $6$ weeks on the floor with beam:}
\begin{itemize}
\item $2$ weeks of detector commissioning
\item $2$ weeks physics run at $2.2$ GeV
\item $2$ weeks physics run at $1.1$ GeV
\end{itemize}
\item{\bf Physics run in 2015, total of $10$ weeks on the floor with beam:}
\begin{itemize}
\item $2$ weeks of detector commissioning
\item $4$ weeks physics run at $2.2$ GeV
\item $4$ weeks physics run at $6.6$ GeV
\end{itemize}
\end{itemize}

The proposed run plan will cover remaining region of parameter space favored by muon \mbox{g-2} anomaly, and will buy in significant territory not only at large couplings ($\alpha^\prime/\alpha>10^{-7}$) but also in the regions of small couplings, down to $\alpha^\prime/\alpha\sim 10^{-10}$. These small coupling regions are not accessible to any other proposed experiment. The HPS is the only experiment that can cover small coupling region using excellent vertexing capability of the Si-tracker. 

In the proposed run plan we have running at a non-standard energy for 12 GeV CEBAF, $E=1.1$ GeV, in 2014. In case that this will not be possible due to schedule conflicts, we will continue to run at $2.2$ GeV instead. We then expect to complete $1.1$ GeV running at some time in 2015, reducing the time spend on $2.2$ GeV running in 2015. The gap between the run periods in 2014 and 2015, on the order of two months, will be used to improve, correct or fix apparatus if necessary.

In summary, we request for the first phase of HPS experiment to run during two run periods, first in the fall of 2014 for a total of 3 weeks of beam time (6 weeks on the floor), with beam energies $1.1$ GeV, two weeks, and $2.2$ GeV, two weeks. For the second part we request 5 weeks of beam (10 weeks on the floor) that will be equally shared between beam energies of $2.2$ GeV and $6.6$ GeV. We expect that the second part of the run will be In 2015.   

