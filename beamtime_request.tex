The Jefferson Laboratory PAC39 graded HPS physics with an "A", approved a commissioning run with electrons, and granted so-called �C1� approval 
for the full HPS experiment. The total requested beam time for the experiment is 180 days. For the upcoming commissioning run we have requested 3 weeks. 
Anticipating early running in Hall-B, we propose to conduct the experiment in two phases. The first phase, expected to run 
in 2014-2015, will complete the commissioning run and begin the production running at 2.2 GeV and 
6.6 GeV beam energies as detailed below. The second phase, which will use remaining beam time, can be scheduled later in 2015 or in 2016 and beyond,
and will continue runs at 2.2 and 6.6 GeV and possibly other energies.

We plan to execute the first phase of the experiment in two run periods using the apparatus described above. First, we will perform 
a commissioning run which should produce our first physics output, and then, after a month or two down time, continue with a longer run at multiple beam 
energies to cover as much parameter space as possible. The experimental apparatus, if funded on time, will be ready to be commissioned 
and take physics data in the fall of 2014 when the first physics quality beams should be available in Hall-B. The proposed run plan for phase
one is as follows: 

\begin{itemize}
\item {\bf Commissioning run in 2014, total of 3 weeks of beam time ($6$ weeks on the floor assuming 50\% for combined efficiency of 
the accelerator and the detector):}
\begin{itemize}
\item $1$ week of detector commissioning
\item $1$ week of physics run at $2.2$ GeV
\item $1$ week of physics run at $1.1$ GeV
\end{itemize}
\item{\bf Physics run in 2015, total of $5$ weeks of time beam ($10$ weeks on the floor assuming 50\% for combined efficiency of the 
accelerator and the detector):}
\begin{itemize}
\item $1$ week of detector commissioning
\item $2$ weeks of physics run at $2.2$ GeV
\item $2$ weeks of physics run at $6.6$ GeV
\end{itemize}
\end{itemize}
If more beam time is available in 2015, HPS will continue data taking at these or other energies.

The proposed run plan will cover the remaining region of parameter space favored by the muon \mbox{g-2} anomaly, and will explore a significant 
region of parameter space, not only at large couplings ($\alpha^\prime/\alpha>10^{-7}$), but also in the regions of small couplings, down 
to $\alpha^\prime/\alpha\sim 10^{-10}$. This small coupling region is not accessible to any other proposed experiment. 
The excellent vertexing capability of the Si-tracker uniquely enables HPS to cover the small coupling region. 

In the proposed run plan we have assumed running at a non-standard energy for 12 GeV CEBAF, $E=1.1$ GeV, in 2014. In case this will 
not be possible due to scheduling conflicts, we will continue to run at $2.2$ GeV instead. We would then expect to complete the $1.1$ 
GeV running at some time in 2015 by reducing the time at $2.2$ GeV. The gap between the run periods in 2014 
and 2015, on the order of two months, will be used to improve, correct or fix apparatus if necessary.

In summary, we request time for the first phase of HPS experiment, first in the fall of 2014 for a 
total of 3 weeks of beam time (6 weeks on the floor), with beam energies $1.1$ GeV, two weeks, and $2.2$ GeV, two weeks. 
Second, we request 5 weeks of beam (10 weeks on the floor) that will be equally shared between beam energies 
of $2.2$ GeV and $6.6$ GeV. We expect that the second part of the run will be in 2015.   

