\appendix
%\renewcommand*{\thesection}{\arabic{section}}
\renewcommand*{\thesubsection}{\thesection.\arabic{subsection}}
\renewcommand*{\thesubsubsection}{\thesubsection.\arabic{subsubsection}}

\section{WBS Tables}
\label{sec:wbs}
%WBS tables:

\clearpage
%\begin{figure*}[ht]
%\centering
%\vspace*{-5mm}
%\includegraphics[angle=0,width=\textwidth]{cost_schedule/HPSV470p1} 
%\caption{HPS WBS.}
%\label{fig:schedulea}
%\end{figure*}

\includepdf[pages=1-9,angle=90,scale=0.8]{cost_schedule/HPSV470.pdf}
%\begin{rotate}{90}

\section{Additions and improvements to the HPS setup using non-DOE sources of funding}

While the proposed baseline equipment will be sufficient to carry out proposed measurements, the HPS collaboration is seeking funding from non-DOE sources to improve and enhance capabilities of the HPS detector. Plans for improvements include light monitoring system and large area APDs for ECal.  The HPS collaborators from IPN Orsay have applied for a number of European grants, including European Research Council (ERC) Advanced Grant 2013 to purchase APDs, for manpower costs to replace the old ones, to design and build 
new preamplifier boards, and to assemble and test the ECal with the new modules. The total cost of replacing all ECal APDs is about $500$K\$. 
These grants also include light monitoring system. If successful European funding will cover most of the ECal modifications, and will significantly reduce the support needed from DOE for the more modest upgrades we have proposed above. 

Besides ECal improvements, collaboration intend to add muon detector to the HPS setup. Collaborators from the Collage of William\&Mary (PI Prof. Keith Griffion) togetehr with collaborators from Rutgers University (PI Prof. Yuri Gerstein) and Old Dominion University (PI Prof. Lary Weinstein) submitted an MRI proposal to NSF for the Muon System, requesting $\sim 300$k.  

Below the details of ECal improvements, and the motivation and description of the muon system are presented. 

\subsection{Improvements to ECal}

\begin{enumerate}
\item {\bf Light monitoring system}

For an experiment like HPS, where backgrounds must be well understood and need to be strongly suppressed, the trigger bias must be minimized. In particular, having stable, known, and uniform thresholds at the trigger readout is necessary in order to avoid  bias in the 
event selection. Such uniformity and stability can be achieved with the installation of an online gain monitoring 
system. This system will introduce short light pulses into the front face of the crystals. The crystals already have fiber holders attached, allowing implementation of this system without having to modify the crystals or wrapping. 

Optical fibers will be used to transmit light from a calibration  source to the crystals to test the response of the APDs. The response of the system could change in time because of 
losses in crystal transparency due to radiation damage or because of gain variations of the APDs. 
Such a calibration system has been developed for several experiments (CMS at CERN for instance) with various light sources. The system for the ECal 
will be developed at IPN Orsay during 2013 and in the first half of 2014, and will be ready for installation at JLAB for the commissioning run in the fall of 2014. Each module will have a red and blue mono-color LED light source for monitoring purposes. 
Blue light transmission, corresponding to the domain of the crystal's emission spectrum, is very sensitive to the presence of color centers which are produced by radiation damage. So the blue light source will monitor variations of the response in the main domain of the spectrum.
 %Impurities can anneal at room temperature and such monitoring can be sensitive to increasing of output as well, when the radiation exposure is reduced for a long period of time. 
The response to red light is less sensitive to the color centers,  and so permits monitoring the APD gains more directly. Thus the use of two colors separates gain variations due to the 
APD and electronics from those due to changes in the crystal transparency, and provides clear information on the state of the electronics. 
The main challenge for the system is to guarantee stability at a level of $2\%$. The test of the system will be carried-out at
IPN Orsay, in order to study its efficacy and also to test the radiation resistance of the various materials.

\item {\bf Modifications to the side brackets to accommodate fiber bundles for the light monitoring system} -
A light monitoring system was not used during the test run. While the design of the ECal enclosure was done in such a way that it can 
accommodate optical fibers attached to the front face of crystals, the side plates that hold the crystal frames do not have inlets for the accompanying fiber bundles.
Space is available on the side plates for a straight-forward modification which will allow the addition of a light monitoring system.  
    
\item {\bf New low-voltage power supply} - The existing low voltage power supply is a manually controlled, single output power supply 
that feeds the four ECal motherboards through a custom-made patch panel. The present system cannot control the voltage supplied to preamplifiers at different
parts of the ECal, and controlling or resetting them remotely has proved to be very inconvenient, requiring frequent access to the Hall, especially during commissioning. Newly available low voltage power supplies are much more flexible. The one that is the most suitable for the
ECal APD preamplifiers is the WIENER MPV 8008LD. This power supply is being used at JLAB and the control software exists, so it will be easy to incorporate it into HPS.     

\item {\bf New preamplifiers} - A low noise, low gain preamplifiers will be needed to take advantage of increased signal on the input of FADC after removing the spliter. The impact of the lower noise/threshold system is twofold: first it 
will improve the ECal's energy resolution; and second it will make the ECal sensitive to minimum ionizing particles which pass through the crystals transversely. With sensitivity to cosmic ray muons, which will pass through the ECal transversely when it is installed in HPS, the Ecal crystals can be calibrated for MIPS, and their effective gains balanced.  HPS collaborators from INFN Genova have shown that with such a low noise, low threshold system, the ECal can distinguish the MIP energy deposition from noise, see left plot on Figure \ref{fig:mip10x10}.

%\begin{figure*}[t]
%\includegraphics[scale=0.4]{ecal/MIP_5x5_APD.png}
%\caption{\small{Charge distribution from readout of the HPS calorimeter crystal with Hamamatsu S8664-55 APD, and the new low noise amplifier board. The red line histogram corresponds to the charge distribution for all triggers coming from the scintillators positioned above and below the crystal. The black line shows the distribution for hits in the crystal within $100$ ns of the trigger signal. }}\label{fig:mip5x5}
%\end{figure*}


\item{\bf Possibilities with new APDs} - Installing new APDs on the existing crystals will significantly improve the ECal performance, but doing so is expensive,  so replacement is only being considered if a funding source beyond DOE HEP is secured. Replacing the old $5\times 5$ mm$^2$ 
Hamamatsu S8664-55 APDs with $10\times 10$ mm$^2$, Hamamatsu S8664-1010 will improve two critical characteristics of the present calorimeter modules. First, the new APDs from Hamamatsu have much better performance than the ones which are currently installed. Data from Hamamatsu shows that APDs made from the same wafer have excellent gain uniformity. With $\pm 10\%$ known uniformity 
at the gain of $100$, the required variations in bias voltage are only $\sim 4.5$ V. Even for large samples of APDs (~1300),  the required bias voltage differences are ~50 V, which is half that of the current APDS. The ECal supplies bias voltage to groups of APDs, so with new APDs, with their smaller voltage-gain variations, it will be possible to achieve much better 
uniformity in the response of the calorimeter modules, and consequently better uniformity in trigger thresholds. 

Secondly, the new APDs have a $4$ times larger readout area, ensuring $4$ times more light collection and therefore $4$ times larger signals. This will allow the use of lower gain amplifier modules which in turn will decrease electronic noise. Tests 
performed for another calorimeter, now in production at INFN Genova for JLAB Hall-B, showed that amplifier boards 
with a factor 2 lower gains have a noise level $<5$ MeV. The energy deposition in the HPS PbWO$_4$ crystals  
from minimum ionizing cosmic muons passing transversely to the crystal axis is $\sim 15$ MeV. Moving the energy thresholds 
close to $5$ MeV will allow the MIP peak to be clearly distinguished, and will let the calorimeter  be calibrated and monitored with cosmic muons. The HPS collaborators 
from the INFN group have performed the first tests of the Hamamtsu $10\times 10$ mm$^2$ APDs and a new amplifier board on HPS crystals. In 
Figure \ref{fig:mip10x10}, the charge distribution of a single crystal system is shown for $5\times 5$ mm$^2$ (left) and $10\times 
10$ mm$^2$ (right) APDs. A coincidence signal from scintillator telescopes positioned above and below the module provides the trigger. 
The crystal was positioned horizontally as it would be in HPS, so the cosmic muons would pass through it perpendicular to the crystal axis. The red line 
histogram is for all events triggered by the scintillation telescope and corresponds to the noise. The black line histogram corresponds 
to the charge detected within $100$ ns of the trigger time. The MIP peak is clearly visible and well isolated from the noise 
for the S8664-1010 APD readout. For  the S8664-55 APD, the MIP signal is also seen, but its charge distribution is under the noise peak and it
does not have well defined peak position. Using MIP calibration in conjunction with the light monitoring system will ensure stable and reliable performance of the ECal and the trigger system. As a bonus, the lower noise will allow the use of lower  thresholds and improve the calorimeter's energy resolution. The new amplifier boards have to be designed to work with new APDs. 

\begin{figure*}[t]
\includegraphics[scale=0.37]{ecal/MIP_5x5_APD.png}
\includegraphics[scale=0.37]{ecal/MIP_10x10_APD.png}
\caption{\small{Charge distribution from readout of the HPS calorimeter crystal with Hamamatsu S8664-55 (left) and S8664-1010 
(right) APDs, and the new low noise amplifier board. The red line histogram corresponds to the charge distribution for all triggers 
comming from the scintillators positioned above and below the crystal. The black line shows the distribution for hits in the crystal 
within $100$ ns of the trigger signal. }}\label{fig:mip10x10}
\end{figure*}

\end{enumerate}


\subsection{Muon system}

\label{sec:muon}


The di-muon decay channel of the A$^\prime$ has the advantage of a greatly reduced electromagnetic background.  In this case, the only particle background in a muon counter would come from photoproduction of $\pi^+$ and $\pi^-$ pairs that are not fully stopped in the ECal or absorber.  A muon detector will match geometrical acceptances of the tracker and ECal, and will be about a cubic meter in size. With such geometrical coverage, efficiency of detecting high mass A$^\prime$s in $\mu^+\mu^-$ decay channel will be higher than for $e^+e^-$ decays, see Figure \ref{fig:muacc}. Expected low background and high efficiency, the di-muon final state is an attractive complement to A$^\prime$ search using $e^+e^-$ final state, and will add substantial territory in the mass and coupling parameter space. With muon system, HPS will be the only experiment proposed to date to search for heavy photons in an alternative to $e^+e^-$ decay mode.

\begin{figure*}[!ht]
\includegraphics[scale=0.4]{muon/acc.pdf}
\caption{\small{A$^\prime$ detection efficiency through $\mu^+\mu^-$ (blue) and $e^+e^-$ (red) decay channels as a function of mass for 6.6 GeV beam energy.}}\label{fig:muacc}
\end{figure*}

The muon system can easily be constructed with layers of scintillator hodoscopes sandwiched between iron absorbers, and can be added downstream from the rest of the HPS apparatus.
The number of layers and the thickness of absorbers is defined by the $\pi/\mu$ rejection factor. The schematic design of the muon detector was optimized using the GEANT-3 model for the ECal with added layers of iron and scintillators.  In the simulation, muons and pions in the momentum range of $1$ to $4$ GeV/c first passed through the 16 cm of lead tungstate in the ECal and then entered a muon counter with various total absorber thicknesses (see \cite{HPS_PROP} for details).  Detection efficiencies for pions ($\epsilon_\pi$) and muons ($\epsilon_\mu$) were then calculated as a function of absorber thickness and particle momentum.  For low-energy particles ($< 1.7$ GeV) detection in all four layers of scintillator hodoscopes was not considered. Depending on the momentum, particles were not traced behind the third, fourth or fifth absorber.  
Figure \ref{fig:pmrej} shows the resulting rejection factor $\epsilon_\pi/\epsilon_\mu$.  The right-hand plot shows the dependence of  $\epsilon_\pi/\epsilon_\mu$ on the total thickness of the iron absorber, with the best rejection at about 75 cm.  The right-hand plot shows $\epsilon_\pi/\epsilon_\mu$ for a 75 cm absorber as a function of muon momentum.  The suppression of individual pions by two orders of magnitude will suppress pion pairs by 4 orders of magnitude.  

\begin{figure*}[!ht]
\includegraphics[scale=0.44]{muon/pmrej.pdf}
\includegraphics[scale=0.44]{muon/pmrej4.pdf}
\caption{\small{Pion-muon rejection factor $\epsilon_\pi/\epsilon_\mu$ versus total iron absorber thickness
(left) and versus particle momentum for a 75 cm absorber (right).}}\label{fig:pmrej}
\end{figure*}


\subsubsection{Conceptual Design}

On the basis of these simulations, we have designed a muon detector composed of four iron absorbers (total length of $30+15+15+15=75$ cm) with a double-layer scintillator hodoscope positioned after each absorber. The muon detector will be mounted behind the ECal.  The front face of the first absorber will be at $\sim 180$ cm from the target. Similar to the Ecal, the muon detector will consist of two halves, one above and one below the beam.  This segmentation is necessary in order to
minimize the effects of the ``sheet-of-flame" that multitude of low-energy particles in the horizontal plane, swept into the detector acceptance by the dipole analyzing magnet.
The vertical gap between the first hodoscope layers of the two halves is about $5$ cm. Dimensions of hodoscopes and absorbers are shown in Table \ref{tb:muon}.  Figure \ref{fig:HPS_view2} shows a CAD
drawing of the HPS detector, with the muon system on the right, which includes the 4 absorbers (gray), the vacuum box (light gray) between the upper and lower sections, and the final set of scintillator paddles (red). The ECal is directly upstream from the muon detector, with its crystals shown in yellow.  In front of the ECal is a large gray vacuum flange.  The silicon tracker is represented by red and gray rectangles and  the red point on the left is the target position.  

\begin{table}[htdp]
\caption{Dimensions (in cm) of muon system scintillation hodoscopes (H) and iron absorbers (A). }
\begin{center}
\begin{tabular}{|c|c|c|c|c|}
\hline
&H1&H2&H3&H4\\
\hline
Distance from target& 212&232&252&272\\
Width&112&125&138.5&152\\
Hight&10.5&11.5&12.5&13.5\\
\hline
\end{tabular}

\begin{tabular}{|c|c|c|c|c|}
\hline
&A1&A2&A3&A4\\
\hline
Distance from target& 207&227&247&267\\
Width&108.5&122&135&148.5\\
Hight&10&11&12&13\\
Thickness & 30 & 15& 15 & 15\\
\hline
\end{tabular}
\end{center}
\label{tb:muon}
\end{table}%


\begin{figure*}[!ht]
\includegraphics[scale=0.22]{muon/HPS_view2.png}
\caption{\small{CAD drawing of the HPS detector setup.  From left to right this consists of the target (red dot), the silicon tracker
(gray and red rectangles), the large shielding wall (gray), the ECal lead tungstate crystals (yellow, two shades), the muon counter absorbers
(gray), and the final muon counter scintillators (red, two shades).}}
\label{fig:HPS_view2}
\end{figure*}

%\begin{figure*}[!ht]
%\includegraphics[scale=0.8]{muon/pmrej4.pdf}
%\caption{\small{Pion-muon rejection factor as a function of the iron absorber thickness.}}\label{fig:pmrejp}
%\end{figure*}

\begin{figure*}[!ht]
\includegraphics[scale=0.22]{muon/Muon2b.png}
\caption{\small{Horizontal scintillator configuration for the muon counter. Scintillators are
shown in red and yellow/brown.  The white/gray structure is the vacuum box.  Each hodoscope layer (top
and bottom) contains three long strips, read out on both ends.
}}
\label{fig:Muon2p}
\end{figure*}

For the hodoscopes we plan to use the same extruded scintillator strips with embedded wavelength-shifting fiber and multi-anode phototube readout as was developed for the CLAS Preshower Calorimeter. These scintillator strips are 45 mm x 10 mm in cross section, and can be cut to any lengths and widths can be reduced as needed for the muon counter.  Each strip contains two, long tunnels, created in the original extrusion process, into which wave-length shifting fibers can be inserted.  Each hodoscope will consist of one x and one y plane.  In Figure \ref{fig:HPS_view2} in two shades vertical strips of the last hodoscope plane is shown. Figure \ref{fig:Muon2p} in different shades horizontal counters of hodoscope planes are shown. The horizontally aligned strip will extend over the length of the detector and will be read out on each end.  The upper and lower hodoscopes in each plane will have their own vertically aligned strips, which will be read out on only the outer end.  The inner end is inaccessible because of the vacuum box, but there is no particular advantage to having a double readout on these short (135 mm) strips.  

The system can be instrumented with 256 readout channels, in which case the requisite electronics will 
fit into a single VME crate.  Signal from each channel (PMT) 
will be sent to a FADC.  We intend to borrow the CLAS12 Preshower Calorimeter electronics and HV system.  Similar to ECal, FADCs will be used to construct a muon trigger for the experiment.  In the current design there will be 3 horizontal strips in each of 8 hodoscope planes (24 total) and a total of 208 vertical strips in 8 hodoscope planes.  The number of vertical strips per plane increases slightly with distance from the target to keep a constant angular coverage.  The maximum is 33 per hodoscope in the back plane.

Full Monte Carlo simulations with realistic event rates are currently underway in order to finalize design details of the muon counter.  The crucial issues are the event rates in the scintillators near the beamline (which already has initiated a redesign of the vacuum chamber to reduce background), the target-to-muon-counter tracking resolution and the detection efficiency.  Any changes to the detector as a result are expected to be minor and will not alter the conceptual design presented here.


\clearpage

\section{Simulation Tools}
\label{app:sim}

The simulation tools play a critical role in simulating the background
environment, optimizing the detector setup, and developing the trigger 
and reconstruction strategies. We use GEANT4 and EGS5 to simulate 
electromagnetic interactions. There is generally good agreement 
between these two codes. In particular, no inconsistencies have been 
found on secondary particle yields or energy spectra. However, we have found 
significant disagreements on the angular distributions in the multiple
scattering, bremsstrahlung and pair production processes.  

%\vspace{1cm}
%\noindent
%{\bf Multiple Scattering}
\subsection*{Multiple Scattering Simulation}

EGS5 simulates the electron elastic scattering using the Moli\`{e}re theory 
\cite{moliere} as formulated by Bethe. \cite{bethe}
It is based on a small angle approximation
($\theta \ll$ 1 radian), and the angular distribution approaches asymptotically
to Gaussian at small angles, and to Rutherford's Coulomb scattering function at 
large angles given by, 
\begin{equation}
F(\theta) \sim  { \frac{1} {\left(1-cos\theta + {\frac{\chi^2} {2}}\right)^2}}.
\label{eq:rutherford}
\end{equation}

Instead of using the complex and time consuming Moli\`{e}re's formula,
GEANT4 uses two functions explicitly, Gaussian at small angles and the
Rutherford function Eq.~\ref{eq:rutherford} at large angles with a requirement that these two
functions and their first derivatives are joined continuously. 
GEANT4, however, uses a different power
in the denominator in Eq.~\ref{eq:rutherford} which is close to 2 but not exactly equal to 2 and is 
dependent on the target material and thickness.

Several comparisons have been made in the angular distribution $F(\theta)$ in the
differential cross section $d\sigma=F(\theta)d(cos\theta) d\phi$ for 2.2 GeV electron
scattering from 0.125\% $X_0$ Tungsten target. 
The EGS5 simulation is compared with Moli\`{e}re's analytical formula 
in Figure \ref{appendix:1}(a), demonstrating a good agreement between EGS5 and
the Moli\`{e}re theory.
While the Moli\`{e}re theory is based on a small angle approximation,
the multiple scattering theory developed by Gaudsmit and Saunderson is valid 
for any angle by means of an expansion in Legendre polynomials. \cite{gs}
The validity of the small angle approximation is checked by comparing the 
Moli\`{e}re integral with 
the Goudsmit-Saunderson theory as shown in Figure \ref{appendix:1}(b),
demonstrating that the Moli\`{e}re theory is accurate in the angular region
of the HPS detector. 

\begin{figure*}[ht]
\includegraphics[height=3 in]{appendix/appendix_1-eps-converted-to.pdf}
\caption{\small{ (a) Moli\`{e}re vs. EGS5 \hspace{1 cm} (b) Moli\`{e}re vs. Goudsmit-Saunderson}}
\label{appendix:1}
\end{figure*}

Figure \ref{appendix:2} shows the angular distribution comparison between the GEANT4 
simulation and the Moli\`{e}re integral. 
GEANT4 is in good agreement with the Moli\`{e}re integral up to about 1 mrad, then it 
deviates at larger angles, predicting roughly twice the cross section at 15 mrad, 
where the HPS tracker sensor edge is located.

D. Attwood et al. measured 170 MeV muon angular distributions and compared with 
GEANT4 simulations and the Moli\`{e}re theory. \cite{attwood} They concluded that GEANT4 
simulation over-estimated the scattering tail by about a factor of two, and the data were consistent
with the Moli\`{e}re theory. G. Shen et al.~ \cite{shen} and B. Gottschalk et al.~ \cite{gottschalk}
also showed that the Moli\`{e}re theory was consistent with the measurements on a wide variety of
target materials.

\begin{figure}[ht]
\includegraphics[height= 3 in]{appendix/appendix_2-eps-converted-to.pdf}
\caption{\small{ Moli\`{e}re vs. GEANT4 }}
\label{appendix:2}
\end{figure}

%\vspace{1cm}
%\noindent
%{\bf Angular distributions in the bremsstrahlung and pair production processes}
\subsection*{Angular Distributions}

While GEANT4 and EGS5 are in good agreement in the production rates and the secondary particle
energy spectra, there are significant differences in the angular distribution in the secondary
particles. In EGS5, the angular distributions are sampled from the following differential
cross section for the bremsstrahlung process, \cite{koch}

$$d\sigma(k,\theta_\gamma) = {\frac{4Z^2r_0^2} {137}} {\frac{dk} {k}} ydy\left\{{\frac{16y^2E} 
{(y^2+1)^4E_0}}
 -{\frac{(E_0+E)^2} {(y^2+1)^2E_0^2}}+\left\{{\frac{E_0^2+E^2} {(y^2+1)^2E_0^2}} -
 {\frac{4y^2E} {(y^2+1)^4E_0}}\right\} lnM(y) \right\}, $$

\noindent
where $k$ is photon energy, $\theta_\gamma$ is photon polar angle, $E_0$ and $E$ are initial and final 
electron energy, and

$$y=E_0\theta_\gamma; {\frac{1} {M(y)}} = \left({\frac{k} {2E_0E}}\right)^2 + \left({\frac{Z^{1/3}} {111(y^2+1)}}\right)^2, $$

\noindent
and for the pair production process, \cite{motz}

$${\frac{d\sigma} {dE_\pm d\Omega_\pm}} = {\frac{2\alpha Z^2r_0^2} {\pi}} {\frac{E_\pm^2} {k^3}}
\left\{-{\frac{(E_+-E_-)^2} {(u^2+1)^2}}-{\frac{16u^2E_+E_-} {(u^2+1)^4}} + \left\{ {\frac{E_+^2+E_-^2} 
{(u^2+1)^2}} + {\frac{4u^2E_+E_-} {(u^2+1)^4}} \right\} lnM(u)\right\},$$

\noindent
where $k$ photon energy, $E_\pm$ $e^{\pm}$ energy, $\theta_\pm$ $e^{\pm}$ polar angle, and

$$u=E_\pm\theta_\pm; {\frac{1} {M(u)}} = \left({\frac{k} {2E_+E_-}}\right)^2+\left({\frac{Z^{1/3}} {111(u^2+1)}}\right)^2.$$

\noindent
GEANT4 uses an approximate function to simulate the angular distributions in the 
bremsstrahlung and pair production processes given by

$$ f(u) = C [ue^{-au} + d u e^{-3au}], $$

\noindent
with $u=E_0\theta_\gamma$ for incident electron energy $E_0$ and the polar angle 
$\theta_\gamma$ of the bremsstrahlung photon, and $u=E_{\pm}\theta_{\pm}$ for the pair 
energy $E_\pm$ and polar angle $\theta_\pm$ in the pair production. Since the production angle
is typically $1/\gamma$, GEANT4's approximations are acceptable
for most of the high energy detector simulations. However, GEANT4
simulations are inconsistent with the data in the following two cases in the HPS Test Run:
\begin{itemize}
\item The bremsstrahlung photon angular distribution is too narrow, 
resulting in too few scatters in the collimator.
\item The prediction on the pair angular distribution is too narrow, resulting in 
too few Ecal trigger rates.
\end{itemize}

%\vspace{1cm}
%\noindent
%{\bf Conclusions}
\subsection*{Simulation Tools Setup in HPS}

Because of the inaccuracies in GEANT4 described above the electromagnetic interactions in the target are simulated 
by EGS5, and all the particles that come out of the target are passed on to the HPS detector 
simulation system based on GEANT4.


%\section{Production and Decay of the $A^\prime$}
\label{app:ProdAndDecay}

\def \ap {A^\prime}
\def \map {m_{A^\prime}}
\def \thap {\theta_{A^\prime}}


$\ap$ particles are generated in electron collisions on a fixed target by a process analogous to ordinary photon bremsstrahlung, see Figure \ref{fig:apdiagram}.  This can be reliably estimated in the Weizsäcker-Williams approximation (see [1-4]).  When the incoming electron has energy $E_0$, the differential cross-section to produce an $\ap$ of mass $m_{\ap}$ with energy $E_{\ap}\equiv x E_0$ is 
\begin{equation}
\frac{d\sigma}{dxd\cos{\theta_{\ap}}}\approx \frac{8Z^2\alpha^3\epsilon^2 E_0^2 x}{U^2}\tilde{\chi}\times\left[\left(1-x+\frac{x^2}{2}\right)-\frac{x(1-x)m_{\ap}^2E_0^2x\theta_{\ap}^2}{U^2}\right]
\end{equation}
where Z is the atomic number of the target atoms, $\alpha = 1/137$,  is the angle in the lab frame between the emitted A' and the incoming electron, 
\begin{equation}
U(x,\theta_{\ap})=E_0^2x\theta_{\ap}^2+m_{\ap}^2\frac{1-x}{x}+m_e^2x
\label{eq:u}
\end{equation}
is the virtuality of the intermediate electron in initial-state bremsstrahlung, and  is the Weizsacker-Williams effective photon flux, with an overall factor of  removed.  The form of  and its dependence on the $\ap$ mass, beam energy, and target nucleus are discussed in \cite{Kim:1973he}.  For HPS with $E_0$ = 6.6 GeV, we find $\tilde{\chi}\sim 7 (4, 1)$ for $m_{\ap}$ = 100 (200, 500) MeV/$c^2$.
The above results are valid for 
\begin{equation}
m_e\ll m_{\ap}\ll E_0  , ~~ x\theta_{\ap}^2\ll 1.
\end{equation}
For $m_e\ll m_{\ap}$, the angular integration gives
\begin{equation}
\frac{d\sigma}{dx}\approx \frac{8Z^2\alpha^3\epsilon^2 x}{m_{\ap}^2}\left(1+\frac{x^2}{3(1-x)}\right)\tilde{\chi} .
\end{equation}

Assuming the $\ap$ decays into Standard Model particles rather than exotic, it's boosted lifetime is
\begin{equation}
l_0 \equiv \gamma c\tau \approx \frac{0.8 cm}{N_{eff}} \left(\frac{E_0}{10 GeV}\right)\left(\frac{10^{-4}}{\epsilon}\right)^2\left(\frac{100 MeV}{\map}\right)^2,
\end{equation}
where we have neglected phase-space corrections, and $N_{eff}$ counts the number of available decay channels.  If the $\ap$ couples only to electrons, then $N_{eff}=1$.  If the $\ap$ mixes kinetically with the photon, the $N_{eff}=1$ for $\map < 2m_\mu$ and $2+R(\map)$ for $\map \geq 2 m_\mu$, where \cite{eehadrons}
\begin{equation}
R  =\left. \frac{\sigma(e^+e^-\rarr hadrons)}{\sigma (e^+e^- \rarr \mu^+\mu^-)}\right|_{E=\map} . 
\end{equation} 
For the ranges of $\epsilon$ and $\map$ probed by this experiment, the mean decay length $l_0$ can be prompt or as large as tens of centimeters.  

The total number of $\ap$ produced when $N_e$ electrons scatter in a target of $T\ll 1 $ radiation lengths is
\begin{equation}
N\sim N_e\frac{N_0 X_0}{A}T\frac{Z^2\alpha^2\epsilon^2}{\map^2}\tilde{\chi}\sim N_e C T \frac{\epsilon m_e^2}{\map^2},
\end{equation}
where $X_0$ is the radiation length of the target in g/cm$^2$, $N_0 \approx 6\times 10^{23} mole^{-1}$ is Avogadro's number, and $A$ is the target atomic mass in g/mole.  The numerical factor $C\approx 5$ is logarithmically dependent on the choice of nucleus (at least in the range of masses where the form-factor is only slowly varying) and on $\map$, because, roughly, $X_0 \propto \frac{A}{Z^2}$ (see \cite{Kim:1973he, Essig:2010xa,Bjorken:2009mm}).  For a Coulomb of incident
electrons, the total number of $\ap$s produced is given by
\begin{equation}
N\sim 10^5 \left(\frac{N_e}{1 C}\right)\tilde{\chi}\left(\frac{T}{0.1}\right)\left(\frac{\epsilon}{10^{-4}}\right)^2\left(\frac{100 MeV}{\map}\right)^2.
\end{equation}



% \section{Test Run SVT Performance}
% \label{app:svt}
% \subsection*{SVT Calibration}

% 
%
%   svt_calibrations.tex
%       author: Omar Moreno <omoreno1@ucsc.edu>
%               Per Hansson <phansson@slac.stanford.edu>
%
%

In order to prepare the SVT for real physics data-taking, the SVT was 
calibrated. This involved the extraction of the mean baseline (pedestal),
baseline noise and gain for each of the 12,780 SVT channels. All measurements
were made with the APV25 readout chips configured to their nominal operating
points \cite{Jones:1069892} and all sensors biased to 180 V. The APV25s were
operated in ``mulit-peak'' mode with six samples being readout per trigger.
This, in turn, allowed for the extraction of the $t_0$ and amplitude of the 
signals being read out.

Figure~\ref{fig:pedestal_noise} shows an intensity plot of the pedestals 
along with the readout noise as a function of channel number for a single
hybrid.  The noise was computed by taking the RMS of the gaussian distributed
\begin{figure}[h]
    \begin{center}
    	\includegraphics[width=0.45\textwidth]{test2012/svtperformance/baseline}
    	\includegraphics[width=0.45\textwidth]{test2012/svtperformance/gain}
        \caption{Something ... }
%    	\caption{\small{The baseline across a hybrid (left) and the measured response as a function of 
%	                    input charge (right). The overall shifts in the baseline are calibrated out where distinct edges 
%	                    are associated with the five APV25 chips on the hybrid. The gain shows good linearity up to 
%	                    about three $mip$s.} {\color{red}Should we show noise instead of baseline?}}
	\label{fig:pedestal_noise}
    \end{center}
\end{figure}
pedestals for each of the channels and was observed to be consistently within 
[Find number] ADC counts ( electrons).  One observed feature are the large
noise values for the channels lying near the chip edges.  This has also been 
reported by the CMS collaboration and the cause is still under investigation.

%  Need to rewrite this ...
Another important aspect for the characterization of the SVT is the response and the associated 
gain. Using the APV25 internal calibration circuit a known fixed charge was injected into all 
channels of the which allows for an accurate determination of the response and its 
scaling with input charge, shown in Fig.~\ref{fig:baseline_and_gain}. The gain uniformity was 
within the expected range across chips and modules and show good linearity of charge 
depositions up to about 3 $mip$s. 

All reconstructed hits in an event were used to form clusters of energy 
depositions using a nearest neighbor algorithm. Fig.~\ref{fig:cluster_pulse}
shows the mean pulse shape of each of the hits associated with a track as a 
function of time.  
\begin{figure}[h]
	\includegraphics[width=\textwidth]{test2012/svtperformance/pulseshape_and_landau}
    \caption{Something else ...}
%	\caption{\small{The distribution of cluster amplitudes (left) showing the characteristic Landau 
%	shape and the pulse shape from the six samples readout (right) {\color{red} Remove one of the pulse shapes}. }}
	\label{fig:pulseshape}
\end{figure}
% Should this be included? If so, it needs to be rewritten a little better
The figure also demostrates that  the trigger system, described below, is well 
timed in with the tracker. 
Fig.~\ref{fig:cluster_pulse} shows the MIP response to be [Find value] electrons.
Taking the MIP response, the signal to noise ratio was calculated to be 
approximately 25.5 which is well matched to the expected behavior.

\bibliography{svt_calib}
 


% \subsection*{SVT Hit Time Resolution}
% As discussed in Sec.~\ref{sec:svt}, the APV25 multi-peak readout is crucial to the time stamping on hits
% in the SVT. This, in turn, allows the effective occupancies to be lowered for pattern recognition
% during electron running. 
% Six samples of the APV25 shaper output for each trigger are fitted to an ideal $CR-RC$ function to 
% extract the amplitude and hit time.  The $\chi^2$ distribution of these fits from test run data is as expected
% for four degrees of freedom.
% %\begin{figure}[]
% %	\includegraphics[width=0.6\textwidth]{test2012/svtperformance/apvfit_chisq}
% %	\caption{\small{Histogram of $\chi^2$ values for pulse fits for all channels on a representative sensor. The peak at 2 is consistent with 4 degrees of freedom (2 fit parameters), as expected. Pileup was not considered due to the very low hit rate in 
% %the SVT in this photon beam test. } }
% %	\label{fig:apvfit}
% %\end{figure}
% After clustering hits on a sensor, the hit time for each cluster is computed as the 
% amplitude-weighted average of the channel hit times. Since we have no measurement of the ``true'' hit time, we study the overall SVT hit 
% timing performance using the average of all cluster times in a track as the ``track time,'' and take the
%  residual of the cluster time relative to that. The observed track time, shown in Fig.~\ref{fig:tracktime}, has the expected amount of trigger jitter due to the readout clock and trigger system jitter. After correcting for offsets for each sensor (time-of-flight, clock phase) the RMS 
%  of the final residual distribution is roughly 2.4~ns for each sensor. 
% Because the track time is calculated using the individual hit times, the hit time is positively correlated 
% with the track time; thus the RMS of the residual is slightly smaller than the true time resolution.
% The standard deviation of this residual for $n$-hit tracks where all hits have the same time resolution 
% is reduced by a factor of $\sqrt{(n-1)/n}$; since most of our tracks have 8 clusters, the true time 
% resolution is 2.6 ns. 
% %\begin{figure}[ht]
% %	\includegraphics[width=\textwidth]{test2012/svtperformance/timeres}
% %	\caption{\small{Histogram and Gaussian fit of residual of cluster times for a representative sensor, relative to the track time. Because the cluster times and track time have positive covariance, the true time resolution is slightly larger than the standard deviation shown here.} }
% %	\label{fig:timeres}
% %\end{figure}
% %\begin{figure}[ht]
% %	\includegraphics[width=\textwidth]{test2012/svtperformance/hit_dt}
% %	\caption{\small{} }
% %	\label{fig:hit_dt}
% %\end{figure}
% This is somewhat worse than the $\approx 2$ ns resolution expected 
% %(see Fig.~\ref{fig:timeres}) 
% in 
% Sec.~\ref{sec:performance}, but we believe this discrepancy is due to our fit function. Our pulse 
% shape fit assumes an ideal CR-RC pulse shape; since the actual pulse shape has a slower rise time, 
% there is a systematic pull on the hit time when a hit comes immediately before the APV clock time. 
% This is visible in Figure \ref{fig:timeres_2D} as a shift in the residual at certain values of track time.
% \begin{figure}[ht]
% 	\includegraphics[width=0.7\textwidth]{test2012/svtperformance/timeres_2D}
% 	\caption{\small{Plot of the time residual for a representative sensor vs. the track time. 
% 		The kinks in the horizontal band are caused by the fitter; without them the time resolution (measured by taking the projection of this histogram) would be better.} }
% 	\label{fig:timeres_2D}
% \end{figure}
% Work is in progress to use the actual pulse shape in time reconstruction; this should improve time resolution to that expected from previous studies. 
% Reducing the APV25 pulse shaping time will also improve time resolution.
% In short, these results show that we can achieve time resolution adequate for pileup rejection during electron running.
% %\vspace{1cm}{\bf Tracking algorithms [Matt/Omar]}


% \subsection*{SVT Track Reconstruction}
% 
%
%   trk_performance.tex
%       author: Omar Moreno <omoreno1@ucsc.edu>
%      created: December 4, 2012
%
 
Clustered hits in Si planes within each layer are combined to form
2-dimensional ``stereo hits'' which are used by the tracking algorithm to 
form tracks.  The determination of the probability that a stereo hit is 
formed, or hit efficiency, provides insight as to the performance of each of 
the SVT layers.

In order to obtained the hit efficiencies, tracks were fitted using only 4 of 
SVT layers. The resulting track was then extrapolated to the layer omitted
\begin{figure}[h]
        % TODO: Need to update the plot so that Layer 2 on the bottom doesn't 
        % look terrible. It's probably easiest to just remove the point altogether.
    	\includegraphics[width=0.95\textwidth]{test2012/svtperformance/trk_performance/hit_efficiency_vs_layer.pdf}
    	%\includegraphics[width=0.49\textwidth]{test2012/svtperformance/trk_performance/track_reco_efficiency.pdf}
        \caption{{\small
                    Average hit efficiency, excluding known bad channels, 
                    from all dedicated photon runs as a function of layer
                    number.  
                }} 
	\label{fig:hit_track_efficiency}
\end{figure}
from the fit. If the track was found to lie within the sensitive volume
of the layer, a search for a stereo hit within the layer acceptance was 
conducted.  The hit efficiency was then determined as
\[
    \varepsilon_{\mbox{hit efficiency}} = \frac{\mbox{Tracks with hit on missing layer}}
                                            {\mbox{Tracks within layer acceptance}}.
\]
The hit efficiencies per layer were calculated using all dedicated runs. It must 
be noted that those tracks found to intersect bad channels as well as those that 
lie on the sensor edges were excluded from the calculation. As 
can be seen from Figure~\ref{fig:hit_track_efficiency}, the average hit efficiency
per layer was found to be $\approx$ 99\%. 

%95\% for the top SVT volume and greater than 92\% for the bottom volume.  The 
%larger hit inefficiencies observed for some layers was simply due to the 
%sensors composing the layer having a greater number of noisy channels.

%As mentioned above, the standard pattern recognition algorithm is designed to 
%find tracks using the reconstructed stereo hits.  A set of tracking strategies
%outline which layers should be used by the track finding algorithm along
%with their role (seeding, extend, confirm), and the $\chi^2$ cut imposed 
%on the fit. Any kinematic constraints are also specified within the strategy.
%A detailed account of the tracking algorithm can be found  here~\ref{}.

%% Track reco efficiency is no longer going to be included so remove this
%% section
%In order to determine the efficiency of the track finding algorithm, a Monte
%Carlo sample containing pair produced electrons from photons incident on 
%a 1.6\% $X_0$ gold target was used.  The energy of the pair produced electrons
%ranged from .5 GeV to 5.5 GeV. An electron falling within the detector 
%acceptance was considered findable by the tracking algorithm if it 
%traversed through at least 4 of the SVT layers. The track reconstruction
%efficiency was then determined as
%\[
%    \varepsilon_{\mbox{track reco efficiency}} = \frac{\mbox{Tracks found}}
%                                            {\mbox{Tracks found to be findable}}.
%\]
%The resulting efficiency to find an electron which passes through the detector
%acceptance is shown in Figure~\ref{fig:hit_track_efficiency}. The average track
%reconstruction efficiency was found to be \% with the bulk of the inefficiency
%coming from the $\chi^2$ cut imposed on the fit to the six samples during
%the clustering stage.

All events containing pairs of oppositely charged tracks were fit to a
common vertex using a simple vertexing algorithm which searches for the distance
of closest approach between the two tracks.  The reconstructed vertex position
along the beam axis for both data and Monte Carlo is shown on 
Figure~\ref{fig:vz_position}.
\begin{figure}[h]
    \begin{center}
    	\includegraphics[width=0.60\textwidth]{test2012/svtperformance/trk_performance/zvertex.pdf}
        \caption{  
                    The reconstructed vertex position along the beam axis for
                    both data (blue) and Monte Carlo (red).
                } 
	\label{fig:vz_position}
    \end{center}
\end{figure}
Because of the geometric setup of the SVT, observation of pairs produced by
the incident photon required both electrons to experience a hard scatter
within the target.  This results in a broadening of the Gaussian 
distributed reconstructed vertex position. 



% \subsection*{SVT Test Run DAQ Performance}
% 
%
%   svt_daq.tex
%       author: Omar Moreno <omoreno1@ucsc.edu>
%               Santa Cruz Institute for Particle Physics
%               University of California, Santa Cruz
%      created: November 13, 2012
%

The expected data rates and event sizes for each of the dedicated photon runs
were estimated using a full simulation of the SVT DAQ and compared to observed
values. As discussed in Section~\ref{sec:testrun_daq}, the digitized samples
from three hybrids were received by a single DPM.  The DPM then required that
at least three of the six samples exceeded a threshold of two times the noise
level for that channel.  An additional ``pile-up'' cut requiring that 
(sample 2 $>$ sample 1) or (sample 3 $>$ sample 2) was also applied. This was
meant to eliminate hits arising from the falling edge of previous hits expected
to occur when running at the highest occupancies.
%Signals from the photon run were unaffected by such a cut. 

All samples were placed into their own container along with the 
channel number, hybrid identifier, chip address and DPM identifier. An 
additional layer of encapsulation or bank was used to store all samples 
emerging from a single DPM along with the DPM identifier, the event number
an error bit and hybrid temperatures. A diagram of the container along with
the sizes of each of the elements is shown on Figure~\ref{fig:data_format}.
\begin{figure}[h]
    \begin{center}
    	\includegraphics[width=0.60\textwidth]{test2012/svtperformance/daq/svt_data_format.pdf}
        \caption{
                    SVT data format. Samples readout from three hybrids are 
                    processed by a single FPGA and are placed within a single
                    container or FPGA bank.  An additional layer of 
                    encapsulation is used to store all of the FPGA banks.
                 } 
	\label{fig:data_format}
    \end{center}
\end{figure}
Overall, the container overhead will contribute a total of 326 bytes to an event
with an additional 16 bytes per hit.

The observed occupancy expected for each of the converter thicknesses along with the 
corresponding data rate are shown on Figure~\ref{fig:data_rates}.
\begin{figure}[h]
    \begin{center}
    	\includegraphics[width=0.49\textwidth]{test2012/svtperformance/daq/n_dead_channels.pdf}
    	\includegraphics[width=0.49\textwidth]{test2012/svtperformance/daq/data_rates.pdf}
        \caption{
                    The plot on the left shows the percentage of bad/noisy channels observed
                    during each of the runs (green) along with the number of noisy/misconfigured 
                    readout chips (blue).  Most of the noisy channels present during runs were
                    due to misconfigured chips.  The  plot on the right shows the occupancies (blue)
                    and data rates (black) observed for each of the targets used.  Once all noisy channels
                    are masked, the observed data rates agree well with those predicted by the
                    SVT readout simulation. 
                } 
	\label{fig:data_rates}
    \end{center}
\end{figure}
The data rates observed during the test run were
much higher than expected.  This can be attributed to a known
noisy sensor and a few noisy chips which appeared during certain runs as can be seen
on Figure~\ref{fig:data_rates}.  The causes
of both these issues are now well understood and will be resolved for future running.

%Table~\ref{table:sim_rates}.
%\begin{table}[h]
%    \scalebox{0.9}{
%    \begin{tabular}{ c | c | c | c }
%    \hline
%
%    Converter Thickness (\%$X_0$) & Sim Occupancy (\%)  & Sim Event Size (kB) &   Sim Data Rate (Mb/s) \\      
%    \hline 
%   1.6                           & .438                & 1.22                &   2.07                 \\
%   0.45                          & .293                & .93                 &   .53                  \\
%    0.18                          & .118                & .56                 &   .24                  \\ 
%    \end{tabular} } 
%    \caption{Occupancy, event size and resulting data rate expected for each of the three 
%             converter thicknesses used in the test run.}
%    \label{table:sim_rates}
%\end{table}

A better comparison between simulated and observed data rates can be obtained
by masking out all known noisy channels found during the commissioning of the 
SVT along with the noisy chips.  The resulting occupancies and data rates are also shown on 
Figure~\ref{fig:data_rates}. The simulated occupancies shown include a contribution
of 0.02\% (3 hits) due to noise and the data rates are estimated using the trigger
rates observed during each of the dedicated photon runs.  As can be seen from the
figure, the occupancies and data rates after most bad channels are masked are well
in agreement with those predicted by Monte Carlo.
%Table~\ref{table:observed_rates} list the observed occupancy, event sizes and 
%\begin{table}[h]
%    \scalebox{0.9}{ 
%        \begin{tabular}{ c | c | c | c }
%            \hline
%            Converter Thickness (\%$X_0$)   & Obs. Occupancy (\%) & Obs. Event Size (kB) & Obs. Data Rate (Mb/s) \\
%            \hline
%            1.6                             & 1.03                & 2.43                 & 4.12                  \\
%            0.45                            & 1.22                & 2.82                 & 1.61                  \\
%            0.18                            & 1.23                & 2.84                 & 1.21                  \\
%        \end{tabular}
%    }
%    \caption{Occupancy, event size and resulting data rate observed for each of the three 
%             converter thicknesses used in the test run.}
%    \label{table:observed_rates}
%\end{table}    


\section{Test Run ECal Calibration}
\label{sec:ecal_calib}

The noise and pedestal of the readout chain are calibrated by running the ECal readout in a mode where the preamplifier output is sampled every 4~ns in a time window of 100 samples: by looking at a part of the window before the hit, we calibrate the readout channel.

We calibrate gain of the individual ECal channels using the SVT measurement of track momentum and comparison to Monte Carlo simulation. 
% The ratio of cluster energy to track momentum is calculated both for Monte Carlo simulation and test run data at each point in the ECal, and we find the value of gain for each channel that brings the two into agreement.
We disable all SVT and ECal channels in the simulation that were inoperable or noisy in the test run, so any efficiency or bias effects that affect the real data should be reflected in the simulation as well; then we use a formula to compute the ``weighted E/p'' for a crystal, representing the average E/p for clusters that include the crystal: $\frac{\sum_j w_{j,i}}{\sum_j\frac{P_j}{E_j}w_{j,i}}$, and iteratively adjust the gains until the weighted E/p is equal for test run data and simulation.

% The calibrated gains are corrected by the ratio between the weighted E/p values from Monte Carlo and real data.
% The E/p in Monte Carlo data is also affected by the gain because the trigger thresholds change, so both Monte Carlo and data reconstruction are rerun with each iteration of gain calibration.
% It takes up to 4 iterations for the gains to stabilize; the final values are shown in Figure \ref{fig:gains}.
\begin{figure}[ht]
	\includegraphics[width=0.45\textwidth]{test2012/ecalperformance/ecalgainplots_corr_sim}
	\includegraphics[width=0.45\textwidth]{test2012/ecalperformance/gains}
	\caption{\small{Weighted E/p from Monte Carlo simulation (left), calibrated values of gain in units of MeV per ADC count (right).}}
	\label{fig:gains}
\end{figure}

These gains can then be used to convert from ADC counts in a channel to the energy deposited into that ECal crystal.
The other information needed to find the energy of an incident particle is the sampling fraction---the ratio of energy read out from crystals to energy of an incident particle.
The conventional sampling fraction---the fraction of incident energy that is deposited in crystals---is approximately 0.9 for our ECal, and less at edges.
For our readout, there is additional energy lost because crystals under the readout threshold are not read out.
The weighted E/p used in calibration (see Figure \ref{fig:gains}) is an approximate measurement of sampling fraction, but the sampling fraction is energy-dependent because of the effect of readout threshold. 
A full computation of sampling fraction can be done using simulation.
