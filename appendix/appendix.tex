\appendix
\section{Simulation tools}

The simulation tools play a critical role in simulating the background
environment, optimizing the detector setup, and developing the trigger 
and reconstruction strategies. We use GEANT4 and EGS5 to simulate 
electromagnetic interactions. There is generally good agreement 
between these two codes. In particular, no inconsistencies have been 
found on secondary particle yields or energy spectra. However, we have found 
significant disagreements on the angular distributions in the multiple
scattering, bremsstrahlung and pair production processes.  

\vspace{1cm}
\noindent
{\bf Multiple Scattering}

EGS5 simulates the electron elastic scattering using the Moli\`{e}re theory 
\cite{moliere} as formulated by Bethe. \cite{bethe}
It is based on a small angle approximation
($\theta \ll$ 1 radian), and the angular distribution approaches asymptotically
to Gaussian at small angles, and to Rutherford's Coulomb scattering function at 
large angles given by, 

$$ F(\theta) \sim  { 1 \over {(1-cos\theta + {\chi^2 \over 2})^2}}.    \hspace{2 cm} (1) $$

Instead of using the complex and time consuming Moli\`{e}re's formula,
GEANT4 uses two functions explicitly, Gaussian at small angles and the
Rutherford function Eq. (1) at large angles with a requirement that these two
functions and their first derivatives are joined continuously. 
GEANT4, however, uses a different power
in the denominator in Eq. (1) which is close to 2 but not exactly equal to 2 and is 
dependent on the target material and thickness.

Several comparisons have been made in the angular distribution $F(\theta)$ in the
differential cross section $d\sigma=F(\theta)d(cos\theta) d\phi$ for 2.2 GeV electron
scattering from 0.125\% $X_0$ Tungsten target. 
The EGS5 simulation is compared with Moli\`{e}re's analytical formula 
in Figure \ref{appendix:1}(a), demonstrating a good agreement between EGS5 and
the Moli\`{e}re theory.
While the Moli\`{e}re theory is based on a small angle approximation,
the multiple scattering theory developed by Gaudsmit and Saunderson is valid 
for any angle by means of an expansion in Legendre polynomials. \cite{gs}
The validity of the small angle approximation is checked by comparing the 
Moli\`{e}re integral with 
the Goudsmit-Saunderson theory as shown in Figure \ref{appendix:1}(b),
demonstrating that the Moli\`{e}re theory is accurate in the angular region
of the HPS detector. 

\begin{figure*}[h]
\includegraphics[height=3 in]{appendix/appendix_1.eps}
\caption{\small{ (a) Moli\`{e}re vs. EGS5 \hspace{1 cm} (b) Moli\`{e}re vs. Goudsmit-Saunderson}}
\label{appendix:1}
\end{figure*}

Figure \ref{appendix:2} shows the angular distribution comparison between the GEANT4 
simulation and the Moli\`{e}re integral. 
GEANT4 is in good agreement with the Moli\`{e}re integral up to about 1 mrad, then it 
deviates at larger angles, predicting roughly twice the cross section at 15 mrad, 
where the HPS tracker sensor edge is located.

D. Attwood et al. measured 170 MeV muon angular distributions and compared with 
GEANT4 simulations and the Moli\`{e}re theory. \cite{attwood} They concluded that GEANT4 
simulation over-estimated the scattering tail by about a factor of two, and the data were consistent
with the Moli\`{e}re theory. G. Shen et al.~ \cite{shen} and B. Gottschalk et al.~ \cite{gottschalk}
also showed that the Moli\`{e}re theory was consistent with the measurements on a wide variety of
target materials.

\begin{figure}[h]
\includegraphics[height= 3 in]{appendix/appendix_2.eps}
\caption{\small{ Moli\`{e}re vs. GEANT4 }}
\label{appendix:2}
\end{figure}

\vspace{1cm}
\noindent
{\bf Angular distributions in the bremsstrahlung and pair production processes}

While GEANT4 and EGS5 are in good agreement in the production rates and the secondary particle
energy spectra, there are significant differences in the angular distribution in the secondary
particles. In EGS5, the angular distributions are sampled from the following differential
cross section for the bremsstrahlung process, \cite{koch}

$$d\sigma(k,\theta_\gamma) = {{4Z^2r_0^2} \over 137} {dk \over k} ydy\{{{16y^2E} \over 
{(y^2+1)^4E_0}}
 -{{(E_0+E)^2} \over {(y^2+1)^2E_0^2}}+\{{{E_0^2+E^2} \over {(y^2+1)^2E_0^2}} -
 {{4y^2E} \over {(y^2+1)^4E_0}}\} lnM(y) \}, $$

\noindent
where $k$ photon energy, $\theta_\gamma$ photon polar angle, $E_0$ and $E$ are initial and final 
electron energy, and

$$y=E_0\theta_\gamma; {1 \over {M(y)}} = ({k \over {2E_0E}})^2 + ({{Z^{1/3}} \over {111(y^2+1)}})^2, $$

\noindent
and for the pair production process, \cite{motz}

$${{d\sigma} \over {dE_\pm d\Omega_\pm}} = {{2\alpha Z^2r_0^2} \over \pi} {E_\pm^2 \over k^3}
\{-{{(E_+-E_-)^2} \over {(u^2+1)^2}}-{{16u^2E_+E_-} \over {(u^2+1)^4}} + \{ {{E_+^2+E_-^2} \over
{(u^2+1)^2}} + {{4u^2E_+E_-} \over {(u^2+1)^4}} \} lnM(u)\},$$

\noindent
where $k$ photon energy, $E_\pm$ $e^{\pm}$ energy, $\theta_\pm$ $e^{\pm}$ polar angle, and

$$u=E_\pm\theta_\pm; {1 \over {M(u)}} = ({k \over {2E_+E_-}})^2+({Z^{1/3} \over {111(u^2+1)}})^2.$$

\noindent
GEANT4 uses an approximate function to simulate the angular distributions in the 
bremsstrahlung and pair production processes given by

$$ f(u) = C [ue^{-au} + d u e^{-3au}], $$

\noindent
with $u=E_0\theta_\gamma$ for incident electron energy $E_0$ and the polar angle 
$\theta_\gamma$ of the bremsstrahlung photon, and $u=E_{\pm}\theta_{\pm}$ for the pair 
energy $E_\pm$ and polar angle $\theta_\pm$ in the pair production. Since the production angle
is typically $1/\gamma$, GEANT4's approximations are acceptable
for most of the high energy detector simulations. However, GEANT4
simulations are inconsistent with the data in the following two cases in the HPS Test Run,

1) GEANT4 prediction on the bremsstrahlung photon angular distribution is too narrow, 
resulting in too few collimator scattering, and

2) GEANT4 prediction on the pair angular distribution is too narrow, resulting in 
too few Ecal trigger rates.

\vspace{1cm}
\noindent
{\bf Conclusions}

Because of these inaccuracies in GEANT4 the electromagnetic interactions in the target are simulated 
by EGS5, and all the particles that come out of the target are passed on to the HPS detector 
simulation system based on GEANT4.


%\section{Production and Decay of the $A^\prime$}
\label{app:ProdAndDecay}

\def \ap {A^\prime}
\def \map {m_{A^\prime}}
\def \thap {\theta_{A^\prime}}


$\ap$ particles are generated in electron collisions on a fixed target by a process analogous to ordinary photon bremsstrahlung, see Figure \ref{fig:apdiagram}.  This can be reliably estimated in the Weizsäcker-Williams approximation (see [1-4]).  When the incoming electron has energy $E_0$, the differential cross-section to produce an $\ap$ of mass $m_{\ap}$ with energy $E_{\ap}\equiv x E_0$ is 
\begin{equation}
\frac{d\sigma}{dxd\cos{\theta_{\ap}}}\approx \frac{8Z^2\alpha^3\epsilon^2 E_0^2 x}{U^2}\tilde{\chi}\times\left[\left(1-x+\frac{x^2}{2}\right)-\frac{x(1-x)m_{\ap}^2E_0^2x\theta_{\ap}^2}{U^2}\right]
\end{equation}
where Z is the atomic number of the target atoms, $\alpha = 1/137$,  is the angle in the lab frame between the emitted A' and the incoming electron, 
\begin{equation}
U(x,\theta_{\ap})=E_0^2x\theta_{\ap}^2+m_{\ap}^2\frac{1-x}{x}+m_e^2x
\label{eq:u}
\end{equation}
is the virtuality of the intermediate electron in initial-state bremsstrahlung, and  is the Weizsacker-Williams effective photon flux, with an overall factor of  removed.  The form of  and its dependence on the $\ap$ mass, beam energy, and target nucleus are discussed in \cite{Kim:1973he}.  For HPS with $E_0$ = 6.6 GeV, we find $\tilde{\chi}\sim 7 (4, 1)$ for $m_{\ap}$ = 100 (200, 500) MeV/$c^2$.
The above results are valid for 
\begin{equation}
m_e\ll m_{\ap}\ll E_0  , ~~ x\theta_{\ap}^2\ll 1.
\end{equation}
For $m_e\ll m_{\ap}$, the angular integration gives
\begin{equation}
\frac{d\sigma}{dx}\approx \frac{8Z^2\alpha^3\epsilon^2 x}{m_{\ap}^2}\left(1+\frac{x^2}{3(1-x)}\right)\tilde{\chi} .
\end{equation}

Assuming the $\ap$ decays into Standard Model particles rather than exotic, it's boosted lifetime is
\begin{equation}
l_0 \equiv \gamma c\tau \approx \frac{0.8 cm}{N_{eff}} \left(\frac{E_0}{10 GeV}\right)\left(\frac{10^{-4}}{\epsilon}\right)^2\left(\frac{100 MeV}{\map}\right)^2,
\end{equation}
where we have neglected phase-space corrections, and $N_{eff}$ counts the number of available decay channels.  If the $\ap$ couples only to electrons, then $N_{eff}=1$.  If the $\ap$ mixes kinetically with the photon, the $N_{eff}=1$ for $\map < 2m_\mu$ and $2+R(\map)$ for $\map \geq 2 m_\mu$, where \cite{eehadrons}
\begin{equation}
R  =\left. \frac{\sigma(e^+e^-\rarr hadrons)}{\sigma (e^+e^- \rarr \mu^+\mu^-)}\right|_{E=\map} . 
\end{equation} 
For the ranges of $\epsilon$ and $\map$ probed by this experiment, the mean decay length $l_0$ can be prompt or as large as tens of centimeters.  

The total number of $\ap$ produced when $N_e$ electrons scatter in a target of $T\ll 1 $ radiation lengths is
\begin{equation}
N\sim N_e\frac{N_0 X_0}{A}T\frac{Z^2\alpha^2\epsilon^2}{\map^2}\tilde{\chi}\sim N_e C T \frac{\epsilon m_e^2}{\map^2},
\end{equation}
where $X_0$ is the radiation length of the target in g/cm$^2$, $N_0 \approx 6\times 10^{23} mole^{-1}$ is Avogadro's number, and $A$ is the target atomic mass in g/mole.  The numerical factor $C\approx 5$ is logarithmically dependent on the choice of nucleus (at least in the range of masses where the form-factor is only slowly varying) and on $\map$, because, roughly, $X_0 \propto \frac{A}{Z^2}$ (see \cite{Kim:1973he, Essig:2010xa,Bjorken:2009mm}).  For a Coulomb of incident
electrons, the total number of $\ap$s produced is given by
\begin{equation}
N\sim 10^5 \left(\frac{N_e}{1 C}\right)\tilde{\chi}\left(\frac{T}{0.1}\right)\left(\frac{\epsilon}{10^{-4}}\right)^2\left(\frac{100 MeV}{\map}\right)^2.
\end{equation}



\section{Test Run SVT Performance}
\subsection*{SVT Calibration}


%
%   svt_calibrations.tex
%       author: Omar Moreno <omoreno1@ucsc.edu>
%               Per Hansson <phansson@slac.stanford.edu>
%
%

In order to prepare the SVT for real physics data-taking, the SVT was 
calibrated. This involved the extraction of the mean baseline (pedestal),
baseline noise and gain for each of the 12,780 SVT channels. All measurements
were made with the APV25 readout chips configured to their nominal operating
points \cite{Jones:1069892} and all sensors biased to 180 V. The APV25s were
operated in ``mulit-peak'' mode with six samples being readout per trigger.
This, in turn, allowed for the extraction of the $t_0$ and amplitude of the 
signals being read out.

Figure~\ref{fig:pedestal_noise} shows an intensity plot of the pedestals 
along with the readout noise as a function of channel number for a single
hybrid.  The noise was computed by taking the RMS of the gaussian distributed
\begin{figure}[h]
    \begin{center}
    	\includegraphics[width=0.45\textwidth]{test2012/svtperformance/baseline}
    	\includegraphics[width=0.45\textwidth]{test2012/svtperformance/gain}
        \caption{Something ... }
%    	\caption{\small{The baseline across a hybrid (left) and the measured response as a function of 
%	                    input charge (right). The overall shifts in the baseline are calibrated out where distinct edges 
%	                    are associated with the five APV25 chips on the hybrid. The gain shows good linearity up to 
%	                    about three $mip$s.} {\color{red}Should we show noise instead of baseline?}}
	\label{fig:pedestal_noise}
    \end{center}
\end{figure}
pedestals for each of the channels and was observed to be consistently within 
[Find number] ADC counts ( electrons).  One observed feature are the large
noise values for the channels lying near the chip edges.  This has also been 
reported by the CMS collaboration and the cause is still under investigation.

%  Need to rewrite this ...
Another important aspect for the characterization of the SVT is the response and the associated 
gain. Using the APV25 internal calibration circuit a known fixed charge was injected into all 
channels of the which allows for an accurate determination of the response and its 
scaling with input charge, shown in Fig.~\ref{fig:baseline_and_gain}. The gain uniformity was 
within the expected range across chips and modules and show good linearity of charge 
depositions up to about 3 $mip$s. 

All reconstructed hits in an event were used to form clusters of energy 
depositions using a nearest neighbor algorithm. Fig.~\ref{fig:cluster_pulse}
shows the mean pulse shape of each of the hits associated with a track as a 
function of time.  
\begin{figure}[h]
	\includegraphics[width=\textwidth]{test2012/svtperformance/pulseshape_and_landau}
    \caption{Something else ...}
%	\caption{\small{The distribution of cluster amplitudes (left) showing the characteristic Landau 
%	shape and the pulse shape from the six samples readout (right) {\color{red} Remove one of the pulse shapes}. }}
	\label{fig:pulseshape}
\end{figure}
% Should this be included? If so, it needs to be rewritten a little better
The figure also demostrates that  the trigger system, described below, is well 
timed in with the tracker. 
Fig.~\ref{fig:cluster_pulse} shows the MIP response to be [Find value] electrons.
Taking the MIP response, the signal to noise ratio was calculated to be 
approximately 25.5 which is well matched to the expected behavior.

\bibliography{svt_calib}
 


\subsection*{SVT Hit Time Resolution}
As discussed in Sec.~\ref{sec:svt} the multi-peak APV25 readout is crucial in order to time stamp
hits in the tracker to lower effective occupancies for pattern recognition during electron running. 
Six samples of the APV25 shaper output for each trigger are fitted to an ideal $CR-RC$ function to 
extract the amplitude and hit time.  The $\chi^2$ distribution of these fits from test run data is as expected
for four degrees of freedom.
%\begin{figure}[]
%	\includegraphics[width=0.6\textwidth]{test2012/svtperformance/apvfit_chisq}
%	\caption{\small{Histogram of $\chi^2$ values for pulse fits for all channels on a representative sensor. The peak at 2 is consistent with 4 degrees of freedom (2 fit parameters), as expected. Pileup was not considered due to the very low hit rate in 
%the SVT in this photon beam test. } }
%	\label{fig:apvfit}
%\end{figure}
After clustering hits on a sensor, the hit time for the each cluster is computed as the 
amplitude-weighted average of the channel hit times. Since we have no measurement of the ``true'' hit time, we study the overall SVT hit 
timing performance using the average of all cluster times in a track as the ``track time,'' and take the
 residual of the cluster time relative to that. The observed track time, shown in Fig.~\ref{fig:tracktime}, has the expected amount of trigger jitter due to the readout clock and trigger system jitter. After correcting for offsets for each sensor (time-of-flight, clock phase) the RMS 
 of the final residual distribution is roughly 2.4~ns for each sensor. 
Because the track time is calculated using the individual hit times, the hit time is positively correlated 
with the track time; thus the RMS of the residual is slightly smaller than the true time resolution.
The standard deviation of this residual for $n$-hit tracks where all hits have the same time resolution 
is reduced by a factor of $\sqrt{(n-1)/n}$; since most of our tracks have 8 clusters, the true time 
resolution is 2.6 ns. 
%\begin{figure}[ht]
%	\includegraphics[width=\textwidth]{test2012/svtperformance/timeres}
%	\caption{\small{Histogram and Gaussian fit of residual of cluster times for a representative sensor, relative to the track time. Because the cluster times and track time have positive covariance, the true time resolution is slightly larger than the standard deviation shown here.} }
%	\label{fig:timeres}
%\end{figure}
%\begin{figure}[ht]
%	\includegraphics[width=\textwidth]{test2012/svtperformance/hit_dt}
%	\caption{\small{} }
%	\label{fig:hit_dt}
%\end{figure}
This is somewhat worse than the $\approx 2$ ns resolution expected 
%(see Fig.~\ref{fig:timeres}) 
in 
Sec.~\ref{sec:performance}, but we believe this discrepancy is due to our fit function. Our pulse 
shape fit assumes an ideal CR-RC pulse shape; since the actual pulse shape has a slower rise time, 
there is a systematic pull on the hit time when a hit comes immediately before the APV clock time. 
This is visible in Figure \ref{fig:timeres_2D} as a shift in the residual at certain values of track time.
\begin{figure}[h]
	\includegraphics[width=0.7\textwidth]{test2012/svtperformance/timeres_2D}
	\caption{\small{Plot of the time residual for a representative sensor vs. the track time. 
		The kinks in the horizontal band are caused by the fitter; without them the time resolution (measured by taking the projection of this histogram) would be better.} }
	\label{fig:timeres_2D}
\end{figure}
Work is in progress to use the actual pulse shape in time reconstruction; this should improve time resolution to that expected from previous studies. 
Reducing the APV25 pulse shaping time will also improve time resolution.
In short, these results show that we can achieve time resolution adequate for pileup rejection during electron running.
%\vspace{1cm}{\bf Tracking algorithms [Matt/Omar]}


\subsection*{SVT Track Reconstruction}

%
%   trk_performance.tex
%       author: Omar Moreno <omoreno1@ucsc.edu>
%      created: December 4, 2012
%
 
Clustered hits in Si planes within each layer are combined to form
2-dimensional ``stereo hits'' which are used by the tracking algorithm to 
form tracks.  The determination of the probability that a stereo hit is 
formed, or hit efficiency, provides insight as to the performance of each of 
the SVT layers.

In order to obtained the hit efficiencies, tracks were fitted using only 4 of 
SVT layers. The resulting track was then extrapolated to the layer omitted
\begin{figure}[h]
        % TODO: Need to update the plot so that Layer 2 on the bottom doesn't 
        % look terrible. It's probably easiest to just remove the point altogether.
    	\includegraphics[width=0.95\textwidth]{test2012/svtperformance/trk_performance/hit_efficiency_vs_layer.pdf}
    	%\includegraphics[width=0.49\textwidth]{test2012/svtperformance/trk_performance/track_reco_efficiency.pdf}
        \caption{{\small
                    Average hit efficiency, excluding known bad channels, 
                    from all dedicated photon runs as a function of layer
                    number.  
                }} 
	\label{fig:hit_track_efficiency}
\end{figure}
from the fit. If the track was found to lie within the sensitive volume
of the layer, a search for a stereo hit within the layer acceptance was 
conducted.  The hit efficiency was then determined as
\[
    \varepsilon_{\mbox{hit efficiency}} = \frac{\mbox{Tracks with hit on missing layer}}
                                            {\mbox{Tracks within layer acceptance}}.
\]
The hit efficiencies per layer were calculated using all dedicated runs. It must 
be noted that those tracks found to intersect bad channels as well as those that 
lie on the sensor edges were excluded from the calculation. As 
can be seen from Figure~\ref{fig:hit_track_efficiency}, the average hit efficiency
per layer was found to be $\approx$ 99\%. 

%95\% for the top SVT volume and greater than 92\% for the bottom volume.  The 
%larger hit inefficiencies observed for some layers was simply due to the 
%sensors composing the layer having a greater number of noisy channels.

%As mentioned above, the standard pattern recognition algorithm is designed to 
%find tracks using the reconstructed stereo hits.  A set of tracking strategies
%outline which layers should be used by the track finding algorithm along
%with their role (seeding, extend, confirm), and the $\chi^2$ cut imposed 
%on the fit. Any kinematic constraints are also specified within the strategy.
%A detailed account of the tracking algorithm can be found  here~\ref{}.

%% Track reco efficiency is no longer going to be included so remove this
%% section
%In order to determine the efficiency of the track finding algorithm, a Monte
%Carlo sample containing pair produced electrons from photons incident on 
%a 1.6\% $X_0$ gold target was used.  The energy of the pair produced electrons
%ranged from .5 GeV to 5.5 GeV. An electron falling within the detector 
%acceptance was considered findable by the tracking algorithm if it 
%traversed through at least 4 of the SVT layers. The track reconstruction
%efficiency was then determined as
%\[
%    \varepsilon_{\mbox{track reco efficiency}} = \frac{\mbox{Tracks found}}
%                                            {\mbox{Tracks found to be findable}}.
%\]
%The resulting efficiency to find an electron which passes through the detector
%acceptance is shown in Figure~\ref{fig:hit_track_efficiency}. The average track
%reconstruction efficiency was found to be \% with the bulk of the inefficiency
%coming from the $\chi^2$ cut imposed on the fit to the six samples during
%the clustering stage.

All events containing pairs of oppositely charged tracks were fit to a
common vertex using a simple vertexing algorithm which searches for the distance
of closest approach between the two tracks.  The reconstructed vertex position
along the beam axis for both data and Monte Carlo is shown on 
Figure~\ref{fig:vz_position}.
\begin{figure}[h]
    \begin{center}
    	\includegraphics[width=0.60\textwidth]{test2012/svtperformance/trk_performance/zvertex.pdf}
        \caption{  
                    The reconstructed vertex position along the beam axis for
                    both data (blue) and Monte Carlo (red).
                } 
	\label{fig:vz_position}
    \end{center}
\end{figure}
Because of the geometric setup of the SVT, observation of pairs produced by
the incident photon required both electrons to experience a hard scatter
within the target.  This results in a broadening of the Gaussian 
distributed reconstructed vertex position. 



\subsection*{SVT Test Run DAQ Performance}

%
%   svt_daq.tex
%       author: Omar Moreno <omoreno1@ucsc.edu>
%               Santa Cruz Institute for Particle Physics
%               University of California, Santa Cruz
%      created: November 13, 2012
%

The expected data rates and event sizes for each of the dedicated photon runs
were estimated using a full simulation of the SVT DAQ and compared to observed
values. As discussed in Section~\ref{sec:testrun_daq}, the digitized samples
from three hybrids were received by a single DPM.  The DPM then required that
at least three of the six samples exceeded a threshold of two times the noise
level for that channel.  An additional ``pile-up'' cut requiring that 
(sample 2 $>$ sample 1) or (sample 3 $>$ sample 2) was also applied. This was
meant to eliminate hits arising from the falling edge of previous hits expected
to occur when running at the highest occupancies.
%Signals from the photon run were unaffected by such a cut. 

All samples were placed into their own container along with the 
channel number, hybrid identifier, chip address and DPM identifier. An 
additional layer of encapsulation or bank was used to store all samples 
emerging from a single DPM along with the DPM identifier, the event number
an error bit and hybrid temperatures. A diagram of the container along with
the sizes of each of the elements is shown on Figure~\ref{fig:data_format}.
\begin{figure}[h]
    \begin{center}
    	\includegraphics[width=0.60\textwidth]{test2012/svtperformance/daq/svt_data_format.pdf}
        \caption{
                    SVT data format. Samples readout from three hybrids are 
                    processed by a single FPGA and are placed within a single
                    container or FPGA bank.  An additional layer of 
                    encapsulation is used to store all of the FPGA banks.
                 } 
	\label{fig:data_format}
    \end{center}
\end{figure}
Overall, the container overhead will contribute a total of 326 bytes to an event
with an additional 16 bytes per hit.

The observed occupancy expected for each of the converter thicknesses along with the 
corresponding data rate are shown on Figure~\ref{fig:data_rates}.
\begin{figure}[h]
    \begin{center}
    	\includegraphics[width=0.49\textwidth]{test2012/svtperformance/daq/n_dead_channels.pdf}
    	\includegraphics[width=0.49\textwidth]{test2012/svtperformance/daq/data_rates.pdf}
        \caption{
                    The plot on the left shows the percentage of bad/noisy channels observed
                    during each of the runs (green) along with the number of noisy/misconfigured 
                    readout chips (blue).  Most of the noisy channels present during runs were
                    due to misconfigured chips.  The  plot on the right shows the occupancies (blue)
                    and data rates (black) observed for each of the targets used.  Once all noisy channels
                    are masked, the observed data rates agree well with those predicted by the
                    SVT readout simulation. 
                } 
	\label{fig:data_rates}
    \end{center}
\end{figure}
The data rates observed during the test run were
much higher than expected.  This can be attributed to a known
noisy sensor and a few noisy chips which appeared during certain runs as can be seen
on Figure~\ref{fig:data_rates}.  The causes
of both these issues are now well understood and will be resolved for future running.

%Table~\ref{table:sim_rates}.
%\begin{table}[h]
%    \scalebox{0.9}{
%    \begin{tabular}{ c | c | c | c }
%    \hline
%
%    Converter Thickness (\%$X_0$) & Sim Occupancy (\%)  & Sim Event Size (kB) &   Sim Data Rate (Mb/s) \\      
%    \hline 
%   1.6                           & .438                & 1.22                &   2.07                 \\
%   0.45                          & .293                & .93                 &   .53                  \\
%    0.18                          & .118                & .56                 &   .24                  \\ 
%    \end{tabular} } 
%    \caption{Occupancy, event size and resulting data rate expected for each of the three 
%             converter thicknesses used in the test run.}
%    \label{table:sim_rates}
%\end{table}

A better comparison between simulated and observed data rates can be obtained
by masking out all known noisy channels found during the commissioning of the 
SVT along with the noisy chips.  The resulting occupancies and data rates are also shown on 
Figure~\ref{fig:data_rates}. The simulated occupancies shown include a contribution
of 0.02\% (3 hits) due to noise and the data rates are estimated using the trigger
rates observed during each of the dedicated photon runs.  As can be seen from the
figure, the occupancies and data rates after most bad channels are masked are well
in agreement with those predicted by Monte Carlo.
%Table~\ref{table:observed_rates} list the observed occupancy, event sizes and 
%\begin{table}[h]
%    \scalebox{0.9}{ 
%        \begin{tabular}{ c | c | c | c }
%            \hline
%            Converter Thickness (\%$X_0$)   & Obs. Occupancy (\%) & Obs. Event Size (kB) & Obs. Data Rate (Mb/s) \\
%            \hline
%            1.6                             & 1.03                & 2.43                 & 4.12                  \\
%            0.45                            & 1.22                & 2.82                 & 1.61                  \\
%            0.18                            & 1.23                & 2.84                 & 1.21                  \\
%        \end{tabular}
%    }
%    \caption{Occupancy, event size and resulting data rate observed for each of the three 
%             converter thicknesses used in the test run.}
%    \label{table:observed_rates}
%\end{table}    


\section{Test Run ECal Calibration}
\label{sec:ecal_calib}

The noise and pedestal of the readout chain are calibrated by running the ECal readout in a mode where the preamplifier output is sampled every 4~ns in a time window of 100 samples: by looking at a part of the window before the hit, we calibrate the readout channel.

For ECal analysis, cluster reconstruction was done using the algorithm described in \cite{HPS_PROP}: build clusters around seed hits (hits above a ``seed'' energy threshold and with greater energy than any neighboring hits), and add all neighboring hits above an ``add'' energy threshold.

We calibrate gain of the individual ECal channels using the SVT measurement of track momentum. 
The ratio of cluster energy to track momentum is calculated both for Monte Carlo simulation and test run data at each point in the ECal, and we find the value of gain for each channel that brings the two into agreement.
We use a formula to compute the ``weighted E/p'' for a crystal, representing the average E/p for clusters that include the crystal: $\frac{\sum_j w_{j,i}}{\sum_j\frac{P_j}{E_j}w_{j,i}}$.
We disable all SVT and ECal channels in the simulation that were inoperable or noisy in the test run, so any efficiency or bias effects that affect the real data should be reflected in the simulation as well.

The calibrated gains are corrected by the ratio between the weighted E/p values from Monte Carlo and real data.
The E/p in Monte Carlo data is also affected by the gain because the trigger thresholds change, so both Monte Carlo and data reconstruction are rerun with each iteration of gain calibration.
It takes up to 4 iterations for the gains to stabilize; the final values are shown in Figure \ref{fig:gains}.
\begin{figure}[ht]
	\includegraphics[width=0.45\textwidth]{test2012/ecalperformance/ecalgainplots_corr_sim}
	\includegraphics[width=0.45\textwidth]{test2012/ecalperformance/gains}
	\caption{\small{Weighted E/p from Monte Carlo simulation (left), calibrated values of gain in units of MeV per ADC count (right).}}
	\label{fig:gains}
\end{figure}

These gains can then be used to convert from ADC counts in a channel to energy deposited into that ECal crystal.
The other information needed to find the energy of an incident particle is the sampling fraction---the ratio of energy read out from crystals to energy of an incident particle.
The conventional sampling fraction---the fraction of incident energy that is deposited in crystals---is approximately 0.9 for our ECal, and less at edges.
For our readout, there is additional energy lost because crystals under the readout threshold are not read out.
The weighted E/p used in calibration (see Figure \ref{fig:gains}) is an approximate measurement of sampling fraction, but the sampling fraction is energy-dependent because of the effect of readout threshold. 
A full computation of sampling fraction can be done using simulation.
