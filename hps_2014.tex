\documentclass[prc,12pt]{revtex4}
%\documentclass[12pt]{article}

%\usepackage{graphics}
\usepackage[dvipdf]{graphicx}
%\usepackage{subfig}  % For subfloats
\usepackage{subfigure}  % For subfloats
\usepackage{color}
\usepackage{multirow}
\usepackage{epsfig}
\usepackage{wrapfig}
\usepackage{rotating}
\usepackage{amsmath}
\usepackage{footmisc}
\usepackage{titlesec}
%\usepackage{subfigure}
\usepackage{hyperref}
\usepackage{float}
\newcommand{\footlabel}[2]{%
    \addtocounter{footnote}{1}
    \footnotetext[\thefootnote]{%
        \addtocounter{footnote}{-1}%
        \refstepcounter{footnote}\label{#1}%
        #2%
    }%
    $^{\ref{#1}}$%
}
\usepackage[center]{background} 
\SetBgContents{DRAFT}
\SetBgPosition{8.5cm,-11.5cm}
%\backgroundsetup{contents=Confidential,color=blue!30}
\def \rarr {\rightarrow}
\def \grinp {\includegraphics}
\def \tw {\textwidth}
\def\dfrac#1#2{\displaystyle{{#1}\over{#2}}}
\def \dstl {\displaystyle}
\definecolor{GREEN}{rgb}{0.,0.8,0}
\definecolor{RED}{rgb}{1,0,0}
\definecolor{ORANGE}{rgb}{1,0.5,0}
%
\newcommand{\red[1]}{{\color{red}{\bf #1}}}
\newcommand{\justifyheading}{\raggedright}

\titleformat{\section}
  {\normalfont\Large\bfseries\justifyheading}{\thesection}{1em}{}
\titleformat{\subsection}
  {\normalfont\large\justifyheading}{\thesubsection}{1em}{}
\titleformat{\subsubsection}
  {\normalfont\justifyheading}{\thesubsubsection}{1em}{}

\renewcommand*{\thesection}{\arabic{section}}
\renewcommand*{\thesubsection}{\thesection.\arabic{subsection}}
\renewcommand*{\thesubsubsection}{\thesubsection.\arabic{subsubsection}}

\makeatletter
\def\p@subsection{}
\def\p@subsubsection{}
\makeatother

\begin{document}

{\color{red} Mon Jan  7 12:04:43 PST 2013 } %GIT_REVISION_TIME

\title{\bf\large{Heavy Photon Search Experiment at Jefferson Laboratory: proposal for 2014-2015 run}}

\newcommand{\JLAB}{Thomas Jefferson National Accelerator Facility, Newport News, Virginia 23606}
\newcommand{\YEREVAN}{Yerevan Physics Institute, 375036 Yerevan, Armenia}
\newcommand{\NSU}{Norfolk State University, Norfolk, Virginia 23504}
\newcommand{\ODU}{Old Dominion University, Norfolk, Virginia 23529}
\newcommand{\genova}{Istituto Nazionale di Fisica Nucleare, Sezione di Genova e Dipartimento di Fisica dell\'Universita, 16146 Genova, Italy}
\newcommand{\ORSAY}{Institut de Physique Nucleaire d'Orsay, IN2P3, BP 1, 91406 Orsay, France}
\newcommand{\UCSC}{University of California, Santa Cruz, CA 95064}
\newcommand{\SUNY}{Stony Brook University, Stony Brook, NY 11794-3800}
\newcommand{\FNAL}{Fermi National Accelerator Laboratory, Batavia, IL 60510-5011}
\newcommand{\HUJ}{Hebrew University of Jerusalem, Jerusalem, Israel}
\newcommand{\UNH}{University of New Hampshire, Department of Physics, Durham, NH 03824}
\newcommand{\PERIMETER}{Perimeter Institute, Ontario, Canada N2L 2Y5}
\newcommand{\RPI}{Rensselaer Polytechnic Institute, Department of Physics, Troy, NY 12181}
\newcommand{\SLAC}{SLAC National Accelerator Laboratory, Menlo Park, CA 94025}
\newcommand{\WNM}{The College of William and Mary, Department of Physics, Williamsburg, VA 23185}
\newcommand{\RTG}{Rutgers University, Department of Physics and Astronomy, Piscataway, NJ 08854}
\newcommand{\GLASGOW}{School of Physics \& Astronomy, University of Glasgow, Glasgow, G12 8QQ, Scotland, UK}
\newcommand{\IDAHO}{Idaho State University, Pocatello, Idaho 83209}
\newcommand{\UCONN}{University of Connecticut, Department of Physics, Storrs, CT 06269}
%, Kelvin Building, Room 505
\newcommand{\Contact}{Contact person}


\author{P. Hansson Adrian, C. Field, N. Graf, M. Graham, G. Haller, R. Herbst, J. Jaros\footnote{\Contact}\footnote{Co-spokesperson\label{spoks}}, 
T. Maruyama, J. McCormick, K. Moffeit, T. Nelson, H. Neal, A. Odian, M. Oriunno, S. Uemura, D. Walz}
\affiliation{\SLAC}                                 
 \author{A. Grillo, V. Fadeyev, O. Moreno}
\affiliation{\UCSC}
\author{W. Cooper}
\affiliation{\FNAL}
\author{S. Boyarinov, V. Burkert, C. Cuevas, A. Deur, H. Egiyan, L. Elouadrhiri, A. Freyberger, F.-X. Girod, 
S. Kaneta, V. Kubarovsky, N. Nganga, B. Raydo, Y. Sharabian, S. Stepanyan\footref{spoks}, M. Ungaro, B. Wojtsekhowski}
\affiliation{\JLAB}
\author{R. Essig}
\affiliation{\SUNY}
\author{M. Holtrop\footref{spoks}, K. Slifer, S. K. Phillips}
\affiliation{\UNH}
\author{R. Dupre, M. Guidal, S. Niccolai, E. Rauly, and P. Rosier}
\affiliation{\ORSAY}
\author{D. Sokhan}
\affiliation{\GLASGOW}
\author{P. Schuster, N. Toro}
\affiliation{\PERIMETER}
\author{N. Dashyan, N. Gevorgyan, R. Paremuzyan, H. Voskanyan}
\affiliation{\YEREVAN}
%\author{C. Salgado}
%\affiliation{\NSU}
\author{M. Khandaker}
\affiliation{\IDAHO}
\author{M. Battaglieri, A. Celentano, R. De Vita}
\affiliation{\genova}
\author{S. Bueltmann, L. Weinstein}
\affiliation{\ODU}
\author{G. Ron}
\affiliation{\HUJ}
\author{A. Kubarovsky}
\affiliation{\UCONN}
\author{K. Griffioen}
\affiliation{\WNM}
\author{Y. Gershtein, J. Reichert}
\affiliation{\RTG}

\date{\today}

\begin{abstract}
%\vspace{2.cm}
\clearpage

The Heavy Photon Search (HPS) experiment will search for a new heavy vector boson, aka Òheavy photonÓ, at Jefferson Laboratory. The HPS proposal was conditionally approved, C1, with "A" rating by the Jefferson Laboratory PAC39  on Jun of 2012. 

\end{abstract}

\maketitle
\clearpage

\tableofcontents
\clearpage

\section{Introduction}


Access to higher and higher luminosities and ever faster detection and recording  techniques enables  searches for new physics at otherwise well-explored energies. This fundamental premise of Intensity Frontier physics has already seen dramatic demonstration at the e$^+$e$^-$ B factories, where high luminosities and impressive data handling capacities have allowed extensive exploration of CP violation in the quark sector. The same principle is being exploited in new proposals to explore neutrino masses, mixings, and CP violation by directing ever more intense neutrino beams at massive detectors to push sensitivity well beyond present limits. At the Intensity Frontier, searches for new physics often rely on the study of rare processes and the search for subtle effects which would indirectly indicate physics beyond the Standard Model. But this is not the rule. New studies of otherwise commonplace phenomena at electron machines, like trident production off heavy nuclear targets, can, with sufficient sensitivity, explore whole new worlds and directly search for hidden sector particles and forces, those with very weak couplings to our Standard Model world. The Heavy Photon Search at Jefferson Laboratory does exactly this, utilizing the high duty factor CEBAF accelerator, intense beams, fast detectors, electronics and triggering, and state of the art data acquisition to explore a very common landscape in search of a most uncommon quarry.

Heavy photons, or �dark� or �hidden sector� photons, may well be part of our universe and related to the Dark Matter. Particles of dark matter, which interact very weakly with normal matter and account for a quarter of the universal mass-energy, are of course not yet detected. The Dark Matter can be thought of as constituting, or inhabiting, a � hidden sector�, since it interacts so weakly with normal, baryonic matter. This sector could include a complex of new forces and particles with which we barely interact.  Stimulated by the observation of very high energy electrons and positrons in the cosmic rays and the difficulty of understanding their production in terms of tried and true SUSY dark matter annihilation, several authors realized that models in which massive dark matter particles annihilate to heavy photons, which in turn decay to high energy electron-positron pairs, could naturally account for the observations. These theories presume heavy photons couple to the dark matter, mediate its interactions, are produced  in its annihilation, and weakly couple to electric charge. Heavy photons in the mass range of 100 to 1000 MeV  can reasonably account for the observed cosmic ray fluxes. 

Many Beyond Standard Model theories generate extra U(1) gauge groups, and the associated gauge bosons could have masses over a very wide range. As Holdum realized in the mid 80�s, it is natural that such �heavy photons� kinetically mix with our own photon, leading to their induced coupling to electric charge. This mixing can be mediated by GUT level particles which carry both Standard Model hypercharge and its hidden sector analogue. Interestingly, the natural scale for this mixing results in heavy photons coupling to Standard model charged particles with couplings of order $10^{-3}$e.  So heavy photons naturally couple to electrons, albeit with couplings much suppressed compared to those in standard QED. It follows that electrons will radiate heavy photons, and heavy photons will decay to electron-positron pairs or pairs of other kinematically accessible charged particles, but at rates significantly below QED trident production, and with lifetimes far longer than those expected from purely electromagnetic interactions.

 HPS distinguishes heavy photons from the copious background of QED tridents by  using both invariant mass and decay length signatures. With good mass resolution, heavy photons will appear as sharp resonances above the QED continuum. For suitable values of mass and coupling, heavy photons will have long lifetimes, resulting in discernible secondary decay vertices.  The Heavy Photon Search employs a large acceptance  forward magnetic spectrometer with precise momentum measurement and vertexing  capability, followed by a highly segmented crystal Electromagnetic Calorimeter for fast triggering and electron identification.  HPS depends on  the 40 MHz readout capability of the silicon microstrip vertex tracker, 250 MHz FADC readout of the  electromagnetic calorimeter, and very high rate triggering and data acquisition systems, to fully exploit CEBAF�s essentially DC beams and  high intensities.  A muon identification system just downstream of the ECal  significantly boosts the experimental reach for heavy photon masses above the dimuon threshold and provides an independent trigger. The beam is transported in vacuum through the entire apparatus to eliminate beam gas backgrounds; and the apparatus is split top-bottom, to avoid electrons  which have multiple Coulomb scattered or radiated  in the target.

HPS  probes a unique region of the mass-coupling parameter space where the heavy photon signal would be lost in the trident background without the vertex signature, and it simultaneously accesses a region at higher coupling strength by relying on bump hunting alone. HPS has sensitivity to a region of parameter space favored by accounting for the discrepancy between  measured and calculated  values for the muon�s g-2 with the existence of  a heavy photon, and probes an extensive region suggested by parameters which could account for dark matter annihilations into heavy photons.  In broader  terms, HPS searches for heavy photons in a region suggested on very general theoretical grounds. As seen above, coupling strengths of order $10^{-3}$e are theoretically natural; masses of order  $\alpha m_W$ are also expected on general grounds.  Interestingly, HPS is also sensitive to the production of �true muonium�, the QED atom comprised of $\mu^+ \mu^-$, which is produced with a well-defined (and detectable) cross-section, and decays with a well-defined  (and observable) lifetime to $e^+e^-$. HPS should discover true muonium, measure some of its properties, and find it a useful calibration signal.

This proposal seeks funding for the Heavy Photon Search (HPS) Experiment at Thomas Jefferson National Accelerator Facility.  This experiment  is the second stage of a program that was initiated with the Heavy Photon Search Test Run Proposal \cite{HPS_PROP, HPS_tPROP}, which was approved by the Jefferson Laboratory Program Advisory Committee PAC37 in January, 2011, and approved and funded by DOE HEP in the late Spring of 2011. PAC37 also conditionally approved the full experiment, contingent upon the Test Run results.  During the remainder of FY2011 and the first half of FY2012, the Test Run apparatus, data acquisition system, and system software were designed, constructed, and tested. On April 19, 2012, the newly constituted HPS Collaboration installed the experiment in Jefferson Lab�s Hall B experimental area, and began commissioning the experiment parasitically, using the HDIce photon beam. Although the Jefferson Lab schedule did not accommodate the electron beam running which had been requested, the apparatus was fully commissioned by running parasitically in the photon beam. The trigger and data acquisition and storage systems worked well, and all systems performed as expected. Efficient track reconstruction in the Silicon Vertex Tracker was demonstrated, measurements of shower energies and positions were made  in the Electromagnetic Calorimeter, and critical assumptions about background rates were tested. The critical test run goals were accomplished.  A status report summarizing HPS�s progress and results was submitted to PAC39 \cite{HPS_PROP_UPD} along with a request for unconditional approval for the full experiment. At its June, 2012  meeting,  PAC39 graded HPS physics with an "A", approved a commissioning run with electrons, and granted us a so-called "C1" approval, which gave Jefferson Laboratory management the final say in granting HPS the running time needed to search comprehensively for Heavy Photons. Since that approval, it has become clear that running time will become available in Hall B in late calendar 2014 for our commissioning run, when the upgraded CEBAF 12 will have been completed, commissioned, and operational. CLAS12 is the large general purpose apparatus being constructed to exploit CEBAF 12 in Hall B. Delays in funding will delay the construction of the CLAS12 magnets and will delay CLAS12 installation beyond 2015, thereby providing HPS the opportunity for  a commissioning run in 2014 and the extended data collection run in 2015. To take full advantage of these scheduling windfalls, the HPS Collaboration has re-visited the original HPS design, and simplified and improved it. The resulting simplifications make it possible to construct and test HPS in time for installation in late 2014. The resulting improvements extend the reach far beyond that of the Test Run experiment, maximize the physics output during this time period, and let HPS begin searching for heavy photons in a large and hitherto unexplored region of parameter space.  

 In the following, this proposal motivates and describes the new HPS Experiment, documents the experience and performance obtained with the Test Run Apparatus, demonstrates that the backgrounds expected in electron running are understood and manageable,  reviews the performance and physics reach of the new experiment, and outlines the budget, schedule, and milestones for constructing  and deploying it. It concludes with a request for beam time.



%% The sections on the heavy photon physics motivation.

\newcommand{\MeV}{\rm MeV}
\section{Motivations for Searching for Heavy Photons}

HPS will search for heavy photons, called $A'$s, which are new hypothesized massive vector bosons that have a small coupling 
to electrically charged matter, including electrons.  
The existence of an $A'$ is theoretically natural and could explain the discrepancy between the measured and 
observed anomalous magnetic moment of the muon and several intriguing dark matter-related anomalies.  
As discussed in the following section, HPS should also have the capability to make the first detection of \emph{True Muonium}, a bound state of a 
$\mu^+ - \mu^-$ pair predicted by Quantum Electrodynamics (QED).  
The search for $A'$s has generated enormous interest in the international physics community.  This is evidenced, for example, 
by its inclusion in the recent Intensity Frontier Workshop \cite{Kamionkowski:2010mi,Hewett:2012ns}, 
many novel searches in colliding beam and fixed-target data (see \cite{Dark2012} for a recent summary of results),  
and by numerous new experiments (in addition to HPS) proposed to search for them, including 
APEX~\cite{Essig:2010xa,Abrahamyan:2011gv},  MAMI~\cite{Merkel:2011ze}, and DarkLight~\cite{Freytsis:2009bh}.
We briefly review the theory and motivation for heavy photons and existing constraints on $A'$.

\subsection{Theory Update}

The $A'$ is a new abelian $U(1)$ gauge boson with a weak coupling 
to electrically charged particles induced by ``kinetic mixing'' with the photon~\cite{Holdom:1985ag,Galison:1983pa}.  
Kinetic mixing produces an effective parity-conserving interaction
$\epsilon e A'_\mu J^\mu_{\rm EM}$ of the $A'$ to the 
electromagnetic current $J^\mu_{EM}$,  suppressed relative to the electron charge 
$e$ by the parameter $\epsilon$, which can naturally be in the range $10^{-12} -10^{-2}$ \cite{Essig:2009nc,Goodsell:2009xc,Cicoli:2011yh,Goodsell:2011wn}. 

More broadly, ``kinetic mixing'' of the photon with new forces offers one of the few 
portals with which ordinary matter can be used to search for light new forces beyond the Standard Model
consistent with known symmetries. 
An $A'$ would also allow ordinary matter to have a small coupling to new particles in a ``hidden sector'' 
that do not interact with the Standard Model's strong, weak, or electromagnetic forces.  
There has been intense speculation over the past three decades about the existence 
of hidden sectors. Theoretical models with dark matter, supersymmetry, and string theory constructions often employ 
hidden sectors with new particle content to resolve various phenomenological questions~\cite{Goodsell:2010ie,NSF-ITP-84-170,PRINT-86-0084 (PRINCETON),Andreas:2011in,arXiv:1002.0329} 
(see \cite{Hewett:2012ns} for a recent review). 
The photon mixing with an $A'$ could provide the only non-gravitational window into their existence. 

While loop level effects can naturally generate $\epsilon$ in an observable range, 
simple theory arguments offer less guidance for what range of $A'$ mass to search for. 
Many mass generating mechanisms have been proposed -- $A'$ masses can arise, for example, 
via the Higgs mechanism as in the models of~\cite{Fayet:2007ua,Cheung:2009qd,ArkaniHamed:2008qp,Morrissey:2009ur},
or via a Stuckelberg mechanism, as often occurs in large volume string compactification models~\cite{Hewett:2012ns}.
In models using a Higgs mechanism, a natural mass range for an $A'$ is near (but beneath) the weak scale, 
in the MeV to GeV range. This mass range has received considerable attention in part because it may also 
allow $A'$s to resolve several anomalies (see below). Existing constraints 
are shown in Fig.~\ref{fig:hspaw-heavy-A'}. HPS will be sensitive to $A'$ masses in between 20--1000~MeV.

%%%%%%%%%%
\begin{figure*}[ht]
\centering
%\vspace*{-5mm}
\includegraphics[width=0.8\textwidth]{limit_g-2_electron.pdf} 
\caption{ Existing constraints on heavy photons ($A'$). 
Shown are existing 90\% confidence level limits from the beam dump experiments 
E141, E774, Orsay, and U70~\cite{Bjorken:2009mm,Blumlein:2011mv,Andreas:2012mt,Riordan:1987aw,Bross:1989mp,Davier:1989wz,Konaka:1986cb}, 
the muon anomalous magnetic moment $a_\mu$~\cite{Pospelov:2008zw},  
KLOE~\cite{Collaboration:2011zc}, 
the test run results reported by APEX~\cite{Abrahamyan:2011gv} and MAMI~\cite{Merkel:2011ze}, 
an updated estimate using a BaBar result~\cite{Bjorken:2009mm,Reece:2009un,Aubert:2009cp}, 
%a constraint from supernova cooling~\cite{Bjorken:2009mm} as updated in~\cite{Dent:2012mx}, 
and an updated constraint from the electron anomalous magnetic moment~\cite{endo:g2e,Davoudiasl:2012ig}. 
In the green band, the $A'$ can explain the observed discrepancy between the
calculated and measured muon anomalous magnetic moment~\cite{Pospelov:2008zw} 
at 90\% confidence level.
%Projected sensitivities are shown for the full APEX run~\cite{Essig:2010xa}, 
%DarkLight~\cite{Freytsis:2009bh}, and VEPP-3~\cite{Wojtsekhowski:2009vz}.  
%MAMI has plans (not shown) to probe similar parameter regions as these experiments. 
%Several projected sensitivities are shown for HPS. Solid red shows the 2$\sigma$ limits from the full HPS experiment, assuming 3 months of running time at each of 2.2 GeV(200 nA) and 6.6 GeV (450 nA). The upper region corresponds to a resonance search, the lower to a combined resonance plus vertexing search. Dashed red shows the limits from 1 week of running the HPS Test Run at 2.2 GeV (200 nA). The dashed blue limits corresponds to 1 week of running the Test Run at 1.1 GeV (200 nA).  
%Existing and future $e^+e^-$ colliders like \babar, BELLE, KLOE, Super$B$, BELLE-2, and KLOE-2  can also probe large 
%parts of the parameter space for $\epsilon\gtrsim 10^{-4}-10^{-3}$ (not shown).  
}
\label{fig:hspaw-heavy-A'}
\end{figure*}
%%%%%%%%%%

\subsubsection{Heavy Photons and Dark Matter}
The possible role of heavy photons in the physics of dark matter~\cite{ArkaniHamed:2008qn,Pospelov:2008jd} has provided an urgent impetus to search directly for heavy photons.  Results from two classes of dark matter searches --- ``indirect'' searches for galactic dark matter annihilation and ``direct'' searches for dark matter scattering off nuclei --- have both been interpreted as potential signals of dark matter interacting through a heavy photon.  Both areas have developed considerably in recent years, but not decisively.  Here we briefly summarize the status of dark matter, the case for its interactions with heavy photons,  and pertinent recent developments in both observation and theory.  The motivation to test these theories of dark matter in a controlled laboratory experiment remains strong.    

The concordance model of big bang cosmology --- the Lambda Cold Dark Matter ($\Lambda$CDM) model --- explains all observations of the cosmic microwave background, large-scale structure formation, and supernovae, see 
e.g.~\cite{LambdaCDMData}. This model suggests that Standard Model particles make up only about 5\% of the energy density in the Universe, while ``dark energy'' and ``dark matter'' make up 68\% and 27\%, respectively, of the Universe's energy density. The concordance model does not require dark matter to have any new interactions beyond gravity with Standard Model particles. However, an intriguing theoretical observation, dubbed the ``WIMP miracle'', suggests that dark matter does have new interactions. In particular, if dark matter consists of ~10 GeV to 10 TeV particles interacting via an electroweak-strength force (weakly interacting massive particles or WIMPs), they would automatically have the right relic abundance consistent with the $\Lambda$CDM model.

If dark matter does interact with ordinary matter, such interactions could produce at least two observable consequences: dark matter particles in the Milky Way Galaxy (and other bound astrophysical systems) can annihilate or decay into visible matter, which could be detectable as energetic cosmic rays and/or gamma rays at Earth (indirect detection).  Dark matter passing through Earth can also scatter off nuclear targets, causing the target to recoil.  This recoil is observable in radio-pure detectors with sufficiently low background rates of nuclear recoil (direct detection).  

%Dark matter annihilating into heavy photons offers one explanation for the cosmic-ray electron and positron excesses observed by PAMELA, Fermi, ATIC, and HESS.  
%Such models predict correlated gamma-ray fluxes from other astrophysical systems where dark matter can annihilate (such as the inner galaxy, distant galaxies and clusters) and an imprint in the cosmic microwave background (CMB) from dark matter annihilation in the early Universe.  Since dark matter annihilation into heavy photons was first proposed, limits on gamma-ray fluxes from the above systems have  constrained the parameter space significantly.   However, large theoretical uncertainties  both the rate of dark matter annihilation and the efficiency for gamma-ray production reduce the power of these limits.    
%the gamma-ray flux depends in part on how electrons lose  medium into which electrons are emitted (magnetic field density, density of starlight) electrons  (firstly, because gamma-ray fluxes depend in part on  bounds suffer large (mainly astrophysical) theoretical uncertainties, and their implications for the dark matter models of interest are disputed in the literature.  However, the dark photon mass range that remains viable when these uncertainties are taken into account overlaps considerably with the projected sensitivity of HPS.   

\paragraph{Indirect Detection}The satellites PAMELA \cite{Adriani:2008zr} and Fermi \cite{Ackermann:2010ij}, the balloon-borne detector ATIC \cite{Chang:2008aa}, the ground-based Cherenkov telescope HESS \cite{Aharonian:2008aa,Aharonian:2009ah}, and other experiments have all reported an excess in the cosmic-ray flux of electrons and/or positrons above backgrounds expected from normal astrophysical processes.  
The evidence for this excess has only grown, with new measurements of the cosmic-ray electron flux by PAMELA \cite{Adriani:2011xv} and confirmation by Fermi and AMS-2 of the positron excess \cite{FermiLAT:2011ab,AMS2:2013}.  While further data from AMS-2 may shed more light on the spectrum of these excess cosmic-rays, the origin of these excess positrons and electrons remains unknown.  It may plausibly arise from any of three possibilities: pair creation in nearby pulsars, acceleration in supernova shocks, or dark matter annihilation or decay.

If the excess arises from dark matter annihilation, two features are incompatible with annihilation of ``conventional'' thermal WIMP dark matter charged under the Standard Model weak interactions, but compatible with an alternative explanation, namely that dark matter is charged under a new $U(1)'$ and annihilates into $A'$ pairs, which decay directly into electrons and positrons, and/or into muons that decay into electrons and positrons (see e.g. \cite{ArkaniHamed:2008qn,Pospelov:2008jd,Cirelli:2008pk,Cholis:2008qq,Cholis:2008wq}): 
\begin{itemize}
\item The annihilation cross-section required to explain the electron signal is $50-1000$ times larger than the cross-section favored for the ``WIMP miracle''.   This can be explained if dark matter interacts with an $\mathcal{O}$(GeV)-mass $A'$, which mediates a new moderate range force and enhances the annihilation rate at low velocities (the relative velocity of dark matter in the Galactic Halo, $v\sim 10^{-3} c$, is much lower than in the early universe, and the relative velocity in self-bound dark matter subhalos is lower still).  We refer the reader to \cite{Finkbeiner:2010sm,Slatyer:2011kg} for a recent discussion.
\item The PAMELA satellite did not see any anti-proton excess \cite{Adriani:2008zq}, which implies that, if dark matter annihilation is responsible for the positron/electron signals, it does not produce baryons.  This contradicts expectations for dark matter annihilating through Standard Model interactions, but is expected if dark matter decays into light $A'$, which (for $m_{A'}\lesssim$ GeV) are kinematically unable to decay into protons and anti-protons.
\end{itemize}

We emphasize that these cosmic-ray excesses do not point to a unique region in the $\epsilon-m_{A'}$ parameter space.  
Firstly, the value of $m_{A'}$ determines the branching ratios of the $A'$ (and hence the dark matter, which here is assumed to annihilate to 
the $A'$) to different Standard Model states, including $e^+e^-$, $\mu^+\mu^-$, pions etc.  Since one is trying to match the $e^-$ and 
$e^+$ flux on Earth from dark matter annihilation in the Milky-Way halo to the measured cosmic-ray spectra, the required dark matter 
mass and annihilation cross section is sensitive to the different branching ratios and, hence, $m_{A'}$.   For example, for 
$m_{A'} < 2 m_{\mu}$, the dark matter would almost exclusively annihilate to $e^+e^-$.   However, for $m_{A'} \sim 700$~MeV near the 
$\rho$ or $\omega'$ resonance, the dark matter would annihilate dominantly to pions, decreasing the energy and yield of 
$e^+$ and $e^-$ per annihilation event; this would require a larger dark matter annihilation cross section and larger dark matter mass 
to fit the cosmic-ray spectra.  A large degeneracy thus exists between $m_{A'}$ and the dark matter mass and cross section.  
The degeneracy can be lessened somewhat, but not removed completely, since various other constraints will prefer some regions 
over others (e.g.~$m_{A'}\sim 100$~MeV over 700~MeV).  A set of benchmarks can be found in e.g.~\cite{Finkbeiner:2010sm}, but it will be of interest to cover the whole $m_{A'}$ region as proposed by HPS (even higher masses near 900 MeV would be of interest).  

The second parameter of interest is $\epsilon$.  Unfortunately, $\epsilon$ is almost completely unconstrained by the cosmic-ray data.  The reason is that $\epsilon$ determines the lifetime of the $A'$, but does not affect the dark matter annihilation cross section nor the decay branching ratios of the $A'$ (and hence the dark matter) to Standard Model final states;  and to explain the cosmic-ray anomalies, it is 
irrelevant if the $A'$ decays promptly or only after traveling for thousands of kilometers.  

If dark matter annihilation produces the high-energy $e^+e^-$ excess, correlated gamma-ray fluxes are expected from more distant astrophysical systems where dark matter can annihilate; such fluxes are not expected for the other possible explanations of the cosmic-ray excesses.  Such gamma-ray fluxes have not been seen by satellite or ground-based gamma-ray telescopes, like the the Fermi Gamma-ray Telescope, MAGIC, HESS, or VERITAS.  Bounds on the gamma ray flux from dwarf spheroidals \cite{Ackermann:2011wa}, the outer Milky Way \cite{DiffuseGalactic}, the Galactic Center (e.g. \cite{Papucci:2009gd,Hutsi:2010ai} and references therein), and distant galaxies \cite{Hutsi:2010ai,Zavala:2011tt} and clusters \cite{Huang:2011xr} can thus be used to constrain dark matter interpretations of the Pamela/FERMI excess.  
In a similar spirit, dark matter annihilation in the epoch of atomic recombination would leave an imprint in the cosmic microwave background radiation, which is similarly constrained \cite{CMBrefs}, and the self-interaction of dark matter via $A'$ exchange could affect the shape of galactic halos \cite{Feng:2009hw,Buckley:2009in}.  Each of these systems can be used to constrain models of the PAMELA/Fermi excess, albeit with large theoretical uncertainties.  
%While the present situation (described below) is quite inconclusive, it should be noted that new CMB data expected from Planck will improve sensitivity to dark matter annihilation at the time of recombination by a factor of 10 over WMAP \cite{CMBrefs}.  Thus Planck should either find evidence for dark matter annihilation with a high cross-section (providing further support for dark matter interpretations of the $e^+e^-$ excess), or more robustly constrain the minimal theories.  

The present situation can perhaps be summed up as follows: corroborating evidence for an explanation of the cosmic-ray excesses 
in terms of annihilating dark matter 
\emph{could} have shown up, but have not.  However, the size of the expected corroborating signals is very uncertain, so that the 
present situation is still inconclusive. 
Perhaps the best hope for a more definitive statement on a dark matter origin of the cosmic-ray excesses will arise from new CMB polarization data 
expected from Planck, which will improve sensitivity to dark matter annihilation at the time of recombination by a factor of 10 over WMAP \cite{CMBrefs}.  Planck should either find evidence for dark matter annihilation with a high cross-section (providing further support for dark matter interpretations of the $e^+e^-$ excess), or more robustly constrain the minimal theories.  
We note that the recent release of the Planck data only included the temperature data, and the resulting constraints will be 
only minimally improved in comparison to the previously available data from WMAP9, the Atacama Cosmology Telescope (ACT), and the 
South Pole Telescope (SPT).  A significant improvement is expected next year, when the CMB polarization power spectrum from 
Planck, ACT, and SPT will become available.  

A very important caveat to the above discussion is that we assumed that dark matter \emph{annihilations} to $A'$s are the origin of the excesses.  
Instead, dark matter \emph{decays} to an $A'$ and other light hidden sector particles are also a viable 
possibility~\cite{Essig:2010ye,Ruderman:2009tj}.  
In this case, an $A'$ mass 
below $\sim 1$~GeV is again motivated by the absence of an antiproton signal, but the size of the $e^+/e^-$ signal is set by the 
dark matter decay lifetime, and independent of the $A'$ mass (recall that in the case of dark matter annihilations, the $A'$ mass was an important 
ingredient in determining the size of the Sommerfeld enhancement and, thus, the annihilation cross section).  
Dark matter decays are less constrained than annihilations as a possible origin to the cosmic-ray excesses, as they produce a smaller 
corroborating gamma-ray signal (this signal is now proportional to the dark matter density $\rho$ and not $\rho^2$).  Also, no 
evidence is expected to show up in the CMB data, since the required dark matter lifetime to explain the cosmic-ray excesses is $\sim 10^{26}$~seconds, much larger than the time of the CMB formation ($\sim 10^{13}$~seconds).  

%Accounting for substructure at the level suggested by e.g.~\cite{Kamionkowski:2010mi},  light-$A'$ regions easily avoid constraints from the CMB, galactic center, dwarf galaxies, and dark matter self-interaction (though improved measurements of the CMB by Planck could severely constrain even these scenarios)~\cite{Slatyer:2011kg}.   Constraints on dark matter annihilation in distant galaxies (which would give rise to a diffuse and isotropic gamma-ray signal over the entire sky) are potentially severe even for these light-$A'$ models, but subject to theoretical uncertainties of several orders of magnitude.  In summary, light-$A'$ models of dark matter annihilation are consistent with all other data, but their viability depends on aspects of the dark matter distribution that are not yet reliably understood.

\paragraph{Direct Detection}
 The search for dark-matter-nuclear scattering has also seen considerable developments recently, but remains equally ambiguous.  Four experiments have reported excesses that \emph{may} be attributable to dark matter, although more mundane explanations 
are certainly possible: DAMA/Libra \cite{Bernabei:2010mq}, CoGeNT \cite{Aalseth:2010vx}, which also reported an annual modulation signal \cite{Aalseth:2011wp},  CRESST \cite{Angloher:2011uu}, and CDMS-Silicon \cite{Agnese:2013dwa}. 
If all or a subset of these signals have a dark matter origin,  they are most readily attributed to light dark matter ($\sim 10$ GeV).  
However, results from CDMS-Germanium \cite{CDMS}, XENON10 \cite{Angle:2011th}, and XENON 100 \cite{Aprile:2011hi} appear to disfavor the same parameter regions.  Experimental and detector uncertainties remain large enough that perhaps some model parameter space remains 
moderately consistent with all of these results \cite{Kelso:2011gd,Frandsen:2013cna}.  In fact, a recent re-analysis of the XENON10 constraint in~\cite{Frandsen:2013cna} found a mistake in the original XENON10 publication \cite{Angle:2011th}, which weakens the published limit by a factor of a few.  The situation is very fluid; more data is forthcoming and will shed light on the current situation.
Though the evidence for light dark matter is controversial, it does raise a puzzle: dark matter with such low masses and high couplings cannot easily interact through Standard Model forces (such as $Z$-boson exchange), without being excluded by measurements of the total $Z$ width at LEP. If indeed dark matter is light, then it seems most likely to interact through a new mediator, a possibility that HPS will probe in the case of an $A'$.

We note that heavy inelastic dark matter ($\sim 100-1000$~GeV) interacting with nuclei through $A'$-exchange was a possible explanation for 
these direct detection anomalies a few years ago, and its annihilation to $A'$s could also have explained the cosmic-ray excesses.  However, 
this possibility is now highly constrained by results from XENON100~\cite{Aprile:2011ts} and CRESST~\cite{Angloher:2011uu}.  Light dark matter, 
as mentioned above, is still viable.  In order to have a unified dark matter explanation of the cosmic-ray excesses and direct detection 
anomalies, one would now likely need two components of dark matter, one light and one heavy component.  Theoretical examples of 
such a possibility have been discussed in the literature, see e.g.~\cite{Essig:2010ye}.


\subsubsection{Heavy Photons and Muon $g-2$}

Besides being theoretically natural and having a possible connection to dark matter, an $A'$ could explain the discrepancy between the measured and 
calculated value of the anomalous magnetic moment of the muon $(a_\mu=g-2)$~\cite{Pospelov:2008zw}.  
This long-standing puzzle has several possible resolutions, but among the simplest new physics explanations
is the existence of a new force mediator that couples to muons, like the $A'$.  The contribution to $a_\mu$ of the $A'$ 
is like that of the photon, but suppressed by the mixing parameter $\epsilon^2$ and dependent on the $A'$ mass.  
The green region in Fig.~\ref{fig:hspaw-heavy-A'} is the 2$\sigma$ band in which the $A'$ can 
explain the discrepancy.  This is an intriguing region, which the HPS experiment will probe.  

\subsection{Update on Experimental Status}

The most recent (as of October 2012) comprehensive update summarizing the experimental status of $A'$ searches 
can be found in the presentations and summary talk of the Frascati ``Dark 2012'' workshop~\cite{Dark2012}.
All relevant measurements and constraints, as of this workshop, are included in Fig.~\ref{fig:hspaw-heavy-A'}.
One important change relative to a year ago is that an improved measurement of the Rydberg energy scale 
has allowed previous measurements of $g-2$ of the electron to constrain the allowed parameter space somewhat (in the low $A'$ mass range)
%has slightly reduced the range of allowed parameter space (on the low mass range) for an $A'$ to explain the $g-2$ of the muon 
%discrepancy
~\cite{endo:g2e,Davoudiasl:2012ig}.  
Additionally, searches for $A'$s in rare $\phi$ decays at KLOE and rare $\pi^0$ decays at WASA have slightly reduced 
the allowed parameter space on the high mass range~\cite{rarek}. 
%An ongoing search in the $e^+e^-$ spectrum of data recently gathered by MAMI~\cite{Merkel:2011ze} should also 
%be sensitive to the $A'$ mass range explored by KLOE, but with sensitivity down to roughly $\epsilon\sim 10^{-3}$. 
Finally, improved theoretical calculations and modeling of the experimental acceptance have led to slightly revised constraints on the 
$A'$ parameter space from past beam dump experiments sensitive to $A'$ production and decay to $e^+e^-$ pairs~\cite{andreas}.

\subsection{HPS physics with True Muonium}

Positronium and muonium, bound states of $(e^+ e^-)$ and $(\mu^+ e^-)$ pairs, respectively, have been
produced and studied \cite{Deutsch:1951zza,Friedman:1957mz,Hughes:1960zz}, but True Muonium has not yet been 
detected (see e.g. \cite{Holvik:1986ty,ArteagaRomero:2000yh,Brodsky:2009gx,Bilenky:1969zd,Hughes:1971,Malenfant:1987tm,Karshenboim:1998we,Owen:1972,Jentschura:1997ma,Jentschura:1997tv,Karshenboim:1998am}. 
Together with tauonium $(\tau^+ \tau^-)$ and tau-muonium $(\tau^{\pm} \mu^{\mp})$, True Muonium is among the most
compact pure QED systems. While $(\tau^+ \tau^-)$ and $(\tau^{\pm} \mu^{\mp})$ are difficult to detect since the $\tau$ has a
weak decay that competes with the QED decay, the $\mu$ is very long lived so that the decay of True
Muonium is purely a QED process. 

The detection of True Muonium would be a significant discovery and would constitute a further important test of QED.   
A number of applications of True Muonium measurements have been highlighted in 
\cite{Brodsky:2009gx}, designed to exploit True Muonium as a perturbative laboratory 
for QCD bound state physics. 
These include measuring dissociation cross-sections as a function of energy and lifetimes of the various states. 
More speculatively, the discrepancy between theory and experiment for $g-2$ of the muon~\cite{Bennett:2006fi} and the discrepant measurement of
the charge radius of the proton using muon bound states~\cite{Pohl:2010zza} suggest that further measurements of 
muon properties would be useful to resolve these puzzles. 

%That HPS will be the first experiment capable of detecting True Muonium is straightforward to understand. 
%The triplet True Muonium states $1^3S_1$, $2^3S_1$, and $2^3P_2$ all eventually decay 
%to $e^+e^-$ final states, with lifetimes long enough to leave a detectable displaced vertex.
%In that important respect, triplet True Muonium states behave just like $A'$s. 
%True Muonium production kinematics is a bit different. 
%In HPS, True Muonium will be produced by electron scattering off the high-$Z$ nuclear target. 
%The so-called ``single-photon'' production mechanisms gives rise to True Muonium states with kinematics extremely similar to $A'$s -- HPS will be most 
%sensitive to these. The ``three-photon'' production mechanism, which is typically larger, gives rise to True Muonium with characteristically lower energy. 
 
Studies of the production and dissociation of True Muonium suggest
that the yields in HPS should be sufficient for observation
\cite{Banburski:2012tk}, and are discussed further in section 3. That
HPS is uniquely suited for detecting True Muonium is straightforward
to understand.
The triplet True Muonium states $1^3S_1$, $2^3S_1$, and $2^3P_2$ all
eventually decay
to $e^+e^-$ final states, with lifetimes long enough to leave a
detectable displaced vertex.
In that important respect, triplet True Muonium states behave just like $A'$s.
True Muonium production kinematics is a bit different.
In HPS, True Muonium will be produced by electron scattering off the
high-$Z$ nuclear target.
The so-called ``single-photon'' production mechanisms gives rise to
True Muonium states with kinematics extremely similar to $A'$s -- HPS
will be most
sensitive to these. The ``three-photon'' production mechanism, which
is typically larger, gives rise to True Muonium with
characteristically lower energy. 
 
In addition to primary production mechanisms, there are a variety of secondary mechanisms
that are important in targets thicker than $\sim0.01\%$ radiation lengths.
This was studied in some detail in \cite{Banburski:2012tk}, where it was 
shown that $1^3S_1$ excitations (in the target) are the main source of 
$2^3S_1$ and $2^3P_2$ production.
The $2S$ and $2P$ state are especially long-lived, so this finding
suggests that HPS may first discover these states as they will comprise 
a sizable fraction of the $e^+e^-$ decays with displaced vertex in the range of $\sim 1$ cm to several cm. 
The $1S$ state will be the main component of the decays in the region of  $\sim 1$ cm and below. 

\subsection{HPS Searches for Hidden Sectors}

As highlighted in the Intensity Frontier Workshop report \cite{Hewett:2012ns},
a well-motivated class of beyond the Standard Model scenarios include new particles that interact
indirectly or very weakly with Standard Model matter (hidden sectors), possibly associated with dark matter. 
Low-energy and high-intensity experiments offer an excellent tool for exploring these
possibilities, complementary to the ongoing efforts at high energy colliders. 

HPS is primarily desinged to look for new sub-GeV $A'$s that decay into lepton pairs.
But if an $A'$ is part of a larger hidden sector, as is often assumed in the literature, 
some fraction of the decays could be more intricate.
For example, an $A'$ might decay into hidden sector particles, which in turn may decay back into Standard Model lepton or photons.  
These decays would tyically have displaced vertices and multiple leptons or photons. The phenomenolgofy of a variety of such scenarios 
have been considered in \cite{Strassler:2006im, Essig:2009nc} (and references therein). 
Search strategies to look for more general decays of $A'$s into hidden sector particles are actively being developed within HPS, 
and we comment in more detail on this physics opportunity in Section \ref{sec:hidden_ex}.

%%%%%%%Additional References to Add%%%%%%%%%%%%




\clearpage


\section{Proposed Measurements}


The primary goal of the proposed experiment will search for a heavy photon (dark photon) in the mass range from 20 MeV to 1000 MeV in at least two settings of beam energy 2.2 GeV and 6.6 GeV. 
HPS  ultimately relies upon the precision measurement of two quantities: the invariant mass of the A$^\prime$ decay products and the position of the decay vertex. By placing a tracking and vertexing detector immediately downstream of the target inside an analyzing magnet, the complete kinematic information required for A$^\prime$ reconstruction can be obtained from a single system, whose proximity to the target naturally maximizes the acceptance of a relatively compact detector and provides excellent momentum and vertexing resolution. A finely segmented, fast electromagnetic calorimeter, just downstream of the tracker,  provides a powerful high rate trigger, identifies electrons, and augments  the electron energy measurement. Behind the ECal a muon system consist of four planes of scintillator hodoscopes sandwiched between iron absorbers will be positioned. The muon system will provide trigger for ($\mu^+\mu^-$) detection and will be used for muon identification. It will extend search for high mass A$^\prime$ in di-muon decay mode where electromagnetic backgrounds are much reduced. Very high rate data acquisition systems, for the tracker, Ecal and muon system, make it possible to trigger and transfer data at $10$s of kHz, and run with negligible dead time.

The HPS experiment also has the potential to discover ``true muonium'', a bound state of a $\mu^+ \mu^-$ pair. 

\subsection{Search for the heavy photon}
\def \ap {A^\prime}
\def \map {m_{A^\prime}}
\def \thap {\theta_{A^\prime}}
\subsubsection{Heavy Photon Signal}
\label{sec:apsignal}

\begin{figure}
\includegraphics[scale=1]{measurements/Aprime-diagram.pdf}
\caption{Diagram of  $\ap$ production by bremstrahlung off of an incoming electron scattering with an atomic nucleus.}
\label{fig:apdiagram}
\end{figure}

$\ap$ particles are generated in electron collisions on a fixed target by a process analogous to ordinary photon bremsstrahlung, see Figure \ref{fig:apdiagram}.  This can be reliably estimated in the Weizsäcker-Williams approximation (see [1-4]).  When the incoming electron has energy E0, the differential cross-section to produce an $\ap$ of mass $m_{\ap}$ with energy $E_{\ap}\equiv x E_0$ is 
\begin{equation}
\frac{d\sigma}{dxd\cos{\theta_{\ap}}}\approx \frac{8Z^2\alpha^3\epsilon^2 E_0^2 x}{U^2}\tilde{\chi}\times\left[\left(1-x+\frac{x^2}{2}\right)-\frac{x(1-x)m_{\ap}^2E_0^2x\theta_{\ap}^2}{U^2}\right]
\end{equation}
where Z is the atomic number of the target atoms, $\alpha = 1/137$,  is the angle in the lab frame between the emitted A' and the incoming electron, 
\begin{equation}
U(x,\theta_{\ap})=E_0^2x\theta_{\ap}^2+m_{\ap}^2\frac{1-x}{x}+m_e^2x
\end{equation}
is the virtuality of the intermediate electron in initial-state bremsstrahlung, and  is the Weizsacker-Williams effective photon flux, with an overall factor of  removed.  The form of  and its dependence on the $\ap$ mass, beam energy, and target nucleus are discussed in\ref{something}.  For HPS with $E_0$ = 6.6 GeV, we find $\tilde{\chi}\sim 7 (4, 1)$ for $m_{\ap}$ = 100 (200, 500) MeV/$c^2$.
The above results are valid for 
\begin{equation}
m_e\ll m_{\ap}\ll E_0  , ~~ x\theta_{\ap}^2\ll 1.
\end{equation}

For $m_e\ll m_{\ap}$, the angular integration gives
\begin{equation}
\frac{d\sigma}{dx}\approx \frac{8Z^2\alpha^3\epsilon^2 x}{m_{\ap}^2}\left(1+\frac{x^2}{3(1-x)}\right)\tilde{\chi} .
\end{equation}
The rate and kinematics of $\ap$ radiation differ from massless bremsstrahlung in several important ways:
\begin{itemize}
\item  {\bf Rate}: The total $\ap$ production rate is controlled by $\alpha^3\epsilon^2 / m_{\ap}^2$.  
 Therefore, it is suppressed relative to photon bremsstrahlung by $\sim \epsilon^2 m_e^2/m_{\ap}^2$.  Additional suppression from small $\tilde{\chi}$  occurs for large $m_{\ap}$  or small $E_0$.
\item {\bf Angle}:  $\ap$ emission is dominated at angles$\theta_{\ap}$ such that $U(x,\theta_{\ap}\lesssim 2 U(x,0)$ (beyond this point, wide-angle emission falls as $\theta_{\ap}^4$).  For near it's median value, the cutoff emission angle is
\begin{equation}
\theta_{\ap,max}\sim max\left(\frac{\sqrt{\map m_e}}{E_0},\left(\map/E_0\right)^{3/2}\right),
\end{equation}
which is parametrically smaller than the opening angle of the $\ap$ decay product, $\sim  \map/E_0$.  Although this opening angle is small, the backgrounds mimicking the signal (discussed in Section \ref{sec:physicsbkgs}) dominate at even smaller angles.
\item {\bf Energy}:  $\ap$ bremsstrahlung is sharply peaked at $\approx 1$, where $U(x,0)$ is minimized.  When an $\ap$ is produced, it carries nearly the entire beam energy.  In fact, the median value of (1-x) is $\sim {\rm max}\left(\frac{m_e}{\map},\frac{\map}{E_0}\right)$.  
\end{itemize}
The  latter two properties are quite important in improving signal significance, and are discussed
further in Section \ref{sec:physicsbkgs}.

Assuming the $\ap$ decays into Standard Model particles rather than exotic, it's boosted lifetime is
\begin{equation}
l_0 \equiv \gamma c\tau \approx \frac{0.8 cm}{N_{eff}} \left(\frac{E_0}{10 GeV}\right)\left(\frac{10^{-4}}{\epsilon}\right)^2\left(\frac{100 MeV}{\map}\right)^2,
\end{equation}
where we have neglected phase-space corrections, and $N_{eff}$ counts the number of available decay channels.  If the $\ap$ couples only to electrons, then $N_{eff}=1$.  If the $\ap$ mixes kinetically with the photon, the $N_{eff}=1$ for $\map < 2m_\mu$ and $2+R(\map)$ for $\map \geq 2 m_\mu$, where \cite{eehadrons}
\begin{equation}
R  =\left. \frac{\sigma(e^+e^-\rarr hadrons)}{\sigma (e^+e^- \rarr \mu^2\mu^-)}\right|_{E=\map} . 
\end{equation} 
For th ranges of $\epsilon$ and $\map$ probed by this experiment, the mean decay length $l_0$ can be prompt or as large as tens of centimeters.  

The total number of $ap$ produced when $N_e$ electrons scatter in a target of $T\ll 1 $ radiation lengths is
\begin{equation}
N\sim N_e\frac{N_0 X_0}{A}T\frac{Z^2\alpha^2\epsilon^2}{\map^2}\tilde{\chi}\sim N_e C T \frac{\epsilon m_e^2}{\map^2},
\end{equation}
where $X_0$ is the radiation length of the target in g/cm$^2$, $N_0 \approx 6\times 10^{23} mole^{-1}$ is Avogadro's number, and $A$ is the target atomic mass in g/mole.  The numerical factor $C\approx 5$ is logarithmically dependent on the choice of nucleus (at least in the range of masses where the form-factor is only slowly varying) and on $\map$, because, roughly, $X_0 \propto \frac{A}{Z^2}$ (see \cite{somerefs}).  For a Coulomb of incident
electrons, the total number of $\ap$s produced is given by
\begin{equation}
N\sim 10^5 \left(\frac{N_e}{1 C}\right)\tilde{\chi}\left(\frac{T}{0.1}\right)\left(\frac{\epsilon}{10^{-4}}\right)^2\left(\frac{100 MeV}{\map}\right)^2.
\end{equation}

\subsubsection{Radiative and Bethe-Heitler Trident backgrounds}
\label{sec:physicsbkgs}
The stark kinematic differences between QED trident backgrounds and the $\ap$ signal can be used to advantage to maximize the signal to background ratio. QED tridents dominate the final event sample, so we consider their properties in some detail here.

\begin{figure}
\includegraphics[scale=1]{measurements/rad-bh-diagrams.pdf}
\caption{Sample diagrams of (left) radiative trident ($\gamma^*$) and (right) Bethe-Heitler trident reactions that comprise the primary background to the $\ap\rarr l^+l^-$  search.}
\label{fig:radbhdiagram}
\end{figure}

The irreducible background rates are given by the diagrams shown in Figure \ref{fig:radbhdiagram}. These trident events can be usefully separated into 'radiative' diagrams (Figure \ref{fig:radbhdiagram} (a)), and 'Bethe-Heitler' diagrams (Figure \ref{fig:radbhdiagram} (b)), that are separately gauge-invariant.

The contribution from the radiative diagrams (Figure \ref{fig:radbhdiagram} (a)) alone is also useful as a guide to the behavior of $\ap$ signals at various masses. Indeed, the kinematics of the A' signal events is identical to the distribution of radiative trident events restricted in an invariant mass window near the A' mass. Moreover, the rate of the A' signal is simply related to the radiative trident cross-section within the spectrometer acceptance and a mass window of width $\delta m$ by \cite{4}
\begin{equation}
\frac{d\sigma\left(e^- Z \rarr e^- Z(\ap\rarr l^+l^-)\right)}{d\sigma\left(e^- Z \rarr e^- Z(\gamma^*\rarr l^+l^-)\right)}=\frac{3\pi\epsilon^2}{2 N_{eff}\alpha}\frac{\map}{\delta m}
\end{equation}
This exact analytic formula was also checked with a MC simulation of both the $\ap$ signal and the radiative trident background restricted to a small mass window $\delta m$, and we find nearly perfect agreement. Thus, the radiative subsample can be used to analyze the signal, which simplifies the analysis considerably.

It is instructive to compare kinematic features of the radiative and Bethe-Heitler distributions, as the most sensitive experiment maximizes acceptance of radiative events and rejection of Bethe-Heitler tridents. Although the Bethe-Heitler process has a much larger total cross-section than either the signal or the radiative trident background, it can be significantly reduced by exploiting its very different kinematics. In particular, the $\ap$ carries most of the beam energy (see discussion in Section \ref{sec:apsignal}), while the recoiling electron is very soft and scatters to a wide angle. In contrast, the Bethe-Heitler process is not enhanced at high pair energies. Moreover, Bethe-Heitler processes have a forward singularity that strongly favors asymmetric configurations with one energetic, forward electron or positron and the other constituent of the pair much softer.
These properties are discussed further in the Appendix of [4], and illustrated in Figure \ref{fig:tridentkinematics}.


\begin{figure}
\includegraphics[scale=1]{measurements/rad-bh-energy.pdf}
\includegraphics[scale=1]{measurements/E1vsE2.pdf}
\caption{ Left: The distribution of Bethe-Heitler background events (black) and $\ap$ signal events (red) as a function of the sum of the electron and positron energy. Note that the signal is peaked at high energies, while the background is peaked at much lower energies. Right: The distribution of the positron versus electron energy for Bethe-Heitler background events (black dots) and A’ signal events (red dots). Note that in both plots neither the signal nor background events have been normalized to the correct number. In reality, the number of background events is much larger than the number of signal events. Also, note that the electron energy here refers to the energy of the electron produced in the reaction, not the recoiling beam electron.}
\label{fig:tridentkinematics}
\end{figure}


\subsection{Search for true muonium}

The proposed HPS experiment has the potential to discover ``true muonium'', a bound state of a
$\mu^+ \mu^-$ pair, denoted here by $(\mu^+ \mu^-)$. 
We expect that HPS will discover the 1S, 2S, and 2P true muonium bound states with its proposed run plan. 
The detection of these states should demonstrate the capability of the HPS experiment 
to identify rare separated vertex decays, and will provide a natural calibration 
tool for improving searches for heavy photons. Details of the production and detection of true muonium using HPS detector can be found in \cite{HPS_PROP_UPD}. 
The $(\mu^+ \mu^-)$ atom is hydrogen-like, and so has a set of excited states characterized by a principal quantum number n. 
The binding energy of these states is E = $-1407$ eV/n$^2$. The $(\mu^+ \mu^-)$ ``atom'' can be produced by an electron beam incident on a target such 
as tungsten \cite{Holvik:1986ty,ArteagaRomero:2000yh}. 

With the existing proposal, HPS will search for true muonium
just as it does for heavy photons with separated vertices, requiring a vertex cut at about 1.5 cm to reject almost all
QED background events, then searching for a resonance at 2 m$_{\mu}$. An additional cut 
on the total energy of the $e^+e^-$ pair of $E_{e^-}+E_{e^+}> 0.8 \ E_{beam}$ will also be required
for triggering. 

Based on \cite{toAppear}, the total production yield for 1S, 2S, and 2P (including secondary production)
leaving a target of thickness $t_b$(or larger) and satisfying the above requirements is,
\begin{equation}
N_{(\mu^+ \mu^-)} = 600 \left( \frac{I}{450 \ nA} \right) \left( \frac{t}{3 \ months} \right)
\end{equation}
%
where a beam energy E$_{beam} = 6.6$ GeV, and the nominal conditions
of 450 nA beam current for 3 months ($\sim 7.8 \times 10^6$ s) on a single foil has been assumed.
The vertices near the cut of $1.5$ cm will be dominated by the 1S state, while 
a tail of vertices extending out beyond a few cm is dominated by 2S and 2P. 

Accounting for all the efficiencies associated with a separated vertex search, we would expect to see about 60--100 true muonium events 
(we caution that the acceptance parameterization here is uncertain at the 50\% level).
The HPS experiment should be able to identify enough events to claim a discovery, and in addition, should be able to measure the mass of true muonium.  There are certainly other properties of true muonium that would be interesting to measure.  A measurement of the lifetimes would be interesting, as the lifetimes are sensitive to physics that couples to leptonic currents.  With enough statistics, it should be possible to perform a measurement of the lifetimes of the 1S, 2S, and 2P states; work is ongoing to investigate this possibility.  

\subsection{Other searches for hidden sector particles}


While the primary motivation for the conception of the HPS experiment is a search for $\ap$ decaying to lepton pairs, following Arkani-Hamed et al, we should make sure that HPS is sensitive to other kinds of Hidden Valley (HV) scenarios. As Strassler and Zurek pointed out, HV provide plethora of natural ways to explain the nature of the Dark Matter. The recent discovery of a boson that seems to have the properties of the Higgs boson of the Standard model also brings an old problem to the fore – why is the Higgs mass is so light compare to the Planck scale? Searches for supersymmetric partners and other particles with strong and/or weak charges at the energy frontier so far came out negative, pushing the range of allowed masses for such particles higher and higher. HPS will be the first in hopefully a series of experiments that complements the energy frontier by looking for light particles that couple to the SM particles through new, very weak forces. Just as exploring the energy frontier at the LHC one should be ready for unexpected, so should we be exploring the coupling frontier at the HPS.

In a general HV scenario, the new fermions may be lighter then the $\ap$ and have their own QCD-like forces forming hidden mesons and baryons some of which could be constituents of Dark Matter. Some of such mesons would decay into SM fermion pairs – either democratically or with mass-dependent branchings.


\begin{figure}
\includegraphics[scale=1]{measurements/multilepton-diagram.pdf}
\caption{Sample diagram of a non-Abelian hidden sector interaction.}
\label{fig:mldiagram}
\end{figure}

As an example, we consider the case where the $\ap$ decays into a pair of dark mesons ($\pi_v$), which in turn promptly decay into electron pairs (see Fig \ref{fig:mldiagram}). In the simulation, we took $\ap$ and $\pi_v$ masses to be 0.3 and 0.1 GeV respectively. We generated 25,000 events at 2.2 GeV incoming beam and put them through the full simulation and reconstruction.

We found that the probability of having all four electrons in the detector acceptance is fairly small. Only XXX events had 4 or more reconstructed tracks and of those more then half had (fake, misreconstructed, other background) tracks. Once we required two $\pi_v$ candidates with consistent masses ($\Delta M<0.025$ GeV) the $\ap$ signal is clearly visible in the two-dimensional plane of average $\pi_v$ mass and $\ap$ mass (see Fig \ref{fig:mlinvmass})     
The resulting signal efficiency is $50/25000 \sim 2\times 10^{-3}$.


\begin{figure}
\includegraphics[scale=1]{measurements/ML-m1vsm2pdf.pdf}
\includegraphics[scale=1]{measurements/ML-mAvsmAvgf.pdf}
\caption{Left:  Invariant masses of the two reconstructed $\pi_v$ candidates.  Right:  The 4-lepton invariant mass verses the average  $\pi_v$ candidate mass.}
\label{fig:mlinvmass}
\end{figure}
 
	In an effort to enhance the signal efficiency we tried to take advantage of the fact that $\ap$ is produced with very little transverse momentum. Therefore, in the events with just three reconstructed electrons, we assume that the fourth electron is out of acceptance, and its transverse momentum is equal to minus the sum of the transverse momenta of the three reconstructed electrons. The longitudinal momentum of the fourth electron is then given, with two-fold ambiguity, by requiring the $\pi_v$ candidate it makes to have the same mass as the fully reconstructed $\pi_v$. We resolve the ambiguity by picking the solution with the largest longitudinal momentum, as long as the total $\ap$ energy does not exceed the incoming beam energy. Thus reconstructed $\pi_v$ and $\ap$ mass distributions are shown in Fig \ref{fig:mlreciolmass} (each event can give up to two entries since there are two possible reconstructed $\pi_v$ hypotheses). The $\ap$ mass resolution is 0.07 GeV, and the signal efficiency is around 3\%.



\begin{figure}
\includegraphics[scale=1]{measurements/ML-recoilMass.pdf}
\caption{Left:  The $\pi_v$ candidate recoill mass.  Right:  The $\ap$ candidate recoil mass.}
\label{fig:mlreciolmass}
\end{figure}

	We estimated the background using the same simulated QED trident sample as for the baseline $\ap$ search… (results of that study here)



\clearpage



\section{Description of the HPS setup}
\label{setup}

\subsection{Overview } 


HPS will utilize a setup located at the upstream end of  experimental Hall-B at Jefferson Lab. The setup will be based on a three-magnet chicane, the second dipole magnet serving as the analyzing magnet for our forward spectrometer. The detector package will include a silicon tracker, an electromagnetic calorimeter, and a muon detector\footnote{The muon system will be designed and built by consortium of three universities, The Collage of William and Mary, Rutgers University, and Old Dominion University, using NSF MRI funding. Details of the system are described in Appendix \ref{sec:muon}.}. High luminosities are needed to search for heavy photons with small couplings and masses in the $20$ to $1000$ MeV range. Utilizing CEBAF's essentially continuous duty cycle, the experiment can simultaneously maximize luminosity and minimize backgrounds by employing detectors with short live times and rapid readout. The HPS setup is designed to run with $> 200$ nA electron beams at energies from $1.1$ GeV to $6.6$ GeV impinging on a tungsten target of up to 0.0025 $X_{0}$ located  10 cm upstream of the first layer of the tracker.

The HPS tracker consists of six double layer planes, 36 microstrip sensors in total. Placing the planes of the tracker in close proximity to the target means that the primary beam must pass directly through the middle of the tracking detector. This has necessitated that the sensors don't encroach on a ``dead zone,'' where multiple Coulomb scattered beam particles and radiative secondaries are bent into the horizontal plane, the so-called ``wall of flame.''  However, since the energy released in the decay of a low mass A$^\prime$ is small relative to its boost, the opening angle between decay daughters can be quite small. To maximize the acceptance for low mass A$^\prime$s, the vertical extent of the dead zone must be minimized and sensors placed as close as possible to the beam, so our design incorporates precision movers that can bring the silicon detectors  to the required positions. Since interactions of the primary beam with air or even helium at atmospheric pressure gives rise to low-momentum secondaries that generate unacceptable occupancies, we have chosen to place the entire tracking and vertexing system in vacuum, in the Hall B pair spectrometer's magnet vacuum chamber. Silicon microstrip sensors are used in the tracker/vertexer because they collect ionization in $10$'s of nanoseconds and produce pulses as short as $50-100$ ns. The sensors are read out continuously at $40$ MHz using the APV25 chip, developed for the CMS experiment at the LHC. Running the high speed silicon module readout in vacuum further requires a vacuum compatible cooling system, and data and power vacuum feedthroughs. All these features are incorporated  in the design of the apparatus, as described below and have been tested in the May 2012 test run.


The electromagnetic calorimeter (Ecal) consists of 442 PbWO$_4$ crystals (reconfigured from the CLAS Inner Calorimeter) that are read out with APDs and amplifiers. The short  pulse widths allow the ECal to run at very high rates. The Ecal data is digitized in the JLAB FADC250, a $250$ MHz flash ADC developed for the JLAB $12$ GeV Upgrade program.  The full analogue information from the FADCs coupled with the fine spatial information of the calorimeter is available to the trigger, which uses energy deposition, position, timing, and energy-position correlations to reduce the trigger rate to a manageable $\sim 30$ kHz. Like the tracker system, the electromagnetic calorimeter is split to avoid impinging on the ``dead zone.'' The beam and radiative secondaries pass between the upper and lower ECal modules, which are housed in temperature-controlled enclosures, needed to stabilize the energy calibration. 

%The muon detection system will be installed behind the ECal, which has absorbed most of the electromagnetic background produced in the target. The remaining backgrounds will be attenuated by the first absorber layer of the muon system. The muon system will consist of four double layers of scintillator hodoscopes sandwiched between iron absorbers. Light from scintillator strips will be transported to photo-multiplier tubes via wave-length shifting fibers embedded within the strips. As in case of the ECal, muon system will be divided into two parts, beam up and beam down. There is a vacuum chamber between the two parts to allow the beam and radiated secondaries to be transported in vacuum.

In the following, the various elements of the experiment are discussed in more detail, beginning with the beamline and experimental controls, continuing with the tracker/vertexer and electromagnetic calorimeter, then data acquisition and trigger systems, and offline computing at the end. 
 
 


 \subsection{HPS beamline}
 \label{setup:beamline}
 
The Hall-B beam line, magnetic elements, beam profile monitors, and beam position and current monitors upstream of the HPS setup will be 
used as is (after slight modifications for 12 GeV). The only modification needed for the upstream part of the beam line is the addition of 
a collimator upstream of the Hall-B tagger magnet. The role of the collimator is to prevent the beam from directly hitting the Si-tracker in an 
event in which the high intensity beam can move up or down from its nominal position. The collimators, which  can consist of  a $1$ cm 
thick tungsten plates with different size oval holes (to match the beam profile), can be incorporated into the Hall-B photon tagger radiator ladder 
to provide vertical alignment. Horizontal alignment of the whole system will also be needed for fine tuning of the collimator position 
relative to the beam. Downstream of the HPS setup, there will be two beamlines, the electron beam line that will transport electron beam to the 
Hall-B electron beam dump, and a photon beam line that will transport the photon beam generated in the target to a photon beam dump mounted 
on the space frame. The electron beam will be transported in vacuum all the way through to the beam dump. Following the vacuum 
chamber in the last chicane dipole, the photon beam will go to the dump in Helium bag. There will be an H-corrector installed on the electron 
line after 
the HPS chicane to compensate any possible mis-steering of the beam in the chicane and to make sure that the electron beam stays on the 
original beamline to the dump. A YAG viewer will be used to monitor beam position jsut before the dump. The beam position on the dump monitor 
must stay unchanged before and after energizing the chicane.  
 
 \subsubsection{Layout of the HPS setup} 
 \label{setup:layout}
 
The HPS experiment will use the same three magnet chicane that was used for the CLAS Two Photon Exchange experiment (TPE). The layout of 
the beam line and the chicane is shown in Figure \ref{fig:ebeam}. The Hall B pair spectrometer magnet, an 18D36 (pole length $91.44$ cm, 
gap $45.72\times 15.24$ cm$^2$, max-field $1.5$ T), will serve as the analyzing magnet. The dipole field direction (Y) is perpendicular 
to the horizontal (XZ) plane. The Hall B �Frascati� H magnets (pole length $50$ cm, gap $21\times 8.25$ cm$^2$, max-field $1.2$ T) 
will be used as the first and the last dipoles of the chicane. The analyzing magnet will be operated at a $0.25$ T-m field for $1.1$ 
GeV, a $0.5$ T-m field for $2.2$ GeV, and a $1.5$ T-m field for $6.6$ GeV running. The two bending magnets will be set to  $0.125$ 
T-m, $0.25$ T-m, and $0.75$ T-m fields, respectively. The distance between the centers of the magnets will be about $50$ cm bigger than 
it was for the TPE run, to accommodate the muon system. The location of the analyzing magnet along the beam will be exactly the same  
as it was for the TPE run. 

The analyzing magnet will be displaced to beam left by $\sim 11$ cm in order to optimize the detector acceptance for e$^+$ 
and e$^-$, see Figure \ref{fig:ebeamt}.
 
 \begin{figure*}[!ht]
%\includegraphics[scale=0.18]{beamline/05055-01_NT.pdf}
\includegraphics[angle=180, width=0.85\textwidth]{beamline/HPS12_66840e051XX_ELEV.pdf}
\caption{\small{Beam line configuration for the HPS test run with electron beams. The chicane configuration is similar to a previously run 
CLAS experiment.}}\label{fig:ebeam}
\end{figure*}

\begin{figure*}[t]
\includegraphics[angle=180., width=0.85\textwidth]{beamline/HPS12_66840e051XX_PLAN.pdf}
\caption{\small{Top view of the beam line configuration for the HPS test run with electron beams. The analyzing magnet is shifted by 4 inches 
(110 mm) to beam's left to get optimal acceptance for both $e^+$s and $e^-$s.}}\label{fig:ebeamt}
\end{figure*}

 
The HPS target is positioned at the upstream edge of the analyzing magnet's pole. The distance from the target to the first layer of the 
silicon tracker is $10$ cm, and to the face of the electromagnetic calorimeter $\sim 137$ cm. There is continuous vacuum for the electron 
beam throughout the entire setup ending in the Hall B electron beam dump. The Si-tracker and the target will be located inside the Hall-B 
pair spectrometer vacuum chamber. The SVT vacuum box is mounted on the upstream end of the analyzing magnet vacuum chamber to provide 
connections for the SVT motion system, the cooling system, power and signal cables, and the target motion system. The Ecal vacuum chamber 
is attached to the downstream end of the analyzing magnet vacuum chamber, above and below which are placed the Ecal modules. Downstream of 
the Ecal vacuum chamber, another vacuum chamber is attached, leading through the muon system and the downstream chicane magnet.

The analyzing magnet, the Hall B pair spectrometer dipole, has its own power supply. The �Frascati� H magnets will use one common power 
supply and will be powered by the Hall B "mini-torus" power supply. There will be a shunt installed between the two �Frascati� magnets 
to allow independent small changes in currents on those two magnets if necessary (as it was done during the TPE experiment, although 
never used). Both power supplies are bipolar, so the magnets can be degaussed when needed. From available field map data at $900$ A, 
the$\int{Bdl}$ of Frascati H magnets along the path of the electron beam is $0.663$ T-m. The specified max current for these magnets 
is $950$ A. In order to get $0.75$ T-m an additional $10\%$ increase in field value will be needed. From initial evaluation of the 
magnet design and characteristics, it should not be a problem to run them at $10\%$ higher currents over their specified max current. 
If this is not possible, reducing the gap by $1/3$ inch can yield the desired $\int{Bdl}$. 

 
\clearpage

\subsubsection{Running Conditions} 
 
The HPS will use $\sim 1.1$ GeV, $\sim 2.2$ GeV, and $\sim 6.6$ GeV electron beams of up to $500$ nA incident on a thin tungsten (W) target. 
Operational experience (with 6 GeV machine) showed that the CEBAF beam is very clean, and is contained within $\pm 0.5$ mm with halo at the level 
of less than $10^{-5}$. It is expected that the beams from the 12 GeV machine will be of comparable quality, at least for up to 3-pass 
beams (up $\sim 6.6$ GeV), so the primary electron beam should cleanly pass through the �dead zone� gap of the HPS setup. 
 
For optimizing the vertexing performance and acquiring physics data, an asymmetric beam profile is desirable. Since the vertex resolution 
in the non-bend plane will be high, beam sizes of $<50 ~\mu$m in the Y direction are preferable. The momentum measurement will not benefit 
from small beam sizes in the X direction, and if the beam sizes in both dimensions are too small, the target foil will overheat. For these reasons the 
required beam sizes for HPS will be $\sigma_X \sim 250 \mu$m and $\sigma_Y < 50 \mu$m.
The HPS beam parameter requirements are presented in Table \ref{tb:beam}. 
 
 \begin{table}[!htb]
 \centering
 \begin{tabular}{|c|c|c|c|c|}
\hline
%Parameter & \multicolumn{3}{c|}{Requirement/Expectation} &Unit \\ \hline 
Parameter & \multicolumn{3}{c|}{Requirement} &Unit \\ \hline 
E & $1100$ & $2200$ & $6600$ & MeV \\ \hline
$\delta$E/E & \multicolumn{3}{c|}{$< 10^{-4}$} & \\ \hline 
Current & $< 200$ & $< 400$ & $< 500$ & nA \\ \hline
Current Instability & \multicolumn{3}{c|}{$< 5$} &\% \\ \hline 
$\sigma_x $& \multicolumn{3}{c|}{$< 300$ } & $\mu$m \\ \hline 
$\sigma_y$&\multicolumn{3}{c|}{$<50$} & $\mu$m \\ \hline
Position Stability & \multicolumn{3}{c|}{$< 30$} &$\mu$m \\ \hline
Divergence& \multicolumn{3}{c|}{$< 100$} & $\mu$rad \\ \hline 
Beam Halo ($> 5\sigma_Y$) & \multicolumn{3}{c|}{$< 10^{-5}$} & \\ \hline
 \end{tabular}
\caption{ Required beam parameters.} 
\label{tb:beam}
\end{table}

The B-line optics in the $6$ GeV era was checked using simulation and a beam test of the system. The optics program ELEGANT \cite{elegant} was 
used to determine the optimized B-line parameters needed to achieve an asymmetric beam size, $\sigma_X \approx 250 \mu$m and $\sigma_Y\approx 
20 \mu$m, at the HPS test run target location. 
Beam tests were conducted in Hall B to validate these optics simulations during the Two Photon Exchange experiment when $2.2$ GeV beam was 
available (February of 2011). Parameters were set for a beam profile of $\sigma_X \approx 300 ~\mu$m and $\sigma_Y\approx 10 \mu$m at the 
Hall B �tagger� beam profiler ($\sim 8$ meters upstream of the proposed HPS target location). Several beam profile scans with different 
scanner and data readout speeds were performed to check the beam stability and the systematics in the measurements. 
One of the scans is shown in Figure \ref{fig:profile_test}. As can be seen from the figure, the required profile can be reliably  achieved. 
Several scans performed over two hours resulted in a consistent and stable beam profile. Based on the width of the Y-profile, beam position 
stability is  $< 20 \mu$m. Note that any beam motion with more than 10 $\mu$m amplitude and faster than 1Hz is included in the scan.

\begin{figure*}[t]
\includegraphics[scale=0.45]{beamline/harp_02.pdf}
\caption{\small{Wire harp scan after loading optics parameters from the ELEGANT program. The wire scan speed was 0.1mm/s, 
readout speed is 15Hz. Based on the width of the Y-profile, the beam position stability is  $< 20 \mu$m. Note: any beam motion with more 
than 10 $\mu$m amplitude and faster than 1Hz is included in the scan.}}\label{fig:profile_test}
\end{figure*}
 
% {\bf {\it ELEGANT for 12 GeV machine} from Arne}
The beamline optimizations have been performed for the 12 GeV CEBAF machine including the proposed changes for Hall-B/CLAS12 operations. 
Using the program ELEGANT and inputting the new locations of magnetic elements and their field maps, the beam profile was optimized at the HPS 
target location. In Figure \ref{fig:hps2014} the beam sizes and the beam angles are shown for $6.7$ GeV setup.  The required beam 
size is achievable within the operataing specifications of all the quadrupoles.  Since HPS will run at beam energies $<6.7$ GeV it is straight 
forward to scale (linearly) the magnets down to the other energies.  The beam size/angle (beam transport) remains the same 
for $1.1$, $2.2$, and $4.4$ GeV energies with the exception of the small emittance increase at $6.7$ GeV.
  
  \begin{figure*}[t]
%\includegraphics[scale=0.35]{beamline/hps_test_300_by_15.jpg}
\includegraphics[scale=0.35]{beamline/hps2014_beamsize.jpg}
\includegraphics[scale=0.35]{beamline/hps2014_beamangle.jpg}
\caption{\small{Beam sizes in X and Y along the B-line in the upstream tunnel and in the region of the HPS test run setup. At the HPS target an 
asymmetric beam profile $\sigma_X=300 \mu$m and $\sigma_Y=20 \mu$m can be achieved with existing B-line optics.}}\label{fig:hps2014}
\end{figure*}


 
 \subsubsection{Beam Diagnostics}
 \label{setup:beam_dignostics}
 
Beam position and current will be controlled using inputs from two sets of cavity beam position monitors (BPMs), that are located in the 
upstream tunnel. 
Sets of corrector dipoles and quadrupoles are routinely used to tune the beam for Hall B (2C21 to 2C24). A pair of BPMs, 2C21 and 2C24, 
will define the incoming trajectory of the beam and are included in the fast feedback loop. Readings from these BPMs will be used to 
maintain stable beam positions and currents. The stability of beam positions at two different locations also ensures the stability of the 
beam inclination.
 
The beam profile will be measured using three wire scanners. Two are installed in the tunnel, the first one at 2C23, and the second one before the 
Hall-B tagger magnet, (2c24 harp, called "tagger harp"), about 8 meters upstream of the HPS target. The third wire harp, 2H00 harp, will 
be located just before the first chicane dipole. The first two profilers will be used to establish the required beam parameters during the 
initial setup. The Hall-B tagger magnet will be energized when beam tune is in progress, diverting the beam away from the HPS apparatus.
After an acceptable beam profile is achieved, the tagger magnet will be degaussed and turned off, and the beam 
will be put through the HPS system and the beam profile will be checked using the 2H00 wire harp. 
The backgrounds in the HPS silicon tracker from beam profiling using the 2H00 harp have been simulated. At $5$ nA beam current, the 
radiation damage is equivalent to about $10$ sec. of production beam current on the HPS target, so is not a concern.

A set of tungsten beam-fiducial wires will be installed immediately in front of the silicon detectors in the experiment's analyzing magnet. One horizontal wire, 20 micron diameter, and one 30 micron wire at 9 degrees to the horizontal, will be mounted on a frame attached to the upper movable silicon support plate, and similarly for the bottom plate. The frames for the wires are wide enough that they do not occlude the silicon active area. The wires can be used to locate the position of the beam relative to the silicon. To accomplish this safely, the vertical separation between the front silicon sensor and its nearest wire is 8 mm. This separation, and the small wire diameters, also means that, when the sensors are positioned for data taking, the wires will have a negligible effect on acceptance. The wires are also available for use as a fairly conventional wire scanner. In particular they can provide information about the minor and major axes, and the tip angle (roll), of a strongly elliptical beam.

An insertable YAG screen beam viewer will be installed in the downstream alcove of Hall-B, before the Faraday cup, $\sim 40$ meters downstream 
of the HPS target. Both the position and profile of the beam will be used to setup the chicane magnets and to monitor beam quality during the run. 
A set of beam halo counters mounted along the beam line provides continuous and fast monitoring of the beam conditions. These counters are 
like those used for beam profile measurements. Excess noise in the beam halo counters triggers the machine fast
shutdown system (FSD) in order to terminate beam in the event of beam excursions which could damage the HPS detectors. The FSD will occur 
in less than 40 $\mu$s. In addition to halo counters, a beam offset monitor (BOM) will be installed upstream of the 2H00 wire harp. It is 
similar to the BOM used in CLAS. A short quartz cylinder, about $6$ mm OD and $4$ mm ID, with optical fibers attached around the edge will be centered 
on the beam. Even a few electrons in the beam tail will generate light in the cylinder that will be detected in a multi-anode PMT attached to 
the readout fibers. Errant beam motion towards the collimator located upstream of the tagger magnet will generate more light and increase the counts 
in the quartz cylinder, signalling a potential problem. The BOM will be wired to FSD as part of the equipment protection system. 
 
\subsubsection{Vacuum chambers} 
\label{vacchamb}
 
The SVT vacuum box will be attached to the existing magnet vacuum chamber as shown in Figure \ref{fig:svtvbox}. Power, high voltage, and data 
signals to and from the hybrids are connected through two $8$" flanges on the sides of the vacuum box. Two vertical linear motion mechanisms 
driven by stepper motors 
are used to position the SVT upper and lower modules with a precision of $1.25 ~\mu$m/step. A third linear motion mechanism is used to position 
the target on or off the beam. All the stepper motors are placed at a large enough distance from the magnet to avoid any ill effect from the 
magnetic fringe field.
An existing stepper motor driver and  EPICS-based control software will be used.   
 
\begin{figure*}[t]
\includegraphics[scale=0.75]{beamline/svt_in.pdf}
\caption{\small{Rendering of the SVT inside the Hall-B pair spectrometer vacuum chamber and the upstream vacuum box with SVT and target connections.}}\label{fig:svtvbox}
\end{figure*}

The scattering chamber between the top and bottom parts of the ECal is a critical beamline element. In order to keep the calorimeter as close as 
possible to the beam plane, include sufficient thermal insulation for the ECal, and maintain as wide a vacuum gap as possible, the top and bottom 
plates of the scattering chamber must be quite thin. At the location where the primary beams  ($e^-$ and $\gamma$) exit, the openings in the 
chamber have been enlarged. In Figure \ref{fig:ecalv} a rendering of the scattering chamber in between the two halves of the ECAL is shown. The 
front flange of the chamber connects directly to the magnet vacuum chamber. Vacuum is maintained only on the electron side (beam right) since 
the backgrounds on the positron side are negligible.  
This design is based on detailed GEANT4 simulations of the background rates and acceptance of the ECal. It places crystals within 20 mm 
from the beam plane to maximize low-mass acceptance. In order to avoid excessive deformation of the thin walls of the vacuum chamber, an aluminum 
honeycomb support is inserted between the upper and lower walls, to beam's right.

The ECal vacuum chamber will be connected to the muon system vacuum chamber, located between the two halves of the muon system. Special openings for 
the photon and electron beams are not needed in the muon system vacuum chamber. The gaps for the radiated secondary electrons are essentially 
projections of those in the ECal vacuum chamber. At its 
upstream end, the muon vacuum chamber will have a gap of $\sim 5$ cm. At the downstream end that gap will be $\sim 7.5$ cm.

The last vacuum chamber, which passes through the third dipole, does not need to have a narrow opening. It will have size of the Frascati H magnet 
gap. At the downstream end of this chamber, there will be flange with two exit windows, a Kapton window for the photon beam to exit the chamber 
and go to 
the photon beam dump through a Helium bag, and a vacuum continuation to the standard beam line for the electron beam to go to the Hall-B 
electron beam dump. 

\begin{figure*}[t]
\includegraphics[scale=0.25]{beamline/ecal_vac.pdf}
\caption{\small{Rendering of the ECal and the ECal vacuum chamber.}}\label{fig:ecalv}
\end{figure*}
 
\subsubsection{Beam dumps and shieldings} 

The Hall-B electron beam dump will be used to terminate the electron beam. Due to its high intensity, the beam will not be dumped in the Faraday 
cup (FC); instead, the existing beam blocker before the FC will be used to terminate the beam. The photon beam will be dumped in a photon beam dump, 
which will be a hut  made of lead bricks located on the space frame. There will be a shielding wall after the last chicane magnet to prevent 
radiation from reaching the detector systems on the downstream side of the Hall.

\subsubsection{Targets} 
\label{target}

A thin tungsten foil is used as the target. High Z material is
chosen for its short radiation length, to minimize the hadronic
production relative to the electromagnetic trident and $A'$
production. The target is located 10 cm in front of the first
plane of silicon strip detectors.

The primary target, 10 mm square, is 0.00125 radiation lengths
(approximately 4 $\mu$m tungsten). Mounted immediately above it
is a similar area of 0.0025 radiation lengths, available for some
of the data taking, adjusting the beam current as appropriate.
The foil can be fully retracted from the beam, and is inserted on
to the beam line from above, using a stepping motor linear
actuator. The bottom edge of the foil is free-standing so there
is no thick support frame to trip the beam when the target is
inserted. Its position is adjustable vertically allowing either
thickness to be selected, and different sections of the
tungsten can be used in the event of beam damage. The support
frame on the beam-right side of the target is made thin enough to
prevent radiation damage to the silicon in the event of an errant
beam caused, for example, by an upstream chicane magnet trip.
               
The target is intended to operate with beam currents up to 500
nA, which produce strong local heating. The strength of tungsten
drops by an order of magnitude with temperature increases in the
range of 1000 C. In addition, the material re-crystallizes above
this range, which increases the tendency for cracking where
thermal expansion has caused temporary dimpling. For these
reasons, it was decided to keep the temperature rise less than
about 1000 degrees, which is accomplished by selecting an
adequately large beam spot area. For example at 200 nA the rms
beam radii will be held above 20 by 250 $\mu$m, or 40 by 250
$\mu$m for 400 nA. Simulations have shown that these beam spot
sizes do not diminish the pair reconstruction resolution of the
experiment.


 

\clearpage

\subsection{Silicon vertex tracker}
\label{sec:svt}

Achieving the best possible acceptance at low A$^\prime$ mass requires positioning the silicon as close as possible above and below the primary beam, where radiation and occupancy are both limiting factors.  As a result, the silicon must be actively cooled to retard radiation damage, hits must be read out quickly and tagged with the best possible time resolution to reduce effective occupancies, and the tracker must operate in a vacuum to eliminate beam-gas secondaries.  At the limit of feasibility from these considerations, the silicon in the first layer is only 0.5 mm from the center of the beam, so prudence dictates that the tracker be retractable from the beam. Meanwhile, achieving the best possible acceptance at high mass means that the active volume of the tracker fill as much of the magnet bore as possible.  Finally, both the mass and vertex resolution that determine the experimental sensitivity are dominated by multiple scattering effects, so minimizing material in the tracker is the principal design goal.

The HPS Test Run SVT, described and discussed in Chapter~\ref{sec:testrun2012}, achieved these goals with the minimal possible apparatus capable of delivering A$^\prime$ physics during a short run. Unlike the initial proposal for the full experiment \cite{HPS_PROP}, this design used a single small module type with small angle stereo arranged into five layers and compromised redundancy, precision, and longevity in order to compress the project timeline and reduce the budget. In the process of developing this design, it was found that this simple system was capable of delivering a surprising fraction of the physics potential anticipated for the full experiment.  With this in mind, we now propose a design for the HPS SVT that builds upon the HPS Test  SVT, principally by addressing the compromises made for HPS Test to ensure the best possible performance for A$^\prime$ physics within the envelope of the existing beam line layout and analyzing magnet.  This design uses the exact same sensors and readout chips, retaining the most successful elements of the HPS Test design and addressing the weaknesses identified during assembly and operation to ensure the success of the experiment.


The key requirements and design principles of the HPS Silicon Vertex Tracker (SVT) are discussed at length in the initial proposal to the JLab PAC~\cite{HPS_PROP}.  

\subsubsection{Layout}

The layout of the HPS SVT is summarized in Table~\ref{table:svt_layout} and shown in Figure~\ref{figure:svt_layout}. There are six measurement stations, or ``layers," placed immediately downstream of the target. Each layer comprises a pair of closely-spaced planes and each plane is responsible for measuring a single coordinate, or ``view''. Introduction of a stereo angle between the two planes of each layer enables three-dimensional tracking and vertexing.
%================
\begin{table}[h]
\begin{center}
\begin{tabular}{lcccccc}   
\hline \hline 
    Layer number & 1 & 2 & 3 & 4 & 5 & 6 \\      
\hline
    nominal $z$, from target (cm)  & 10 & 20 & 30 & 50 & 70  & 90 \\ 
    Stereo Angle (mrad)  & 100 & 100 & 100 & 50 & 50 & 50 \\ 
    Bend-plane resolution ($\mu m$)  & $\approx$60 & $\approx$60 & $\approx$60 & $\approx$120 & $\approx$120 & $\approx$120 \\ 
    Non-bend resolution ($\mu m$)  & $\approx$6 & $\approx$6 & $\approx$6 & $\approx$6 & $\approx$6  & $\approx$6 \\ 
    Number of sensors  & 4 & 4 & 4 & 8 & 8 & 8 \\ 
    Nominal dead zone in $y$ (cm)  & $\pm1.5$  & $\pm3.0$  & $\pm4.5$  & $\pm7.5$  & $\pm10.5$ & $\pm12.0$  \\ 
    Module power consumption (W) & 6.9 & 6.9 & 6.9 & 13.8 & 13.8 & 13.8 \\
\hline \hline
\end{tabular}
\caption[]{Layout of the HPS SVT.}
\label{table:svt_layout} 
\end{center}
\end{table}
%=================
\begin{figure}[ht]
    \includegraphics[width=\textwidth]{svt/figures/10dec6.jpg}
\caption{\small{A rendering of the concept for the HPS SVT.  The beam enters from the left through the vacuum box providing services to the SVT.  The silicon is shown in red and the hybrid readout boards in green.} }
\label{figure:svt_layout}
\end{figure}
%================

The layout of the first three layers is the same as in the HPS Test SVT, with a single sensor of coverage both above and below the beam and 100 milliradian stereo angle to balance acceptance against the resolution required for vertexing.  The last three layers are two sensors wide in the bend direction to better match the ECal acceptance and use 50 milliradian stereo, as in HPS Test, to break the tracking degeneracy that creates fake tracks from ghost hits in layers with the same stereo angle.  The first five layers cover the full acceptance of the ECal with one redundant layer.  The sixth layer has only slightly less acceptance than the ECal and results in an extra safety factor in tracking purity and improved momentum resolution for the vast majority of tracks.  Altogether, this layout comprises 36 sensors and 180 readout chips for a total of 23040 readout channels.

Acceptance for larger A$^\prime$ masses is limited by the size of the magnet but low-mass sensitivity depends on reconstructing tracks as close to the primary beam as possible; minimizing the so-called ``dead zone" surrounding the degraded primary beam in the mid-plane of the detector.  For the tracker, there are a number of considerations including proximity to beam halo and radiation damage in the first layer, ability to resolve time-overlapping hits, and the wall of pattern recognition errors at very high occupancies. With sensors capable of operation to $1.5 \times 10^{15}$ 1 MeV neq., readout with single-hit resolution of $\sim$2 ns and two-hit resolution of $\sim$50 ns, and pattern recognition robust to 2\% occupancy; the closest tolerable position of Layer 1 results in tracking acceptance outside of a 15 mrad dead zone around the beam plane.  In this configuration, the edge of the silicon in Layer 1 is 0.5 mm away from the center of the beam where occupancy from beam-gas curlers would be unacceptable. Therefore, along with the drive to minimize multiple scattering errors, the desire to maximize low-mass acceptance creates the principal design challenges for the SVT: a lightweight, movable, liquid-cooled tracker with superior time resolution and operated in vacuum.

\subsubsection{Module Design}

One strength of the HPS Test SVT is exceptionally low material budget, an average of 0.7\% $X_0$ per double-sided layer in the tracking volume with only 10\% of this from support material.  The HPS Test sensor modules achieve this figure by compromising the straightness, mechanical stability and cooling of the sensors.  The module design for HPS aims to maintain this material budget but eliminate these compromises by retaining the design of the carbon fiber half-modules but mounting them in a more robust way. Furthermore, this design allows the existing half-modules built for HPS Test to be reused for the first three layers of HPS, enabling the development of this module concept and the assembly of HPS to commence immediately and with a very small initial investment. 

A half module for the HPS Test SVT consists of a single sensor and a hybrid electronic readout board glued to a polyimide-laminated carbon fiber composite backing.  A window is machined in the carbon fiber leaving only a frame around the periphery of the silicon to minimize material. A 50 $\mu$m sheet of polyimide is laminated to the surface of the carbon fiber with 1 mm overhang at all openings to ensure good isolation between the back side of the sensor, carrying high-voltage bias, and the carbon fiber which is held near ground.  The sensors are single-sided, radiation tolerant, p$^+$ in n bulk, AC coupled, poly-biased sensors fabricated on $<$100$>$ silicon manufactured by the Hamamatsu Photonics Corporation for the RunIIb upgrade of the D\O\ silicon detector. The sensors are read out by APV25 chips operating in multi-peak mode, allowing tagging of individual hit times with a precision of approximately 2 ns.

For HPS Test, the half-modules were sandwiched back-to-back around an aluminum cooling block at the hybrid end and a similar PEEK spacer block at the other.  To allow for module rework, the modules were assembled with hardware and thermal compound instead of adhesive and have no stiffening material between the two sensors. For simplicity, only the cooling block at the hybrid end is supported, resulting in deviations in the planarity of the sensors as large as a few hundred microns at the cantilevered end. Furthermore, the lack of cooling at the unsupported end where there is no heat source from readout electronics limits the temperature achievable at the most highly irradiated portion of the sensor, a very small spot along the edge of the sensor adjacent to the dead zone.  Improved cooling is necessary to achieve the longevity required for the longer running time envisioned for the HPS detector.

For layers 1-3 of HPS, these same half-module structures will be tensioned, like bicycle spokes, between a pair of cooling blocks held by a grooved aluminum base, as shown in Figure~\ref{fig:newmodule_L1-3}. 
%=================
\begin{figure}[ht]
    \includegraphics[width=\textwidth]{svt/figures/10dec3.jpg}
\caption{\small{A rendering of the concept for for the new modules for Layers 1-3 of the HPS SVT.  A cutaway at the left shows the spring and lever mechanism that maintains tension on the carbon fiber of the half modules.} }
\label{fig:newmodule_L1-3}
\end{figure}
%================
Rather than building manifolds to provide cooling to these blocks and attempt to isolate them thermally from the underlying support structure as in HPS Test, the entire aluminum support structure will be cooled.  The block at the hybrid end of the module is fixed, while the other pivots on a shoulder screw with a small stainless compression spring providing tension of approximately 40 N and taking up the 60 micron differential contraction between the half module and the support structure during a 30 $^\circ$C cool down. The same low-viscosity thermal compound used in HPS Test will provide the thermal contact in the pivot joint between the grooved base and the pivoting block and generates a negligible temperature drop across the gap. This arrangement results in much flatter silicon, much better mechanical stability and much better cooling that provided by the previous design for layers 1-3. With temperature stability during running better than 1 $^\circ$C, dimensional stability of the tracker will be better than intrinsic measurement resolutions and more than an order of magnitude better than the resolution limitation from multiple scattering effects.

More importantly, this scheme allows the ultra-thin design to be extended to the longer, double-sensor half-modules of layers 4-6 that have a pair of silicon sensors in the middle and a readout hybrid at each end, as shown in Figure~\ref{fig:newmodule_L4-6}.  
%=================
\begin{figure}[ht]
    \includegraphics[width=\textwidth]{svt/figures/10dec1}
\caption{\small{A rendering of the concept for for the new modules for Layers 4-6 of the HPS SVT.  A cutaway at the left shows the mechanism responsible for keeping the half modules under tension.} }
\label{fig:newmodule_L4-6}
\end{figure}
%================
With a larger heat load to transmit, the temperature drop through the pivot joint of the moving block will be approximately 0.5 $^\circ$C. To accommodate the length of these double modules across the width of the vacuum chamber, the hybrids will be shortened by approximately 20\%.  Through the use of polyimide flex cables instead of twisted pair and the elimination of superfluous circuitry, this footprint can be achieved with little effort.  Aside from this, only minor changes to the conceptual design of the half module concept of HPS Test are envisioned to reduce assembly effort.

\subsubsection{Mechanical Support, Cooling and Services}

Sag of the aluminum support plates, when subjected to the bending load of the long motion levers, was the largest source of mechanical imprecision in the HPS Test SVT.  For HPS, this motion system will be reused but with changes to eliminate this weakness. First, only layers 1-3 will retract, reducing the length of the support plate by a factor of two.  Replacing the twisted pair readout with flex cables eliminates the largest external load on the plate.  Finally, the 0.5 inch plate will be replaced by a 0.25 inch plate with 0.25 inch sides, forming a u-channel for increased stiffness, as shown in Figure~\ref{fig:newsupport}. 
%=================
\begin{figure}[ht]
    \includegraphics[width=\textwidth]{svt/figures/10dec8}
\caption{\small{A rendering of the support concept the HPS SVT.  The modules are mounted in cooled channels.  The channels for Layers 1-3 pivot on a downstream "C-support" and are moved by lever extending upstream to linear shifts on the vacuum box, as in HPS Test.  Layers 4-6 are fixed in place. The DAQ boards, shown in green, are mounted to a separate cooling plate located in a low-neutron region upstream on the positron side of the detector.} }
\label{fig:newsupport}
\end{figure}
%================
The walls of this support channel will extend almost to the dead zone and the entire structure will be cooled by large, embedded copper tubes. Surrounding the modules over most of the solid angle, these support channels will nearly eliminate thermal radiation from the walls of the vacuum chamber, the primary heat load on the sensors.  Layers 4-6 will be mounted inside similar cooled channels but will be fixed since they are already far enough from the dead zone for safety during beam tuning. These two support structures will be mounted to a single baseplate as before, with complete adjustability relative to the vacuum chamber.  

Readout and power for HPS Test required 30 conductors per hybrid, or a total of 600 lines, with even that count requiring elimination of sense for DVDD on the hybrids.  It does not appear feasible to scale this solution by nearly a factor of two for HPS.  Instead, we plan to provide digitization and optical readout of the APV25 data as well as shared power for the hybrids on boards placed inside the vacuum chamber, as described in Section~\ref{sec:svt_daq}.  In consideration of the DAQ requirements and the environment inside of the vacuum chamber, it appears that there is a natural location for support and cooling of the necessary boards adjacent to layers 1-3 on the positron side.  This structure consists of a single vertically-oriented plate with an embedded cooling loop, shown in Figure~\ref{fig:newsupport}. By separating the readout boards on the outside of this plate by the same 20 cm separation of layers 4-6, a single cable solution can be employed to connect the hybrids of layers 4-6 to these boards.  This leaves layers 1-3 equidistant from the pair of readout boards on the inner side, where a second cable type can connect to the existing pigtail connectors for those hybrids.  The feedthroughs required for power and data in this design fit easily on one of the two flanges in the existing vacuum box, leaving the other flange for additional cooling, eliminating the need for a cooling manifold inside of the vacuum chamber.

\clearpage

\subsection{Electromagnetic Calorimeter} 
\label{sec:ecal}

The Ecal, depicted in Figure \ref{fig:ecal}, consists of $442$ lead-tungstate PbWO$_4$ crystals with avalanche photodiode (APD) readout and amplifiers. Those have all similarly short pulse widths, so that they can run at very high rates. Indeed, the expected high radiation and high rate environment, together with a high magnetic field in close proximity, essentially imposed lead-tungstate (PbWO$_4$) crystals with APD readout. The lead-tungstate modules, see Figure \ref{fig:module}, are taken from the Inner Calorimeter (IC) of the JLab CLAS detector, which was originally built by a collaboration of many institutions where IPN Orsay (France) played a key role in the design and fabrication of the support frames, thermal enclosure, amplifiers, and connection and motherboards. CLAS calorimeter successfully ran for 10 years in various high energy electroproduction experiments. The PbWO$_4$ crystals are $16$ cm long and tapered. The cross section of the front face is $1.3\times 1.3$ cm$^2$, at the back end it is $1.6\times 1.6$ cm$^2$. Modules in the ECal are arranged in two formations, as shown in Figure \ref{fig:ecal}. There are 5 layers in each formation; four layers have $46$ crystals and one has $37$. The ECal is mounted downstream of the analyzing dipole magnet at the distance of about $137$ cm from the upstream edge of the magnet. The two ECal modules are positioned just above and below the ECal vacuum chamber, through which the beam and the wall of flame passes in vacuum. At its closest point, the edge of the crystal is at $2$ cm from the beam. In order to maintain stable performance of the calorimeter, the crystals, APDs, and amplifiers are enclosed within a temperature stabilized environment, held constant at the level of 1\!\char23F. The expected energy resolution of the system from operational experience with the IC is $\sigma_E/E \sim 4.5\%/\sqrt{E}$ (GeV). As in the IC, PbWO$_4$ modules are connected to a motherboard that provides power and transmits signals from individual APDs and amplifier boards. The ECal data is digitized using the JLAB FADC250, a 250 MHz flash ADC developed for the 12 GeV Upgrade. The full analogue information from the FADCs coupled with the spatial and time information of each module are available to the trigger system, which uses energy deposition, position, timing, and energy-position correlation to reduce the trigger rate to a manageable $\sim 30$ kHz (see Section \ref{sec:triggerdaq} for details).

\begin{figure*}[t]
\includegraphics[width=\textwidth]{ecal/ECal.png}
\caption{\small{Arrangement of Ecal crystals. Two modules are positioned above and below the beam plane. Each module has 5 layers. There are 46 crystals in each layer, with exception of layers closest to the beam plane where 9 crystals are missing to allow larger opening for outgoing electron and photon beams.}}\label{fig:ecal}
\end{figure*}

\begin{figure*}[t]
\includegraphics[width=\textwidth]{ecal/ecal_module.png}
\caption{\small{ECal module composed of a 16 cm long lead-tungstate crystal, Avalanche Photo Diode, and a amplifier board.}}\label{fig:module}
\end{figure*}

Calorimeter described above was built and used during the HPS test run in April-May of 2012. For the first time, for the readout and the trigger system for a calorimeter in real experiment JLAB FADC250s were used. The trigger algorithm was designed to satisfy HPS event selection criteria and was based on a newly developed FPGA based trigger processors. With photon beams, only limited aspects of the whole system were tested. Critical parts of the ECal performed well during the test ran and the min goals of the run were achieved (see Section \ref{sec:testrun2012} for details). While ECal performance during the test run was satisfactory, several areas are in need for improvements. In the following subsections planned improvements to the existing ECal are described.   

\subsubsection{Improvements to the existing calorimeter}

There are no plans for major mechanical changes. The crystals, support frames, and thermal enclosure will stay unchanged. The most of changes are related to the signal readout chain and they are based on the operational experience with ECal and FADC250. Below details of modifications/improvements are given: 

\begin{enumerate}
\item {\bf Replacing the ECal mounting system} - 
During the test run ECal was mounted on the Hall-B pair spectrometer mount rails together with the pair spectrometer hodoscopes. The ECal vacuum chamber was not installed and the fine alignment of ECal was not necessary. Due to limited space on the mounting rails and the absence for a need for a mounting system with fine adjustments, ECal was hang from the mount rails using simply long threaded rods. This system must be replaced with a more sturdy and better adjustable (horizontally and vertically) system in order to properly align ECal in close contact with the ECal vacuum chamber

\item {\bf Modifications to the side brackets to accommodate fiber bundles for light monitoring system} -
The light monitoring system was not used during the test run. While design of the ECal enclosure is done in such a way that it will accommodate optical fibers attached to the front face of crystals, the side plates that hold crystal frames do not have inlets for light monitoring fiber bundle. Space is available on the side plates. Modification requires making holes and placing transition connectors  
    
\item {\bf Modification of motherboards} - One of the issues we faced during the test run was the noise on the motherboards. Missing channels seen on the performance figures in Section \ref{sec:testrun2012} are largely due to switching off noisy channels since there was no time for debugging them. The new motherboards will be designed and build to resolve issues encountered during the test run. One of options in discussion is to replace long motherboards with shorter ones with power and signal connectors located on the top (for the top module) and the bottom (for the bottom module) of the thermal enclosure

\item {\bf Signal splitting} - From experience gained with the JLAB FADC250 by HPS group as well as other groups at JLAB working on detector developments, it is evident that with the appropriate firmware FADCs can be used both for time measurements, and as real-time scalers. Present ECal readout configuration uses signal splitters to split APD signal after preamplifier in 2:1 ratio and sends $2/3$ of the signal to a discriminator that has built-in scalers and feeds the TDC channel. The other $1/3$ of the signal goes to FADC for energy measurement. Removing the split will increase the signal on the FADC input by $\times 3$. This will allow to lower the amplifier gain. Using the FADC250 for time measurements and as a scaler is planned for other detectors at JLAB 
 
\item {\bf New preamplifiers} - with increased signal strength at the FADC input by $\times 3$ after eliminating the signal splitter, amplifier gains can be lowered by the same amount without loosing in resolution. Lowering the amplifier gains will reduce the noise and will allow to bring down thresholds on individual channels. The impact of the lower noise/threshold system is twofold - first it will improve resolution, second it will allow us to setup MIP calibration system for ECal modules while they are installed in the detector (horizontal orientation of the crystals). As shown in studies performed by HPS collaborators from INFN Genova, with low noise, low threshold system MIP energy deposition can be seen in the module for transverse cosmic muons, see Figure \ref{fig:mip10x10} and explanations below

\item {\bf Light monitoring system}

For an experiment like HPS, where backgrounds must be well understood and need to be strongly suppressed, the trigger bias can be an important issue. In particular, having stable and known thresholds at the trigger readout is necessary in order to avoid  bias in the event selection. Uniformity of the trigger response and stability can be achieved with the installation of an online gain monitoring system. This system will introduce short light pulses in the front of the crystals. Crystals have fiber holder glued onto the  front face, allowing implementation of the system without modification of crystals and wrapping. 

Optical fibers will be used to transmit light from the source to the crystals to test response of APDs. The response of the system will therefore be sensitive to both transparency losses of crystals due to a possible radiation damage and gain variations of APDs. Such a system has been developed for several experiments (CMS at CERN for instance) with various light sources. The system for ECal will be developed in IPN Orsay during 2013 and in the first half of 2014, to be ready for installation at JLAB for the commissioning run in the fall of 2014. Each module will be supplied with a red and blue mono-color LED as light sources for monitoring purposes. Blue light, corresponds to the domain of the crystal emission spectrum and is very sensitive to the presence of color centers, produced by radiation damage. This light source is very useful to test variations of the response in the main domain of the spectrum. 
%Impurities can anneal at room temperature and such monitoring can be sensitive to increasing of output as well, when the radiation exposure is reduced for a long period of time. 
A red colored light, less sensitive to the color centers, permits monitoring the APD gains more directly and thereby separates effects of APD and electronics from the crystal transparency, and provides clear information on the state of the electronics. The main challenge for the system is to guarantee stability at a level of $2\%$. The test of the system will be carried-out at IPN Orsay, in order to guarantee its efficiency and also to test radiation resistance of the various materials

\item {\bf New low-voltage power supply} - the existing low voltage power supply is a manually controlled, single output power supply that feeds four motherboards through custom-made patch panel. Inability of controlling voltage supply to preamplifiers at different parts of the ECal and remotely controlling/resetting them is a big disadvantage, requires frequent access to the Hall, especially in the commissioning phase. Available new low voltage power supplies are much more flexible. The one that is the most suitable for ECal APD preamplifiers is WIENER MPV 8008LD. This power supply is being used at JLAB and the control software exists.     

\end{enumerate}
  

\subsubsection{Possibilities with new APDs} 

One major improvement that can be done for the calorimeter is the implementation of new APDs. Replacing the old $5\times 5$ mm$^2$ Hamamatsu S8664-55 APDs with $10\times 10$ mm$^2$, Hamamatsu S8664-1010 will resolve two issues with present modules of the HPS calorimeter. First, new APDs from Hamamatsu have much better performance than the ones which are now installed on lead-tungstate crystals. Data from Hamamatsu shows that APDs made from the same wafer have excellent uniformity. With $\pm 10\%$ known uniformity at the gain of $100$, variations in bias voltage are only $\sim 4.5$ V. Moreover, data provided for samples of 1300 the bias voltage difference is ~50 V, when for APDs now we have ($442$ pieces) the voltage difference is more than $100$ V. In th eECal the bias voltage to APDs supplied in  groups and therefore with new APDs that have much smaller voltage-gain variations much better uniformity in the response of the calorimeter modules in the trigger can be achieved. 

The second, a $4$ times larger readout area will ensure $4$ times more light collection and therefore $4$ time larger signal from APDs. This will allow the use of different amplifier modules with lower gain that in turn will decrease electronic noise. Tests performed for another calorimeter, now in production at INFN Genova for JLAB Hall-B, showed that the same type of amplifier boards with factor 2 lower gains have noise level on the order of $<5$ MeV. Minimum ionizing energy deposition in PbWO$_4$ crystals of HPS calorimeter from cosmic muons passing through perpendicular to the crystal axis is $\sim 15$ MeV. If energy thresholds can be moved close to $5$ MeV, then MIP peak will be seen and calorimeter can be calibrated and monitored with cosmic muons. The HPS collaborators from INFN group made the first tests with HPS crystals using Hamamtsu $10\times 10$ mm$^2$ APDs and a new amplifier board. In Figure \ref{fig:mip10x10}, the charge distribution of a single crystal system is shown for $5\times 5$ mm$^2$ (left) and $10\times 10$ mm$^2$ (right) APDs. As a trigger, coincidence signal from scintillator telescopes positioned above and bellow the module is used. The crystal was positioned horizontally so the cosmic muons will pass through it perpendicular to the crystal axis. The red line histogram is for all events triggered by scintillation telescope and corresponds to the noise. The black line histogram corresponds to the charge detected within $100$ ns of the trigger time. The MIP peak is clearly visible and well isolated from the noise for S8664-1010 APD readout. For  S8664-55 APD MIP signal is also seen, but the MIP charge distribution is under the noise peak and does not have well defined peak position. Possibility of MIP calibration together with light monitoring system will ensure stable and reliable performance of the ECal and the trigger system. In addition, the lower noise will allow to lower the acquisition thresholds and improve the energy resolution. The new amplifier boards have to be designed to work with new APDs. 

\begin{figure*}[t]
\includegraphics[scale=0.37]{ecal/MIP_5x5_APD.png}
\includegraphics[scale=0.37]{ecal/MIP_10x10_APD.png}
\caption{\small{Charge distribution on QDC from readout of the HPS calorimeter crystal with Hamamatsu S8664-55 (left) and S8664-1010 (right) APDs, and the new low noise amplifier board. The red line histogram corresponds to charge distribution for all triggers comming from telescopes positioned above and bellow the crystal. The black line distribution is for hits in the crystal within $100$ ns of the trigger signal. }}\label{fig:mip10x10}
\end{figure*}

The total cost of replacing all ECal APDs is about $500$K\$. The HPS collaborators from IPN Orsay applied for  European Research Council (ERC) Advanced Grant 2013 to purchase APDs, for manpower costs to replace the old ones, design and build new preamplifier boards, and assemble the ECal with the new modules. This grant will also include light monitoring system. If successful the grant will cover most of the ECal modifications. 

\clearpage

\subsection{Muon system}

\label{sec:muon}


The di-muon decay channel of the A$^\prime$ has the advantage of a greatly reduced electromagnetic background.  In this case, the only particle background in a muon counter would come from photoproduction of $\pi^+$ and $\pi^-$ pairs that are not fully stopped in the ECal or absorber.  A muon detector will match geometrical acceptances of the tracker and ECal, and will be about a cubic meter in size. With such geometrical coverage, efficiency of detecting high mass A$^\prime$s in $\mu^+\mu^-$ decay channel will be higher than for $e^+e^-$ decays, see Figure \ref{fig:muacc}. Expected low background and high efficiency, the di-muon final state is an attractive complement to A$^\prime$ search using $e^+e^-$ final state, and will add substantial territory in the mass and coupling parameter space. With muon system, HPS will be the only experiment proposed to date to search for heavy photons in an alternative to $e^+e^-$ decay mode.

\begin{figure*}[!ht]
\includegraphics[scale=0.4]{muon/acc.pdf}
\caption{\small{A$^\prime$ detection efficiency through $\mu^+\mu^-$ (blue) and $e^+e^-$ (red) decay channels as a function of mass for 6.6 GeV beam energy.}}\label{fig:muacc}
\end{figure*}

The muon system can easily be constructed with layers of scintillator hodoscopes sandwiched between iron absorbers, and can be added downstream from the rest of the HPS apparatus.
The number of layers and the thickness of absorbers is defined by the $\pi/\mu$ rejection factor. The schematic design of the muon detector was optimized using the GEANT-3 model for the ECal with added layers of iron and scintillators.  In the simulation, muons and pions in the momentum range of $1$ to $4$ GeV/c first passed through the 16 cm of lead tungstate in the ECal and then entered a muon counter with various total absorber thicknesses (see \cite{HPS_PROP} for details).  Detection efficiencies for pions ($\epsilon_\pi$) and muons ($\epsilon_\mu$) were then calculated as a function of absorber thickness and particle momentum.  For low-energy particles ($< 1.7$ GeV) detection in all four layers of scintillator hodoscopes was not considered. Depending on the momentum, particles were not traced behind the third, fourth or fifth absorber.  
Figure \ref{fig:pmrej} shows the resulting rejection factor $\epsilon_\pi/\epsilon_\mu$.  The right-hand plot shows the dependence of  $\epsilon_\pi/\epsilon_\mu$ on the total thickness of the iron absorber, with the best rejection at about 75 cm.  The right-hand plot shows $\epsilon_\pi/\epsilon_\mu$ for a 75 cm absorber as a function of muon momentum.  The suppression of individual pions by two orders of magnitude will suppress pion pairs by 4 orders of magnitude.  

\begin{figure*}[!ht]
\includegraphics[scale=0.44]{muon/pmrej.pdf}
\includegraphics[scale=0.44]{muon/pmrej4.pdf}
\caption{\small{Pion-muon rejection factor $\epsilon_\pi/\epsilon_\mu$ versus total iron absorber thickness
(left) and versus particle momentum for a 75 cm absorber (right).}}\label{fig:pmrej}
\end{figure*}


\subsubsection{Conceptual Design}

On the basis of these simulations, we have designed a muon detector composed of four iron absorbers (total length of $30+15+15+15=75$ cm) with a double-layer scintillator hodoscope positioned after each absorber. The muon detector will be mounted behind the ECal.  The front face of the first absorber will be at $\sim 180$ cm from the target. Similar to the Ecal, the muon detector will consist of two halves, one above and one below the beam.  This segmentation is necessary in order to
minimize the effects of the ``sheet-of-flame" that multitude of low-energy particles in the horizontal plane, swept into the detector acceptance by the dipole analyzing magnet.
The vertical gap between the first hodoscope layers of the two halves is about $5$ cm. Dimensions of hodoscopes and absorbers are shown in Table \ref{tb:muon}.  Figure \ref{fig:HPS_view2} shows a CAD
drawing of the HPS detector, with the muon system on the right, which includes the 4 absorbers (gray), the vacuum box (light gray) between the upper and lower sections, and the final set of scintillator paddles (red). The ECal is directly upstream from the muon detector, with its crystals shown in yellow.  In front of the ECal is a large gray vacuum flange.  The silicon tracker is represented by red and gray rectangles and  the red point on the left is the target position.  

\begin{table}[htdp]
\caption{Dimensions (in cm) of muon system scintillation hodoscopes (H) and iron absorbers (A). }
\begin{center}
\begin{tabular}{|c|c|c|c|c|}
\hline
&H1&H2&H3&H4\\
\hline
Distance from target& 212&232&252&272\\
Width&112&125&138.5&152\\
Hight&10.5&11.5&12.5&13.5\\
\hline
\end{tabular}

\begin{tabular}{|c|c|c|c|c|}
\hline
&A1&A2&A3&A4\\
\hline
Distance from target& 207&227&247&267\\
Width&108.5&122&135&148.5\\
Hight&10&11&12&13\\
Thickness & 30 & 15& 15 & 15\\
\hline
\end{tabular}
\end{center}
\label{tb:muon}
\end{table}%


\begin{figure*}[!ht]
\includegraphics[scale=0.22]{muon/HPS_view2.png}
\caption{\small{CAD drawing of the HPS detector setup.  From left to right this consists of the target (red dot), the silicon tracker
(gray and red rectangles), the large shielding wall (gray), the ECal lead tungstate crystals (yellow, two shades), the muon counter absorbers
(gray), and the final muon counter scintillators (red, two shades).}}
\label{fig:HPS_view2}
\end{figure*}

%\begin{figure*}[!ht]
%\includegraphics[scale=0.8]{muon/pmrej4.pdf}
%\caption{\small{Pion-muon rejection factor as a function of the iron absorber thickness.}}\label{fig:pmrejp}
%\end{figure*}

\begin{figure*}[!ht]
\includegraphics[scale=0.22]{muon/Muon2b.png}
\caption{\small{Horizontal scintillator configuration for the muon counter. Scintillators are
shown in red and yellow/brown.  The white/gray structure is the vacuum box.  Each hodoscope layer (top
and bottom) contains three long strips, read out on both ends.
}}
\label{fig:Muon2p}
\end{figure*}

For the hodoscopes we plan to use the same extruded scintillator strips with embedded wavelength-shifting fiber and multi-anode phototube readout as was developed for the CLAS Preshower Calorimeter. These scintillator strips are 45 mm x 10 mm in cross section, and can be cut to any lengths and widths can be reduced as needed for the muon counter.  Each strip contains two, long tunnels, created in the original extrusion process, into which wave-length shifting fibers can be inserted.  Each hodoscope will consist of one x and one y plane.  In Figure \ref{fig:HPS_view2} in two shades vertical strips of the last hodoscope plane is shown. Figure \ref{fig:Muon2p} in different shades horizontal counters of hodoscope planes are shown. The horizontally aligned strip will extend over the length of the detector and will be read out on each end.  The upper and lower hodoscopes in each plane will have their own vertically aligned strips, which will be read out on only the outer end.  The inner end is inaccessible because of the vacuum box, but there is no particular advantage to having a double readout on these short (135 mm) strips.  

The system can be instrumented with 256 readout channels, in which case the requisite electronics will 
fit into a single VME crate.  Signal from each channel (PMT) 
will be sent to a FADC.  We intend to borrow the CLAS12 Preshower Calorimeter electronics and HV system.  Similar to ECal, FADCs will be used to construct a muon trigger for the experiment.  In the current design there will be 3 horizontal strips in each of 8 hodoscope planes (24 total) and a total of 208 vertical strips in 8 hodoscope planes.  The number of vertical strips per plane increases slightly with distance from the target to keep a constant angular coverage.  The maximum is 33 per hodoscope in the back plane.

Full Monte Carlo simulations with realistic event rates are currently underway in order to finalize design details of the muon counter.  The crucial issues are the event rates in the scintillators near the beamline (which already has initiated a redesign of the vacuum chamber to reduce background), the target-to-muon-counter tracking resolution and the detection efficiency.  Any changes to the detector as a result are expected to be minor and will not alter the conceptual design presented here.


\clearpage

\subsection{Trigger and DAQ }
The HPS experiment data acquisition and trigger system handles the acquisition of data and 
distribution of triggers for three sub-detectors; the SVT, ECal and the muon detector. There 
are two front end electronic systems; the SVT is readout with 128-channel integrated circuits
 supported by Advanced Telecom Communications Architecture~\cite{atca} (ATCA) hardware
 while the ECal and muon detectors signals are processed by VXS based hardware. The 
Level 1 trigger receives input from the ECal and muon detectors only to form a decision 
 on which events to be read out.  The triggered events are acquired from the three subsystems and are processed in the data acquisition system outlined in 
 Fig.~\ref{fig:daq_hardware_overview}.
\begin{figure*}[t]
%\includegraphics[ scale=0.25]{test2012/ecal_mounted.JPG}
\includegraphics[ scale=0.3]{daq_trigger/figures/daq_schem.pdf}
\caption{\small{Schematic block diagram of the data acquisition system.}}
\label{fig:daq_hardware_overview}
\end{figure*}
Every VXS and VME crate contains the Readout Controller (ROC) that collects information 
from the front end electronics boards, processes it and sends it through the network to the 
Event Builder. ROC is the single blade Intel-based CPU module running DAQ software under CentOS Linux OS. The ROC for the SVT runs on 
a processor blade in the ATCA crate which handles the SVT part of the DAQ system. 
The Event Builder assembles 
information from the SVT, ECal and muon detector ROCs and assembles it into a single 
event which is passed to the Event Recorder which writes it to the data storage system capable of handling up to 100MBytes/s. The Event Builder and other 
critical components run on multicore Intel-based multi-CPU servers which is a 
sufficient configuration to handle HPS. The backbone of the DAQ network system is a 
Foundry router providing 1Gbit and 10Gbit connections between DAQ components and 
the JLab computing facility. The SVT ROC has a 10Gbit link to the Foundry router while the 
ROCs for the ECal and muon detector connect through a 1Gbit network switch. The long 
term data storage is handled by a 10Gbit uplink to the JLab computing facility. 

Section~\ref{sec:svt_daq} describes the SVT DAQ in more detail. The VXS based readout for 
the ECal and muon detector is described in Sec.~\ref{sec:fadc_daq} and the trigger 
system is explained in more detail in Sec.~\ref{sec:triggerdaq}.

\subsubsection{SVT Data Acquisition}
\label{sec:svt_daq}
The goal of the SVT DAQ is to support the continuous 40~MHz readout and processing of signals from 
the 32 silicon strip sensors of the SVT and select those events that were identified by the 
Level 1 trigger system for transfer to the JLab DAQ for further event processing at rates 
close to 50~kHz. 
Due to the difficult environment of the SVT with extreme occupancy and pile-up from multiple bunches the number of noise hits has to be low to keep total data rates under control and 
each pulse from an energy deposition in the silicon needs to be sampled in order to facilitate 
reconstruction of the hit time to high accuracy. 

To meet these demanding requirements each of the 32 silicon strip sensors is connected to a 
hybrid board housing five 128-channel APV125 front-end ASICs~\ref{Jones:1069892,Raymond:2002yr}, 
see Fig.~\ref{fig:hybrid_and_apv25}.
The APV25 ASIC, initially developed for the Compact Muon Solenoid silicon
 tracker  at the Large Hadron Collider at CERN, was 
 chosen based on their good match to the HPS requirements. They provide amplification, 
 pipelining, and analog storage for trigger accepted events. 
  \begin{figure*}[t]
\includegraphics[ scale=0.3]{daq_trigger/figures/hybrid.jpg}
\includegraphics[ scale=0.3]{daq_trigger/figures/apvs-on-hybrid.jpg}
\caption{\small{Picture of a hybrid board from the test run in 2012 holding five 
APV25 ASICs that are wire bonded to the silicon sensor.}}
\label{fig:hybrid_and_apv25}
\end{figure*}
Each hybrid board has five analog output lines where analog data from each APV25 ASIC are 
sent using low power LVDS differential current signals over about 1~m of flex cable to a 
front-end readout board. A pre-amplifier scales the APV25 differential current output to match 
the range of a 14-bit Analog to Digital Converter (ADC). Each front-end board has four sections 
that each service four hybrids. The ADC operates at the system clock frequency of 41.667~MHz.
The digitized output from the front-end board is sent through compact 8-pair mini-SAS cables to 
the vacuum flanges to connect to the upstream DAQ outside the vacuum chamber. 
The front-end readout board houses a radiation resistant FPGA and buffers to allow for the control of the 
distribution of clock, trigger and I$^{2}$C communication with the APV25 ASICs. To further simplify the services, and minimize cabling, that enter through the vacuum flanges it contains 
linear regulators to distribute and regulate three low voltage power lines to the APV25 ASICs in addition 
to the high voltage bias. Figure~\ref{fig:svt_daq_flange_fe_boards} shows a schematics layout of the 
downstream readout chain of the SVT.
The digitized signals from the 20,480 channels are converted to optical signals at the custom built 
flange boards. Each flange board houses optical drivers to handle the electrical-optical 
conversion and transmit the optical signals over 10~m fibers to the upstream SVT DAQ. 
It also interfaces the low- and high voltage power transmission from the CAEN power supplies 
to the vacuum chamber.  
 \begin{figure*}[t]
\includegraphics[ scale=0.6]{daq_trigger/figures/daq_hps_2014_schematics_fe_flange_boards.pdf} 
\caption{\small{Schematic overview of the front end and flange boards of the downstream part of 
SVT DAQ.}}
\label{fig:svt_daq_overview}
\end{figure*}

The main SVT DAQ crate is developed and built at SLAC using 
the Advanced Telecom Communication Architecture (ATCA) system for high speed data transfer.
The optical signals from four hybrids, one half flange board, are received 
at one of four sections of the Rear Transition Module (RTM) boards of the ATCA crate, see Fig.~\ref{fig:svt_daq_overview}.
\begin{figure*}[]
\includegraphics[ scale=0.6]{daq_trigger/figures/daq_hps_2014_schematics_atca.pdf} 
\caption{\small{Schematic block diagrams of the SVT data acquisition system.}}
\label{fig:svt_daq_overview}
\end{figure*}
The main COB board has four FPGA units that each interfaces with a single section of the 
the RTM.  Each FPGA is housed on a separate daughter board called 
Data Processing Module (DPM). The modular ATCA design allows to re-use architecture and functionality 
from other DAQ system. Figure~\ref{fig:rtm} shows a RTM- and COB board designed and used for 
the ATLAS muon system which components are similar to that used by HPS.{\color{red} Need to be checked.} 
\begin{figure*}[]
\includegraphics[ scale=0.25]{daq_trigger/figures/rtm.png}
\includegraphics[ scale=0.4]{daq_trigger/figures/svt_daq_module_noted.png}
\caption{\small{Picture of a RTM (top) and COB board (bottom) used in the HPS test run 2012. {\color{red}Can we use a picture from ATLAS DAQ here?}}}
\label{fig:rtm}
\end{figure*}
In order to minimize complexity of the system inside the vacuum chamber, the signal processing is 
solely done at the DPM and not at the front end board which simply digitizes the signal. 
Each DPM receives the digitized signals from the hybrids 
from the RTM, applies thresholds for data reduction and organizes the sample data 
into UDP datagrams. Each DPM also includes an I$^{2}$C controller to configure and monitor the 
APV25 chips. One of the DPMs functions as the trigger interface which receives trigger 
signals from the optical fiber module on the RTM, handles distribution of clock and trigger 
and handles communication with the JLab trigger supervisor and the RCE. The 
RCE (Reconfigurable Cluster Element) is a generic computational building block 
on the trigger interface DPM running a 450~MHz PPC processor with 4GB of DDR3 
memory. Four COBs housed in a two ATCA crates is sufficient to handle 
the 36 hybrids of the SVT.

The RCE receives and buffers UDP datagrams from the data and trigger DPMs and
 assembles them into full event frames. The RCE also runs an implementation of the JLab ROC application that handles the integration of the SVT event frames into the JLab DAQ 
 system described above. The RCE node transfer data to the JLab DAQ  
 through a 10~Gbit Ethernet backend interface. The maximum readout rate of the SVT is approximately 
 43~kHz, limited by the APV25 readout rate. 
%The maximum readout rate of the SVT DAQ  is limited by the readout time 
%of the APV25 chip. Using overlapping trigger and readout functionality, where the 
%APV25 chip can buffer up to 5 triggers, the maximum average readout rate expected for 
%HPS is 45kHz {\color{red} need verification}.   







\subsubsection{ECal and Muon Detector FADC Readout}
\label{sec:fadc_daq}
The main part of the readout of the ECal and muon detector are identical. Signals from each 
readout unit are sent to a signal splitter. For the ECal the charge signal from the APDs are 
shaped and amplified as described in Sec.~\ref{sec:ecal_readout} before feed into the 
splitter. One of the outputs of every splitter (for both the ECal and muon detector channels) 
feeds a separate channel on Flash ADC (FADC) readout boards that are packaged in 
16-channel VXS modules. Two 20-slot VXS crates are used to accommodate the two ECal 
halves, each one with 221 channels, and one 20-slot VXS crate is used to readout the muon 
detector. 

The FADCs stores the 12-but digitized samples of each channel in 8~$\mu$s deep pipelines. 
After a trigger is received a corresponding part of the pipeline is processed (usually few hundreds of nanoseconds). 
If the signal passes a predefined threshold, X number of samples before and Y samples after are summed. Transferring only 
the time it passed threshold and a channel sum as energy estimate greatly reduces the data rate. 
During data analysis value '(X+Y)*pedestal' will be subtracted to obtain actual pulse integral.

The FADCs are an integral part of the HPS calorimeter trigger system. Pulse energies 
and times from each FADC channel in the same VXS crate is collected by a crate trigger
 processor board (CTP) which performs cluster finding. The result from each create trigger processor (i.e. each half of 
the ECal) is combined in the sub-system processor module (SSP) which applies further selections 
based on single- and combination of clusters to form a trigger decision passed to the trigger 
supervisor. The trigger process has a pulse timing resolution of 4~ns. This allows a narrow 
coincidence window of 8~ns to be used when searching for clusters. 
This is an important part of improving 
the pattern recognition to reduce confusion in the tracker which has considerable longer 
integration times. 





The main characteristics of the Jefferson Lab Flash ADC are as follows:
\begin{itemize}
\item 12 bits digitizer with 250Msps
\item 50$\Omega$ termination input
\item Front-end input range  -0.5V, -1V or -2V.  Input range has to be above maximum pulse height to ensure no signal clipping
\end{itemize}

The FADC charge resolution as a function of the front-end input range is presented in Table~\ref{tab:charge_resolution}.
\begin{table}[h]
\centering
\begin{tabular}{| l | l |}
\hline
Input Range & Nominal Charge Resolution\\\hline
-0.5V & \ 9.76 fc per ADC count \\\hline
-1.0V & 19.53 fc per ADC count \\\hline
-2.0V & 39.06 fc per ADC count \\\hline
\end{tabular}
\caption{FADC charge resolution for different front-end input ranges.}
\label{tab:charge_resolution}
\end{table}

FADC data paths for the readout and trigger operation  are presented in Fig.~\ref{fig:hps_trigger_data}.
There are two FADC operation modes: the readout mode and trigger mode.

 In readout mode FADC determines the energy of the one ECal channel that will be reported. 
The channel integration occurs only if the input signal crosses the programmable threshold level.  Then a programmable number of samples around the threshold crossing are added together to form the reported integral.  The readout  mode has the following parameters for every FADC channel (see Fig.~\ref{fig:hps_trigger_data}, top panel):
 \begin{itemize}
 \item Number of samples integrated before the threshold crossing (NSB)
 \item Number of samples integrated after the  threshold crossing (NSA)
 \item Readout threshold, measured in ADC counts.
 \end{itemize}
 
The number of samples for a given channel integration  is the sum of NSB+NSA samples that will be stored in  
 the 17-bit FADC register. It is a fixed gate width pulse integration and there is no pedestal subtraction in the sum (pedestal subtraction happens offline).
 

 

\begin{figure}[t]
\includegraphics[scale=0.4]{daq_trigger/figures/hps_trigger_data}
\caption{\small{FADC data paths}}
\label{fig:hps_trigger_data}
\end{figure}

A block diagram of the HPS  trigger processing is shown in Fig.~\ref{fig:hps_trigger_data}, bottom panel. 
The trigger processing mode has the following parameters for every FADC channel:
 \begin{itemize}
 \item Number of samples integrated before the threshold crossing (NSB)
 \item Number of samples integrated after the  threshold crossing (NSA)
 \item Readout threshold, measured in ADC counts.
 \item Pedestal
 \item Conversion factor (gain) that converts  ADC channel to MeV, from 0 to 8191 MeV, 13 bits
 \item Energy discriminator (minimum energy cutoff)
 \end{itemize}
The parameters NSB, NSA and readout threshold are the same as in the readout mode.
The pedestal value is then subtracted from the integrated sum over NSB+NSA samples and this value is converted to MeV units using the gain conversion factor. The energy can be discriminated to cut off low energy pulses before reporting to the CTP. The value reported to the CTP is a 13bit pulse energy with a 4ns timing resolution where it crossed the readout threshold. Pulse data for every channel is sent to the CTP every 32ns (if there is no hit a 0 energy pulse is sent still). This sets a worst case double pulse resolution of 32ns per channel, but it can be as less than this if pulses occur in different 32ns windows, but close together).





See Sec.~\ref{sec:triggerdaq} for more details on the operation of the trigger system.


\subsubsection{Trigger System}
\label{sec:triggerdaq}

%\subsubsection{ECal Trigger overview}

The trigger system is designed to efficiently select e$^{+}$e$^{-}$ and $\mu^{+}\mu^{-}$ events by 
using information from the ECal and Muon System. For e$^{+}$e$^{-}$ events, the trigger looks for time coincidences of clusters in the top and bottom half of the ECal. The clusters also have to satisfy loose kinematic selections optimized on A$^{\prime}$ events to further reduce the rate. 
For $\mu^{+}\mu^{-}$ events\footnote{see Appendix \ref{sec:muon} for details.}, signals from at least the first two layers of the muon hodoscopes are combined with an ECal signal consistent with a minimum ionizing particle (MIP).

As described above in Sec.~\ref{sec:fadc_daq}, the first stage of the trigger logic is incorporated into the FPGA's on the FADC boards which sends crystal energy and time information to the CTP. With the available 3-bit time information, the CTP can in principle look for time coincidence of crystal signals with 4~ns resolution (HPS will use a 8~ns time coincidence interval). The final trigger decision is made in the CTP and Sub-System Processor (SSP). The Trigger Supervisor generates all necessary signals and controls the entire DAQ system readout through the Trigger Interface (TI) units. The TI units are installed in every crate that participate in the readout process. 

The trigger system is free-running and driven by the 250~MHz global clock and has essentially zero dead time at occupancies expected by HPS. The Trigger Supervisor can apply dead time if necessary, for example on a `busy' or `full' condition from front-end electronics. The system is designed to handle trigger rates above 50~kHz and a latency set to $\approx 3~\mu$s to match that required by the SVT APV25 chip. 


 


% The proposed trigger system is nearly deadtimeless. 
%The trigger logic will search for a time coincidence between two clusters in opposite halves of the ECal for ($e^+e^-$) trigger and two MIP signatures in opposite halves of ECal and at least in the first 2 layers of the muon hodoscopes for ($\mu^+\mu^-$) trigger. The coincidence time window is programmable with 4ns resolution. The maximum trigger decision time (latency) is currently set to 3 $\mu$s for Level 1. That value is defined by the SVT readout APV25 chip.


%The first stage components of the trigger logic are incorporated into FPGAs of the Flash ADC board's, while the final decision is made in a Crate Trigger Processors (CTPs) and  Sub-System Processor (SSP) . As described above, 
%FADC sends 13-bit pulse energy and time information to CTP. With available 3-bit time information, CTP can in principal look for coincidence between different channels in 4 ns time window. For the HPS L1 trigger formation, the time window for channel coincidence will be set to 8 ns.



\vspace{1cm}
{\bf $e^+e^-$ Trigger} 


The trigger system for e$^+$e$^-$ events can be broken down into three sections (see Fig.~\ref{fig:hps_trigger_cal}):
 \begin{itemize}
 \item FADC (pulse finding): samples and digitizes the signal pulses from each detector channel. Sends the measured pulse energy and arrival time to the CTP.
\item CTP (cluster finding): groups FADC pulses from each half of the ECal into clusters. The cluster energy, arrival time, and hit pattern are sent to the SSP.
 \item SSP (cluster pair finding): searches for time coincidence of pairs of clusters from the top and bottom half of the ECal and applies topological selections.
\end{itemize}
 \begin{figure}[t]
 \includegraphics[scale=0.25]{daq_trigger/figures/hps_trigger_cal}
\caption{\small{Block diagram of the ECAL trigger system consisting of the FADC that samples and digitizes the detector channel signals and sends them for cluster finding in the CTP. The CTP clusters are sent to the SSP where the final trigger decision is taken based on pairs of clusters in both halves of the ECal. The decision is sent to the Trigger Supervisor (TS) that generates the necessary signals to readout the sub-detectors.}}
 \label{fig:hps_trigger_cal}
 \end{figure}
The time coincidence window of pairs of clusters in the top and bottom half of the ECal are programmable with 4~ns resolution. 
%The trigger system to look for  e$^+$e$^-$ events can be broken down into the three sections (see Fig.~ref{fig:hps_trigger_cal}):
% \begin{itemize}
 %\item FADC (pulse finding): Samples the detector channel to find pulses above the preset trigger threshold. Pulse energy and times are sent to CTP
% \item CTP (cluster finding): Searches FADC pulses (from half of calorimeter) to find clusters. Cluster energy, time, and hit pattern sent to SSP
% \item SSP (cluster pair finding): Searches CTP clusters (from top and bottom) to find cluster pairs and creates the trigger. Trigger cuts on pairs decide final trigger.
% \end{itemize}
The cluster finding algorithm is very fast and makes use of the parallel processing nature of FPGA's by simultaneously searching for 125 clusters, up to 3x3 in size, across the calorimeter crystal array, see 
Fig.~\ref{fig:hps_trigger_3x3}. 
\begin{figure}[h]
\includegraphics[scale=0.4]{daq_trigger/figures/hps_trigger_3x3}
\caption{\small{Cluster finding algorithm.}}
\label{fig:hps_trigger_3x3}
\end{figure}
It performs the following tasks:
\begin{itemize}
\item Adds energy from hits together for every 3x3 square of channels in ECal.
\item Hits are added together if they occur (leading edge) within a programmable number of 4~ns clock cycles of each other (HPS will use 8~ns time coincidence time interval).
\item If the 3x3 energy sum is larger than the programmable cluster energy threshold and the sum is greater than any neighboring 3x3 windows, the CTP reports the cluster parameters to the SSP. 
\end{itemize}
The CTP evaluates all hits in its half of the calorimeter every 4~ns. A programmable time window is used to allow hits that are out of time with each other to be considered as part of a cluster sum. This is done by reporting hits when they occur and then reporting them again for the next $N$ number of 4~ns clock cycles, where $N \in [0,7]$. This is useful to deal with skew and jitter that develop from the detector, cabling, and electronics. As described above, the CTP only selects the 3x3 window with the highest energy sum of its neighbors. This filtering is applied to deal with overlapping clusters and cases where the cluster is larger than a 3x3 window.
%A 3x3 window with an energy sum above threshold and a time stamp within a programmable time window the cluster processor will report this cluster to the SSP only if the energy is greater than energies of all neighboring (up, down, left, right, and diagonals) 3x3 window clusters for that clock cycle. If the energy is not greater than its neighbor it will not be reported, instead the neighbor with larges energy will be reported. The reason for this filtering is because there are several 3x3 windows that overlap and see the sample crystals and also many clusters are larger than a 3x3 window.


The CTP sends the following information about the clusters to the SSP:
\begin{itemize}
\item 13-bit cluster energy (in MeV)
\item Cluster position (crystal index: x,y)
\item Cluster time (with 4~ns resolution)
\item Cluster 3x3 hit pattern (the detector channels reporting a hit in the cluster)
\end{itemize}
The cluster position is the coordinate of the peak crystal energy as seen from a 3x3 view. The 3x3 cluster hit pattern can be used by the SSP to help filter strange cluster patterns and/or make a low resolution cluster centroid computation.
%{\bf Sub-System Processor} 
%The cluster's time, energy, position and 3x3 pattern found in two VXS crates are reported to the Sub-System Processor. 
The SSP collects the cluster information from the two halves of the calorimeter and applies selections optimized to further reduce background rates with small impact on the A' signal:
\begin{itemize}
\item Energy sum,  
$E_{min}\le E_{top}+E_{bottom}\le E_{max}$
\item Pair time coincidence, 
$|t_{top}-t_{bottom}|\le \Delta t_{max}$ 
\item Energy difference, 
$|E_{top}-E_{bottom}|\le \Delta E_{max}$ 
\item Energy slope,
$E_{cluster\_with\_min\_energy}+R_{cluster\_with\_min\_energy}\times F_{energy}\le Threshold_{slope}$
\item Co-planarity, 
$|
tan^{-1}(\frac{X_{top}}{Y_{top}})-
tan^{-1}(\frac{X_{bottom}}{Y_{bottom}}) |\le Coplanarity_{angle}$
\item Number of hits in 3x3 window, 
\#$hits_{3\times 3}\ge HitThreshold$
\end{itemize}
\noindent
where $ E_{max}$,  $\Delta t_{max}$, $ \Delta E_{max}$ , $Threshold_{slope}$, 
$F_{energy}$, $Coplanarity_{angle}$
and
$HitThreshold$ are programmable parameters.

The SSP can also create a trigger decision based on the existence of a single cluster in the ECal exceeding the energy threshold which is  useful for commissioning and calibration runs. 

Online event analysis will be provided to be compared against trigger event data for immediate verification (on each trigger cut: cluster energies, positions, timing, energy slope, coplanarity and hit threshold). With identical trigger and data readout paths and high energy resolution, very precise agreement can be expected between trigger and readout.



\vspace{1cm}
{\bf Diagnostic Tools}

Previous experience with similar (but much simpler) trigger systems showed that diagnostic tools are necessary to make sure that the calorimeter and trigger electronics work as expected. 

Scalers will be implemented for every ECal channel. An example of such a diagnostic tool is presented in Fig.~\ref{fig:dvcs_beam}
from the previous version of the ECal. Hot or dead channels are easily identified and disabled online.
\begin{figure}[h]
\includegraphics[scale=0.52]{daq_trigger/figures/dvcs_beam}
\caption{\small{Scalers (example from the previous version of the calorimeter).}}
\label{fig:dvcs_beam}
\end{figure}
A diagnostic scope permits to analyze the trigger logic online. The goal is to have a Two-Dimensional Analyzer
 to provide a remote debug interface to identify bad channels, verify cluster finding algorithms and check timing.
 The details of this logic analyzer are as following:
 
\begin{itemize}
\item Runs in parallel, non-intrusive, to the calorimeter trigger
\item Can setup trigger on any ECal pixel arrangement and/or cluster count
\item Can move forward/backward in time by ~250~ns to see timing details
\item Will be customized for HPS geometry and hardware
\end{itemize}

An example of the 2D analyzer is presented in Fig.~\ref{fig:dvcs_2_cluster}. Two clusters are displayed
in the picture. The red color displays the hits in the calorimeter and  the center of clusters is displayed in yellow.

In addition to scalers, the distributions of individual ADC channel pulse energies will be monitored and  cluster hit position and energy from the SSP processor will be histogrammed as well. Two histograms (accepted and rejected) will be provided for each trigger cut used in the trigger logic.



\begin{figure}[t]
\includegraphics[scale=0.8]{daq_trigger/figures/dvcs_2_cluster}
\caption{\small{Diagnostic scope (example with two clusters found from the previous version of the calorimeter). Green - no hits, red - tower with hit, yellow - cluster found.}}
\label{fig:dvcs_2_cluster}
\end{figure}






\subsubsection{Event Size and Data Rates}

The high occupancies in the detector requires a high readout bandwidth to be able to transfer hits from the 
detectors to disk. The event sizes and rates are based on estimates from full Geant4-based simulations 
including all known backgrounds. As expected the SVT dominate the expected rates. 
The noise hit occupancy in the SVT is kept low by requiring that three of the six samples are above an
effective threshold of three times the noise level. The dominant contribution to the occupancy is instead 
the high rate of beam background hits estimated. This is estimated 
from detailed full simulation resulting in an occupancy of around 0.3\% or an average of 61 channels above threshold.  
%Background studies (see Sec.~\ref{sec:hps_perf}) show that 
%there are on average 10 tracks per event at a beam energy of 2.2~GeV and current of 
%200~nA. With each track 
%having on average 2 strips above threshold for each sensor there are on average 160 channels above threshold. Each of these channels will result in six digitized samples of the 
%pulse shape giving in total of 1084 samples per event for the SVT.
Each SVT channel has, in addition to the six digitized samples,  header information that identifies the 
the channel number and it's chip address. The complete SVT event size also 
include the overhead from each FPGA and the JLab data stream bank header.  
The maximum average event size increased with decreasing beam energy since a larger 
fraction of backgrounds get larger opening angles and thus potentially higher than the 15~mrad 
vertical dead zone angle. For a beam energy of 1.1~GeV, the average SVT event size is 2.5~kBytes and 
the rate is 43Mbytes/s, well within the SVT DAQ capabilities. 
The ECal and muon detectors, with occupancies between 3-10\%, each contribute with an event size of 
approximately 0.3kBytes and maximum rates of about 12~MBytes/s for the 1.1~GeV run. 
%Each calorimeter or muon hit consist of 8 bytes (4 byte energy, 4 byte time)
 %with a 12 byte header (4 byte trigger number, 8 byte trigger time) for each FADC board. 
 Such rates are well within the 100~MBytes/s limit for each VXS crate used in the ECal and muon 
DAQ system.
 % for both the The main limitation is of the order of 100Mbytes/s from each VXS crate. For a 
% 10\% occupancy estimated in Sec.~\ref{sec:trig_rate} the ECal event size is approximately 0.7~kbytes which translates to a total data rate of approximately 31.5~Mbytes/s 
%(split between the two VXS crates), well within the DAQ system design. 
%The contribution from the muon system is small due to it's significantly lower number of channels. The system is readout by nine FADC boards in a single VXS crate. The event 
%size for a 10\% occupancy level is 0.2~kbytes which translates to a data rate of 10~Mbytes/s. 
Table~\ref{tab:data_rates} summarizes the event size and data rates. The highest overall rate, for a 1.1~GeV run, and that needs to be written to disk is 56~MBytes/s which is within the current 
DAQ system design limit of 100~MBytes/s. 
\begin{table}[]
\centering
\begin{tabular}{|l|ccc|ccc|ccc|}
\hline
 & \multicolumn{3}{|c|}{Occupancy(\%)} &  \multicolumn{3}{|c|}{Event size (kB)} &  \multicolumn{3}{|c|}{Data rate (MB/s)} \\
\hline
Beam energy (GeV) & 1.1 & 2.2 & 6.6 & 1.1 & 2.2 & 6.6 & 1.1 & 2.2 & 6.6 \\
\hline
SVT & 0.5 & 0.3  & 0.3  & 2.5 & 1.7 & 1.5 & 43.1 & 27.2 & 18.9\\
ECal & 3.0 & 4.2  & 4.7 & 0.3 & 0.3  & 0.3 & 12.1 & 4.8  & 3.9 \\
Muon & 10 &  10 & 10  & 0.3 & 0.3 & 0.3 & 5.5 & 4.9 & 3.9 \\
\hline
Total& \multicolumn{3}{|c|}{-} & 3.0 & 2.3 & 2.1 & 53.6 & 36.9 & 26.8 \\
\hline
\end{tabular}
\caption{{\small Summary of the occupancy, event size and data rate expected for the runs at  runs at the three beam 
energies in the run plan. }}
\label{tab:data_rates}
\end{table}


\subsection{Offline Computing Model}

The following is an outline of the offline computing model envisioned for satisfying the analysis needs of the HPS experiment. The raw data collected over the running periods must be processed through calibration passes, reconstructed into calibrated objects useful for analysis and separated into analysis streams. Corresponding Monte Carlo will also need to be produced and separated into the same analysis streams.

The raw data must be processed to produce physics data objects that can be analyzed. 
This reconstruction process will also include filters to select events of physics interest. We use the event size estimates 
in Table~\ref{tab:raw_data_size}, which are based on Table~\ref{tab:data_rates} from the previous section and object sizes 
in EVIO (raw data) and LCIO (processed data) formats. 
\begin{table}[ht]
\centering
\begin{tabular}{|l|c|c|}
\hline
Beam energy & Raw (EVIO) event size (kB) & Reconstructed (LCIO) event size (kB) \\ 
\hline
1.1 GeV  &  2.2 & 4.8 \\
2.2 GeV  &  2.3 & 5.0 \\
6.6 GeV  &  2.1 & 4.0 \\
\hline
\end{tabular}
\caption{{\small Data event sizes. }}
\label{tab:raw_data_size}
\end{table}

Table~\ref{tab:data_volume} shows the expected number of triggered events and
%the average trigger rate and 
the total amount of data expected over the 
different runs. 
We assume that the experiment collects data for all of its available beam time and the time allocated for detector commissioning, even though the experiment reach only assumes 50\% availability; this gives a conservative estimate of computing requirements. 
For trigger rate estimates, we use the ECal trigger rate from Section \ref{sec:ecaltrigg}; based on Appendix \ref{sec:muontrigg}, the muon system trigger rate is expected to be negligible.
\begin{table}[ht]
\centering
\begin{tabular}{|l|c|c|c|c|c|c|}
\hline
Run & $E_{beam}$ (GeV) & Time (days) & Events ($\times 10^9$) & Raw data (TB) & Processed data (TB)\\
\hline
2014 & 1.1 & 21 & 33 & 73 & 159 \\
2014 & 2.2 & 21 & 29 & 67 & 145  \\
\hline
Total & - & 42 & 62 & 140 & 304 \\
\hline
2015 & 2.2 & 35 & 48 & 112 & 241 \\
2015 & 6.6 & 35 & 38 & 80 & 153 \\
\hline
Total & - & 70 & 86 & 192 & 394 \\
\hline
\end{tabular}
\caption{{\small Summary of the raw and processed data expected from the HPS runs. }}
\label{tab:data_volume}
\end{table}


For modeling signals, estimating backgrounds and confirming the understanding of the detector 
performance, extensive Monte Carlo simulation is needed. 
Three types of events will be simulated: general beam background, trident background, and A' events.
% Table~\ref{tab:mc_event_size} summarizes 
% the typical event size at the various stages of the simulation. 
% \begin{table}[]
% \centering
% \begin{tabular}{|lccc|}
% \hline
% Event type & Sim. stage & Size/triggered event (kB) & Mass points  \\
% \hline
% Beam bkg.  & evgen	& 37.0	& 1	\\
% A' signal & evgen	& 0.5	& 10	\\
% A'+beam bkg & evgen	& 37.4	& 10	\\
% \hline
% Beam bkg.  & MC output	& 79.5	& 1	\\
% A' signal & MC output	& 2.5	& 10	\\
% A'+beam bkg & MC output	& 82.0	& 10	\\
% \hline
% \end{tabular}
% \caption{{\small Event sizes in kB per triggered event, including pileup events for beam background. }}
% \label{tab:mc_event_size}
% \end{table}

General beam background events will be generated by fully simulating beam background as described in \ref{sec:backgrounds} and simulating the HPS trigger. 
Because this is a compute-intensive process, only 1 million triggered events will be simulated at each beam energy; this is adequate for trigger and DAQ studies.

Trident background and A' events will be generated by using MadGraph to make trident or A' events 
with enhanced trigger probability, overlaying beam background, and simulating the trigger.
The amount of triggered trident events to be simulated is 10\% of the amount expected in actual data; 
the number of triggered A' events to be simulated is 100 million at each of 10 mass points at each beam energy. 
These will be used to test the analysis.

In total 472 (618)~TB of storage for data (raw, reconstructed and simulated) is needed for the 2014 (2015) run.
Tape is currently the only economical storage
solution for storing all of the raw, simulated and processed data.

The processing of the raw data is foreseen to occur at JLab. Given a
typical bandwidth between sites of 3 to 4 TB/day, 
only data summaries of events satisfying
pre-selection criteria for targeted analyses will be exported to remote
sites. Likewise, the size of the simulated data samples suggests that
the simulation should be processed and stored at JLAB and that 
only data summaries or small samples of the full data will be
exported.

Analyses needing
access to the hit level information will need to be run at JLab or run
on small samples of exported data unless they can take advantage of the
data summaries.

Data summaries will be written as ROOT trees. 65 (89) TB of DSTs will be generated 
in the 2014 (2015) run. These will be generated and stored on tape at JLab, and mirrored on tape at SLAC.

Disk space at JLAB will be 
needed for staging, code releases and scratch areas. 
Disk space will also be needed at SLAC for code releases and scratch areas. 
Both needs are covered by existing computing infrastructure.

The HPS storage requirements are summarized in Tab.~\ref{tab:datastorage}.
\begin{table}[ht]
\centering
\begin{tabular}{|l|c|c|}
\hline
Storage category & 2014 (TB) & 2015 (TB) \\
\hline
Raw data & 140 & 192 \\
Reconstructed data & 304 & 394 \\
Simulated data (raw and reconstructed) & 27 & 31 \\
\hline
Total data & 472 & 618 \\
\hline
DST (run data) & 62 & 86 \\
DST (simulated data) & 3 & 3 \\
\hline
Total DST  & 65  & 89 \\
\hline
\end{tabular}
\caption{{\small Data storage summary; data storage is at JLab only, while DST storage is common to JLab and SLAC.}}
\label{tab:datastorage}
\end{table}

% Approximately 0.015~CPU seconds are needed to reconstruct a data event on 
% a typical 2.4~GHz core. This would require a total of 0.26 million CPU hours of processing for the 
% entire 2014 dataset at the JLab processing center.  To simulate events, approximately 0.02 CPU seconds 
% are needed for a beam event and approximately 0.7 seconds for an A' event. In total 8.8 million CPU hours are needed for Monte Carlo 
% simulation for the 2014 run. 
% Based on experience with previous experiments, it is reasonable to estimate that the net CPU needed for 
% analysis work (batch and interactive) will be comparable to that needed for production. 
Simulation production and data reconstruction will be done on the batch farm at JLab. The CPU requirements 
are summarized in Tab.~\ref{tab:computing}.
\begin{table}[ht]
\centering
\begin{tabular}{|l|c|c|}
\hline
Computing category & 2014& 2015 \\
\hline
Raw data processing ($\times 10^{6}$~CPUh)  & 0.26 & 0.36 \\
Simulation production ($\times 10^{6}$~CPUh) & 0.84 & 0.99 \\
\hline
Total ($\times 10^{6}$~CPUh) & 1.10 & 1.35 \\
\hline
\end{tabular}
\caption{{\small Computing needs summary in CPU hours using typical 2.4~GHz cores.}}
\label{tab:computing}
\end{table}

The Jefferson Lab Computing Center provides computing and storage for experiments at JLab. 
A request will be submitted for data storage (tape and disk), computing resources (CPU hours for simulation and production), and data transfers to/from JLab.


%\clearpage
\section{May-2012 Test Run}
\label{sec:testrun2012}
The HPS Test run was proposed to DOE early in 2011 as the first stage of the HPS experiment. Its
purposes included demonstrating that the apparatus and data acquisition systems are feasible and
that the trigger rates and occupancies encountered in electron-beam running are as simulated.  It also
provided valuable experience to the HPS Collaboration in all aspects of designing, building, installing,
and running the experiment at JLab. Furthermore, in the case that HPS Test detector met all of the 
performance goals and was given dedicated running time with an electron beam, the HPS Test Run 
could provide new sensitivity to heavy photons. The HPS Test apparatus was installed on April 19, 2012, and 
ran parasitically with the HDice experiment, using a photon beam, until May 18. The JLab run schedule 
precluded any dedicated electron beam running, but the HPS Test Run was allowed a short and valuable
dedicated run with the photon beam.

This section briefly reviews the HPS Test Run apparatus, a simplified version of that 
planned for the full HPS experiment, and demonstrates the feasibility of the detector
technologies proposed for silicon tracker, ECal, and data acquisition systems. It documents the
performance of the trigger, data acquisition, silicon tracker, and ECal
and shows that the performance assumed in calculating the physics reach of the experiment is realistic.
Of particular importance, data from dedicated photon beam running has been used to compare
the observed trigger rates with that expected in simulation. The trigger rate is almost entirely due to
photons which have converted to e$^+$e$^-$ upstream of HPS and is sensitive to the multiple
Coulomb scattering of electrons and positrons in the conversion target.  Since scattered primary beam is the dominant 
source of occupancy in running HPS in an electron beam, good agreement between data and simulation 
confirm the background simulation used to benchmark the physics reach of the HPS experiment.

In addition to this important test of our background simulation, the test run accomplished the following goals which are explained below in Sec.~\ref{sec:testrun_performance}:
\begin{enumerate}
	\item More than 97\% of SVT channels functioned properly
	\item SVT readout signal to noise of 25.5
	\item SVT hit time resolution of 2.6 ns; this proves hit time reconstruction will work for HPS
	\item SVT hit efficiency greater than 98\%
	%\item SVT track reconstruction efficiency greater than 98\%
	\item Survey-based SVT alignment performs as expected and will allow track-based alignment
	\item 87\% of ECal crystals functioned properly, and defects will be corrected by planned ECal upgrades
	\item Calibrated ECal using SVT tracks
	\item Fully integrated SVT and JLab DAQ
	\item Trigger functions as designed; FADC trigger rate tested to greater than 100~kHz
	% deficiencies found in test run trigger performance are addressed for the 2014-2015 run
\end{enumerate}

\subsection{HPS Test Run Apparatus } 

In Figure \ref{fig:hpstest_layout}, the layout of the parasitic run is shown. The silicon vertex tracker was installed inside the Hall B pair spectrometer magnet vacuum chamber with the electromagnetic calorimeter mounted downstream.
Both the tracker and the ECal were retracted off the beam plane to allow clean passage of the photon beam through the system.
 
\begin{figure}[ht]
    \includegraphics[width=\textwidth]{test2012/HPS_dimensions}
\caption{\small{Layout of the HPS parasitic run.} }
\label{fig:hpstest_layout}
\end{figure}

For dedicated HPS running the photon beam was generated in the interaction of the $5.5$ GeV electrons with a gold radiator of $10^{-4}$ r.l., located $\approx 8$ meters upstream of the PS pair converter. After collimation ($D=6.4$ mm), the photon beam passes through the pair converter and through the HPS system. The converter was located $\approx 77$~cm upstream of the first layer of the silicon vertex tracker. Data were taken on different converters (empty, $1.8\times 10^{-3}$ r.l., $4.5\times 10^{-3}$ r.l., and $1.6\times 10^{-2}$ r.l.). These measurements were repeated for the reverse field setting of the pair spectrometer dipole.

\subsubsection{Test Run SVT}

The silicon tracking and vertexing detector for HPS Test, or SVT, operates in an existing vacuum chamber inside the pair spectrometer analyzing magnet in Hall B at JLab.  The design principles of the SVT are described in further detail in the HPS Test Run Proposal  \cite{HPS_tPROP}. There are five measurement stations, or ``layers,'' placed immediately downstream of the target. Each layer comprises a pair of closely-spaced planes and each plane is responsible for measuring a single coordinate, or ``view.'' Introduction of a stereo angle between the two planes of each layer provides three-dimensional tracking and vertexing throughout the acceptance of the detector with one redundant layer. 

In order to accommodate the dead zone, the SVT is built in two halves that are mirror reflections of one another about the plane of the nominal electron beam.  Each half consists of five double-sided modules mounted on a support plate that provides services to the modules and allows them to be moved as a group relative to the dead zone. Each module places a pair of silicon microstrip sensors back-to-back at a specified stereo angle with independent cooling and support.

Modules with 100 milliradian stereo are used in the first three layers to provide higher-resolution 3-d space points for vertexing. The 50 milliradian stereo of the last two layers breaks the tracking degeneracy of having five identical layers and minimizes fakes from ghost hits, improving pattern recognition while still providing sufficient pointing resolution into Layer 3 for robust hit association in the denser environment there. These stereo angles are the same as those proposed in Section \ref{sec:svt} for the new SVT. The details of the five layers are shown in Table \ref{tab:trk} and a rendering of the detector layout is shown in Figure \ref{fig:tracker_model}.  Figure \ref{fig:tracker_halves} shows a photograph of both completed detector halves prior to final assembly.  Altogether, this layout comprises 20 sensors and hybrids and 100 APV25 chips for a total of 12780 readout channels. 

\begin{table}[h]
\begin{center}
\begin{tabular}{lccccc}   
\hline \hline 
    Layer & 1 & 2 & 3 & 4 & 5 \\      
\hline
    Nominal $z$, from target (cm)  & 10 & 20 & 30 & 50 & 70  \\ 
    Stereo angle (mrad)  & 100 & 100 & 100 & 50 & 50 \\ 
    Bend plane resolution ($\mu m$)  & $\approx$60 & $\approx$60 & $\approx$60 & $\approx$120 & $\approx$120  \\ 
    Non-bend plane resolution ($\mu m$)  & $\approx$6 & $\approx$6 & $\approx$6 & $\approx$6 & $\approx$6  \\ 
    \# sensors  & 4 & 4 & 4 & 4 & 4  \\ 
    Nominal dead zone (mm)  & $\pm1.5$  & $\pm3.0$  & $\pm4.5$  & $\pm7.5$  & $\pm10.5$  \\ 
    Power consumption (W) & 6.9 & 6.9 & 6.9 & 6.9 & 6.9 \\
\hline \hline
\end{tabular}
\caption[]{Layout of the HPS Test SVT. }
\label{tab:trk} 
\end{center}
\end{table}

\begin{figure}[ht]
    \includegraphics[width=\textwidth]{test2012/HPS_nocables_nowires}
\caption{\small{A partial rendering of the HPS Test SVT solid model showing the modules of the upper and lower half-detectors on their support plates, the hinged C-support, the motion levers, the cooling manifolds on their strain relief plate and the baseplate with its adjusters.  The sensors are shown in red and the hybrids in green. The beam enters from the right.} }
\label{fig:tracker_model}
\end{figure}

\begin{figure}[ht]
    \includegraphics[width=\textwidth]{test2012/2012-101-PHOTON-DETECTOR-001.jpg}
\caption{\small{Both halves of the HPS Test SVT fully assembled at SLAC.} }
\label{fig:tracker_halves}
\end{figure}

Power is provided to each hybrid using CAEN power supplies on loan from Fermilab. Three low voltages are supplied for the APV25 and one high voltage to reverse bias the sensor. The supplies that provide sensor bias are capable of 500V operation and can be used to test operation at high voltage in close proximity to an electron beam. The total power consumption of each hybrid during normal operation is approximately 1.7 W, which is removed by the cooling system. Care was exercised in selecting power and data cables to ensure vacuum compatibility and sufficient radiation hardness. A custom junction box interfaces the CAEN power supply output channels to the SVT hybrids. Control of the supplies is provided via an EPICS graphical user interface, which allows monitoring of the detector and interlock protection.

The linear shifts that define the opening of the SVT are controlled by a pair of stepper motors located in low field regions at the ends of the analyzing magnet.  For photon running, these are locked in the open position, but for electron running they will be connected and controlled through EPICS so that the distance between the beam and the sensors can be adjusted to balance detector occupancies and acceptance.

\subsubsection{Test Run ECal}

The electromagnetic calorimeter (ECal) for HPS, as described in Section \ref{sec:ecal}, was built and tested in the test run.
The only differences between the test run ECal and what is proposed here for HPS are in the position and the vacuum chamber.
The vacuum chamber between the two ECal modules was not used for the photon test run; instead a  2'' beam pipe was used to transport photon beam from the pair spectrometer vacuum chamber to the HDIce target.  
The ECal was mounted downstream of the analyzing dipole magnet at the distance of $\sim 148$ cm from the upstream edge of the magnet. The two ECal modules were positioned just above and below the beam pipe such that the edge of the crystal closest to the beam was $3.7$ cm from it. 


%In order to maintain stable performance of the PbWO$_4$ calorimeter, the crystals and APDs are enclosed within a temperature stabilized environment, held constant to a precision of $1~^o$F. The expected energy resolution of the system from operational experience with the IC is $\sigma_E/E \sim 4.5\% / \sqrt{E (GeV)}$. As in the IC, PbWO$_4$ modules are connected to a motherboard that provides power to and transmits signals from individual APDs and pre-amplifier boards. Crystals inside the box are supported by aluminum frames as shown in Figure \ref{fig:ecal_assembly}.

For the test run, the ECal made use of the existing low and high voltage systems from the CLAS IC, as well as signal cables and splitters. Connectors on the existing signal cables were rearranged to accommodate the new layout of the channels. 
 
Assembly of the bottom half to the ECal is shown in Fig.~\ref{fig:ecal_assembly}.
\begin{figure*}[t]
\includegraphics[ scale=0.25]{test2012/ecal_assembly.jpg}
\caption{\small{Assembly of the ECal bottom module.}}\label{fig:ecal_assembly}
\end{figure*}
Figure \ref{fig:ecal_mounted} shows the ECal in its installed position for the parasitic run with photon beam.
 
\begin{figure*}[t]
\includegraphics[ scale=0.25]{test2012/ecal_mounted}
\caption{\small{ECal mounted downstream of the Hall-B pair spectrometer for the parasitic run with photon beams. Hoses for the cooling system, and the power and signal cables for beam-right side of both modules are visible.}}\label{fig:ecal_mounted}
\end{figure*}



\subsubsection{Test Run Data Acquisition}
\label{sec:testrun_daq}
The test run served as a proof of principle for the proposed DAQ and the system was very 
close to that proposed for HPS. Since the systems are similar only the main differences will be highlighted here,  with more details in Section~\ref{sec:daq} and with results and experiences discussed below in Section~\ref{sec:testrun_performance}. 


A generic layout of the hardware from the test run DAQ is laid out in Figure~\ref{fig:daq}.
 \begin{figure*}[h]
\includegraphics[scale=0.35]{test2012/daq/daq_schem.pdf}
\caption{\small{Readout and processing system block diagram.}}\label{fig:daq}
\end{figure*}
The front-end systems of the ECal and SVT are similar. 
The differences of the DAQ and trigger w.r.t. to the HPS are fleshed out in greater detail below but 
are all related to the following areas:
\begin{enumerate}
\item a lower ECal cluster resolution and no calibration available at the trigger level together with simpler trigger logic, 
\item a smaller SVT without the need for optical readout and power distribution inside the vacuum,
and
\item lower bandwidth links.
\end{enumerate}

The two front-end electronics systems for the ECal and SVT are essentially unchanged. The ECal provides 
input to the Level~1 trigger system after which an accepted event is acquired from the two sub-systems 
and are processed in the data acquisition and processing system. The Readout Crate Controllers (ROCs)
 described for HPS are unchanged and installed in every VME, VME64X, VXS crates running 
 mvme6100 controllers with a prpmc880 or pmc280 co-processor modules. A hybrid approach was 
 used for the SVT DAQ in the test run where the ROC ran on a external PC connected to the ATCA crate. 
 Similar to HPS, a Foundry Router was used as the backbone of the DAQ system, providing 1Gbps and 
 10Gbps connections between components and to the JLAB Computer Center.

The Event Builder, Event Recorder, and other critical DAQ components ran on 
4-CPU Opteron-based servers, which was sufficient for the test run. The RAID5 test run storage system 
had a 100~MByte/sec capability which was well enough for the data rates attained in the test run as 
described below. 

%\subsubsection{SVT DAQ}



The SVT DAQ was a rapidly built DAQ  designed to readout data continuously at 40 MHz from the silicon detector modules, and transfer data to the JLab DAQ once a trigger signal is received. It is built using the same 
basic architecture and layout as the HPS DAQ but without the optical readout components and 
power distribution system inside the vacuum described in Sec.~\ref{sec:svt_daq}. 

The test run had a total of 20 silicon strip sensors, each one connected to an onboard hybrid readout card 
which is similar to the HPS hybrid, each one holding five 128-channel APV25 integrated circuits. The test 
run hybrid readout card is shown in Fig.~\ref{fig:hybrid_and_apv25_testrun} 
 \begin{figure*}[t]
\includegraphics[ scale=0.3]{test2012/daq/hybrid.jpg}
\includegraphics[ scale=0.3]{test2012/daq/apvs-on-hybrid.jpg}
\caption{\small{Picture of a test run hybrid readout board holding five APV25 ASICs. The wire bonds to the 
silicon sensors can be seen as well.}}
\label{fig:hybrid_and_apv25_testrun}
\end{figure*}

Contrary to the HPS DAQ where opto-boards digitizes and converts the APV25 analog output signals to 
optical signals inside the vacuum chamber, the hybrids here carry analog signal directly to the 
Rear Transistion Module (RTM) via a multi-twisted-pair cable. The amplification and digitization of the 
analog differential voltage output of the APV25 output are therefore carried out on the RTM board 
which was designed specifically for the HPS test run. Figure~\ref{fig:svtdaq} shows an overall layout of 
the SVT test run DAQ system (compare to Fig.~\ref{fig:svt_daq_overview}).
 \begin{figure}[t]
\includegraphics[scale=0.9]{test2012/daq/svt_daq_diagram.png}
\caption{\small{Schematic of the SVT DAQ showing input from the hybrids mounted on the silicon detector to the RTM, its connection to the COB, and the Ethernet switch used to transfer data at 1 Gbps to the 
DAQ PC and ultimately to the JLAB DAQ.}}
\label{fig:svtdaq}
\end{figure}
On the RTM, a pre-amplifier converts the APV25 differential current output to a different voltage output 
scaled to the sensitive range of a 14-bit ADC. The RTM is organized into four sections with each section 
supporting 3 hybrids (15 channels). 
The ADC is operated at the system clock of 41.667 Mhz. 
%The RTM also includes a 4-channel Fiber Optic module and supporting logic which can be used to interface to the JLAB trigger supervisor card.
The ATCA main board (the Cluster On Board or COB) is similar to the HPS DAQ with the important exception that one of the DPM's functions as the trigger interface only and there is no RCE module. 
Instead, the DPMs package and send the data from the hybrids to an external PC through a 1Gbps 
ethernet connection which serve the same purpose as the RCE module in the HPS DAQ. 
The ATCA crate hosts two COB cards, one supporting four data processing DPMs and the other supporting three data processing DPMs and one trigger DPM to support a total of 21 hybrids, one more than required. 
The test run RTM and COB can be seen in Fig.~\ref{fig:rtm_testrun}. 
\begin{figure*}[t]
\includegraphics[ scale=0.25]{test2012/daq/rtm.png}
\includegraphics[ scale=0.4]{test2012/daq/svt_daq_module_noted.png}
\caption{\small{Picture of a RTM (top) and COB board (bottom) used in the HPS test run 2012.}}
\label{fig:rtm_testrun}
\end{figure*}
The external PC supports three network interfaces, 2 standard 1G-bit Ethernet and one custom low latency data reception card. The first Ethernet interface is used for slow control and monitoring of the 8 DPM modules. The second Ethernet interface serves as the interface to the JLAB data acquisition system. The third custom low latency network interface is used to receive data from the SVT ATCA crate and supports a low latency, reliable TTL trigger acknowledge interface to the trigger DPM. This PC hosts the SVT control and monitoring software as well as the JLAB ROC (Read Out Controller) application.


The ECal DAQ system used in the test run is very similar to that described for HPS in Sec.~\ref{sec:fadc_daq}. 
The only significant difference is that in the test run, the signals from the ECal modules were sent to a signal splitter. One of the outputs of the splitter is fed to a 
discriminator that also has an internal scaler, and then to a TDC channel. The other output is sent to the 
JLab FADC250 VXS module, shown in Fig.~\ref{fig:fadc}.
%, is based on information from the FADC boards and includes a cluster 
%finding algorithm using FPGA modules. With the FADC-based system, the energy of clusters used to 
%make a trigger decision is determined at the crate level. 






\subsubsection{Test Run Trigger System}
\label{sec:testrun_trigger}

A block diagram of the HPS test run trigger processing is shown in Fig.~\ref{fig:trigger_diagram}.  A gigabit bandwidth is used to transport all the individual FADC250 channel sums (5-bits) and clock (3-bits) encoding bits to resolve a $4$~ns period within a $32$~ns frame.  The clock encoding bits report the time when the input signal crosses the programmable threshold within the $32$~ns frame.  If the input signal does not cross threshold for a given $32$~ns frame, then the channel data is reported as zero.
\begin{figure}[t]
\includegraphics[scale=0.6]{test2012/trigger/HPSChanSum_001.jpg}
\caption{\small{Block diagram for the trigger system.}}\label{fig:trigger_diagram}
\end{figure}

The reported 5-bit channel sum value is extracted from the 17-bit register that contains the integrated (sum) signal value of the input channel.  The channel integration occurs only if the input signal crosses the programmable threshold level.  The samples that are included in the channel integration are those that are above threshold.  The number of samples for a given channel integration will not be larger than the frame report latency time (128~ns or 32 samples). Samples of the input signal are shown in Fig.~\ref{fig:trigsamples}. The point where the input signal crosses threshold determines which frame the integrated value is reported.  The time where the input signal crosses threshold is captured within the frame and reported with the three clock encoding bits to a 4~ns time stamp of the threshold crossing time.
\begin{figure}[t]
\includegraphics[scale=0.9]{test2012/trigger//trigger_pulse_samples}
\caption{\small{Example of input signals, and how they are integrated for the test run trigger.}}\label{fig:trigsamples}
\end{figure}
In the example, the pulse for channel~2 crosses multiple frames.  The point where the signal crosses threshold determines the frame where the integration value will be reported for the given channel.  The number of points that are above threshold will be limited to 32. If multiple pulses arrive within a 32~ns frame, they will not be resolved and thus create a pile-up effect.  The trigger application will only process a single falling edge or single rising edge per 32~ns frame. Multiple pulses per frame can be recovered from the readout data offline.

Information from each FADC channel were reported to the Crate Trigger Processor (CTP) through Gigabit serial data streams. The sixteen serial data streams will be processed on a frame by frame basis, and the cluster finding algorithm (same as for HPS in Sec.~\ref{sec:triggerdaq}) will produce a serial data stream that will be processed by the Sub-System Processor (SSP) to create a readout trigger signal that can be distributed to the front end FADC250 for Physics event readout. The system was designed for maximum trigger accept rate of 50~kHz. 

For the test run, the trigger decision in the SSP was a simple threshold: the trigger would fire on a single cluster with energy exceeding the threshold. The full trigger described in \cite{HPS_tPROP} was however implemented and tested.



\subsection{Multiple Coulomb Scattering Measurement}
\label{sec:angular_measurement}

One of the main parameters that has the 
largest impact on the main design of the HPS experiment is the occupancies expected 
close to the beam.  The largest contribution comes from electrons scattered to large angles 
in the target. It is worth remembering that HPS is sensitive to scattering angles very far out 
on the tail which has been less explored in past experiments. One of the main goals of the 
test run in 2012 was to evaluate the description of the tails of the multiple scattering in order 
to gain further confidence in our expected detector occupancy. As will be shown below, the 
test run was sensitive to the multiple coulomb scattering description despite the fact that 
the data was taken with a photon beam. The details of the test run conditions are described in Sec.~\ref{sec:testrun2012}. 

Figure~\ref{fig:schematic_testrun_vs_erun} gives a schematic view of the main differences 
between the photon and electron beam setup. 
\begin{figure*}[t]
\includegraphics[ scale=0.5]{test2012/angular_measurement/pictures/photon_vs_electron_beam_schematic.png}
\caption{\small{Schematic comparison of the photon beam setup to the electron beam.}}\label{fig:schematic_testrun_vs_erun}
\end{figure*}
In particular, the angular distribution of the pair produced electron and positions emerging 
from the target has contributions from two sources: {\it i} the pair production angle $\alpha$ 
and {\it ii} the resulting angle from multiple scattering in the target after production. This is 
schematically described in Fig.~\ref{fig:schematic_pair_prod}. 
\begin{figure*}[t]
\includegraphics[ scale=0.7]{test2012/angular_measurement/pictures/pair_prod_schematics.png}
\caption{\small{Schematic description of the relevant angles for pair production in the 
test run.}}\label{fig:schematic_pair_prod}
\end{figure*}
The contribution from both sources to the final angular distribution are comparable. FigureX 
shows the expected distribution of the vertical angle $\theta_y$ for the $e^+e^-$ pair  coming 
out of the converter compared to the pair production angle. {\color{red} Need this figure from Takashi.} 


{\bf Sample Composition}\newline
Since we are primarily interested in measuring the angular 
distributions for the $e^+e^-$ pair we checked that the contribution from photons are negligible. Table~\ref{tab:sample_composition} shows the sample composition. The fraction 
of photons that would deposit energy to reach threshold in the ECal crystals are much less than 2\% without any angular which will further reduce the fraction of photons. 
\begin{table}[]
\centering
\begin{tabular}{c|c|c|c}
Type & Nominal & $E>0.2$~GeV & $E>0.5$~GeV \\
\hline
electron & 7150 & 4938 & 3186 \\
positron & 6019 & 4568 & 2874 \\
$e^+e^-$ & 13169 & 9506 & 6060 \\
photon & 2984 & 640 & 151 \\
\hline
\end{tabular}
\caption{Sample composition for the photon test run. {\color{red} This needs to be updated 
or removed.}}
\label{tab:sample_composition}
\end{table}

The photon beam line during the test run presented a relatively large fraction of events 
originating upstream of the converter. This contribution was measured during data taking by 
removing the converter and thus taking data without any target but with all other conditions 
the same. Figure~\ref{fig:tracks_at_converter} shows the vertical position of 
reconstructed tracks in the SVT during data taking with the 1.6\% radiation length converter. 
Note the small satellite peaks visible at about $\pm 8$~mm. 
The same figure also overlays the same distribution for the empty converter 
run which clearly shows that these satellite peaks are due to the upstream backgrounds.
\begin{figure*}[t]
\includegraphics[ scale=0.6]{test2012/angular_measurement/pictures/tracks_at_converter_Y_top.png}
\includegraphics[ scale=0.6]{test2012/angular_measurement/pictures/tracks_at_converter_Y_bottom.png}
\caption{\small{Vertical position of extrapolated tracks from the SVT to the converter position.} {\color{red}Need update.}}\label{fig:tracks_at_converter}
\end{figure*}

In order to properly normalize the rates the integrated current was measured for the different 
runs. Typically the current was about 30~nA. Table.~\ref{tab:currents} shows the measured integrated currents. {\color{red} Describe how these where measured?}. The uncertainty 
of the measurement is estimated to approximately 5\%. 
\begin{table}
\centering
\begin{tabular}{l|c|c|c}
%Run & Target thickness [\%r.l.] &   start time[s]      & end time [s] & Duration [s] &       integrated beam current (nC)    \\                thickness (%r.l.)       Rate(Hz)     Recorded(Hz)  Magnet Polarity
Run & Target thickness & Duration &  $e^-$ on converter \\
 &  (\%r.l.) & (s) & (nC)    \\   
\hline\hline
1351 & 1.6   & 911 &     24385.9     \\
\hline
1353 & 0.18   & 2640 &    193508.9  \\
\hline
1354 & 0.45  & 2149 &       140709.9  \\
\hline
1358 & 0    & 1279  &   88079.6  \\
%1349    1337323714      1337324625      51344.0926551819        54879.7343788147        1.6                     1262.120     1174.728      -1
%1351    1337324962      1337325268      24385.9185791016        26928.0426635742        1.6                     1933.479     1696.808      -1
%1353    1337325717      1337328357      193508.881838322        204325.132622242        0.18                    436.895      425.659       -1
%1354    1337328521      1337330670      140709.898532331        148839.141475141        0.45                    596.055      570.870       -1
%1358    1337331152      1337332431      88079.5567516331        92523.9428218845        0                       309.785      304.253       -1
%1359    1337332615      1337334014      91653.0026320741        91761.4541434497        0                       318.640      311.540       1
%1360    1337334136      1337336898      198670.590789914        209883.979889035        0.18                    451.067      446.510       1
%1362    1337337264      1337338713      105642.70688653         110298.553449392        1.6                     1864.090     1659.675      1
%1363    1337340178      1337340456      8556.8459701538         8556.8459701538         1.6                     1864.090     1659.675      1
\hline
\end{tabular}
\caption{{\small Measured integrated currents for the runs used to measure the angular distributions.}}
\end{table}
Taking into account the luminosity the upstream background fraction was 16\%, 52\% and 71\% 
for converter thickness of 1.6\%, 0.45\% and 0.18\%, respectively. 

The measured angular distribution in the ECal for the three target thicknesses are shown in 
Fig.~\ref{fig:ang_distr_data} before and after normalization and subtraction of the upstream 
background contribution.
\begin{figure*}[t]
\includegraphics[ scale=0.65]{test2012/angular_measurement/pictures/rate_ecalrow_raw.png}
\includegraphics[ scale=0.65]{test2012/angular_measurement/pictures/rate_ecalrow_norm_subtr.png}
\caption{\small{Measured raw vertical angular distributions before (left) and after (right) 
normalization and background subtraction.} {\color{red} Need update.}}
\label{fig:ang_distr_data}
\end{figure*}

These measured angular distributions was compared to simulation and 
while the bulk of multiple coulomb scattering and pair production angles 
are well studies in the past {\color{red} (reference to appendix)} HPS 
is sensitive only to the tails of these distributions that are less explored. Note that for HPS 
we are only interested in validating our description of the multiple coulomb scattering as that 
is our main background when running in an electron beam. Since 
what we measure in ~\ref{fig:ang_distr_data} is a convolution of both we use a special event 
generation procedure in the simulation to separate description of the multiple scattering and 
pair production angle: We use EGS5~\cite{egs5} to generate the pair produced $e^+e^-$ pair 
and then pass these four vectors to either {\sc EGS5} or {\sc Geant4}~\cite{geant4} which 
models the multiple Coulomb scattering in the target. Thus the {\sc EGS5} is used to normalize 
the pair production angles and the remaining difference would be coming from the 
multiple scattering description in the two different simulation programs. There exist an unlikely 
scenario that the difference in multiple scattering between the two programs would cancel the difference in multiple scattering. To mitigate this possibility the angular distributions are 
measured for each converter thickness. Since the pair production angle is independent of 
the thickness the relative change between data and simulation for different converter thicknesses only depends on the multiple scattering contribution. 

Figure~\ref{fig:ang_distr_dataMC} shows a comparison between data and the {\sc EGS5} 
simulation where the rates have been normalized to 1~s of beam at a current of 90nA. 
\begin{figure*}[t]
\includegraphics[ scale=0.25]{test2012/angular_measurement/pictures/dataMC_1351_Hit_Y_top_norm_bkgsub.png}
\includegraphics[ scale=0.25]{test2012/angular_measurement/pictures/dataMC_1354_Hit_Y_top_norm_bkgsub.png}
\includegraphics[ scale=0.25]{test2012/angular_measurement/pictures/dataMC_1353_Hit_Y_top_norm_bkgsub.png}
\caption{\small{Comparison between the observed and simulated angular 
distribution using {\sc EGS5} for a converter thickness of 1.6\% (left), 0.45\% (middle) and 0.18\% 
(right).  Only statistical uncertainties are included. }}
\label{fig:ang_distr_dataMC}
\end{figure*}
The total rate prediction for simulation and data are compared in Fig.~\ref{rate_vs_thickness} 
and Tab.~\ref{tab:ang_distr_dataMC} summarizes the result. 
\begin{figure*}[t]
%\includegraphics[ scale=0.65]{test2012/angular_measurement/pictures/rate_vs_thickness_dataMC.png}
\includegraphics[ scale=0.3]{test2012/angular_measurement/pictures/dataMC_geant4.png}
\includegraphics[ scale=0.3]{test2012/angular_measurement/pictures/dataMC_egs.png}
\caption{\small{The measured rate in the ECal as a function of target thickness comparing 
to the multiple scattering models from {\sc Geant4} (left) and {\sc EGS5} (right)}.} 
\label{fig:ang_distr_data}
\end{figure*}
A few systematic uncertainties was estimated which include; a 5\% uncertainty on the integrated 
current normalization, limited Monte Carlo statistics, alignment of the ECal and uncertainty 
from the background normalization. The uncertainty from the initial gain calibration of the 
ECal described in Sec.{\color{red} X} was estimated to be less than {\color{red} Y\%, need to check this calibration 
systematic.}.

\begin{table}
\begin{tabular}{|l|c|c|c|}
Converter (\% r.l.) & 1.60\% & 0.45\% &	0.18\% \\
\hline
{\sc EGS5} &	1162 $\pm$ 112 &	255 $\pm$ 28 &	94 $\pm$ 17	\\
\hline
{\sc Geant4} & 2633 $\pm$ 250 & 	371 $\pm$ 38 &	114 $\pm$ 18 \\
\hline
Observed 	& 1064 $\pm$ 2 & 196 $\pm$ 1 &	92 $\pm$ 1 \\						
%Beam gap	58	13	5	132	19	6
%	EGS			G4		
%Target thickness	1.60%	0.45%	0.18%	1.60%	0.45%	0.18%
%Data [/90nC]	1064	196	92	1064	196	92
%Pred. [/90nC]	1162	255	94	2633	371	114
%Total uncertainty	112	28	17	250	38	18
%Stat	2	1	1	2	1	1
%						
%Stat MC	11	3	1	16	3	1
%Bkg norm.	14	14	14	14	14	14
%Current norm.	94	21	8	212	30	9
%Beam gap	58	13	5	132	19	6
\hline
\end{tabular}
\caption{ {\small Observed and predicted events for 1~s of beam at 90nA for different converter 
thicknesses. The uncertainty on the predictions is the total uncertainty including estimated 
systematic uncertainties. }{\color{red} Cross-check numbers}}
\end{table}
With the set of conservative systematics the total systematic uncertainty was 
between 10 and 18\%. 

In summary it's clear that the version of  
{\sc Geant4} with the default physics lists overestimates the large angle multiple scattering tail. 
This preliminary result further strengthens our confidence that {\sc EGS5} describes the multiple scattering in the target which is important to understand our expected occupancy and trigger 
rates for HPS. As a side note, since {\sc EGS5} was used to generate the pair 
angle distribution for both simulation described in the result it's interesting to see that the 
ratio of data to that the prediction varied from 0.91 (0.40), 0.77 (0.53) and 0.98 (0.81) for {\sc EGS} ({\sc Geant4}) at 1.6\%, 0.45\% and 0.18\% converter thickness, respectively. If the pair angle distribution were responsible 
for the difference between {\sc Geant4}  and {\sc EGS5} that would show up as a large shift in 
this ratio since the multiple scattering contribution varies. 

There are many steps needed to go from this preliminary result to a real 
measurement. However, this preliminary result further strengthens our confidence that 
{\sc EGS5} is able to properly describe the large angle multiple scattering events in the target 
which is important to estimate our occupancy and trigger rates for HPS discussed in Sec.~\ref{sec:performance}.




\subsection{Test Run Apparatus Performance} 
\label{sec:testrun_performance}
As previously noted, all running of the HPS Test apparatus was with photon beams, using the Hall-B pair 
spectrometer (PS) pair converter as a target.  This target,  located $\sim77$~cm from the first layer 
of the tracker, produced a sufficient flux of electrons and positrons to test the principles of running the HPS experiment.
This section will report on a few selected preliminary results that demonstrate our understanding of the system. 

\subsubsection{SVT Performance}
\vspace{1cm}{\bf Calibration [Omar]}


Description of hit amplitude, baseline/gain calibration, noisy channels/chips, pulse shape cuts, occupancy. 

Plots: Response plot, gain, noisy channels vs run nr?, data/MC of noise hits,Data/MC plot of occupancy for some layers  

\vspace{1cm}{\bf Cluster reconstruction [Omar]}


Description of the cluster reconstruction.

Plots: mip distribution

\vspace{1cm}{\bf SVT timing [Sho]}

The time reconstruction algorithm described in \ref{sec:svtŧ} was used to fit a single hit to each SVT channel in each event.
Pileup was not considered due to the very low hit rate in the SVT.

Values of fit $\chi^2$ fell in the distribution of $\chi^2$ for 4 degrees of freedom (6 points -- 2 fit parameters), as expected.

After clustering these hits, the hit time for the cluster is computed as the amplitude-weighted average of the channel hit times. 

Description of the SVT hit time reconstruction. 

Plots: example time fits, mean hit times across tracker, plot of simulated time resolution vs S/N? 

\vspace{1cm}{\bf Tracking algorithms [Matt/Omar]}


Pattern recognition/Stereo hit reconstruction and description of the tracking algorithm. 

 Plots: tracking efficiency vs run nr, hit efficiency vs run nr for data. Overlay MC.
 
\vspace{1cm}{\bf Tracking algorithms [Matt]}


Analysis of two track events. 

Plots: invariant mass, vertex position, 2-track event multiplicity. Compare with MC for all these 

\label{sec:svtperformance}

%Good alignment of the SVT is critical to achieving the expected tracking performance and physics reach. 
%The sensors must be aligned internally and with respect to the target and other beam line components for optimal performance. 
%The alignment of the test run apparatus proceeds in several steps which must be tied together to achieve the final alignment.
%These include survey measurements of various SVT assemblies, a beam line survey at JLab, and finally 
%a track-based alignment. 

The SVT was aligned using a combination of optical, laser and touch probe surveys at SLAC and JLab. The 
optical survey of individual modules with precision of a few microns are combined with a touch-prove survey 
of the overall SVT support structure, with 25-100 microns precision, to locate the silicon sensor layers with 
respect to the support plates and the mechanical survey balls on the base plate.
%Mechanical surveys using touch probes was performed on the two SVT tracker planes before 
%shipping to JLab. This survey measured reference points on the base plate, C-support and on the surface of the tracker support plates. These positions where then tied to very precise optical survey of each sensor module, referencing the silicon sensor position w.r.t. to the cooling blocks. 
%An important aspect of the mechanical survey is the relatively large sag of 
%the 70~cm long support plate which is supported by the C-support hinge on one end and the 
%extension bar attached to the linear shaft at the other. The measured sag without all services 
%(cooling manifold and cables was not dressed at this point) was more than $250~\mu$m.  
%For HPS, only the first three layers are being supported from each end which will reduce this 
%sag with at least a factor of four. {\color{red} check this}. The goal of the mechanical surveys is 
%to reach a relative alignment of the 
%silicon to about $100-200~\mu$m where alignment using tracks become feasible and the 
%improvements from mechanical surveys become harder. 
After full assembly and installation of the SVT at JLab, a mechanical survey of the SVT base plate position 
inside the pair spectrometer vacuum chamber is used to determine the global position of the SVT with respect 
to CEBAF beam line. 
%at the as-built alignment.  The sag
%of the long support plates and lever arms used in the test run are the dominant error in determining
%relative modules positions, a defect addressed in the proposed design for the SVT.
%Finally at JLab, with a fully assembled 
%tracker, an optical and touch probe survey was performed to locate 
%the SVT inside the vacuum chamber, using the measurements of the base plate, in the reference coordinate system of 
%the analyzing magnet. 
The resulting survey-based alignment has the position of the silicon sensors correct to within a few hundred 
microns as shown in the mean of the biased track residuals in Fig.~\ref{fig:res_top}.  The large multiple scattering contribution can be seen by the large increase in the width of the residuals in the later layers. The agreement with simulation is reasonable; a slight track reconstruction algorithm bias can be seen in the mean for the simulation in later layers which will be fixed in the future. 
%This level of internal alignment 
%shows that the survey approach, while not perfect, is adequate as a starting point to bootstrap the SVT 
%alignment. 
%At this level of internal 
%alignment   an internal SVT alignment with track residuals less than a few 
%hundred microns shown in .  is then studied using reconstructed tracks in the SVT. The main observable of the internal alignment of the silicon 
%sensors is the so-called track residual. It is 
%defined as the difference of the measured and predicted track 
%position at that sensor. Figure~\ref{fig:res_top} 
%shows representable 3D space point track residuals, relative to track parameters determined at the target, for tracks reconstructed in the top half of the tracker.
\begin{figure*}[]
\includegraphics[ scale=0.3]{test2012/alignment/pictures/res_top/h_trk_top_res_y_mean_h_trk_top_res_y_mean_dataMC_trigseltwotrksel4hit_recoilmc_twotrkfilt.png}
\includegraphics[ scale=0.3]{test2012/alignment/pictures/res_top/h_trk_top_res_z_mean_h_trk_top_res_z_mean_dataMC_trigseltwotrksel4hit_recoilmc_twotrkfilt.png}
\includegraphics[ scale=0.3]{test2012/alignment/pictures/res_top/h_trk_top_res_y_sigma_h_trk_top_res_y_sigma_dataMC_trigseltwotrksel4hit_recoilmc_twotrkfilt.png}
\includegraphics[ scale=0.3]{test2012/alignment/pictures/res_top/h_trk_top_res_z_sigma_h_trk_top_res_z_sigma_dataMC_trigseltwotrksel4hit_recoilmc_twotrkfilt.png}
\caption{\small{Mean (top) and standard deviation (bottom) of biased residuals (i.e. hits are included in the track fit) between the actual hit position and the predicted position from the reconstructed tracks in the bend (left) and non-bend (right) plane in the top half of the SVT after mechanical survey. The smaller width for the 5th layer in the bend plane is an effect from mixing tracks with four or five hit tracks.}}
\label{fig:res_top}
\end{figure*}
%These are compared to the residuals from an ideally aligned tracker with residuals centered at zero.
%Note the larger width for the downstream layers,highlighting the large 
%multiple scattering contribution in the track reconstruction. 
%The intrinsic single hit resolution of $\approx 6~\mu$m is negligible for layers beyond the second. 
%Fig.~\ref{fig:res_top_summary} shows a summary of the mean residuals for each layer of the tracker 
%after the mechanical survey alignment constants have been applied.
%\begin{figure*}[t]
%\includegraphics[ scale=0.7]{test2012/alignment/pictures/res_top/res_top_summary-1.png}
%\includegraphics[ scale=0.7]{test2012/alignment/pictures/res_top/res_top_summary-2.png}\caption{\small{Mean of the track residuals for each detector layer in the top tracker half 
%in the bend (left) and non-bend (right) plane after mechanical survey constants are applied.}}\label{fig:res_top_summary}
%\end{figure*}
%Note that these pull distributions come from biased 
%residuals (the hit was used in the track fit) and are thus not expected to have a 
%width of one. 
%\begin{figure*}[]
%\includegraphics[ scale=1.2]{test2012/alignment/pictures/res_pull_top/res_pull_top-1.png}
%\includegraphics[ scale=0.5]{test2012/alignment/pictures/res_pull_top/res_pull_top-2.png}
%\includegraphics[ scale=0.5]{test2012/alignment/pictures/res_pull_top/res_pull_top-3.png}
%\includegraphics[ scale=1.2]{test2012/alignment/pictures/res_pull_top/res_pull_top-4.png}
%\includegraphics[ scale=0.5]{test2012/alignment/pictures/res_pull_top/res_pull_top-5.png}
%\includegraphics[ scale=1.2]{test2012/alignment/pictures/res_pull_top/res_pull_top-6.png}
%\includegraphics[ scale=0.5]{test2012/alignment/pictures/res_pull_top/res_pull_top-7.png}
%\includegraphics[ scale=0.5]{test2012/alignment/pictures/res_pull_top/res_pull_top-8.png}
%\includegraphics[ scale=1.2]{test2012/alignment/pictures/res_pull_top/res_pull_top-9.png}
%\includegraphics[ scale=0.5]{test2012/alignment/pictures/res_pull_top/res_pull_top-10.png}
%\caption{\small{Track residual pulls in the bend (top) and non-bend (bottom) plane 
%for tracks reconstructed in the 
%top half of the tracker.  }}
%\label{fig:res_pull_top_nonbend}
%\end{figure*}

%In electron running, the beam spot can be used as a constraint in the global track-based alignment. 
We also extrapolate the reconstructed tracks back to the converter located $\approx 77$~cm 
from our first silicon layer to understand the tracker alignment w.r.t. to the other components on the 
beam line. Figure~\ref{fig:extrapol_converter} shows good agreement of the reconstructed track position at the converter with that predicted from simulation using the measured field map of the analyzing magnet to take into account the fringe field. The offset of the horizontal position simply reflects the fact that the positions are reconstructed in an SVT-centered coordinate system, which is tilted with respect to the beam coordinate system.
\begin{figure*}[t]
\includegraphics[ scale=0.25]{test2012/alignment/pictures/extrapolation_converter/h_trk_top_fr_conv_y_h_trk_top_conv_y_dataMC_twotrksel.png}
\includegraphics[ scale=0.25]{test2012/alignment/pictures/extrapolation_converter/h_trk_top_fr_conv_z_h_trk_top_conv_z_dataMC_twotrksel.png}
%\includegraphics[ scale=0.5]{test2012/alignment/pictures/extrapolation_converter/extrapolation_Y_converter_top.png}
%\includegraphics[ scale=0.5]{test2012/alignment/pictures/extrapolation_converter/extrapolation_Y_converter_bot.png}
\caption{\small{Extrapolated track positions for reconstructed e$^{+}$e$^{-}$ pairs in the SVT taking into account the measured fringe field of the analyzing magnet. 
The filled histograms show the prediction from simulation using an ideal geometry. 
A shift in the bend-plane coordinate for tracks in the bottom half (top right) is likely due to alignment or incomplete description of the
magnetic field at the edge of the magnet.
%The extra bumps in the data at $\pm10$~mm arise from backgrounds originating upstream of the 
converter.
}}
\label{fig:extrapol_converter}
\end{figure*}
%The width is roughly consistent with between data and simulation with a shift in the bend-plane 
%coordinate for tracks in the bottom half which is likely due to alignment or incomplete description of the 
%magnetic field at the edge of the magnet. 
%There are two small bumps in the vertical position in the data arising from backgrounds 
%originating upstream of the converter verified using the run without a converter.
%The luminous region, inferred from harp scans of the photon beam profile, has a width of about 1~mm 
%%(best described by a double Gaussian: $0.71e^{\frac{x}{0.366}}\times 0.29e^{\frac{x}{1.111}}$)
%and a total beam envelope of around 7~mm. The small bumps in data at $\pm10$~mm 
%are from particles produced upstream of the converter. The width and position of the tracks 
%are roughly consistent with the expected distribution from an ideal geometry as shown by the simulated 
%tracks in the same figures. The larger shift in the bend direction for bottom tracks 
%is still under investigation. 

With initial residuals less than $\sim 500~\mu$m across all layers of 
the tracker and a reconstructed beam profile similar to that expected from simulation, it appears these survey techniques 
are adequate to bootstrap the SVT alignment. 
For HPS, we are developing a more sophisticated global track-based alignment technique to reach 
the final alignment precision. This framework will also enable us to explore and understand important details 
such as weak modes and how dedicated alignment runs 
(e.g. with magnetic field off or with different targets) may shape operational procedures during HPS running.
%Fig.~\ref{fig:test_harpscan} shows a HARP scan taken during the test run. 
%\begin{figure*}[t]
%\includegraphics[ scale=0.5]{test2012/alignment/pictures/harp_scan_testrun.png}
%\caption{\small{Photon beam profile HARP scan close to the converter.}}\label{fig:testrun_harpscan}
%\end{figure*}
%The width of the beam can be described by a double Gaussian $0.71e^{\frac{x}{0.366}}\times 0.29e^{1.111}$ which is also used in the simulations. The beam envelope extends out to 
%about 7mm. 


By selecting e$^{+}$e$^{-}$ pairs from the triggered events we're able to study basic distributions of pair production kinematics and in particular those related to our vertex performance. Pairs of opposite charge tracks, one in the top and one in the bottom half of the SVT, with larger than 400~MeV was selected. The pair production kinematics are relatively well reproduced given the alignment of the tracker; Fig.~\ref{fig:pair_kin} shows the invariant mass and ratio of electron momentum over the sum of electron and positron. 
\begin{figure}[ht]
%   \includegraphics[ width=0.4\textwidth]{test2012/vertexing/figures/h_invM_h_invM_dataMC_0016x0_oneclselgoodquadranttwotrksel}
   \includegraphics[scale=0.25]{test2012/vertexing/figures/h_invM_h_invM_dataMC_0016x0_twotrksel.png}
   \includegraphics[scale=0.25]{test2012/vertexing/figures/h_ratioEsum_h_ratioEsum_dataMC_0016x0_twotrksel.png}
\caption{\small{The reconstructed invariant mass (left) and ratio of electron momentum over the momentum sum for pairs (right) of opposite charge tracks selected in the top and bottom half of the tracker.}}
\label{fig:pair_kin}
\end{figure}
At this moment, we are still working on understanding the relative normalization of pair events in the Test 
Run. 


For the vertexing 
performance the foremost difference compared to electron beam running is that the target was 
located $\sim67$~cm from our nominal target position; giving almost collinear tracks in the detector. This 
degrades the vertex resolution along the 
beam line compared to that expected in an electron beam with tracks from the nominal target position. 
Furthermore, tails of the vertex distributions are hard to study with the finite sample of events from the 
Test Run. 
Nevertheless, useful information can still be 
obtained by studying the vertex distributions. Figure~\ref{fig:vtx_pos} shows the distance of closest 
approach of the momentum vectors extrapolated in the 
upstream direction from our analyzing magnet, taking into account the measured fringe field. 
 \begin{figure*}[t]
\includegraphics[ scale=0.25]{test2012/vertexing/figures/h_vtx_fr_x_h_vtx_x_dataMC_twotrksel.png}
\includegraphics[ scale=0.25]{test2012/vertexing/figures/h_vtx_fr_y_h_vtx_y_dataMC_twotrksel.png}
\includegraphics[ scale=0.25]{test2012/vertexing/figures/h_vtx_fr_z_h_vtx_z_dataMC_twotrksel.png}
\caption{\small{Vertex position represented by the distance of closest approach of the extrapolated momentum vectors upstream of the analyzing magnet. The overall shift from zero is due to a 30~mrad rotation of the SVT with respect to the beam line.}}\label{fig:vtx_pos}
\end{figure*}
While the tails of the vertex distribution expected in electron beam running is not accessible here the 
fact that the core is relatively well described provides confidence of the description of the amount of 
material and the multiple scattering description; both crucial for benchmarking the physics reach of the 
HPS detector.





%\subsubsection{ECal \& Trigger Performance}

\vspace{1cm}{\bf ECal performance [Sho]}

The ECal preamplifiers shape the APD signal into a CR-RC pulse of rise time $\approx 14$ ns; this is sampled every 4 ns and stored in a pipeline on the FADC readout board.
On receiving a trigger, the FADC searches for rising threshold crossings in the pipeline, and integrates pulses by summing 5 samples before and 30 samples after each threshold crossing.

The noise and pedestal of the readout chain are calibrated by running the

Of 442 crystals/channels, 39 were disabled or disconnected and were not read out by the DAQ. 
13 of these were not read out because of a shortage of FADC readout boards.
The remainder either had no HV bias on the APD, or were disabled in the FADC software due to noise.

In the data, we identified two types of abnormal channels. 
One FADC was not sending trigger signals correctly, resulting in low efficiency. This affected the 13 channels read out by that FADC.
5 channels were diagnosed as noisy because they had a high incidence of hits out of coincidence with the trigger.

A large number of channels were originally misidentified as noisy because they had much higher hit occupancy than neighboring channels.
Gain calibration (described in the next section) shows that these channels have high gain (and thus lower energy threshold) but are otherwise normal.

The abnormal channels were ignored in analysis in order to simplify comparison with Monte Carlo. This leaves 385 useful channels---87\% of the ECal.

We found that one quadrant of the ECal had been miswired in such a way as to flip the horizontal coordinate---the column of crystals nearest the center was connected to the readout channels for the rightmost column, and vice versa.


Describe clustering, thresholds, occupancy, dead/noisy crystals, flip in readout?

Plots: Occupancy map

\begin{figure}[ht]
	\includegraphics[width=0.5\textwidth]{test2012/ecalperformance/hitrates}
	\caption{\small{}}
	\label{fig:hitrates}
\end{figure}

\vspace{1cm}{\bf ECal Calibration [Sho]}


Description of the gain calibration. Relate to how good it needs to be which should be in the performance section. Discussion of sampling fraction. Relation to what calibration that is needed for the trigger in 2014?

Plots: E/p map before and after calibration, spread in gain, E/p data/MC


\vspace{1cm}{\bf Trigger performance [Sho/Ben]}

%\begin{figure}[ht]
%	\includegraphics[width=\textwidth]{test2012/ecalperformance/trigtimes}
%	\caption{\small{}}
%	\label{fig:trigtimes}
%\end{figure}

As described in Section \ref{sec:tesrun_trigger}, the trigger and DAQ integrate pulses differently to measure hit energy. The trigger integrates using a time-over-threshold window, and the DAQ readout integrates using a constant window (5 samples before and 30 samples after a threshold crossing). 

For every event, the trigger reports as a bitmask the trigger decision (top trigger, bottom trigger, or both) and the time the trigger fires.

We study trigger performance by simulating the trigger for each event and comparing actual To study trigger performance, we first convert from readout hits (constant integration window) to trigger hits (time-over-threshold integration). 
This is done by converting from the readout hit to pulse amplitude, then applying the time-over-threshold algorithm to the simulated ECal pulse shape. 
We then simulate the CTP clustering algorithm and the trigger decision, and compare the trigger decision and trigger time reported by the simulation to what was reported by the real trigger.

To eliminate trigger bias in checking the trigger decision, we use a tag and probe method: to check trigger performance in one half of the ECal, we tag events where there was a trigger in the other half, and exactly one probe cluster in the ECal half under test. 
We then measured trigger efficiency (proportion of tagged events where there was a trigger) as a function of ADC counts and energy of the probe cluster.

These turn-on curves are shown for the top half of the ECal in Figure \ref{fig:turnon}. 
The trigger threshold is seen to be 1280 ADC counts as expected; the threshold is not perfectly sharp in this analysis because of uncertainties in the conversion from constant-window to time-over-threshold integrals, but based on comparisons with Monte Carlo simulation we believe the trigger worked exactly as specified. 
The trigger threshold in terms of cluster energy is very uneven for two reasons: gain variations between different ECal crystals lead to threshold variations, and the nonlinearity of the time-over-threshold integral means that the effective threshold is higher for clusters that span multiple crystals.

\begin{figure}[ht]
	\includegraphics[width=0.4\textwidth]{test2012/ecalperformance/top_turnon_adc}
	\includegraphics[width=0.4\textwidth]{test2012/ecalperformance/top_turnon_e}
	\caption{\small{Trigger turn-on as a function of probe cluster ADC counts (left) and probe cluster energy in MeV (right). Both plots are for the top half of the ECal; bottom is similar. 
	Energy is not corrected for sampling fraction.}}
	\label{fig:turnon}
\end{figure}

Overall the trigger appears to have functioned exactly as intended. Changes planned for the next run (constant integration window and per-crystal gain calibration constants for the trigger) will solve both of the issues that led to threshold variations in the test run.

What were the rates, lessons learned?

Plots: Compare observed and expected trigger time {\it Sho}, Tag\&probe {\it Sho}, rates vs time {\it Ben/Sho/Pelle}



\label{sec:ecalperformance}






\clearpage

\section{HPS Performance Studies}
\label{sec:performance}
\def\etal{{\it et al.\/}}


We use the HPS detector simulation system based on SLAC's org.lcsim infrastructure for full GEANT4
simulation of the passage and interaction of charged and neutral particles through the SVT 
and the ECal to the muon detector. In the SVT, it creates realistic energy deposits in the silicon 
microstrip detectors, accounts for dead material, simulates APV25 signal sampling every 25 ns, 
creates clusters, and peforms track finding and reconstruction.
In the ECal, the geometry for the flange and vaccum chamber is based on a tessellated 
represantation imported directly from the CAD drawings. It creates energy deposits in individual 
trapezoidal-shape $PbWO_4$ crystals, simulates FADC signal time evolution and sampling every 4 ns, and 
generates triggers based on the  FPGA trigger algorithm implementation.
To maintain the chicane beamline configuration, the field strength of the
chicane magnets must scale with the beam energy. The performance studies were 
made using the field strength of the 
analyzing magnet of 0.25 Tesla at 1.1 GeV, 0.5 Tesla at 2.2 GeV, and 
1.5 Tesla at 6.6 GeV.
Figure  \ref{fig:lcsim} shows a lcsim rendering of the HPS detector.

\begin{figure}[h]
\includegraphics[width=\textwidth]{performance/lcsimDetector}
\caption{\small{ Rendering of the HPS detector simulation}}
\label{fig:lcsim}
\end{figure}

\subsection{Simulation of backgrounds and detector occupancies}

\subsubsection{Simulation of backgrounds}
\label{sec:backgrounds}

The multiple Coulomb scattering and bremsstrahlung processes in the target will generate high 
intensity fluxes of electrons and photons in the very forward direction, while the large
M{\o}ller interaction cross section with atomic electrons will generate high intensity low energy
electrons. We use the high energy interaction simulation tools GEANT4 and EGS5 to simulate 
these backgrounds. In the original HPS proposal to JLab PAC37, we described a siginificant 
disageement beween these tools. GEANT4 predicted a broader angular
distribution of multiple scattered electrons than EGS5, resulting in twice the occupancy in the
tracker near the dead zone and much higher ECal trigger rates. 
To resolve this disagreement, we proposed a test run, and the outcome of the test run is described 
in the previous section. The algorithms used 
in the codes to simulate the multiple scattering have been studied, and the findings are summarized in
Appendix. The test run result and the algorithm studies have concluded that EGS5 can describe the multiple scattering 
tails more correctly than GEANT4. All the electromagnetic interactions in the target are simulated using EGS5.   

When bound electrons in the target are ionized by incoming electrons or secondary photons, outer 
shell electron will fill the vacancy and characteristic X-rays are emitted. 
These X-rays can contribute background hits in
the SVT when a conversion takes place in the silicon sensors via the photoelectric effect 
or pair productioncon. While the X-ray production is simulated in EGS5, the X-ray intensity at SVT layer 1
can be estimated using  the impact ionization 
cross section, $\sigma_I$, \cite{hoffmann}, the fluorescence yield, $\omega$, \cite{hubbell},
the photoabsorption length in Tungsten, $\lambda_W$, to account for the self-absorption, and the solid 
angle of the SVT layer 1.
Table \ref{tab:xray} summarizes these parameters and the expected X-ray
fluxes at SVT Layer 1 for 0.25\% $X_0$ Tungsten and 100 nA beam current in 8 nsec time window. 

\begin{table}[h]
\begin{center}
\begin{tabular}{|c|c|c|c|c|c|} \hline
  & Energy (keV) & $\sigma_I$ (barns) & \hspace{0.5 cm} $\omega$ \hspace{0.5 cm} & $\lambda_W$ ($\mu$m) & $N_\gamma$ at Layer 1 in 8 ns   \\ \hline
K-shell & 60 & 40 & 0.95 & 100 & 0.5 \\ \hline
L-shell  & 10 & 1000 & 0.30 & 5 & 2 \\ \hline
M-shell  & 2 & 20000 & 0.02 & 0.2 & 0.1 \\ \hline
\end{tabular}
\end{center}
\caption{\small{X-ray intensities}}
\label{tab:xray}
\end{table}

Hadrons are also produced in the target. The hadron production is at least three order
of magnitude smaller than the electromagnic interaction. The polar angle of the hadron productions
is predominantly at larger than 100 mrad, whereas the HPS detector acceptance is limitted to less than
100 mrad. Furthermore, the hadron energy spectrum is soft as they are produced from the 1/k bremsstrahlung
spectrum and more than 90\% of the hadrons are swept away by the analysing magnet before reaching the ECal.
 The hadron production is simulated using GEANT4 and FLUKA. In the target thinner than
about 5\% $X_0$, the ``virtual'' photon interaction is dominant. \cite{mohring} The inclusive hadron
production ${\sigma (eA\rightarrow X)}$ is simulated from the photonuclear process ${\sigma (\gamma A
\rightarrow X)}$ using the equivalent photon approximation,

$$ \sigma (eA \rightarrow X) = \int \sigma_k(\gamma A \rightarrow X) dn(k), $$

\noindent
where $dn(k)$ is the number of equivalent photons with energy $k$ \cite{budnev} and there are 
approximately $8 \times 10^{10} $ photons in 6.6 GeV 100 nA beam. 
Table \ref{tab:pion} summarizes the pion single rates from 1\% $X_0$ Tungsten target
and 6.6 GeV 100 nA beam. While the pion production is higher in GEANT4, the energy spectrum is softer and
consequently the single rate of pions reaching the ECal is lower in GEANT4. While the pions look like a minimum 
ionizing particle in the ECal most of the time, they can deposit a significant energy when ${\pi^0}$ are
produced in the ECal crystals, and together with the beam background, they contribute accidental coincident triggers. 

\begin{table}[h]
\begin{center}
\begin{tabular}{|c|c|c|} \hline
  & Total production rate (kHz) & Single rate reaching the ECal (kHz) \\ \hline
GEANT4 & 410 & 8 \\ \hline
FLUKA  & 240 & 15 \\ \hline
\end{tabular}
\end{center}
\caption{\small{Pion single rates from 1\% $X_0$ Tungsten target at 6.6 GeV 100 nA}}
\label{tab:pion}
\end{table}

\pagebreak
\noindent
Other beam induced background we considered are:

\begin{itemize}
\item
Beam halo

Beam halo was measured using a large dynamic range halo monitor during the 6 GeV era. The beam halo 
that extends to 2 mm was found at the level of $10^{-7}$. At this level, the halo contribution in 
the SVT occupancy is negligible. It is expected that behavior of the 12 GeV machine will be
understood at the same level.

\item
Synchrotron radiations

Synchrotron radiations are produced from the last dipole magnet in the beam line in the vertical 
plane, and from the chicane magnets in the horizontal plane. Since the characteristic energy is 
proportional to $E_{beam}^2$, synchroton radiations are of concern 
only at 6.6 GeV. The characteristic energy ($k_c$),
the average energy ($k_{ave}$), and the power of synchrotron radiations are summarized in 
Table \ref{tab:sync}.
None of the radiations from the last dipole will enter the HPS detector as the radiations will be intersected 
by the beamline collimator. The radiations from the chicane magnets are in the dead zone, and
none of the detector components are designed to intersect the beam plane.   

\begin{table}[h]
\begin{center}
\begin{tabular}{|l|c|c|c|c|} \hline
  Source & $k_c$ (keV) & $k_{ave}$ (keV) & $N_\gamma$ per e- & Power (mW) at 100 nA \\ \hline
  virtical bend & 19 & 5.9 & 4.0 & 2.4 \\ \hline
  Frascati Magnet & 52 & 16 & 4.6 & 7.4 \\ \hline
  PS magnet   & 44 & 14 & 9.3 & 13 \\ \hline
\end{tabular}
\end{center}
\caption{\small{Synchrotron radiations at 6.6 GeV}}
\label{tab:sync}
\end{table}


\item
EM induced backgrounds

Electromagnetic fields induced by the high intensity beam could interfere with the SVT and its electronics
as the detector is located as close as 0.5 mm from the beam. We have evaluated the direct beam field and its wake 
field, the diffraction radiations from the beamline apertures, and the transition radiations from
the target. The intensities of these EM induced backgrounds are small and no interferance with the SVT
is expected.
 
\end{itemize}

\subsubsection{Simulated Tracker Occupancies}

Figure \ref{fig:scatt} shows the distribution of charged particle hits in Si tracker layer 1 
which is located 
10 cm from the target. The beam energy is 6.6 GeV, and the target thickness is 
0.25\% $X_0$. Multiple Coulomb scattered beam electrons are confined within 0.5 cm of the beam axis
(x=y=0), while the low energy M{\o}ller electrons are distributed in a parabolic shape. There are
very few positrons. From these distributions, the detector occupancy in the horizontal Si strip
sensor in the 8 ns time window is calculated for a 400 nA beam current and five different target
thicknesses, 1.0\% $X_0$, 0.5\% $X_0$, 0.25\% $X_0$, 0.1\% $X_0$, and 0.05\% $X_0$, and is shown
in Figure \ref{fig:occup}. As described in Section xx, the dead zone is defined by using 
a criterion that the
maximum occupancy in Layer 1 is 1\%. For a 0.25\% $X_0$ target and 430 nA beam, the occupancy is 
1\% at a distance of 1.5mm from the beam in Layer 1, which corresponds to a dead zone of $\pm$ 15
mrad. As long as the product of target thickness (T) and beam current (I) is constant, the same 
$A'$ production rate is maintained. Since the multiple scattering and hence the effective beam size 
is reduced in a thinner target, it is advantageous to use a thinner target and a higher current.
Using the constraint that the occupancy is 1\% at 15 mrad, we find the beam current $I$ which 
gives this occupancy for each of several potential target thicknesses $T$. The quantity 
$(I\cdot T)^{1/2}$, which is approximately proportional to the sensitivity $S/\sqrt{B}$, is
given in Table \ref{tab:occup}, showing how the sensitivity improves as the target thickness 
decreases.

\begin{figure}[h]
\includegraphics[width= 4 in]{performance/scatterplot.pdf}
\caption{\small{Charged particle distribution in SVT layer 1.}}
\label{fig:scatt}
\end{figure}

\begin{figure}[t]
\includegraphics[width=0.8\textwidth]{performance/occupancy.pdf}
\caption{\small{Sillicon sensor layer 1 occupancy at 400 nA vs. distance from the
beam in mm.}}
\label{fig:occup}
\end{figure}

\begin{table}[h]
\begin{center}
\begin{tabular}{|c|c|c|} \hline
  Target thickness (\% $X_0$) & Beam Current (nA) & $\propto S/\sqrt{B}$ \\ \hline
  1.0 & 60 & 7.7 \\ \hline
  0.5 & 170 & 9.1 \\ \hline
  0.25 & 430 & 10.4 \\ \hline
  0.10 & 1330 & 11.6 \\ \hline
  0.05 & 2860 & 11.9 \\ \hline
\end{tabular}
\end{center}
\caption{\small{Beam current yielding 1\% occupancy in SVT layer 1 for various target 
thicknesses at 6.6 GeV, and the relative experimental sensitivities which result.}}
\label{tab:occup}
\end{table}

The run conditions for other possible beam energies are studied using the same criterion that the maximum occupancy 
in SVT Layer 1 does not exceed 1\%. Table \ref{tab:runc} summarizes the target thickness and proposed beam current. 

\begin{table}[h]
\begin{center}
\begin{tabular}{|c|c|c|} \hline
  Beam Energy & Target thickness (\% $X_0$) & Beam Current (nA) \\ \hline
  1.1 & 0.125 & 50 \\ \hline
  2.2 & 0.125 & 200 \\ \hline
  4.4 & 0.25  & 300 \\ \hline
  6.6 & 0.25 & 450 \\ \hline
\end{tabular}
\end{center}
\caption{\small{Run Conditions}}
\label{tab:runc}
\end{table}

\subsubsection{Simulated ECal occupancies}

There are two factors limiting the allowable ECal occupancy. First, the ECal 
readout algorithm uses a window of fixed size to integrate hit energy. This 
window was set to 140 ns ($35 \times 4$ ns) for the test run, and so the 
number of hits above readout threshold in a 140-ns time window should be well 
below 1. Figure \ref{fig:ecal_rate} shows that the maximum rate in any crystal 
is 500 kHz, which translates to 0.07 hits in 140 ns.

Second, because the FADC only reads out on a rising threshold crossing, each 
hit above threshold causes dead time for that crystal until the preamp output 
falls back below threshold. Figure \ref{fig:ecal_deadtime} shows the fraction 
of time each crystal spends above threshold. The maximum dead time is 0.03, 
meaning that even the hottest crystal is sensitive to new hits 97\% of the time.

\begin{figure}[ht]
	\includegraphics[width=0.5\textwidth]{performance/ecal_rate_100mev_22}

	\includegraphics[width=0.5\textwidth]{performance/ecal_rate_100mev_22_log}
	\caption{\small{Rate of hits over 100 MeV (units of kHz) per crystal (X and Y axes are the crystal index), 
for 2.2 GeV beam at 200 nA. Top plot uses linear scale for the Z-axis; bottom plot is log scale.}}
	\label{fig:ecal_rate}
\end{figure}

\begin{figure}[ht]
	\includegraphics[width=0.5\textwidth]{performance/ecal_deadtime_22}
	\caption{\small{ECal readout deadtime fraction for 2.2 GeV beam at 200 nA, 
with a threshold of 75 MeV for each crystal.}}
	\label{fig:ecal_deadtime}
\end{figure}

\subsection{Trigger rates: ECal and muon detector}

\subsubsection{ECal trigger performance}
The proposed ECal trigger was simulated to test trigger cuts, verify
that the trigger has acceptable efficiency for A' events, and verify 
that the trigger rate is compatible with the HPS DAQ in all running conditions. 

The CEBAF beam bunch structure was simulated by sending one bunch 
equivalent of electrons, 
625 (1.1 GeV), 2,500 (2.2 GeV) and 5,625 $e^-$'s (6.6 GeV), through 
the target, and total 50 million bunches of beam backgrounds 
(equivalent to 100 ms of beam) were 
generated at each beam energy. The details of the target interactions are given in Section \ref{sec:backgrounds}.
Since the trident production process 
was not in EGS5, trident events were generated with MadGraph/MadEvent 
and overlaid on the beam background bunches with average rate 
expected from the trident cross section.
For the trigger acceptance studies, A' events were generated with 
MadGraph/MadEvent at 1.1, 2.2, and 6.6 GeV.

The complete chain of signal evolution in the ECal crystals and 
signal processing through the trigger system was simulated
by following closely the ECal trigger description in Section \ref{sec:triggerdaq}. 
Starting from the energy deposits in the ECal crystals, signals were 
generated using the CR-RC shaper function with a time constant of 15 ns
measured with the ECal crystals, amplitudes were sampled and pulse data 
evaluated every 4 ns (simulating FADC), and the cluster 
finding algorithm and trigger logic were applied (simulating CTP and SSP). 
The simulation has been tested against the actual performance of the test run detector and DAQ: see Section \ref{sec:ecalperformance}.

The trigger parameters described in Section \ref{sec:triggerdaq} are 
chosen by running the simulation and plotting the relevant variables 
for beam background and A' events. This is done for each beam energy 
and a set of A' masses for each beam energy. 
Figure \ref{fig:coplanarity} shows the coplanarity angle vs. the azimuthal 
angle of the lower-energy cluster, indicating that A' events 
tend to have small coplanarity angles. Figure \ref{fig:energy-distance} shows
the distance from beam axis vs. energy of the lower-energy cluster,
indicating that the energy-distance cut can reduce the beam background effectively.
Figure \ref{fig:ediff} shows the cluster energy difference vs. energy sum,
indicating that the energy sum cut can retain A' events effectively.       


These cuts are chosen to lie between the loosest reasonable values (accept as many A' events as possible) and the tightest (reject as many background events as possible). In some cases this leads to different cut values at different beam energies---for example, the coplanarity cut is looser at 1.1 GeV because the background events are clustered at large uncoplanarity and a relatively loose cut rejects most of them, but the cut is tighter at higher beam energies.

\begin{figure}[ht]
	\includegraphics[width=0.4\textwidth]{performance/trigger/coplanarity_22}
	\includegraphics[width=0.4\textwidth]{performance/trigger/coplanarity_22_075mev}
	\caption{\small{Deviation of cluster pairs from coplanarity (units of degrees) for 2.2 GeV beam; background and 75 MeV A' tridents are shown. The X-axis is the azimuth around the beam axis ($\phi_1$) of the lower-energy cluster, such that 0 degrees is the positron side of the detector and 180 degrees is the electron side; the Y-axis is the difference between the azimuth angles ($\phi_1-\phi_2 - 180$) of the two clusters. The coplanarity cut's acceptance region is the space between the red lines.}}
	\label{fig:coplanarity}
\end{figure}

\begin{figure}[ht]
	\includegraphics[width=0.4\textwidth]{performance/trigger/energy-distance_22}
	\includegraphics[width=0.4\textwidth]{performance/trigger/energy-distance_22_075mev}
	\caption{\small{Energy and distance from beam axis of the lower-energy cluster, for 2.2 GeV beam; background and 75 MeV A' tridents are shown. The energy-distance cut's acceptance region is above the red line.}}
	\label{fig:energy-distance}
\end{figure}

\begin{figure}[ht]
	\includegraphics[width=0.4\textwidth]{performance/trigger/ediff_22}
	\includegraphics[width=0.4\textwidth]{performance/trigger/ediff_22_075mev}
	\caption{\small{Energy sum and difference of cluster pairs, for 2.2 GeV beam; background and 75 MeV A' tridents are shown. The energy difference cut's acceptance region is left of the red line.}}
	\label{fig:ediff}
\end{figure}

The following trigger parameters were determined to be independent of beam energy:
\begin{itemize}
	\item Minimum cluster energy ($E_{min}$): 0.1 GeV
	\item Distance ($r_{edist}$) in the energy-distance cut: 200 mm
	\item Energy ($E_{edist}$) in the energy-distance cut: $0.5\times E_{beam}$
\end{itemize}

Table \ref{tab:trigcuts} summarizes the trigger parameters that were found dependent on the beam energy. 
The remaining trigger parameters given in Section \ref{sec:triggerdaq} did not have a significant 
effect on specificity of the trigger.

\begin{table}
	\begin{tabular}{|l|r|r|r|}
		\hline
		Beam energy [GeV] & $E_{max}$ [GeV] & $Esum_{max}$ [GeV] & $\Delta\phi_{max}$ [$^\circ$] \\
		\hline
		1.1	&	0.7	&	0.8	&	90\\
		2.2	&	1.6	&	1.7	&	45\\
		6.6	&	5.0	&	5.5	&	60\\
		\hline
	\end{tabular}
	\caption{ {\small Trigger parameters optimized for different beam energies.}
	\label{tab:trigcuts}}
\end{table}

Trigger rates are shown in Table \ref{tab:trigrates}. These rates are safely under the maximum readout rate of 43 kHz set by the SVT DAQ. 
Furthermore, tightening the coplanarity and energy-distance cuts lowers trigger rates to $\approx 10$ kHz at 1.1 and 2.2 GeV and $\approx 5$ kHz at 6.6 GeV, while reducing the A' efficiency by no more than 2 percentage points; this provides further safety margin in case trigger or data rates are higher than expected.
The addition of pions to the 6.6 GeV background sample has only a small effect on the trigger rate.

\begin{table}
	\begin{tabular}{|l|r|}
		\hline
		Sample &  Rate (kHz)\\
		\hline
		1.1 GeV	beam background 				& 15.7 $\pm$ 0.4	\\
		1.1 GeV beam background+tridents			& 18.3 $\pm$ 0.4	\\
		2.2 GeV	beam background 				& 11.2 $\pm$ 0.3	\\
		2.2 GeV beam background+tridents			& 15.8 $\pm$ 0.4	\\
		6.6 GeV	beam background 				& 10.2 $\pm$ 0.3	\\
		6.6 GeV beam background+tridents			& 12.6 $\pm$ 0.4	\\
		6.6 GeV beam background+tridents+pions (FLUKA)	& 13.4 $\pm$ 0.4	\\
		6.6 GeV beam background+tridents+pions (G4)	& 13.5 $\pm$ 0.4	\\
		\hline
	\end{tabular}
	\caption{ {\small Trigger rates using various background samples, with statistical uncertainties. }
	\label{tab:trigrates}}
\end{table}

Trigger efficiency for A' events is defined as the fraction of A' tridents (generated without fiducial cuts) that produce a trigger.

For the performance of the experiment, we are interested in the combined efficiency of the trigger and tracker: the fraction of A' tridents that produce a trigger and leave enough hits in the tracker for a pair of tracks to be reconstructed.
We simulate charge deposition and readout of the tracker (turning off the generation of noise hits), and check each sensor for hits. 
If the DAQ reads out hits in four stereo pairs in each half of the tracker, the event is in the combined acceptance.

Figure \ref{fig:trigeff} shows the ECal trigger efficiency and the ECal/SVT-combined efficiencies for A' events at 1.1, 2.2, and 6.6 GeV. 
Both trigger and tracker acceptances are dominated by the geometric acceptances of the ECal and tracker.
A' prompt decays are assumed.

\begin{figure}[ht]
	\includegraphics[width=\textwidth]{performance/trigger/ap_eff}
	\caption{\small{Trigger efficiency (solid lines) and combined efficiency (dashed lines) as a function of A' mass, at beam energies of 1.1, 2.2 and 6.6 GeV (red, green and blue respectively).}}
	\label{fig:trigeff}
\end{figure}



\subsubsection{Muon trigger performance}

\subsection{Track reconstruction}
\label{sec:trkperf}


In order to study the tracking performance of the detector, we use samples of $A'$ events 
at a variety of energies and decay lengths.  On top of each event, we overlay backgrounds 
produced by the passage of  beam electrons equivalent to our optimized run conditions at
different beam energies and with a W target and a beamspot with a Gaussian sigma of 40$\mu$m in the vertical direction and 
200$\mu$m in the horizontal. The beam energies, currents, target thickness and analyzing magnetic field  used for these simulations are:
\begin{itemize}
\item 50nA at 1.1~GeV with $X_0=0.125$\% and  B=0.25 T
\item 200nA at 2.2~GeV with $X_0=0.125$\% and  B=0.5 T
\item 300nA at 4.4~GeV with $X_0=0.25$\% and  B=1.0 T
\item 450nA at 6.6~GeV with $X_0=0.25$\% and  B=1.5 T
\end{itemize}
At each energy, we evaluate momentum, invariant mass, and vertex resolution.  The plots shown in the following section typically use the 2.2~GeV beam as an example.  

\subsection{Tracking Efficiency, Pattern Recognition and Fake Rates}

Due to the requirements imposed on the tracks, the efficiency for finding tracks in the 
geometric acceptance is not 1. The average track reconstruction efficiency is 98\% (Fig.~\ref{fig:trkeffic}) and 
the bulk of the inefficiency comes from the cut on the total $\chi^2$ of the track. 
Of the reconstructed tracks, a small percentage include a hit that is not from 
the correct electron.  These ``bad'' hits may be from one of the high energy beam 
electrons scattered from the target into the detector or from a lower energy secondary.  
The left plot of Fig.~\ref{fig:badhits} shows the number of bad hits/track for both the electron 
and positron from the A' decay.  The number of tracks with 0 bad hits is $>$ 98\%.
% and 
%the positrons are slightly cleaner since occupancy of the positron side of the detector 
%is smaller.  
The right plot of Fig.~\ref{fig:badhits} shows the layer number of the bad hit.  
The rate of mishits are slightly higher in the downstream 3 modules due to the larger stereo angle % and are larger for positrons because they dip into the dead zone???%.  
We'll show how these bad hits affect the track parameters in the next section.


\begin{figure}
\includegraphics[scale=0.8]{performance/tracking_performance/pzE-Effic.pdf}
\caption{ Track reconstruction efficiency versus track momentum (right axis). The black histrogram (left axis) show the track momentum distribution. }
\label{fig:trkeffic}
\end{figure}

\begin{figure}
\includegraphics[scale=0.4]{performance/tracking_performance/nBadHits.pdf}
\includegraphics[scale=0.4]{performance/tracking_performance/BadLayer.pdf}
\caption{ The number of bad hits (left) and the layer number of the bad hit (right) 
for electron (black) and positron (blue) tracks.   }
\label{fig:badhits}
\end{figure}


\subsection{Track Momentum and Spatial Resolution}

The momentum resolution is shown in Fig.~\ref{fig:trkmom} as a function of momentum for tracks with 
0 bad hits and for tracks with one or more.  The momentum resolution for well-reconstructed 
tracks is $\delta p/p$ = 4.5\% for B=0.5T (appropriate for a beam energy of 2.2~GeV) and is roughly inversely proportional to B.  
%This momemtum 
%resolution is considerably worse than that in the full HPS proposal (~1.5\%) because small 
%angle stereo, which is used in the test run, provides much less precision in the bend plane 
%than the 90 degree stereo which is used in full HPS. The lower resolution still provides 
%adequate invariant mass resolution for this experiment.


\begin{figure}
\includegraphics[scale=0.8]{performance/tracking_performance/pz2pt2GeV-MomRes-Tracks.pdf}
\caption{  Fractional momentum resolution versus momentum for a beam energy of 2.2~GeV and a magnetic field of 0.5 T. } 
\label{fig:trkmom}
\end{figure}


One quantity we use to determine track quality is the distance of closest approach (DOCA) 
to the beam axis.  We use this instead of the DOCA to the target beam spot since we are 
interested in long-lived decays and tracks from those will not point back to the target. 
We separate the distance into the bend plane (XOCA) and non-bend plane (YOCA) distances.  
Below, in Fig.~\ref{fig:doca}, is the resolution of these quantities as function of momentum.  
The resolution is, on average, about 200$\mu$m (400 $\mu$m) 
in the non-bend (bend) direction but increases significantly at low momentum.  The position 
resolution for tracks with one or more bad hits is somewhat worse, depending on which layer 
the bad hit is.  Tracks with bad hits in 
layers 1 or 2 are a major contribution to the tail of the vertex position distribution. 
    
For long lived  $A^\prime$ decays, the position of the decay vertex is an important discriminating 
variable.  The dominant background to $A^\prime$ production is radiative events which originate 
in the target. Distinguishing $A^\prime$ decays from the background therefore depends on the vertex 
resolution and in particular on the tails of the vertex distribution. In order to study 
the tails, we use large samples of $A^\prime$ events decaying promptly overlaid on top of the 
simulated beam background events.
   
\begin{figure}
\includegraphics[scale=0.4]{performance/tracking_performance/yoca2pt2GeV-MomResolution.pdf}
\includegraphics[scale=0.4]{performance/tracking_performance/xoca2pt2GeV-MomResolution.pdf}
\caption{The resolution of the position of closest approach to the beam axis 
versus track momentum in the (left) non-bend direction and (right) bend direction.  
The dots represent tracks with 0 bad hits and triangles with one or more. }
 \label{fig:doca}
\end{figure}

Each pair of oppositely charged tracks is fit to a common vertex using a Kalman filtering 
method first suggested by Billoir \cite{bf}, \cite{bq} and used in many experiments.  The method 
uses the measured helix parameters and their correlations to determine the most likely 
decay position of the $A^\prime$ and also returns fitted momenta for each particle.  We actually 
fit each pair twice with different hypotheses of their origin.  We constrain either 
the vertex to be consistent with an $A^\prime$:

\begin{itemize}
\item which originates in the 200$\mu$m $\times$ 40$\mu$m beamspot at the target, and moves off 
in the direction given by the measured $A^\prime$ momentum.  This fit will be used for the vertexing search.  
\item which originates and decays at the target within the 200$\mu$m $\times$ 40$\mu$m beamspot.  
This fit will be used for the bump-hunt only search.  
\end{itemize}

For each electron/positron pair reconstructed in the tracker, we compute the invariant mass based 
on the fitted momenta of the tracks.  The mass resolution depends on the invariant mass of the pair 
and is shown in Fig.~\ref{fig:massres}.  The closed circles  in Fig.~\ref{fig:massres} shows the improvement 
in the resolution for the second fit, where the decay is assumed to occur in the target.  

\begin{figure}
\includegraphics[scale=0.8]{performance/tracking_performance/massRes-2pt2.pdf}
\caption{The gaussian width of the mass distributions (MeV/c2) vs generated $A^\prime$ mass (MeV/c2). 
 The open circles are the resolutions when the decay is constrained to the target beamspot 
and the closed circles are without this constraint.    }
\label{fig:massres}
\end{figure}



Even for prompt decays, the z vertex position (Vz) distribution of all reconstructed $e^+e^-$ pairs
  (solid black histogram, Fig.~\ref{fig:vtxResolutionRaw}) shows a long tail, still significant beyond 5cm.   
This tail is primarily comprised of events where one or both of the tracks use one or 
more bad hits.  Fortunately there are a number of quantities we can use to minimize the tails.  
Namely, for purposes of this proposal, we make the following cuts:

\begin{itemize}
\item The $\chi^2$  of each track is less than 20
\item The total momentum of the $A^\prime$ candidate is less than the beam energy
\item A very loose cut on the reconstructed vertex position $|V_x|<400\mu$m and $|V_y|<400\mu$m
\item The clusters in layer 1 of each track must be isolated from the next closest cluster by at least 500 $\mu$m 
\item A $\chi^2$ cut on the vertex fit of less than 5.
\end{itemize}

The vertex resolution depends on the invariant mass of the particles being vertexed. 
Lower masses have worse Gaussian resolutions as shown in Fig.~\ref{fig:vtxResolutionGauss}.  This is expected 
since the error on the opening angle ($\theta$), due to multiple scattering, scales like: 
 $\sigma (\theta )/\theta \sim (1/E)/(m/E) \sim 1/m$.
 
Figure~\ref{fig:vtxResolutionRaw} shows the vertex resolution for samples of 80 MeV and 160 MeV $A^\prime$ events. 
The cuts above remove almost all of the tail past ~1.5cm (points with errors in Fig.~\ref{fig:vtxResolutionRaw}) 
while retaining ~50\% of the $e^+e^-$ pairs from the $A^\prime$ candidate. The events on the tail are 
enhanced with vertices where there are one or more bad hits on the track (represented by the 
blue histogram in Fig.~\ref{fig:vtxResolutionRaw}), although there is still a contribution from well-reconstructed 
tracks.  The rejection of tracks with bad hits depends strongly on the precision of the virtual $A^\prime$ 
trajectory, which in turn depends on the size of the beamspot. Having a beamspot significantly 
smaller than the intrinsic tracker resolution, 100$\mu$m in the non-bend and 300$\mu$m in 
the bend directions, is important.  

In practice, there is much more we can do to clean up the vertex and mass resolution both at 
the track level (e.g. remove hits that are clearly from scattered beam electrons) and at 
the vertex level.  These will be pursued in the near future.

\begin{figure}
\includegraphics[scale=0.8]{performance/tracking_performance/vertexRes-1pt1-2pt2-6pt6.pdf}
\caption{ The Gaussian resolution dependence versus $A^\prime$ mass for signal-only events.   }
\label{fig:vtxResolutionGauss}
\end{figure}
   

\begin{figure}
\includegraphics[width=0.8\textwidth]{performance/tracking_performance/vtx2pt2-40mev.pdf}
\includegraphics[width=0.8\textwidth]{performance/tracking_performance/vtx2pt2-80mev.pdf}
\caption{Distribution of the reconstructed vertex position along the beam axis for 
2.2GeV 40MeV (top) and 80MeV (bottom) $A^\prime$ events before (solid black) and after (points 
with errors) selection.  The blue histogram shows the distribution for pairs that have 
at least one bad hit after selection.    }
\label{fig:vtxResolutionRaw}	
\end{figure}
  
%\bibliographystyle{unsrt}
%\begin{thebibliography}{99}
%\end{thebibliography}



%\subsection{Cluster reconstruction}

%\subsection{Muon identification}



\clearpage


%\clearpage


\section{Run Plan and Beam Time Request}

\section{Schedule and Cost Baseline}
%\section{Schedule and Cost Baseline}
\label{sec:schcost}

Cost estimates for engineering, designing, fabricating, assembling, testing, and installing the Heavy Photon Search detector are given below. The costs reflect considerable savings coming from the reuse of many parts of the Test Run, which was already made from the donation of the silicon microstrip sensors from Fermilab, the use of some DAQ crates and equipment from SLAC, and many contributions from JLab, including PbWO$_4$ calorimeter crystals, the chicane and analyzing magnets, magnet power supplies, and beam diagnostic apparatus. Much of the calorimeter readout electronics utilizes designs which are already in place for the Hall B 12 GeV upgrade, eliminating engineering and design expense. The SVT DAQ benefits from SLAC's development of an ATCA readout system, and incorporates many of its designs.Very significant cost savings come from utilizing the FADCs and data acquisition system being developed for the upgraded CLAS12 detector, which will be available to the Test Run. The Orsay group is contributing engineering and design efforts for the ECal and its vacuum chamber, affording additional savings. 

The costs are given in an accompanying WBS summary table, below, which itemizes the major items subsystem by subsystem, and indicates whether JLab (J) or SLAC (S) takes responsibility for construction. Engineering, design, and technician labor rates include lab overheads, and differ between the two laboratories. Contingencies have been set at 20\%. The cost estimates are based on the experience of the real costs incurred for the construction and installation of the HPS Test Run Experiment. Our DAQ and beamline cost estimates have been made by engineering groups at SLAC and JLab which are experienced in cost estimation and actively involved in many related projects. The SVT estimates came from physicists and engineers on the project, with experience in designing and fabricating silicon detector systems. The Ecal estimates come from physicists and engineers at JLab and Orsay who have constructed a similar system, the CLAS IC, in the recent past. 

The schedule for the overall project is included in a Project Summary table below. A brief description of the schedule for the different subsystems is also given. The overall schedule contingency is about 20\%, and depends critically on the assumption that funding is available by mid 2013.. The HPS has been organized into a Work Breakdown Structure (WBS) for purposes of planning, managing and reporting project activities. Work elements are defined to be consistent with discrete increments of project work. Project Management efforts are distributed throughout the project, including conceptual design and R\&D. The HPS has 18 WBS Level-2 elements: 

\begin{table}[htdp]
\caption{Project WBS structure.}
\begin{center}
\begin{tabular}{|c|c|}
\hline
WBS& NAME \\
\hline\hline
1.1 & Beamline \\
\hline
1.2 & SVT \\
\hline
1.3 & SVT DAQ \\
\hline
1.4 & ECAL \\
\hline
1.5 & Muon \\
\hline
1.6 & TDAQ \\
\hline
1.7 & Slow Controls \\
\hline
1.8 & Installation \& Commissioning \\
\hline
1.9 & 2014 Commissioning Run \\
\hline
1.10 & SLAC Travel \\
\hline
1.11 & SLAC Travel 2015 Data Run \\
1.12 & Project Management  \\
\hline
\hline
\end{tabular}
\end{center}
\label{tb:wbs}
\end{table}%

\subsection{Cost}

The costs include Labor and M\&S. The labor is accounted only if provided by professional centers (engineering) at SLAC or JLAB and it does not include labor provided by physicists which is a dominant contribution to the  project. Labor rates have been applied following the official shop rates at SLAC and JLAB, which include already ~31\% fringe benefits. M\&S have been calculated by the best estimation of the commercially available parts and based on the real cost incurred for the TEST RUN. The overheads have been added to both labor and M\&S being respectively 53\% and 7.65\% at SLAC, 57.15\% and 49\% at JLAB. SLAC travel includes 53\% overheads. A uniform cost contingency of 20\% has been applied. Since the project is stage over three years, an annual inflation rate of 2.5\%  included in the FY14 and FY15 costs.

Beamline expenses for HPS are held to a minimum by using the 18D36 magnet currently installed in Hall B as the analyzing magnet, the two JLab Frascati chicane magnets and the Test Run vacuum box with the SVT vacuum box.  Some overall engineering and design will be required, beam pipes fabricated, the Muon System vacuum chamber designed and built, , a vacuum chamber built for the downstream Frascati magnet, and a photon dump and shielding inserted behind the second chicane magnet. Total beamline expenses are about \$300k.

Three out of the five planes of the SVT Test Run will be reused after modification of their supports which will provide improved mechanical stability and better cooling.. Three new planes with double sensors and their supports will be designed and built from scratch. Fermilab will donate the needed silicon microstrip detectors, as it had for the HPS Test Run. The tracker/vertexer for the test run will cost about \$590k.

The SVT DAQ requires small modifications to the  hybrid, readout  and flange board engineering design,prototyping, and production,, APV25 and chip procurement, and  fabrication and testing. The SVT DAQ also requires designing and prototyping the Trigger Interrupt ACTA card and new firmware for the APV25 to provide event buffering to accommodate higher trigger rates.. ATCA crates, and standard RCE cards are also required.. The expenses are dominated by engineering development, and total \$610k.  

JLab will donate the PbWO$_4$ crystals used in the ECal. Orsay will donate engineering and design for a new enclosure for the crystals, but Jlab will need to fabricate the enclosure, the crystal support structure, the readout motherboard and connection board, and support fixtures. It will also make repairs to the existing motherboards and acquire new power supplies. The total expense will be roughly \$270k, including fabrication, assembly, and testing.. Trigger and DAQ electronics for the ECAL are being developed for the CLAS upgrade, so relatively little engineering and technician time will be needed in preparation of the HPS Test Run except for providing special purpose firmware.. Components, including the 250 MHz FADC boards and crates will be provided at no cost since they can be borrowed from the CLAS upgrade. The system test expenses will also be borne by JLab Hall B. The total cost is \$300k. 

The Muon system costs are in purchasing  photomultipliers and scintillator, designing and fabricating  the absorbers and Muon System vacuum chamber, and building support stands and providing cables.The total is \$270k. 

The Slow Control is needed to monitor the operations of the three sub-detectors. In addition, they will control and interlock the movements of the SVT respect the beamline and provide beam protection interlocks.. The total cost is \$200K being essentially labor required to integrate the HPS with the existing Slow Control system in the Hall-B.

Travel and lodging expenses for SLAC trips to JLab are also included in this proposal. During design and construction, there will be a small number of trips to solidify and review designs, and to work together to begin DAQ integration of the SLAC and JLab systems. Funds are reserved for a collaboration meetings to be held during calendar 2013 and 2014 at JLab. Funds are also reserved  to staff installation, commissioning, and data taking runs. The total is \$190K.

{\bf The total cost for HPS is \$2.8 M. }

HPS is seeking funding from other sources for the Muon System and the Ecal.
William\&Mary will submit an MRI proposal to NSF for the Muon System, requesting $\sim \$130$k. IPN ORSAY (France) has submitted a proposal to a French funding agency for the ECal Light Monitoring System (\$100k) and for a new, high performance APDs to improve crystal readout (\$500k) and other expenses related to ECal fabrication and test. Note that the new APDs are not part of this proposal. If the requests are approved the corresponding funding will be subtracted from the total cost of the HPS. 

\subsection{Schedule}

Our goal is to be ready to install the HPS in Hall-B at JLAB  by September 2014, and proceed with the commissioning on beam with the CEBAF early physics beam window opportunity in October 2014. The data taking will continue in 2015 until Summer.. Meeting this schedule  will require approval and funding as soon as possible, preferably by June 2013.. Schedules for each of the major subsystems of the experiment are attached below, and summarized here. The total construction schedule extends over  16 months, assuming the funding available mid-2013. The schedule contingency is about 20\%. 

The conceptual design of the beamline will be done during 2013. Formal beamline engineering will start when funding is secured.  A Beamline Engineering Design Review will be held in February 2014 to validate the concept before the start the major spending. Final Engineering and Construction will start in Spring 2014 and end well before the installation time in October 2014, wlproviding substantial float. Using keep-alive funds, the Test Run SVT will be shipped back to SLAC by early February 2013 to rework the modules for the first three layers of the HPS and commission the motion control systems.. The conceptual design of the Layers 1-2-3 and Layers 4-5-6 will start already in spring 2013 using keep alive funding.. An Engineering Design Review of the SVT will be held in November 2013, after the funding has been released and before the final production. Preliminary engineering for the SVT DAQ is already underway, using the same keep alive funds. The SVT DAQ will formally begin work in  the second half 2013, after funding and an Engineering Review. The assembly and integration test at SLAC are expected in Spring 2014 and the SVT will be ready for shipping on July 2014. The SVT will be ready for tinstallation at mid-august 2014. SVT  installation in the analyzing magnet vacuum chamber will occur inSeptember, depending on the schedule of the Hall-B.  The SVT schedule has 1 month of float between the shipping and the test at JLAB, which will be eventually used as contingency for  construction work at SLAC.

The Ecal work will start in the second half 2013 and run through June 2014. The ECAL will be ready for installation by June 2014. The Muon System work will start in July 2013 with a design phase which will be validated in an Engineering Design Review in September 2013. It will be ready for installation on August 2014, with one month of float with respect to  the installation date.
The schedule includes seven milestones to track the progress of each subssytem. They will monitor the subdetector readiness after testing at the respective assembly sites, and the readiness for installation at JLAB.  Also, ad-hoc Engineering Design Reviews will be called by the PM for each subsystem before major costs are incurred.

\begin{table}[htdp]
\caption{Project Milestones.}
\begin{center}
\begin{tabular}{|c|c|c|}
\hline
WBS & Milestones & Date\\
\hline\hline
1.1.21&Beamline Installation&26-Sep-14 \\
\hline
1.2.1	&SVT returns from JLAB	&18-Feb-13 \\
\hline
1.2.12&	SVT Shipped at JLAB	&16-Jun-14\\
\hline
1.2.14&	SVT Ready For Installation&	15-Aug-14\\
\hline
1.4.12&	ECAL Ready for the installation&	8-Aug-14\\
\hline
1.5.9	&Muon Ready for installation&	30-May-14\\
\hline
1.10.7&	HPS ready for the beam&	12-Sep-14\\
\hline
\hline
\end{tabular}
\end{center}
\label{tb:milestones}
\end{table}%

\begin{table}[htdp]
\caption{Planned Review.}
\begin{center}
\begin{tabular}{|c|c|c|}
\hline
WBS&Engineering Reviews& Data\\
\hline
\hline
1.1.2 &	Beamline  Review&	3-Feb-14\\
\hline
1.2.2	&SVT Design Review	&14-Oct-13\\
\hline
1.5.1	&Muon Design Review&	2-Sep-13\\
\hline
1.10.1&	Installation Review	&21-Jul-14\\
\hline
\hline
\end{tabular}
\end{center}
\label{tb:reviews}
\end{table}%

\subsection{Manpower}

The manpower needed to design, fabricate, assemble, test, install, and commission the HPS is captured in the WBS tables. The HPS Collaboration has the personnel needed to realize this project. 
Beamline design work will be done at JLab by Arne Freyberger and Stepan Stepanyan and at SLAC by Ken Moffeit; engineering at SLAC by Marco Oriunno, Dieter Walz, and Clive Field; fabrication in the JLab shops; and installation by the Hall B crew. 

Engineering for the Ecal is being done by Philippe Rosier at Orsay in consultation with Marco Oriunno. 

Beam diagnostics and slow control will be supported by Nerses Gevorgyan (Yerevan) and Hovanes Egiyan.

 
The Tracker/Vertexer will be engineered and designed by Marco Oriunno, Tim Nelson and Per Hansson, with additional help from Vitaly Fadeyev,Alex Grillo, and Bill Cooper,, all experienced with silicon detector systems. Others at SLAC and UCSC will help with test and assembly, including Matt Graham, Takashi Maruyama, John Jaros,, a post doc, and graduate students Sho Uemura and Omar Moreno.  Matt McCulloch will serve as the  technician at SLAC. 

The Ecal is being designed by the Orsay Group, especially Philippe Rosier, Emmanuel Rindel, Emmanuel Rauly, Raphael Dupre, and Michel Guidal, with participation by the Jlab group, especially Stepan Stepanyan, and F.-X. Girod. Others at JLab and in the collaboration will help in assembly and test of the ECal, especially groups from Norfolk State University (Carlos Salgado), and INFN Genova(Italy). 
 
The SVT DAQ is being done by Haller's group at SLAC, including Gunther Haller, Ryan Herbst, Tung Phan, and Raghuveer Ausoori. SVT Physicists Per Hanssen, Alex Grillo, Vitaliy Fadeyev, and Tim Nelson will collaborate closely. Postdocs and students will help debug, test, and certify DAQ electronics. 

The Ecal Trigger/DAQ work is done in Sergey Boyarinov's group, which supports Hall B activities at JLAB, and with Chris Cuevas' group, which has designed the FADC250. R. Dupre and V. Kubarovsky will collaborate with this group in assembling and testing the electronics, programming the trigger, and integrating it with the Ecal hardware.

 The HPS collaboration is nearly 60 strong, so has adequate manpower for overall installation, commissioning, and data taking.  Simulation work is supported by Maurik Holtrop, Matt Graham, M.Ungaro, and Takashi Maruyama, along with help from students and Norman Graf  and Jeremy McCormiock at SLAC. Data management and storage and computing infrastructure will be overseen by Sergey Boyarinov and Maurik Holtrop and Homer Neal, all very experienced professionals. Analysis and simulation studies have been initiated by Maurik Holtrop, Matt Graham, and Takashi Maruyama. Students are actively being engaged. 
The HPS collaboration is managed by its three spokespersons, Maurik Holtrop, John Jaros, and Stepan Stepanyan and its Executive Committee, which consists of the spokespeople along with Takashi Maruyama, Matt Graham, Tim Nelson, and F-X Girod . Ten working groups supervise the progress of each sub-system. The Project Manager is Marco Oriunno.

\begin{table}[htdp]
\caption{Working groups.}
\begin{center}
\begin{tabular}{|c|c|c|}
\hline
HPS working Groups	& Chair (Deputy)\\
\hline\hline
Beamline	&K. Moffeit (FX Girod)\\
\hline
SVT	&T.Neslon (V.Fedayev)\\
\hline
ECAL	& R. Dupre (S.Stepanyan)\\
\hline
DAQ	 & S. Boiarinov (P.Hansson)\\
\hline
Trigger &	V. Kubarovsky (T.Maruyama)\\
\hline
Slow Control	& H. Egiyan (N. Gevorgyan)\\
\hline
Muon &	K.Griffioen (Y.Gershtein)\\
\hline
Software	& M.Haltrop (S. Uemura)\\
\hline
Analysis &	M. Graham (O. Moreno)\\
\hline
Project Management &	M. Oriunno (S. Stepanyan, J.Jaros)\\
\hline
\end{tabular}
\end{center}
\label{tb:groups}
\end{table}%

\begin{table}[htdp]
\caption{Distribution of resources.}
\begin{center}
\begin{tabular}{c|ccc}
Res.	 &SLAC	&JLAB	&ORSAY\\
\hline\hline
ME	&1,513	&194	&0 \\
MD	&720	&750	&0\\
MT	&1,130	&420	&40\\
EE 	&2,400	&2,352&	800\\
\end{tabular}
\end{center}
\label{tb:resources}
\end{table}%

\begin{table}[htdp]
\caption{Total Engineering Labor (hrs) required for three years. Only SLAC and JLAB charge costs.}
\begin{center}
\begin{tabular}{c|cccc}
 FTE	&FY13	&FY14	&FY15&	TOT\\
 \hline\hline
ME	&0.42	&0.44&	0.00&	0.86\\
MD&	0.31	&0.10&	0.00&	0.41\\
MT&	0.41	&0.24&	0.00	&0.64\\
EE &	1.16	&0.20&	0.00	&1.36\\
ET&	0.02	&0.00&	0.00	&0.02\\
Physicist	&4.22	&3.92	&4.15	&12.24\\
\end{tabular}
\end{center}
\label{tb:engin}
\end{table}%

<<<<<<< HEAD
{\bf {\large Cost Breakdown HPS}

\begin{figure*}[h]
\centering
%\vspace*{-5mm}
\includegraphics[angle=90,width=0.6\textwidth]{cost_schedule/cost_systems_table.jpg} 
\caption{Cost breakdown by sub-systems.}
\label{fig:cost}
\end{figure*}

\begin{figure*}[h]
\centering
%\vspace*{-5mm}
\includegraphics[width=0.9\textwidth]{cost_schedule/cost_systems.jpg} 
\caption{Cost breakdown by sub-systems.}
\label{fig:cost}
\end{figure*}

\begin{figure*}[h]
\centering
%\vspace*{-5mm}
\includegraphics[width=0.9\textwidth]{cost_schedule/spending.jpg} 
\caption{Spending profile (costs after overheads and contingency).}
\label{fig:spending}
\end{figure*}

\begin{figure*}[h]
\centering
%\vspace*{-5mm}
\includegraphics[angle=0,width=0.85\textwidth]{cost_schedule/HPSV70} 
\caption{HPS schedule.}
\label{fig:schedulea}
\end{figure*}




\clearpage

\appendix
%\renewcommand*{\thesection}{\arabic{section}}
\renewcommand*{\thesubsection}{\thesection.\arabic{subsection}}
\renewcommand*{\thesubsubsection}{\thesubsection.\arabic{subsubsection}}

\section{WBS Tables}
\label{sec:wbs}
%WBS tables:

\clearpage
%\begin{figure*}[ht]
%\centering
%\vspace*{-5mm}
%\includegraphics[angle=0,width=\textwidth]{cost_schedule/HPSV470p1} 
%\caption{HPS WBS.}
%\label{fig:schedulea}
%\end{figure*}

\includepdf[pages=1-9,angle=90,scale=0.8]{cost_schedule/HPSV470.pdf}
%\begin{rotate}{90}

\section{Additions and improvements to the HPS setup using non-DOE sources of funding}

While the proposed baseline equipment will be sufficient to carry out proposed measurements, the HPS collaboration is seeking funding from non-DOE sources to improve and enhance capabilities of the HPS detector. Plans for improvements include light monitoring system and large area APDs for ECal.  The HPS collaborators from IPN Orsay have applied for a number of European grants, including European Research Council (ERC) Advanced Grant 2013 to purchase APDs, for manpower costs to replace the old ones, to design and build 
new preamplifier boards, and to assemble and test the ECal with the new modules. The total cost of replacing all ECal APDs is about $500$K\$. 
These grants also include light monitoring system. If successful European funding will cover most of the ECal modifications, and will significantly reduce the support needed from DOE for the more modest upgrades we have proposed above. 

Besides ECal improvements, collaboration intend to add muon detector to the HPS setup. Collaborators from the Collage of William\&Mary (PI Prof. Keith Griffion) togetehr with collaborators from Rutgers University (PI Prof. Yuri Gerstein) and Old Dominion University (PI Prof. Lary Weinstein) submitted an MRI proposal to NSF for the Muon System, requesting $\sim 300$k.  

Below the details of ECal improvements, and the motivation and description of the muon system are presented. 

\subsection{Improvements to ECal}

\begin{enumerate}
\item {\bf Light monitoring system}

For an experiment like HPS, where backgrounds must be well understood and need to be strongly suppressed, the trigger bias must be minimized. In particular, having stable, known, and uniform thresholds at the trigger readout is necessary in order to avoid  bias in the 
event selection. Such uniformity and stability can be achieved with the installation of an online gain monitoring 
system. This system will introduce short light pulses into the front face of the crystals. The crystals already have fiber holders attached, allowing implementation of this system without having to modify the crystals or wrapping. 

Optical fibers will be used to transmit light from a calibration  source to the crystals to test the response of the APDs. The response of the system could change in time because of 
losses in crystal transparency due to radiation damage or because of gain variations of the APDs. 
Such a calibration system has been developed for several experiments (CMS at CERN for instance) with various light sources. The system for the ECal 
will be developed at IPN Orsay during 2013 and in the first half of 2014, and will be ready for installation at JLAB for the commissioning run in the fall of 2014. Each module will have a red and blue mono-color LED light source for monitoring purposes. 
Blue light transmission, corresponding to the domain of the crystal's emission spectrum, is very sensitive to the presence of color centers which are produced by radiation damage. So the blue light source will monitor variations of the response in the main domain of the spectrum.
 %Impurities can anneal at room temperature and such monitoring can be sensitive to increasing of output as well, when the radiation exposure is reduced for a long period of time. 
The response to red light is less sensitive to the color centers,  and so permits monitoring the APD gains more directly. Thus the use of two colors separates gain variations due to the 
APD and electronics from those due to changes in the crystal transparency, and provides clear information on the state of the electronics. 
The main challenge for the system is to guarantee stability at a level of $2\%$. The test of the system will be carried-out at
IPN Orsay, in order to study its efficacy and also to test the radiation resistance of the various materials.

\item {\bf Modifications to the side brackets to accommodate fiber bundles for the light monitoring system} -
A light monitoring system was not used during the test run. While the design of the ECal enclosure was done in such a way that it can 
accommodate optical fibers attached to the front face of crystals, the side plates that hold the crystal frames do not have inlets for the accompanying fiber bundles.
Space is available on the side plates for a straight-forward modification which will allow the addition of a light monitoring system.  
    
\item {\bf New low-voltage power supply} - The existing low voltage power supply is a manually controlled, single output power supply 
that feeds the four ECal motherboards through a custom-made patch panel. The present system cannot control the voltage supplied to preamplifiers at different
parts of the ECal, and controlling or resetting them remotely has proved to be very inconvenient, requiring frequent access to the Hall, especially during commissioning. Newly available low voltage power supplies are much more flexible. The one that is the most suitable for the
ECal APD preamplifiers is the WIENER MPV 8008LD. This power supply is being used at JLAB and the control software exists, so it will be easy to incorporate it into HPS.     

\item {\bf New preamplifiers} - A low noise, low gain preamplifiers will be needed to take advantage of increased signal on the input of FADC after removing the spliter. The impact of the lower noise/threshold system is twofold: first it 
will improve the ECal's energy resolution; and second it will make the ECal sensitive to minimum ionizing particles which pass through the crystals transversely. With sensitivity to cosmic ray muons, which will pass through the ECal transversely when it is installed in HPS, the Ecal crystals can be calibrated for MIPS, and their effective gains balanced.  HPS collaborators from INFN Genova have shown that with such a low noise, low threshold system, the ECal can distinguish the MIP energy deposition from noise, see left plot on Figure \ref{fig:mip10x10}.

%\begin{figure*}[t]
%\includegraphics[scale=0.4]{ecal/MIP_5x5_APD.png}
%\caption{\small{Charge distribution from readout of the HPS calorimeter crystal with Hamamatsu S8664-55 APD, and the new low noise amplifier board. The red line histogram corresponds to the charge distribution for all triggers coming from the scintillators positioned above and below the crystal. The black line shows the distribution for hits in the crystal within $100$ ns of the trigger signal. }}\label{fig:mip5x5}
%\end{figure*}


\item{\bf Possibilities with new APDs} - Installing new APDs on the existing crystals will significantly improve the ECal performance, but doing so is expensive,  so replacement is only being considered if a funding source beyond DOE HEP is secured. Replacing the old $5\times 5$ mm$^2$ 
Hamamatsu S8664-55 APDs with $10\times 10$ mm$^2$, Hamamatsu S8664-1010 will improve two critical characteristics of the present calorimeter modules. First, the new APDs from Hamamatsu have much better performance than the ones which are currently installed. Data from Hamamatsu shows that APDs made from the same wafer have excellent gain uniformity. With $\pm 10\%$ known uniformity 
at the gain of $100$, the required variations in bias voltage are only $\sim 4.5$ V. Even for large samples of APDs (~1300),  the required bias voltage differences are ~50 V, which is half that of the current APDS. The ECal supplies bias voltage to groups of APDs, so with new APDs, with their smaller voltage-gain variations, it will be possible to achieve much better 
uniformity in the response of the calorimeter modules, and consequently better uniformity in trigger thresholds. 

Secondly, the new APDs have a $4$ times larger readout area, ensuring $4$ times more light collection and therefore $4$ times larger signals. This will allow the use of lower gain amplifier modules which in turn will decrease electronic noise. Tests 
performed for another calorimeter, now in production at INFN Genova for JLAB Hall-B, showed that amplifier boards 
with a factor 2 lower gains have a noise level $<5$ MeV. The energy deposition in the HPS PbWO$_4$ crystals  
from minimum ionizing cosmic muons passing transversely to the crystal axis is $\sim 15$ MeV. Moving the energy thresholds 
close to $5$ MeV will allow the MIP peak to be clearly distinguished, and will let the calorimeter  be calibrated and monitored with cosmic muons. The HPS collaborators 
from the INFN group have performed the first tests of the Hamamtsu $10\times 10$ mm$^2$ APDs and a new amplifier board on HPS crystals. In 
Figure \ref{fig:mip10x10}, the charge distribution of a single crystal system is shown for $5\times 5$ mm$^2$ (left) and $10\times 
10$ mm$^2$ (right) APDs. A coincidence signal from scintillator telescopes positioned above and below the module provides the trigger. 
The crystal was positioned horizontally as it would be in HPS, so the cosmic muons would pass through it perpendicular to the crystal axis. The red line 
histogram is for all events triggered by the scintillation telescope and corresponds to the noise. The black line histogram corresponds 
to the charge detected within $100$ ns of the trigger time. The MIP peak is clearly visible and well isolated from the noise 
for the S8664-1010 APD readout. For  the S8664-55 APD, the MIP signal is also seen, but its charge distribution is under the noise peak and it
does not have well defined peak position. Using MIP calibration in conjunction with the light monitoring system will ensure stable and reliable performance of the ECal and the trigger system. As a bonus, the lower noise will allow the use of lower  thresholds and improve the calorimeter's energy resolution. The new amplifier boards have to be designed to work with new APDs. 

\begin{figure*}[t]
\includegraphics[scale=0.37]{ecal/MIP_5x5_APD.png}
\includegraphics[scale=0.37]{ecal/MIP_10x10_APD.png}
\caption{\small{Charge distribution from readout of the HPS calorimeter crystal with Hamamatsu S8664-55 (left) and S8664-1010 
(right) APDs, and the new low noise amplifier board. The red line histogram corresponds to the charge distribution for all triggers 
comming from the scintillators positioned above and below the crystal. The black line shows the distribution for hits in the crystal 
within $100$ ns of the trigger signal. }}\label{fig:mip10x10}
\end{figure*}

\end{enumerate}


\subsection{Muon system}

\label{sec:muon}


The di-muon decay channel of the A$^\prime$ has the advantage of a greatly reduced electromagnetic background.  In this case, the only particle background in a muon counter would come from photoproduction of $\pi^+$ and $\pi^-$ pairs that are not fully stopped in the ECal or absorber.  A muon detector will match geometrical acceptances of the tracker and ECal, and will be about a cubic meter in size. With such geometrical coverage, efficiency of detecting high mass A$^\prime$s in $\mu^+\mu^-$ decay channel will be higher than for $e^+e^-$ decays, see Figure \ref{fig:muacc}. Expected low background and high efficiency, the di-muon final state is an attractive complement to A$^\prime$ search using $e^+e^-$ final state, and will add substantial territory in the mass and coupling parameter space. With muon system, HPS will be the only experiment proposed to date to search for heavy photons in an alternative to $e^+e^-$ decay mode.

\begin{figure*}[!ht]
\includegraphics[scale=0.4]{muon/acc.pdf}
\caption{\small{A$^\prime$ detection efficiency through $\mu^+\mu^-$ (blue) and $e^+e^-$ (red) decay channels as a function of mass for 6.6 GeV beam energy.}}\label{fig:muacc}
\end{figure*}

The muon system can easily be constructed with layers of scintillator hodoscopes sandwiched between iron absorbers, and can be added downstream from the rest of the HPS apparatus.
The number of layers and the thickness of absorbers is defined by the $\pi/\mu$ rejection factor. The schematic design of the muon detector was optimized using the GEANT-3 model for the ECal with added layers of iron and scintillators.  In the simulation, muons and pions in the momentum range of $1$ to $4$ GeV/c first passed through the 16 cm of lead tungstate in the ECal and then entered a muon counter with various total absorber thicknesses (see \cite{HPS_PROP} for details).  Detection efficiencies for pions ($\epsilon_\pi$) and muons ($\epsilon_\mu$) were then calculated as a function of absorber thickness and particle momentum.  For low-energy particles ($< 1.7$ GeV) detection in all four layers of scintillator hodoscopes was not considered. Depending on the momentum, particles were not traced behind the third, fourth or fifth absorber.  
Figure \ref{fig:pmrej} shows the resulting rejection factor $\epsilon_\pi/\epsilon_\mu$.  The right-hand plot shows the dependence of  $\epsilon_\pi/\epsilon_\mu$ on the total thickness of the iron absorber, with the best rejection at about 75 cm.  The right-hand plot shows $\epsilon_\pi/\epsilon_\mu$ for a 75 cm absorber as a function of muon momentum.  The suppression of individual pions by two orders of magnitude will suppress pion pairs by 4 orders of magnitude.  

\begin{figure*}[!ht]
\includegraphics[scale=0.44]{muon/pmrej.pdf}
\includegraphics[scale=0.44]{muon/pmrej4.pdf}
\caption{\small{Pion-muon rejection factor $\epsilon_\pi/\epsilon_\mu$ versus total iron absorber thickness
(left) and versus particle momentum for a 75 cm absorber (right).}}\label{fig:pmrej}
\end{figure*}


\subsubsection{Conceptual Design}

On the basis of these simulations, we have designed a muon detector composed of four iron absorbers (total length of $30+15+15+15=75$ cm) with a double-layer scintillator hodoscope positioned after each absorber. The muon detector will be mounted behind the ECal.  The front face of the first absorber will be at $\sim 180$ cm from the target. Similar to the Ecal, the muon detector will consist of two halves, one above and one below the beam.  This segmentation is necessary in order to
minimize the effects of the ``sheet-of-flame" that multitude of low-energy particles in the horizontal plane, swept into the detector acceptance by the dipole analyzing magnet.
The vertical gap between the first hodoscope layers of the two halves is about $5$ cm. Dimensions of hodoscopes and absorbers are shown in Table \ref{tb:muon}.  Figure \ref{fig:HPS_view2} shows a CAD
drawing of the HPS detector, with the muon system on the right, which includes the 4 absorbers (gray), the vacuum box (light gray) between the upper and lower sections, and the final set of scintillator paddles (red). The ECal is directly upstream from the muon detector, with its crystals shown in yellow.  In front of the ECal is a large gray vacuum flange.  The silicon tracker is represented by red and gray rectangles and  the red point on the left is the target position.  

\begin{table}[htdp]
\caption{Dimensions (in cm) of muon system scintillation hodoscopes (H) and iron absorbers (A). }
\begin{center}
\begin{tabular}{|c|c|c|c|c|}
\hline
&H1&H2&H3&H4\\
\hline
Distance from target& 212&232&252&272\\
Width&112&125&138.5&152\\
Hight&10.5&11.5&12.5&13.5\\
\hline
\end{tabular}

\begin{tabular}{|c|c|c|c|c|}
\hline
&A1&A2&A3&A4\\
\hline
Distance from target& 207&227&247&267\\
Width&108.5&122&135&148.5\\
Hight&10&11&12&13\\
Thickness & 30 & 15& 15 & 15\\
\hline
\end{tabular}
\end{center}
\label{tb:muon}
\end{table}%


\begin{figure*}[!ht]
\includegraphics[scale=0.22]{muon/HPS_view2.png}
\caption{\small{CAD drawing of the HPS detector setup.  From left to right this consists of the target (red dot), the silicon tracker
(gray and red rectangles), the large shielding wall (gray), the ECal lead tungstate crystals (yellow, two shades), the muon counter absorbers
(gray), and the final muon counter scintillators (red, two shades).}}
\label{fig:HPS_view2}
\end{figure*}

%\begin{figure*}[!ht]
%\includegraphics[scale=0.8]{muon/pmrej4.pdf}
%\caption{\small{Pion-muon rejection factor as a function of the iron absorber thickness.}}\label{fig:pmrejp}
%\end{figure*}

\begin{figure*}[!ht]
\includegraphics[scale=0.22]{muon/Muon2b.png}
\caption{\small{Horizontal scintillator configuration for the muon counter. Scintillators are
shown in red and yellow/brown.  The white/gray structure is the vacuum box.  Each hodoscope layer (top
and bottom) contains three long strips, read out on both ends.
}}
\label{fig:Muon2p}
\end{figure*}

For the hodoscopes we plan to use the same extruded scintillator strips with embedded wavelength-shifting fiber and multi-anode phototube readout as was developed for the CLAS Preshower Calorimeter. These scintillator strips are 45 mm x 10 mm in cross section, and can be cut to any lengths and widths can be reduced as needed for the muon counter.  Each strip contains two, long tunnels, created in the original extrusion process, into which wave-length shifting fibers can be inserted.  Each hodoscope will consist of one x and one y plane.  In Figure \ref{fig:HPS_view2} in two shades vertical strips of the last hodoscope plane is shown. Figure \ref{fig:Muon2p} in different shades horizontal counters of hodoscope planes are shown. The horizontally aligned strip will extend over the length of the detector and will be read out on each end.  The upper and lower hodoscopes in each plane will have their own vertically aligned strips, which will be read out on only the outer end.  The inner end is inaccessible because of the vacuum box, but there is no particular advantage to having a double readout on these short (135 mm) strips.  

The system can be instrumented with 256 readout channels, in which case the requisite electronics will 
fit into a single VME crate.  Signal from each channel (PMT) 
will be sent to a FADC.  We intend to borrow the CLAS12 Preshower Calorimeter electronics and HV system.  Similar to ECal, FADCs will be used to construct a muon trigger for the experiment.  In the current design there will be 3 horizontal strips in each of 8 hodoscope planes (24 total) and a total of 208 vertical strips in 8 hodoscope planes.  The number of vertical strips per plane increases slightly with distance from the target to keep a constant angular coverage.  The maximum is 33 per hodoscope in the back plane.

Full Monte Carlo simulations with realistic event rates are currently underway in order to finalize design details of the muon counter.  The crucial issues are the event rates in the scintillators near the beamline (which already has initiated a redesign of the vacuum chamber to reduce background), the target-to-muon-counter tracking resolution and the detection efficiency.  Any changes to the detector as a result are expected to be minor and will not alter the conceptual design presented here.


\clearpage

\section{Simulation Tools}
\label{app:sim}

The simulation tools play a critical role in simulating the background
environment, optimizing the detector setup, and developing the trigger 
and reconstruction strategies. We use GEANT4 and EGS5 to simulate 
electromagnetic interactions. There is generally good agreement 
between these two codes. In particular, no inconsistencies have been 
found on secondary particle yields or energy spectra. However, we have found 
significant disagreements on the angular distributions in the multiple
scattering, bremsstrahlung and pair production processes.  

%\vspace{1cm}
%\noindent
%{\bf Multiple Scattering}
\subsection*{Multiple Scattering Simulation}

EGS5 simulates the electron elastic scattering using the Moli\`{e}re theory 
\cite{moliere} as formulated by Bethe. \cite{bethe}
It is based on a small angle approximation
($\theta \ll$ 1 radian), and the angular distribution approaches asymptotically
to Gaussian at small angles, and to Rutherford's Coulomb scattering function at 
large angles given by, 
\begin{equation}
F(\theta) \sim  { \frac{1} {\left(1-cos\theta + {\frac{\chi^2} {2}}\right)^2}}.
\label{eq:rutherford}
\end{equation}

Instead of using the complex and time consuming Moli\`{e}re's formula,
GEANT4 uses two functions explicitly, Gaussian at small angles and the
Rutherford function Eq.~\ref{eq:rutherford} at large angles with a requirement that these two
functions and their first derivatives are joined continuously. 
GEANT4, however, uses a different power
in the denominator in Eq.~\ref{eq:rutherford} which is close to 2 but not exactly equal to 2 and is 
dependent on the target material and thickness.

Several comparisons have been made in the angular distribution $F(\theta)$ in the
differential cross section $d\sigma=F(\theta)d(cos\theta) d\phi$ for 2.2 GeV electron
scattering from 0.125\% $X_0$ Tungsten target. 
The EGS5 simulation is compared with Moli\`{e}re's analytical formula 
in Figure \ref{appendix:1}(a), demonstrating a good agreement between EGS5 and
the Moli\`{e}re theory.
While the Moli\`{e}re theory is based on a small angle approximation,
the multiple scattering theory developed by Gaudsmit and Saunderson is valid 
for any angle by means of an expansion in Legendre polynomials. \cite{gs}
The validity of the small angle approximation is checked by comparing the 
Moli\`{e}re integral with 
the Goudsmit-Saunderson theory as shown in Figure \ref{appendix:1}(b),
demonstrating that the Moli\`{e}re theory is accurate in the angular region
of the HPS detector. 

\begin{figure*}[ht]
\includegraphics[height=3 in]{appendix/appendix_1-eps-converted-to.pdf}
\caption{\small{ (a) Moli\`{e}re vs. EGS5 \hspace{1 cm} (b) Moli\`{e}re vs. Goudsmit-Saunderson}}
\label{appendix:1}
\end{figure*}

Figure \ref{appendix:2} shows the angular distribution comparison between the GEANT4 
simulation and the Moli\`{e}re integral. 
GEANT4 is in good agreement with the Moli\`{e}re integral up to about 1 mrad, then it 
deviates at larger angles, predicting roughly twice the cross section at 15 mrad, 
where the HPS tracker sensor edge is located.

D. Attwood et al. measured 170 MeV muon angular distributions and compared with 
GEANT4 simulations and the Moli\`{e}re theory. \cite{attwood} They concluded that GEANT4 
simulation over-estimated the scattering tail by about a factor of two, and the data were consistent
with the Moli\`{e}re theory. G. Shen et al.~ \cite{shen} and B. Gottschalk et al.~ \cite{gottschalk}
also showed that the Moli\`{e}re theory was consistent with the measurements on a wide variety of
target materials.

\begin{figure}[ht]
\includegraphics[height= 3 in]{appendix/appendix_2-eps-converted-to.pdf}
\caption{\small{ Moli\`{e}re vs. GEANT4 }}
\label{appendix:2}
\end{figure}

%\vspace{1cm}
%\noindent
%{\bf Angular distributions in the bremsstrahlung and pair production processes}
\subsection*{Angular Distributions}

While GEANT4 and EGS5 are in good agreement in the production rates and the secondary particle
energy spectra, there are significant differences in the angular distribution in the secondary
particles. In EGS5, the angular distributions are sampled from the following differential
cross section for the bremsstrahlung process, \cite{koch}

$$d\sigma(k,\theta_\gamma) = {\frac{4Z^2r_0^2} {137}} {\frac{dk} {k}} ydy\left\{{\frac{16y^2E} 
{(y^2+1)^4E_0}}
 -{\frac{(E_0+E)^2} {(y^2+1)^2E_0^2}}+\left\{{\frac{E_0^2+E^2} {(y^2+1)^2E_0^2}} -
 {\frac{4y^2E} {(y^2+1)^4E_0}}\right\} lnM(y) \right\}, $$

\noindent
where $k$ is photon energy, $\theta_\gamma$ is photon polar angle, $E_0$ and $E$ are initial and final 
electron energy, and

$$y=E_0\theta_\gamma; {\frac{1} {M(y)}} = \left({\frac{k} {2E_0E}}\right)^2 + \left({\frac{Z^{1/3}} {111(y^2+1)}}\right)^2, $$

\noindent
and for the pair production process, \cite{motz}

$${\frac{d\sigma} {dE_\pm d\Omega_\pm}} = {\frac{2\alpha Z^2r_0^2} {\pi}} {\frac{E_\pm^2} {k^3}}
\left\{-{\frac{(E_+-E_-)^2} {(u^2+1)^2}}-{\frac{16u^2E_+E_-} {(u^2+1)^4}} + \left\{ {\frac{E_+^2+E_-^2} 
{(u^2+1)^2}} + {\frac{4u^2E_+E_-} {(u^2+1)^4}} \right\} lnM(u)\right\},$$

\noindent
where $k$ photon energy, $E_\pm$ $e^{\pm}$ energy, $\theta_\pm$ $e^{\pm}$ polar angle, and

$$u=E_\pm\theta_\pm; {\frac{1} {M(u)}} = \left({\frac{k} {2E_+E_-}}\right)^2+\left({\frac{Z^{1/3}} {111(u^2+1)}}\right)^2.$$

\noindent
GEANT4 uses an approximate function to simulate the angular distributions in the 
bremsstrahlung and pair production processes given by

$$ f(u) = C [ue^{-au} + d u e^{-3au}], $$

\noindent
with $u=E_0\theta_\gamma$ for incident electron energy $E_0$ and the polar angle 
$\theta_\gamma$ of the bremsstrahlung photon, and $u=E_{\pm}\theta_{\pm}$ for the pair 
energy $E_\pm$ and polar angle $\theta_\pm$ in the pair production. Since the production angle
is typically $1/\gamma$, GEANT4's approximations are acceptable
for most of the high energy detector simulations. However, GEANT4
simulations are inconsistent with the data in the following two cases in the HPS Test Run:
\begin{itemize}
\item The bremsstrahlung photon angular distribution is too narrow, 
resulting in too few scatters in the collimator.
\item The prediction on the pair angular distribution is too narrow, resulting in 
too few Ecal trigger rates.
\end{itemize}

%\vspace{1cm}
%\noindent
%{\bf Conclusions}
\subsection*{Simulation Tools Setup in HPS}

Because of the inaccuracies in GEANT4 described above the electromagnetic interactions in the target are simulated 
by EGS5, and all the particles that come out of the target are passed on to the HPS detector 
simulation system based on GEANT4.


%\section{Production and Decay of the $A^\prime$}
\label{app:ProdAndDecay}

\def \ap {A^\prime}
\def \map {m_{A^\prime}}
\def \thap {\theta_{A^\prime}}


$\ap$ particles are generated in electron collisions on a fixed target by a process analogous to ordinary photon bremsstrahlung, see Figure \ref{fig:apdiagram}.  This can be reliably estimated in the Weizsäcker-Williams approximation (see [1-4]).  When the incoming electron has energy $E_0$, the differential cross-section to produce an $\ap$ of mass $m_{\ap}$ with energy $E_{\ap}\equiv x E_0$ is 
\begin{equation}
\frac{d\sigma}{dxd\cos{\theta_{\ap}}}\approx \frac{8Z^2\alpha^3\epsilon^2 E_0^2 x}{U^2}\tilde{\chi}\times\left[\left(1-x+\frac{x^2}{2}\right)-\frac{x(1-x)m_{\ap}^2E_0^2x\theta_{\ap}^2}{U^2}\right]
\end{equation}
where Z is the atomic number of the target atoms, $\alpha = 1/137$,  is the angle in the lab frame between the emitted A' and the incoming electron, 
\begin{equation}
U(x,\theta_{\ap})=E_0^2x\theta_{\ap}^2+m_{\ap}^2\frac{1-x}{x}+m_e^2x
\label{eq:u}
\end{equation}
is the virtuality of the intermediate electron in initial-state bremsstrahlung, and  is the Weizsacker-Williams effective photon flux, with an overall factor of  removed.  The form of  and its dependence on the $\ap$ mass, beam energy, and target nucleus are discussed in \cite{Kim:1973he}.  For HPS with $E_0$ = 6.6 GeV, we find $\tilde{\chi}\sim 7 (4, 1)$ for $m_{\ap}$ = 100 (200, 500) MeV/$c^2$.
The above results are valid for 
\begin{equation}
m_e\ll m_{\ap}\ll E_0  , ~~ x\theta_{\ap}^2\ll 1.
\end{equation}
For $m_e\ll m_{\ap}$, the angular integration gives
\begin{equation}
\frac{d\sigma}{dx}\approx \frac{8Z^2\alpha^3\epsilon^2 x}{m_{\ap}^2}\left(1+\frac{x^2}{3(1-x)}\right)\tilde{\chi} .
\end{equation}

Assuming the $\ap$ decays into Standard Model particles rather than exotic, it's boosted lifetime is
\begin{equation}
l_0 \equiv \gamma c\tau \approx \frac{0.8 cm}{N_{eff}} \left(\frac{E_0}{10 GeV}\right)\left(\frac{10^{-4}}{\epsilon}\right)^2\left(\frac{100 MeV}{\map}\right)^2,
\end{equation}
where we have neglected phase-space corrections, and $N_{eff}$ counts the number of available decay channels.  If the $\ap$ couples only to electrons, then $N_{eff}=1$.  If the $\ap$ mixes kinetically with the photon, the $N_{eff}=1$ for $\map < 2m_\mu$ and $2+R(\map)$ for $\map \geq 2 m_\mu$, where \cite{eehadrons}
\begin{equation}
R  =\left. \frac{\sigma(e^+e^-\rarr hadrons)}{\sigma (e^+e^- \rarr \mu^+\mu^-)}\right|_{E=\map} . 
\end{equation} 
For the ranges of $\epsilon$ and $\map$ probed by this experiment, the mean decay length $l_0$ can be prompt or as large as tens of centimeters.  

The total number of $\ap$ produced when $N_e$ electrons scatter in a target of $T\ll 1 $ radiation lengths is
\begin{equation}
N\sim N_e\frac{N_0 X_0}{A}T\frac{Z^2\alpha^2\epsilon^2}{\map^2}\tilde{\chi}\sim N_e C T \frac{\epsilon m_e^2}{\map^2},
\end{equation}
where $X_0$ is the radiation length of the target in g/cm$^2$, $N_0 \approx 6\times 10^{23} mole^{-1}$ is Avogadro's number, and $A$ is the target atomic mass in g/mole.  The numerical factor $C\approx 5$ is logarithmically dependent on the choice of nucleus (at least in the range of masses where the form-factor is only slowly varying) and on $\map$, because, roughly, $X_0 \propto \frac{A}{Z^2}$ (see \cite{Kim:1973he, Essig:2010xa,Bjorken:2009mm}).  For a Coulomb of incident
electrons, the total number of $\ap$s produced is given by
\begin{equation}
N\sim 10^5 \left(\frac{N_e}{1 C}\right)\tilde{\chi}\left(\frac{T}{0.1}\right)\left(\frac{\epsilon}{10^{-4}}\right)^2\left(\frac{100 MeV}{\map}\right)^2.
\end{equation}



% \section{Test Run SVT Performance}
% \label{app:svt}
% \subsection*{SVT Calibration}

% 
%
%   svt_calibrations.tex
%       author: Omar Moreno <omoreno1@ucsc.edu>
%               Per Hansson <phansson@slac.stanford.edu>
%
%

In order to prepare the SVT for real physics data-taking, the SVT was 
calibrated. This involved the extraction of the mean baseline (pedestal),
baseline noise and gain for each of the 12,780 SVT channels. All measurements
were made with the APV25 readout chips configured to their nominal operating
points \cite{Jones:1069892} and all sensors biased to 180 V. The APV25s were
operated in ``mulit-peak'' mode with six samples being readout per trigger.
This, in turn, allowed for the extraction of the $t_0$ and amplitude of the 
signals being read out.

Figure~\ref{fig:pedestal_noise} shows an intensity plot of the pedestals 
along with the readout noise as a function of channel number for a single
hybrid.  The noise was computed by taking the RMS of the gaussian distributed
\begin{figure}[h]
    \begin{center}
    	\includegraphics[width=0.45\textwidth]{test2012/svtperformance/baseline}
    	\includegraphics[width=0.45\textwidth]{test2012/svtperformance/gain}
        \caption{Something ... }
%    	\caption{\small{The baseline across a hybrid (left) and the measured response as a function of 
%	                    input charge (right). The overall shifts in the baseline are calibrated out where distinct edges 
%	                    are associated with the five APV25 chips on the hybrid. The gain shows good linearity up to 
%	                    about three $mip$s.} {\color{red}Should we show noise instead of baseline?}}
	\label{fig:pedestal_noise}
    \end{center}
\end{figure}
pedestals for each of the channels and was observed to be consistently within 
[Find number] ADC counts ( electrons).  One observed feature are the large
noise values for the channels lying near the chip edges.  This has also been 
reported by the CMS collaboration and the cause is still under investigation.

%  Need to rewrite this ...
Another important aspect for the characterization of the SVT is the response and the associated 
gain. Using the APV25 internal calibration circuit a known fixed charge was injected into all 
channels of the which allows for an accurate determination of the response and its 
scaling with input charge, shown in Fig.~\ref{fig:baseline_and_gain}. The gain uniformity was 
within the expected range across chips and modules and show good linearity of charge 
depositions up to about 3 $mip$s. 

All reconstructed hits in an event were used to form clusters of energy 
depositions using a nearest neighbor algorithm. Fig.~\ref{fig:cluster_pulse}
shows the mean pulse shape of each of the hits associated with a track as a 
function of time.  
\begin{figure}[h]
	\includegraphics[width=\textwidth]{test2012/svtperformance/pulseshape_and_landau}
    \caption{Something else ...}
%	\caption{\small{The distribution of cluster amplitudes (left) showing the characteristic Landau 
%	shape and the pulse shape from the six samples readout (right) {\color{red} Remove one of the pulse shapes}. }}
	\label{fig:pulseshape}
\end{figure}
% Should this be included? If so, it needs to be rewritten a little better
The figure also demostrates that  the trigger system, described below, is well 
timed in with the tracker. 
Fig.~\ref{fig:cluster_pulse} shows the MIP response to be [Find value] electrons.
Taking the MIP response, the signal to noise ratio was calculated to be 
approximately 25.5 which is well matched to the expected behavior.

\bibliography{svt_calib}
 


% \subsection*{SVT Hit Time Resolution}
% As discussed in Sec.~\ref{sec:svt}, the APV25 multi-peak readout is crucial to the time stamping on hits
% in the SVT. This, in turn, allows the effective occupancies to be lowered for pattern recognition
% during electron running. 
% Six samples of the APV25 shaper output for each trigger are fitted to an ideal $CR-RC$ function to 
% extract the amplitude and hit time.  The $\chi^2$ distribution of these fits from test run data is as expected
% for four degrees of freedom.
% %\begin{figure}[]
% %	\includegraphics[width=0.6\textwidth]{test2012/svtperformance/apvfit_chisq}
% %	\caption{\small{Histogram of $\chi^2$ values for pulse fits for all channels on a representative sensor. The peak at 2 is consistent with 4 degrees of freedom (2 fit parameters), as expected. Pileup was not considered due to the very low hit rate in 
% %the SVT in this photon beam test. } }
% %	\label{fig:apvfit}
% %\end{figure}
% After clustering hits on a sensor, the hit time for each cluster is computed as the 
% amplitude-weighted average of the channel hit times. Since we have no measurement of the ``true'' hit time, we study the overall SVT hit 
% timing performance using the average of all cluster times in a track as the ``track time,'' and take the
%  residual of the cluster time relative to that. The observed track time, shown in Fig.~\ref{fig:tracktime}, has the expected amount of trigger jitter due to the readout clock and trigger system jitter. After correcting for offsets for each sensor (time-of-flight, clock phase) the RMS 
%  of the final residual distribution is roughly 2.4~ns for each sensor. 
% Because the track time is calculated using the individual hit times, the hit time is positively correlated 
% with the track time; thus the RMS of the residual is slightly smaller than the true time resolution.
% The standard deviation of this residual for $n$-hit tracks where all hits have the same time resolution 
% is reduced by a factor of $\sqrt{(n-1)/n}$; since most of our tracks have 8 clusters, the true time 
% resolution is 2.6 ns. 
% %\begin{figure}[ht]
% %	\includegraphics[width=\textwidth]{test2012/svtperformance/timeres}
% %	\caption{\small{Histogram and Gaussian fit of residual of cluster times for a representative sensor, relative to the track time. Because the cluster times and track time have positive covariance, the true time resolution is slightly larger than the standard deviation shown here.} }
% %	\label{fig:timeres}
% %\end{figure}
% %\begin{figure}[ht]
% %	\includegraphics[width=\textwidth]{test2012/svtperformance/hit_dt}
% %	\caption{\small{} }
% %	\label{fig:hit_dt}
% %\end{figure}
% This is somewhat worse than the $\approx 2$ ns resolution expected 
% %(see Fig.~\ref{fig:timeres}) 
% in 
% Sec.~\ref{sec:performance}, but we believe this discrepancy is due to our fit function. Our pulse 
% shape fit assumes an ideal CR-RC pulse shape; since the actual pulse shape has a slower rise time, 
% there is a systematic pull on the hit time when a hit comes immediately before the APV clock time. 
% This is visible in Figure \ref{fig:timeres_2D} as a shift in the residual at certain values of track time.
% \begin{figure}[ht]
% 	\includegraphics[width=0.7\textwidth]{test2012/svtperformance/timeres_2D}
% 	\caption{\small{Plot of the time residual for a representative sensor vs. the track time. 
% 		The kinks in the horizontal band are caused by the fitter; without them the time resolution (measured by taking the projection of this histogram) would be better.} }
% 	\label{fig:timeres_2D}
% \end{figure}
% Work is in progress to use the actual pulse shape in time reconstruction; this should improve time resolution to that expected from previous studies. 
% Reducing the APV25 pulse shaping time will also improve time resolution.
% In short, these results show that we can achieve time resolution adequate for pileup rejection during electron running.
% %\vspace{1cm}{\bf Tracking algorithms [Matt/Omar]}


% \subsection*{SVT Track Reconstruction}
% 
%
%   trk_performance.tex
%       author: Omar Moreno <omoreno1@ucsc.edu>
%      created: December 4, 2012
%
 
Clustered hits in Si planes within each layer are combined to form
2-dimensional ``stereo hits'' which are used by the tracking algorithm to 
form tracks.  The determination of the probability that a stereo hit is 
formed, or hit efficiency, provides insight as to the performance of each of 
the SVT layers.

In order to obtained the hit efficiencies, tracks were fitted using only 4 of 
SVT layers. The resulting track was then extrapolated to the layer omitted
\begin{figure}[h]
        % TODO: Need to update the plot so that Layer 2 on the bottom doesn't 
        % look terrible. It's probably easiest to just remove the point altogether.
    	\includegraphics[width=0.95\textwidth]{test2012/svtperformance/trk_performance/hit_efficiency_vs_layer.pdf}
    	%\includegraphics[width=0.49\textwidth]{test2012/svtperformance/trk_performance/track_reco_efficiency.pdf}
        \caption{{\small
                    Average hit efficiency, excluding known bad channels, 
                    from all dedicated photon runs as a function of layer
                    number.  
                }} 
	\label{fig:hit_track_efficiency}
\end{figure}
from the fit. If the track was found to lie within the sensitive volume
of the layer, a search for a stereo hit within the layer acceptance was 
conducted.  The hit efficiency was then determined as
\[
    \varepsilon_{\mbox{hit efficiency}} = \frac{\mbox{Tracks with hit on missing layer}}
                                            {\mbox{Tracks within layer acceptance}}.
\]
The hit efficiencies per layer were calculated using all dedicated runs. It must 
be noted that those tracks found to intersect bad channels as well as those that 
lie on the sensor edges were excluded from the calculation. As 
can be seen from Figure~\ref{fig:hit_track_efficiency}, the average hit efficiency
per layer was found to be $\approx$ 99\%. 

%95\% for the top SVT volume and greater than 92\% for the bottom volume.  The 
%larger hit inefficiencies observed for some layers was simply due to the 
%sensors composing the layer having a greater number of noisy channels.

%As mentioned above, the standard pattern recognition algorithm is designed to 
%find tracks using the reconstructed stereo hits.  A set of tracking strategies
%outline which layers should be used by the track finding algorithm along
%with their role (seeding, extend, confirm), and the $\chi^2$ cut imposed 
%on the fit. Any kinematic constraints are also specified within the strategy.
%A detailed account of the tracking algorithm can be found  here~\ref{}.

%% Track reco efficiency is no longer going to be included so remove this
%% section
%In order to determine the efficiency of the track finding algorithm, a Monte
%Carlo sample containing pair produced electrons from photons incident on 
%a 1.6\% $X_0$ gold target was used.  The energy of the pair produced electrons
%ranged from .5 GeV to 5.5 GeV. An electron falling within the detector 
%acceptance was considered findable by the tracking algorithm if it 
%traversed through at least 4 of the SVT layers. The track reconstruction
%efficiency was then determined as
%\[
%    \varepsilon_{\mbox{track reco efficiency}} = \frac{\mbox{Tracks found}}
%                                            {\mbox{Tracks found to be findable}}.
%\]
%The resulting efficiency to find an electron which passes through the detector
%acceptance is shown in Figure~\ref{fig:hit_track_efficiency}. The average track
%reconstruction efficiency was found to be \% with the bulk of the inefficiency
%coming from the $\chi^2$ cut imposed on the fit to the six samples during
%the clustering stage.

All events containing pairs of oppositely charged tracks were fit to a
common vertex using a simple vertexing algorithm which searches for the distance
of closest approach between the two tracks.  The reconstructed vertex position
along the beam axis for both data and Monte Carlo is shown on 
Figure~\ref{fig:vz_position}.
\begin{figure}[h]
    \begin{center}
    	\includegraphics[width=0.60\textwidth]{test2012/svtperformance/trk_performance/zvertex.pdf}
        \caption{  
                    The reconstructed vertex position along the beam axis for
                    both data (blue) and Monte Carlo (red).
                } 
	\label{fig:vz_position}
    \end{center}
\end{figure}
Because of the geometric setup of the SVT, observation of pairs produced by
the incident photon required both electrons to experience a hard scatter
within the target.  This results in a broadening of the Gaussian 
distributed reconstructed vertex position. 



% \subsection*{SVT Test Run DAQ Performance}
% 
%
%   svt_daq.tex
%       author: Omar Moreno <omoreno1@ucsc.edu>
%               Santa Cruz Institute for Particle Physics
%               University of California, Santa Cruz
%      created: November 13, 2012
%

The expected data rates and event sizes for each of the dedicated photon runs
were estimated using a full simulation of the SVT DAQ and compared to observed
values. As discussed in Section~\ref{sec:testrun_daq}, the digitized samples
from three hybrids were received by a single DPM.  The DPM then required that
at least three of the six samples exceeded a threshold of two times the noise
level for that channel.  An additional ``pile-up'' cut requiring that 
(sample 2 $>$ sample 1) or (sample 3 $>$ sample 2) was also applied. This was
meant to eliminate hits arising from the falling edge of previous hits expected
to occur when running at the highest occupancies.
%Signals from the photon run were unaffected by such a cut. 

All samples were placed into their own container along with the 
channel number, hybrid identifier, chip address and DPM identifier. An 
additional layer of encapsulation or bank was used to store all samples 
emerging from a single DPM along with the DPM identifier, the event number
an error bit and hybrid temperatures. A diagram of the container along with
the sizes of each of the elements is shown on Figure~\ref{fig:data_format}.
\begin{figure}[h]
    \begin{center}
    	\includegraphics[width=0.60\textwidth]{test2012/svtperformance/daq/svt_data_format.pdf}
        \caption{
                    SVT data format. Samples readout from three hybrids are 
                    processed by a single FPGA and are placed within a single
                    container or FPGA bank.  An additional layer of 
                    encapsulation is used to store all of the FPGA banks.
                 } 
	\label{fig:data_format}
    \end{center}
\end{figure}
Overall, the container overhead will contribute a total of 326 bytes to an event
with an additional 16 bytes per hit.

The observed occupancy expected for each of the converter thicknesses along with the 
corresponding data rate are shown on Figure~\ref{fig:data_rates}.
\begin{figure}[h]
    \begin{center}
    	\includegraphics[width=0.49\textwidth]{test2012/svtperformance/daq/n_dead_channels.pdf}
    	\includegraphics[width=0.49\textwidth]{test2012/svtperformance/daq/data_rates.pdf}
        \caption{
                    The plot on the left shows the percentage of bad/noisy channels observed
                    during each of the runs (green) along with the number of noisy/misconfigured 
                    readout chips (blue).  Most of the noisy channels present during runs were
                    due to misconfigured chips.  The  plot on the right shows the occupancies (blue)
                    and data rates (black) observed for each of the targets used.  Once all noisy channels
                    are masked, the observed data rates agree well with those predicted by the
                    SVT readout simulation. 
                } 
	\label{fig:data_rates}
    \end{center}
\end{figure}
The data rates observed during the test run were
much higher than expected.  This can be attributed to a known
noisy sensor and a few noisy chips which appeared during certain runs as can be seen
on Figure~\ref{fig:data_rates}.  The causes
of both these issues are now well understood and will be resolved for future running.

%Table~\ref{table:sim_rates}.
%\begin{table}[h]
%    \scalebox{0.9}{
%    \begin{tabular}{ c | c | c | c }
%    \hline
%
%    Converter Thickness (\%$X_0$) & Sim Occupancy (\%)  & Sim Event Size (kB) &   Sim Data Rate (Mb/s) \\      
%    \hline 
%   1.6                           & .438                & 1.22                &   2.07                 \\
%   0.45                          & .293                & .93                 &   .53                  \\
%    0.18                          & .118                & .56                 &   .24                  \\ 
%    \end{tabular} } 
%    \caption{Occupancy, event size and resulting data rate expected for each of the three 
%             converter thicknesses used in the test run.}
%    \label{table:sim_rates}
%\end{table}

A better comparison between simulated and observed data rates can be obtained
by masking out all known noisy channels found during the commissioning of the 
SVT along with the noisy chips.  The resulting occupancies and data rates are also shown on 
Figure~\ref{fig:data_rates}. The simulated occupancies shown include a contribution
of 0.02\% (3 hits) due to noise and the data rates are estimated using the trigger
rates observed during each of the dedicated photon runs.  As can be seen from the
figure, the occupancies and data rates after most bad channels are masked are well
in agreement with those predicted by Monte Carlo.
%Table~\ref{table:observed_rates} list the observed occupancy, event sizes and 
%\begin{table}[h]
%    \scalebox{0.9}{ 
%        \begin{tabular}{ c | c | c | c }
%            \hline
%            Converter Thickness (\%$X_0$)   & Obs. Occupancy (\%) & Obs. Event Size (kB) & Obs. Data Rate (Mb/s) \\
%            \hline
%            1.6                             & 1.03                & 2.43                 & 4.12                  \\
%            0.45                            & 1.22                & 2.82                 & 1.61                  \\
%            0.18                            & 1.23                & 2.84                 & 1.21                  \\
%        \end{tabular}
%    }
%    \caption{Occupancy, event size and resulting data rate observed for each of the three 
%             converter thicknesses used in the test run.}
%    \label{table:observed_rates}
%\end{table}    


\section{Test Run ECal Calibration}
\label{sec:ecal_calib}

The noise and pedestal of the readout chain are calibrated by running the ECal readout in a mode where the preamplifier output is sampled every 4~ns in a time window of 100 samples: by looking at a part of the window before the hit, we calibrate the readout channel.

We calibrate gain of the individual ECal channels using the SVT measurement of track momentum and comparison to Monte Carlo simulation. 
% The ratio of cluster energy to track momentum is calculated both for Monte Carlo simulation and test run data at each point in the ECal, and we find the value of gain for each channel that brings the two into agreement.
We disable all SVT and ECal channels in the simulation that were inoperable or noisy in the test run, so any efficiency or bias effects that affect the real data should be reflected in the simulation as well; then we use a formula to compute the ``weighted E/p'' for a crystal, representing the average E/p for clusters that include the crystal: $\frac{\sum_j w_{j,i}}{\sum_j\frac{P_j}{E_j}w_{j,i}}$, and iteratively adjust the gains until the weighted E/p is equal for test run data and simulation.

% The calibrated gains are corrected by the ratio between the weighted E/p values from Monte Carlo and real data.
% The E/p in Monte Carlo data is also affected by the gain because the trigger thresholds change, so both Monte Carlo and data reconstruction are rerun with each iteration of gain calibration.
% It takes up to 4 iterations for the gains to stabilize; the final values are shown in Figure \ref{fig:gains}.
\begin{figure}[ht]
	\includegraphics[width=0.45\textwidth]{test2012/ecalperformance/ecalgainplots_corr_sim}
	\includegraphics[width=0.45\textwidth]{test2012/ecalperformance/gains}
	\caption{\small{Weighted E/p from Monte Carlo simulation (left), calibrated values of gain in units of MeV per ADC count (right).}}
	\label{fig:gains}
\end{figure}

These gains can then be used to convert from ADC counts in a channel to the energy deposited into that ECal crystal.
The other information needed to find the energy of an incident particle is the sampling fraction---the ratio of energy read out from crystals to energy of an incident particle.
The conventional sampling fraction---the fraction of incident energy that is deposited in crystals---is approximately 0.9 for our ECal, and less at edges.
For our readout, there is additional energy lost because crystals under the readout threshold are not read out.
The weighted E/p used in calibration (see Figure \ref{fig:gains}) is an approximate measurement of sampling fraction, but the sampling fraction is energy-dependent because of the effect of readout threshold. 
A full computation of sampling fraction can be done using simulation.


\begin{thebibliography}{99}

%Intro%\cite{PAC37Proposal}

\bibitem{HPS_PROP} A. Grillo {\it et al.} [HPS Collaboration], HPS Proposal to JLab PAC37 PR-11-006,
 http://www.jlab.org/exp\_prog/PACpage/PAC37/proposals/Proposals/

\bibitem{HPS_tPROP} A. Grillo {\it et al.} [HPS Collaboration], HPS Test Run Proposal to DOE, 
https://confluence.slac.stanford.edu/display/hpsg/Project+Overview

\bibitem{HPS_PROP_UPD} P. Hansson {\it et al.} [HPS Collaboration], HPS Update PAC 39, 
https://confluence.slac.stanford.edu/display/hpsg/Project+Overview

%motivation
%\begin{thebibliography}{100}
%\cite{Hewett:2012ns}
\bibitem{Hewett:2012ns} 
  J.~L.~Hewett, H.~Weerts, R.~Brock, J.~N.~Butler, B.~C.~K.~Casey, J.~Collar, A.~de Govea and R.~Essig {\it et al.}, arXiv:1205.2671 [hep-ex].
  %``Fundamental Physics at the Intensity Frontier,''
  
  %%CITATION = ARXIV:1205.2671;%%

\bibitem{Kamionkowski:2010mi} 
  M.~Kamionkowski, S.~M.~Koushiappas and M.~Kuhlen,
  %``Galactic Substructure and Dark Matter Annihilation in the Milky Way Halo,''
  Phys.\ Rev.\ D {\bf 81}, 043532 (2010)
  [arXiv:1001.3144 [astro-ph.GA]].
  %%CITATION = ARXIV:1001.3144;%%

\bibitem{Dark2012} 
  ``Dark2012: Dark Forces at Accelerators'', http://www.lnf.infn.it/conference/dark/index.php


%\cite{Essig:2010xa}
\bibitem{Essig:2010xa} 
  R.~Essig, P.~Schuster, N.~Toro and B.~Wojtsekhowski,
  %``An Electron Fixed Target Experiment to Search for a New Vector Boson A' Decaying to e+e-,''
  JHEP {\bf 1102}, 009 (2011)
  [arXiv:1001.2557 [hep-ph]].

%\cite{Abrahamyan:2011gv}
\bibitem{Abrahamyan:2011gv} 
  S.~Abrahamyan {\it et al.}  [APEX Collaboration],
  %``Search for a New Gauge Boson in Electron-Nucleus Fixed-Target Scattering by the APEX Experiment,''
  Phys.\ Rev.\ Lett.\  {\bf 107}, 191804 (2011)
  [arXiv:1108.2750 [hep-ex]].

%\cite{Merkel:2011ze}
\bibitem{Merkel:2011ze} 
  H.~Merkel {\it et al.}  [A1 Collaboration],
  %``Search for Light Gauge Bosons of the Dark Sector at the Mainz Microtron,''
  Phys.\ Rev.\ Lett.\  {\bf 106}, 251802 (2011)
  [arXiv:1101.4091 [nucl-ex]].

%\cite{Freytsis:2009bh}
\bibitem{Freytsis:2009bh}
M.~Freytsis, G.~Ovanesyan and J.~Thaler,
%``Dark Force Detection in Low Energy E-P Collisions,''
JHEP {\bf 1001} (2010) 111
[arXiv:0909.2862 [hep-ph]].

%\cite{Galison:1983pa}
\bibitem{Galison:1983pa}
P.~Galison and A.~Manohar,
%``Two Z's Or Not Two Z's?,''
Phys.\ Lett.\ B {\bf 136} (1984) 279.
%%CITATION = PHLTA,B136,279;%%

%\cite{Goodsell:2009xc}
\bibitem{Goodsell:2009xc}
 M.~Goodsell, J.~Jaeckel, J.~Redondo and A.~Ringwald,
% ``Naturally Light Hidden Photons in LARGE Volume String Compactifications,''
 JHEP {\bf 0911}, 027 (2009)
 [arXiv:0909.0515 [hep-ph]].
 %%CITATION = JHEPA,0911,027;%%
 
 
%\cite{Cicoli:2011yh}
\bibitem{Cicoli:2011yh}
  M.~Cicoli, M.~Goodsell, J.~Jaeckel and A.~Ringwald,
 % ``Testing String Vacua in the Lab: From a Hidden CMB to Dark Forces in Flux Compactifications,''
  JHEP\ {\bf 1107}, 114  (2011)
  [arXiv:1103.3705 [hep-th]].
  %%CITATION = JHEPA,1107,114;%%

%\cite{Goodsell:2011wn}
\bibitem{Goodsell:2011wn}
 M.~Goodsell, S.~Ramos-Sanchez and A.~Ringwald,
 %``Kinetic Mixing of U(1)s in Heterotic Orbifolds,''
 JHEP {\bf 1201}, 021 (2012)
 [arXiv:1110.6901 [hep-th]].
 %%CITATION = ARXIV:1110.6901;%%

%\cite{Essig:2009nc}
 \bibitem{Essig:2009nc}
R.~Essig, P.~Schuster and N.~Toro,
%``Probing Dark Forces and Light Hidden Sectors at Low-Energy $e^+e^-$ Colliders,''
Phys.\ Rev.\ D {\bf 80} (2009) 015003
[%arXiv:0903.3941 [hep-ph]].
%%CITATION = PHRVA,D80,015003;%%

%\cite{Goodsell:2010ie}
\bibitem{Goodsell:2010ie} 
  M.~Goodsell and A.~Ringwald,
%  ``Light hidden-sector U(1)s in string compactifications,''
  Fortsch.\ Phys.\  {\bf 58}, 716 (2010)
  [arXiv:1002.1840 [hep-th]].
  %%CITATION = ARXIV:1002.1840;%%
 
%\cite{NSF-ITP-84-170}
\bibitem{NSF-ITP-84-170} 
  P.~Candelas, G.~T.~Horowitz, A.~Strominger and E.~Witten,
%  ``Vacuum Configurations for Superstrings,''
  Nucl.\ Phys.\ B\ {\bf 258}, 46  (1985).
  %%CITATION = NUPHA,B258,46;%% 

%\cite{PRINT-86-0084 (PRINCETON)}
\bibitem{PRINT-86-0084 (PRINCETON)} 
  E.~Witten,
%  ``New Issues in Manifolds of SU(3) Holonomy,''
  Nucl.\ Phys.\ B\ {\bf 268}, 79  (1986).
  %%CITATION = NUPHA,B268,79;%%

%\cite{Andreas:2011in}
\bibitem{Andreas:2011in}
 S.~Andreas, M.~D.~Goodsell and A.~Ringwald,
% ``Dark matter and Dark Forces from a supersymmetric hidden sector,''
 arXiv:1109.2869 [hep-ph].
 %%CITATION = ARXIV:1109.2869;%%

%\cite{arXiv:1002.0329}
\bibitem{arXiv:1002.0329} 
  J.~Jaeckel and A.~Ringwald,
 % ``The Low-Energy Frontier of Particle Physics,''
  Ann.\ Rev.\ Nucl.\ Part.\ Sci.\ \ {\bf 60}, 405  (2010)
  [arXiv:1002.0329 [hep-ph]].
  %%CITATION = ARNUA,60,405;%%

%\cite{Fayet:2007ua}
\bibitem{Fayet:2007ua}
P.~Fayet,
%``U-Boson Production in E+ E- Annihilations, Psi and Upsilon Decays, and Light Dark Matter,''
Phys.\ Rev.\ D {\bf 75} (2007) 115017
[arXiv:hep-ph/0702176].
%%CITATION = PHRVA,D75,115017;%%

%\cite{Cheung:2009qd}
\bibitem{Cheung:2009qd}
C.~Cheung, J.~T.~Ruderman, L.~T.~Wang and I.~Yavin,
%``Kinetic Mixing as the Origin of Light Dark Scales,''
Phys.\ Rev.\ D {\bf 80} (2009) 035008
[arXiv:0902.3246 [hep-ph]].
%%CITATION = PHRVA,D80,035008;%%

%\cite{ArkaniHamed:2008qp}
\bibitem{ArkaniHamed:2008qp}
  N.~Arkani-Hamed and N.~Weiner,
  %``LHC Signals for a SuperUnified Theory of Dark Matter,''
  JHEP {\bf 0812}, 104 (2008)
  [arXiv:0810.0714 [hep-ph]].
  %%CITATION = JHEPA,0812,104;%%


%\cite{Morrissey:2009ur}
\bibitem{Morrissey:2009ur}
D.~E.~Morrissey, D.~Poland and K.~M.~Zurek,
%``Abelian Hidden Sectors at a Gev,''
JHEP {\bf 0907} (2009) 050
[arXiv:0904.2567 [hep-ph]].
%%CITATION = JHEPA,0907,050;%%

%\cite{Andreas:2012mt}
\bibitem{Andreas:2012mt}
S.~Andreas, C.~Niebuhr and A.~Ringwald,
%``New Limits on Hidden Photons from Past Electron Beam Dumps,''
Phys.\ Rev.\ D {\bf 86} (2012) 095019
[arXiv:1209.6083 [hep-ph]].
%%CITATION = ARXIV:1209.6083;%%

%\cite{Davier:1989wz}
\bibitem{Davier:1989wz}
M.~Davier and H.~Nguyen Ngoc,
%``An Unambiguous Search for a Light Higgs Boson,''
Phys.\ Lett.\ B {\bf 229} (1989) 150.
%%CITATION = PHLTA,B229,150;%%

%\cite{Konaka:1986cb}
\bibitem{Konaka:1986cb}
A.~Konaka, K.~Imai, H.~Kobayashi, A.~Masaike, K.~Miyake, T.~Nakamura, N.~Nagamine and N.~Sasao {\it et al.},
%``Search for Neutral Particles in Electron Beam Dump Experiment,''
Phys.\ Rev.\ Lett.\ {\bf 57} (1986) 659.
%%CITATION = PRLTA,57,659;%%

%\cite{Blumlein:2011mv}
\bibitem{Blumlein:2011mv}
J.~Blumlein and J.~Brunner,
``New Exclusion Limits for Dark Gauge Forces from Beam-Dump Data,''
Phys.\ Lett.\ B {\bf 701} (2011) 155
[arXiv:1104.2747 [hep-ex]].
%%CITATION = ARXIV:1104.2747;%%

%\cite{Davoudiasl:2012ig}
\bibitem{Davoudiasl:2012ig}
H.~Davoudiasl, H.~-S.~Lee and W.~J.~Marciano,
``Dark Side of Higgs Diphoton Decays and Muon G-2,''
Phys.\ Rev.\ D {\bf 86} (2012) 095009
[arXiv:1208.2973 [hep-ph]].
%%CITATION = ARXIV:1208.2973;%%


\bibitem{endo:g2e}
M.~Endo, K.~ Hamaguchi, and G.~ Mishima, [arXiv:1209.2558 [hep-ph]]. 

%\cite{Bjorken:2009mm}
\bibitem{Bjorken:2009mm} 
  J.~D.~Bjorken, R.~Essig, P.~Schuster and N.~Toro,
  %``New Fixed-Target Experiments to Search for Dark Gauge Forces,''
  Phys.\ Rev.\ D {\bf 80}, 075018 (2009)
  [arXiv:0906.0580 [hep-ph]].

\bibitem{Bjorken:1988as}
J.~D.~Bjorken {\it et al.},
%``Search for Neutral Metastable Penetrating Particles Produced in the SLAC Beam Dump,''
Phys.\ Rev.\ D {\bf 38} (1988) 3375.
%%CITATION = PHRVA,D38,3375;%%

%\cite{Riordan:1987aw}
\bibitem{Riordan:1987aw}
E.~M.~Riordan {\it et al.},
%``A Search for Short Lived Axions in an Electron Beam Dump Experiment,''
Phys.\ Rev.\ Lett.\ {\bf 59} (1987) 755.
%%CITATION = PRLTA,59,755;%%

%\cite{Bross:1989mp}
\bibitem{Bross:1989mp}
A.~Bross, M.~Crisler, S.~H.~Pordes, J.~Volk, S.~Errede and J.~Wrbanek,
%``A Search for Shortlived Particles Produced in an Electron Beam Dump,''
Phys.\ Rev.\ Lett.\ {\bf 67} (1991) 2942.
%%CITATION = PRLTA,67,2942;%%


%\cite{Pospelov:2008zw}
\bibitem{Pospelov:2008zw}
M.~Pospelov,
%``Secluded U(1) Below the Weak Scale,''
Phys.\ Rev.\ D {\bf 80} (2009) 095002
[arXiv:0811.1030 [hep-ph]].
%%CITATION = PHRVA,D80,095002;%%

%\cite{Collaboration:2011zc}
\bibitem{Collaboration:2011zc}
KLOE-2~Collaboration,
%``Search for a Vector Gauge Boson in Phi Meson Decays with the KLOE Detector,''
Phys.\ Lett.\ B {\bf 706} (2012) 251
[arXiv:1110.0411 [hep-ex]].
%%CITATION = PHLTA,B706,251;%%

%\cite{Reece:2009un}
\bibitem{Reece:2009un}
M.~Reece and L.~T.~Wang,
%``Searching for the Light Dark Gauge Boson in Gev-Scale Experiments,''
JHEP {\bf 0907} (2009) 051
[arXiv:0904.1743 [hep-ph]].
%%CITATION = JHEPA,0907,051;%%

\bibitem{Aubert:2009cp}
  B.~Aubert {\it et al.}  [BABAR Collaboration], Phys.\ Rev.\ Lett.\  {\bf 103}, 081803 (2009) [arXiv:0905.4539 [hep-ex]].

%\cite{Dent:2012mx}
%%\bibitem{Dent:2012mx}
%%J.~B.~Dent, F.~Ferrer and L.~M.~Krauss,
%``Constraints on Light Hidden Sector Gauge Bosons from Supernova Cooling,''
%%arXiv:1201.2683 [astro-ph.CO].
%%CITATION = ARXIV:1201.2683;%%

%%%%%%%%%%%%%%%%%%%%%%%%%%%%%%%%%%%%
%%%%%%%%%%%%%%%%%%%%%%%%%%%%%%%%%%%%

\bibitem{LambdaCDMData} 
  E.~Komatsu {\it et al.}  [WMAP Collaboration],
  ``Seven-Year Wilkinson Microwave Anisotropy Probe (WMAP) Observations: Cosmological Interpretation,''
  Astrophys.\ J.\ Suppl.\  {\bf 192}, 18 (2011)
  [arXiv:1001.4538 [astro-ph.CO]];
  %%CITATION = ARXIV:1001.4538;%%
%\bibitem{Eisenstein:2005su} 
  D.~J.~Eisenstein {\it et al.}  [SDSS Collaboration],
  %``Detection of the baryon acoustic peak in the large-scale correlation function of SDSS luminous red galaxies,''
  Astrophys.\ J.\  {\bf 633}, 560 (2005)
  [astro-ph/0501171];
  %%CITATION = ASTRO-PH/0501171;%%
%\bibitem{Perlmutter:1998np} 
  S.~Perlmutter {\it et al.}  [Supernova Cosmology Project Collaboration],
  %``Measurements of Omega and Lambda from 42 high redshift supernovae,''
  Astrophys.\ J.\  {\bf 517}, 565 (1999)
  [astro-ph/9812133];
  %%CITATION = ASTRO-PH/9812133;%%
%\bibitem{Riess:1998cb} 
  A.~G.~Riess {\it et al.}  [Supernova Search Team Collaboration],
  %``Observational evidence from supernovae for an accelerating universe and a cosmological constant,''
  Astron.\ J.\  {\bf 116}, 1009 (1998)
  [astro-ph/9805201];
  %%CITATION = ASTRO-PH/9805201;%%
%\bibitem{Kowalski:2008ez} 
  M.~Kowalski {\it et al.}  [Supernova Cosmology Project Collaboration],
  %``Improved Cosmological Constraints from New, Old and Combined Supernova Datasets,''
  Astrophys.\ J.\  {\bf 686}, 749 (2008)
  [arXiv:0804.4142 [astro-ph]].
  %%CITATION = ARXIV:0804.4142;%%

%\cite{Adriani:2008zr}
\bibitem{Adriani:2008zr} 
  O.~Adriani {\it et al.}  [PAMELA Collaboration],
  %``An anomalous positron abundance in cosmic rays with energies 1.5-100 GeV,''
  Nature {\bf 458}, 607 (2009)
  [arXiv:0810.4995 [astro-ph]].
  %%CITATION = ARXIV:0810.4995;%%

%\cite{Ackermann:2010ij}
\bibitem{Ackermann:2010ij} 
  M.~Ackermann {\it et al.}  [Fermi LAT Collaboration],
  %``Fermi LAT observations of cosmic-ray electrons from 7 GeV to 1 TeV,''
  Phys.\ Rev.\ D {\bf 82}, 092004 (2010)
  [arXiv:1008.3999 [astro-ph.HE]].
  %%CITATION = ARXIV:1008.3999;%%

%\cite{Chang:2008aa} 
\bibitem{Chang:2008aa} 
  J.~Chang, J.~H.~Adams, H.~S.~Ahn, G.~L.~Bashindzhagyan, M.~Christl, O.~Ganel, T.~G.~Guzik and J.~Isbert {\it et al.},
  %``An excess of cosmic ray electrons at energies of 300-800 GeV,''
  Nature {\bf 456}, 362 (2008).
  %%CITATION = NATUA,456,362;%%

%\cite{Aharonian:2008aa}
\bibitem{Aharonian:2008aa} 
  F.~Aharonian {\it et al.}  [H.E.S.S. Collaboration],
  %``The energy spectrum of cosmic-ray electrons at TeV energies,''
  Phys.\ Rev.\ Lett.\  {\bf 101}, 261104 (2008)
  [arXiv:0811.3894 [astro-ph]].
  %%CITATION = ARXIV:0811.3894;%%
%\cite{Aharonian:2009ah}

\bibitem{Aharonian:2009ah} 
  F.~Aharonian {\it et al.}  [H.E.S.S. Collaboration],
  %``Probing the ATIC peak in the cosmic-ray electron spectrum with H.E.S.S,''
  Astron.\ Astrophys.\  {\bf 508}, 561 (2009)
  [arXiv:0905.0105 [astro-ph.HE]].
  %%CITATION = ARXIV:0905.0105;%%

%\cite{Adriani:2011xv}
\bibitem{Adriani:2011xv} 
  O.~Adriani {\it et al.}  [PAMELA Collaboration],
  %``The cosmic-ray electron flux measured by the PAMELA experiment between 1 and 625 GeV,''
  Phys.\ Rev.\ Lett.\  {\bf 106}, 201101 (2011)
  [arXiv:1103.2880 [astro-ph.HE]].
  %%CITATION = ARXIV:1103.2880;%%

%\cite{FermiLAT:2011ab}
\bibitem{FermiLAT:2011ab} 
  M.~Ackermann {\it et al.}  [The Fermi LAT Collaboration],
  %``Measurement of separate cosmic-ray electron and positron spectra with the Fermi Large Area Telescope,''
  Phys.\ Rev.\ Lett.\  {\bf 108}, 011103 (2012)
  [arXiv:1109.0521 [astro-ph.HE]].
  %%CITATION = ARXIV:1109.0521;%%

\bibitem{AMS2:2013} 
 M. Aguilar {\it et al.} [AMS Collaboration]
  %``First Result from the Alpha Magnetic Spectrometer on the International Space Station: Precision Measurement of the Positron Fraction in Primary Cosmic Rays of 0.5�350 GeV,''
  Phys.\ Rev.\ Lett.\  {\bf 110}, 141102 (2013). 

%\cite{Cirelli:2008pk}
\bibitem{Cirelli:2008pk} 
  M.~Cirelli, M.~Kadastik, M.~Raidal and A.~Strumia,
  %``Model-independent implications of the e+-, anti-proton cosmic ray spectra on properties of Dark Matter,''
  Nucl.\ Phys.\ B {\bf 813}, 1 (2009)
  [arXiv:0809.2409 [hep-ph]].
  %%CITATION = ARXIV:0809.2409;%%
%\cite{Cholis:2008qq}
\bibitem{Cholis:2008qq} 
  I.~Cholis, D.~P.~Finkbeiner, L.~Goodenough and N.~Weiner,
  %``The PAMELA Positron Excess from Annihilations into a Light Boson,''
  JCAP {\bf 0912}, 007 (2009)
  [arXiv:0810.5344 [astro-ph]].
  %%CITATION = ARXIV:0810.5344;%%
%\cite{Cholis:2008wq}
\bibitem{Cholis:2008wq} 
  I.~Cholis, G.~Dobler, D.~P.~Finkbeiner, L.~Goodenough and N.~Weiner,
  %``The Case for a 700+ GeV WIMP: Cosmic Ray Spectra from ATIC and PAMELA,''
  Phys.\ Rev.\ D {\bf 80}, 123518 (2009)
  [arXiv:0811.3641 [astro-ph]].
  %%CITATION = ARXIV:0811.3641;%%

\bibitem{Slatyer:2011kg} 
  T.~R.~Slatyer, N.~Toro and N.~Weiner,
  %``The Effect of Local Dark Matter Substructure on Constraints in Sommerfeld-Enhanced Models,''
  arXiv:1107.3546 [hep-ph].
  %%CITATION = ARXIV:1107.3546;%%



%\cite{Adriani:2008zq}
\bibitem{Adriani:2008zq} 
  O.~Adriani, G.~C.~Barbarino, G.~A.~Bazilevskaya, R.~Bellotti, M.~Boezio, E.~A.~Bogomolov, L.~Bonechi and M.~Bongi {\it et al.},
  %``A new measurement of the antiproton-to-proton flux ratio up to 100 GeV in the cosmic radiation,''
  Phys.\ Rev.\ Lett.\  {\bf 102}, 051101 (2009)
  [arXiv:0810.4994 [astro-ph]].
  %%CITATION = ARXIV:0810.4994;%%

\bibitem{Ackermann:2011wa} 
  M.~Ackermann {\it et al.}  [Fermi-LAT Collaboration],
  %``Constraining Dark Matter Models from a Combined Analysis of Milky Way Satellites with the Fermi Large Area Telescope,''
  Phys.\ Rev.\ Lett.\  {\bf 107}, 241302 (2011)
  [arXiv:1108.3546 [astro-ph.HE]].
  %%CITATION = ARXIV:1108.3546;%%

\bibitem{DiffuseGalactic}
%\bibitem{Kistler:2009xf} 
  M.~D.~Kistler and J.~M.~Siegal-Gaskins,
  %``Gamma-ray signatures of annihilation to charged leptons in dark matter substructure,''
  Phys.\ Rev.\ D {\bf 81}, 103521 (2010)
  [arXiv:0909.0519 [astro-ph.HE]].
  %%CITATION = ARXIV:0909.0519;%%
%\cite{Abazajian:2010zb}
%\bibitem{Abazajian:2010zb} 
  K.~N.~Abazajian, S.~Blanchet and J.~P.~Harding,
  %``Current and Future Constraints on Dark Matter from Prompt and Inverse-Compton Photon Emission in the Isotropic Diffuse Gamma-ray Background,''
  Phys.\ Rev.\ D {\bf 85}, 043509 (2012)
  [arXiv:1011.5090 [hep-ph]].
  %%CITATION = ARXIV:1011.5090;%%


\bibitem{Papucci:2009gd} 
  M.~Papucci and A.~Strumia,
  %``Robust implications on Dark Matter from the first FERMI sky gamma map,''
  JCAP {\bf 1003}, 014 (2010)
  [arXiv:0912.0742 [hep-ph]].
  %%CITATION = ARXIV:0912.0742;%%

%\cite{Hutsi:2010ai}
\bibitem{Hutsi:2010ai} 
  G.~Hutsi, A.~Hektor and M.~Raidal,
  %``Implications of the Fermi-LAT diffuse gamma-ray measurements on annihilating or decaying Dark Matter,''
  JCAP {\bf 1007}, 008 (2010)
  [arXiv:1004.2036 [astro-ph.HE]].
  %%CITATION = ARXIV:1004.2036;%%

%\cite{Zavala:2011tt}
\bibitem{Zavala:2011tt} 
  J.~Zavala, M.~Vogelsberger, T.~R.~Slatyer, A.~Loeb and V.~Springel,
  %``The cosmic X-ray and gamma-ray background from dark matter annihilation,''
  Phys.\ Rev.\ D {\bf 83}, 123513 (2011)
  [arXiv:1103.0776 [astro-ph.CO]].
  %%CITATION = ARXIV:1103.0776;%%

%\bibitem{extragalactic}
%\cite{Abdo:2010dk}
%\bibitem{Abdo:2010dk} 
%  A.~A.~Abdo {\it et al.}  [Fermi-LAT Collaboration],
  %``Constraints on Cosmological Dark Matter Annihilation from the Fermi-LAT Isotropic Diffuse Gamma-Ray Measurement,''
 % JCAP {\bf 1004}, 014 (2010)
  %[arXiv:1002.4415 [astro-ph.CO]].
  %%CITATION = ARXIV:1002.4415;%%
%\bibitem{Zavala:2011tt} 
 % J.~Zavala, M.~Vogelsberger, T.~R.~Slatyer, A.~Loeb and V.~Springel,
  %``The cosmic X-ray and gamma-ray background from dark matter annihilation,''
  %Phys.\ Rev.\ D {\bf 83}, 123513 (2011)
  %[arXiv:1103.0776 [astro-ph.CO]].
  %%CITATION = ARXIV:1103.0776;%%

%\cite{Huang:2011xr}
\bibitem{Huang:2011xr} 
  X.~Huang, G.~Vertongen and C.~Weniger,
  %``Probing Dark Matter Decay and Annihilation with Fermi LAT Observations of Nearby Galaxy Clusters,''
  JCAP {\bf 1201}, 042 (2012)
  [arXiv:1110.1529 [hep-ph]].
  %%CITATION = ARXIV:1110.1529;%%

\bibitem{CMBrefs}
%\cite{Galli:2009zc}
%\bibitem{Galli:2009zc} 
  S.~Galli, F.~Iocco, G.~Bertone and A.~Melchiorri,
  %``CMB constraints on Dark Matter models with large annihilation cross-section,''
  Phys.\ Rev.\ D {\bf 80}, 023505 (2009)
  [arXiv:0905.0003 [astro-ph.CO]];
  %%CITATION = ARXIV:0905.0003;%%
%\cite{Slatyer:2009yq}
%\bibitem{Slatyer:2009yq} 
  T.~R.~Slatyer, N.~Padmanabhan and D.~P.~Finkbeiner,
  %``CMB Constraints on WIMP Annihilation: Energy Absorption During the Recombination Epoch,''
  Phys.\ Rev.\ D {\bf 80}, 043526 (2009)
  [arXiv:0906.1197 [astro-ph.CO]];
  %%CITATION = ARXIV:0906.1197;%%
%\cite{Galli:2011rz}
%\bibitem{Galli:2011rz} 
  S.~Galli, F.~Iocco, G.~Bertone and A.~Melchiorri,
  %``Updated CMB constraints on Dark Matter annihilation cross-sections,''
  Phys.\ Rev.\ D {\bf 84}, 027302 (2011)
  [arXiv:1106.1528 [astro-ph.CO]];
  %%CITATION = ARXIV:1106.1528;%%
%\cite{Finkbeiner:2011dx}
%\bibitem{Finkbeiner:2011dx} 
  D.~P.~Finkbeiner, S.~Galli, T.~Lin and T.~R.~Slatyer,
  %``Searching for Dark Matter in the CMB: A Compact Parameterization of Energy Injection from New Physics,''
  Phys.\ Rev.\ D {\bf 85}, 043522 (2012)
  [arXiv:1109.6322 [astro-ph.CO]].
  %%CITATION = ARXIV:1109.6322;%%
%\cite{Feng:2009hw}

\bibitem{Feng:2009hw} 
  J.~L.~Feng, M.~Kaplinghat and H.~-B.~Yu,
  %``Halo Shape and Relic Density Exclusions of Sommerfeld-Enhanced Dark Matter Explanations of Cosmic Ray Excesses,''
  Phys.\ Rev.\ Lett.\  {\bf 104}, 151301 (2010)
  [arXiv:0911.0422 [hep-ph]].
  %%CITATION = ARXIV:0911.0422;%%

	%\cite{Buckley:2009in}
\bibitem{Buckley:2009in} 
  M.~R.~Buckley and P.~J.~Fox,
  %``Dark Matter Self-Interactions and Light Force Carriers,''
  Phys.\ Rev.\ D {\bf 81}, 083522 (2010)
  [arXiv:0911.3898 [hep-ph]].
  %%CITATION = ARXIV:0911.3898;%%


%\cite{Pieri:2009je}
%\bibitem{Pieri:2009je}  L.~Pieri, J.~Lavalle, G.~Bertone and E.~Branchini,
  %``Implications of High-Resolution Simulations on Indirect Dark Matter Searches,''
%  Phys.\ Rev.\ D {\bf 83}, 023518 (2011)
 % [arXiv:0908.0195 [astro-ph.HE]].
  %%CITATION = ARXIV:0908.0195;%%
%\bibitem{Kistler:2009xf} 
 % M.~D.~Kistler and J.~M.~Siegal-Gaskins,
  %``Gamma-ray signatures of annihilation to charged leptons in dark matter substructure,''
 % Phys.\ Rev.\ D {\bf 81}, 103521 (2010)
  %[arXiv:0909.0519 [astro-ph.HE]].
  %%CITATION = ARXIV:0909.0519;%%
  
%\cite{Ruderman:2009tj}
\bibitem{Ruderman:2009tj}
J.~T.~Ruderman and T.~Volansky,
``Decaying into the Hidden Sector,''
JHEP {\bf 1002} (2010) 024
[arXiv:0908.1570 [hep-ph]].
%%CITATION = ARXIV:0908.1570;%%

%\cite{Essig:2010ye}
\bibitem{Essig:2010ye}
R.~Essig, J.~Kaplan, P.~Schuster and N.~Toro,
``On the Origin of Light Dark Matter Species,''
Submitted to: Submitted to Physical Review D
[arXiv:1004.0691 [hep-ph]].
%%CITATION = ARXIV:1004.0691;%%

%\cite{Bernabei:2010mq}
\bibitem{Bernabei:2010mq} 
  R.~Bernabei {\it et al.}  [DAMA and LIBRA Collaborations],
  %``New results from DAMA/LIBRA,''
  Eur.\ Phys.\ J.\ C {\bf 67}, 39 (2010)
  [arXiv:1002.1028 [astro-ph.GA]].
  %%CITATION = ARXIV:1002.1028;%%

%\cite{Aalseth:2010vx}
\bibitem{Aalseth:2010vx} 
  C.~E.~Aalseth {\it et al.}  [CoGeNT Collaboration],
  %``Results from a Search for Light-Mass Dark Matter with a P-type Point Contact Germanium Detector,''
  Phys.\ Rev.\ Lett.\  {\bf 106}, 131301 (2011)
  [arXiv:1002.4703 [astro-ph.CO]].
  %%CITATION = ARXIV:1002.4703;%%

%\cite{Aalseth:2011wp}
\bibitem{Aalseth:2011wp} 
  C.~E.~Aalseth, P.~S.~Barbeau, J.~Colaresi, J.~I.~Collar, J.~Diaz Leon, J.~E.~Fast, N.~Fields and T.~W.~Hossbach {\it et al.},
  %``Search for an Annual Modulation in a P-type Point Contact Germanium Dark Matter Detector,''
  Phys.\ Rev.\ Lett.\  {\bf 107}, 141301 (2011)
  [arXiv:1106.0650 [astro-ph.CO]].
  %%CITATION = ARXIV:1106.0650;%%

%\cite{Angloher:2011uu}
\bibitem{Angloher:2011uu} 
  G.~Angloher, M.~Bauer, I.~Bavykina, A.~Bento, C.~Bucci, C.~Ciemniak, G.~Deuter and F.~von Feilitzsch {\it et al.},
  %``Results from 730 kg days of the CRESST-II Dark Matter Search,''
  arXiv:1109.0702 [astro-ph.CO].
  %%CITATION = ARXIV:1109.0702;%%
\bibitem{Agnese:2013dwa}
R. Agnese {\it et al.} [CDMS Collaboration], arXiv1204.3706 [stro-ph.CO].


%\cite{Ahmed:2010wy}
\bibitem{CDMS} 
  Z.~Ahmed {\it et al.}  [CDMS-II Collaboration],
  %``Results from a Low-Energy Analysis of the CDMS II Germanium Data,''
  Phys.\ Rev.\ Lett.\  {\bf 106}, 131302 (2011)
  [arXiv:1011.2482 [astro-ph.CO]].
  %%CITATION = ARXIV:1011.2482;%%%\cite{Ahmed:2012vq}
%\bibitem{Ahmed:2012vq} 
  Z.~Ahmed {\it et al.}  [CDMS Collaboration],
  %``Search for annual modulation in low-energy CDMS-II data,''
  arXiv:1203.1309 [astro-ph.CO].
  %%CITATION = ARXIV:1203.1309;%%

\bibitem{Angle:2011th} 
  J.~Angle {\it et al.}  [XENON10 Collaboration],
  %``A search for light dark matter in XENON10 data,''
  Phys.\ Rev.\ Lett.\  {\bf 107}, 051301 (2011)
  [arXiv:1104.3088 [astro-ph.CO]].
  %%CITATION = ARXIV:1104.3088;%%
%\cite{Aprile:2011hi}
\bibitem{Aprile:2011hi} 
  E.~Aprile {\it et al.}  [XENON100 Collaboration],
  %``Dark Matter Results from 100 Live Days of XENON100 Data,''
  Phys.\ Rev.\ Lett.\  {\bf 107}, 131302 (2011)
  [arXiv:1104.2549 [astro-ph.CO]].
  %%CITATION = ARXIV:1104.2549;%%

%\cite{Kelso:2011gd}
\bibitem{Kelso:2011gd} 
  C.~Kelso, D.~Hooper and M.~R.~Buckley,
  %``Toward A Consistent Picture For CRESST, CoGeNT and DAMA,''
  Phys.\ Rev.\ D {\bf 85}, 043515 (2012)
  [arXiv:1110.5338 [astro-ph.CO]].
  %%CITATION = ARXIV:1110.5338;%%

%\cite{Aprile:2011ts}
\bibitem{Aprile:2011ts}
E.~Aprile {\it et al.} [XENON100 Collaboration],
``Implications on Inelastic Dark Matter from 100 Live Days of Xenon100 Data,''
Phys.\ Rev.\ D {\bf 84} (2011) 061101
[arXiv:1104.3121 [astro-ph.CO]].
%%CITATION = ARXIV:1104.3121;%%

\bibitem{rarek}
T. ~Beranek, and ~M. Vanderhaeghen, [	arXiv:1209.4561 [hep-ph]].

\bibitem{andreas}
S.~Andreas, C.~Niebuhr, and A. ~Ringwald,  [arXiv:1209.6083 [hep-ph]].


%% END NATALIA'S CITES FOR DM SECTION
%%%%%%%%%%%%%%%%%%%%%%%%%%%%%%%%%%%%
%%%%%%%%%%%%%%%%%%%%%%%%%%%%%%%%%%%%

%%%%%%%%%\cite{Bross:1989mp}
%%%%%%%%\bibitem{HPS}
%%%%%%%%The Heavy Photon Search Collaboration (HPS), \\
%%%%%%%%{\tt https://confluence.slac.stanford.edu/display/hpsg/}
%%%%%%%%
%\end{thebibliography}

%\begin{thebibliography}{100}

%\cite{Deutsch:1951zza}
\bibitem{Deutsch:1951zza} 
  M.~Deutsch,
  %``Evidence for the Formation of Positronium in Gases,''
  Phys.\ Rev.\  {\bf 82}, 455 (1951).
  %%CITATION = PHRVA,82,455;%%

 
%\cite{Friedman:1957mz}
\bibitem{Friedman:1957mz} 
  J.~I.~Friedman and V.~L.~Telegdi,
  %``Nuclear Emulsion Evidence For Parity Nonconservation In The Decay Chain Pi+ Mu+ E+,''
  Phys.\ Rev.\  {\bf 105}, 1681 (1957).
  %%CITATION = PHRVA,105,1681;%%


%\cite{Hughes:1960zz}
\bibitem{Hughes:1960zz} 
  V.~W.~Hughes, D.~W.~McColm, K.~Ziock and R.~Prepost,
  %``Formation of Muonium and Observation of its Larmor Precession,''
  Phys.\ Rev.\ Lett.\  {\bf 5}, 63 (1960).
  %%CITATION = PRLTA,5,63;%%



%\cite{Holvik:1986ty}
\bibitem{Holvik:1986ty} 
  E.~Holvik and H.~A.~Olsen,
  %``Creation of Relativistic Fermionium in Collisions of Electrons with Atoms,''
  Phys.\ Rev.\ D {\bf 35}, 2124 (1987).
  %%CITATION = PHRVA,D35,2124;%%


%\cite{ArteagaRomero:2000yh}
\bibitem{ArteagaRomero:2000yh} 
  N.~Arteaga-Romero, C.~Carimalo and V.~G.~Serbo,
  %``Production of bound triplet mu+ mu- system in collisions of electrons with atoms,''
  Phys.\ Rev.\ A {\bf 62}, 032501 (2000)
  [hep-ph/0001278].
  %%CITATION = HEP-PH/0001278;%%


%\cite{Brodsky:2009gx}
\bibitem{Brodsky:2009gx} 
  S.~J.~Brodsky and R.~F.~Lebed,
  %``Production of the Smallest QED Atom: True Muonium (mu+ mu-),''
  Phys.\ Rev.\ Lett.\  {\bf 102}, 213401 (2009)
  [arXiv:0904.2225 [hep-ph]].
  %%CITATION = ARXIV:0904.2225;%%

%\cite{Bilenky:1969zd}
\bibitem{Bilenky:1969zd} 
  S.~M.~Bilenky, V.~H.~Nguyen, L.~L.~Nemenov and F.~G.~Tkebuchava,
  %``Production and decay of (muon-plus muon-minus)-atoms,''
  Yad.\ Fiz.\  {\bf 10}, 812 (1969). 
  %{\color{red} \bf S. Bilen’kii, N. van Hieu, L. Nemenov, and F. Tkebuchava, Sov. J. Nucl. Phys., 10, 469 (1969). in original proposal. -- couldn't get copy to check reference.}
  %%CITATION = YAFIA,10,812;%%


%\cite{Hughes:1971}
\bibitem{Hughes:1971} 
  V.~W.~Hughes and B.~Maglic, 
  Bull.\ Am.\ Phys.\ Soc.\ 16, 65 (1971). 


%\cite{Malenfant:1987tm}
\bibitem{Malenfant:1987tm} 
  J.~Malenfant,
  %``Cancellation Of The Divergence Of The Wave Function At The Origin In Leptonic Decay Rates,''
  Phys.\ Rev.\ D {\bf 36}, 863 (1987).
  %%CITATION = PHRVA,D36,863;%%


%\cite{Karshenboim:1998we}
\bibitem{Karshenboim:1998we} 
  S.~G.~Karshenboim, U.~D.~Jentschura, V.~G.~Ivanov and G.~Soff,
  %``Next-to-leading and higher order corrections to the decay rate of dimuonium,''
  Phys.\ Lett.\ B {\bf 424}, 397 (1998).
  %%CITATION = PHLTA,B424,397;%%


%\cite{Owen:1972}
\bibitem{Owen:1972} 
  D.~A.~Owen and W.~W.~Repko, 
  Phys.\ Rev.\ A 5, 1570 (1972).                 


%\cite{Jentschura:1997ma}
\bibitem{Jentschura:1997ma} 
  U.~D.~Jentschura, G.~Soff, V.~G.~Ivanov and S.~G.~Karshenboim,
  %``Decay rates and hyperfine structure of the bound mu+ mu- system,''
  hep-ph/9706401.
  %%CITATION = HEP-PH/9706401;%%

%\cite{Jentschura:1997tv}
\bibitem{Jentschura:1997tv} 
  U.~D.~Jentschura, G.~Soff, V.~G.~Ivanov and S.~G.~Karshenboim,
  %``The Bound mu+ mu- system,''
  Phys.\ Rev.\ A {\bf 56}, 4483 (1997)
  [physics/9706026].
  %%CITATION = PHYSICS/9706026;%%


%\cite{Karshenboim:1998am}
\bibitem{Karshenboim:1998am} 
  S.~G.~Karshenboim, V.~G.~Ivanov, U.~D.~Jentschura and G.~Soff,
  %``Bound states of the muon antimuon system: Lifetimes and hyperfine splitting,''
  J.\ Exp.\ Theor.\ Phys.\  {\bf 86}, 226 (1998)
  [Zh.\ Eksp.\ Teor.\ Fiz.\  {\bf 113}, 409 (1998)].
  %%CITATION = JTPHE,86,226;%%


%\cite{Bennett:2006fi}
\bibitem{Bennett:2006fi} 
  G.~W.~Bennett {\it et al.}  [Muon G-2 Collaboration],
  %``Final Report of the Muon E821 Anomalous Magnetic Moment Measurement at BNL,''
  Phys.\ Rev.\ D {\bf 73}, 072003 (2006)
  [hep-ex/0602035].
  %%CITATION = HEP-EX/0602035;%%



%\cite{Pohl:2010zza}
\bibitem{Pohl:2010zza} 
  R.~Pohl, A.~Antognini, F.~Nez, F.~D.~Amaro, F.~Biraben, J.~M.~R.~Cardoso, D.~S.~Covita and A.~Dax {\it et al.},
  %``The size of the proton,''
  Nature {\bf 466}, 213 (2010).
  %%CITATION = NATUA,466,213;%%

%%%
%\cite{toAppear}
\bibitem{Banburski:2012tk} 
  A.~Banburski and P.~Schuster,
  %``The Production and Discovery of True Muonium in Fixed-Target Experiments,''
  Phys.\ Rev.\ D {\bf 86}, 093007 (2012)
  [arXiv:1206.3961 [hep-ph]].
  %%CITATION = ARXIV:1206.3961;%%


%\cite{APS}
%\bibitem{APS}
 %A. Banburski,
 %``Discovering True Muonium Physics with Fixed-Target Experiments''
 %APS Talk, March 31 2012, APR12-2012-020018

%\cite{Strassler:2006im}
\bibitem{Strassler:2006im} 
  M.~J.~Strassler and K.~M.~Zurek,
  %``Echoes of a hidden valley at hadron colliders,''
  Phys.\ Lett.\ B {\bf 651}, 374 (2007)
  [hep-ph/0604261].
  %%CITATION = HEP-PH/0604261;%%
  
  %\cite{Essig:2009nc}
%\bibitem{Essig:2009nc} 
% R.~Essig, P.~Schuster and N.~Toro,
%``Probing Dark Forces and Light Hidden Sectors at Low-Energy e+e- Colliders,''
  %Phys.\ Rev.\ D {\bf 80}, 015003 (2009)
  %[arXiv:0903.3941 [hep-ph]].
  %%CITATION = ARXIV:0903.3941;%%





%\end{thebibliography}

%%
 
%\bibliographystyle{unsrt}
%\begin{thebibliography}{99}
\bibitem{hoffmann}
D.H.H. Hoffmann \etal, Z. Physik {\bf A293}, 187 (1979)

\bibitem{hubbell}
J.H. Hubbell \etal, J. Phys. Chem. Ref. Data, Vol. {\bf 23}, 339 (1994) 

\bibitem{mohring} H.-J. M\"{o}hring, Nucl. Instrum. Methods {\bf A292}, 482 (1990)

\bibitem{budnev} Y.M. Budnev \etal, Phys. Rep. {\bf 15C}, 181 (1975)

%\end{thebibliography}

%\bibliographystyle{unsrt}
%\begin{thebibliography}{99}
\bibitem{bf}
P. Billoir, R. Fruhwirth, and M. Regler, Nucl. Instr. And Meth. {\bf A241}, 115 (1985). 

\bibitem{bq}
P. Billoir and S. Qian, Nucl. Instr. And Meth. {\bf A311}, 139 (1991). 

%\end{thebibliography}


\bibitem{elegant} M. Borland, A Flexible SDDS-Compliant Code for Accelerator Simulation, ANL, Argonne, IL 60439, USA
 
\bibitem{slic}  http://www.lcsim.org/software/slic/
 
\bibitem{trackAlgo} P. Billoir, R. Fruhwirth, and M. Regler, Nucl. Instr. And Meth. A241 (1985) 115. 

\bibitem{vertexAlgo} P. Billoir and S. Qian, Nucl. Instr. And Meth. A311 (1991) 139. 

\bibitem{Jones:1069892} L. Jones, APV25-S1: User guide version 2.2, RAL Microelectronics Design Group, 2011.

\bibitem{Raymond:2002yr} M. Raymond et al., APV25 production testing and quality assurance, 8th Workshop on Electronics for LHC
	              Experiments, Colmar, France, 9-13 Sep 2002
	               

%\bibliographystyle{unsrt}
%\begin{thebibliography}{99}
\bibitem{moliere}
G. Moli\'{e}re, Z. Naturforsch. {\bf 3a}, 78 (1948)

\bibitem{bethe}
H. A. Bethe, Phys. Rev. {\bf 89}, 1256 (1953)
\bibitem{gs}
S. A. Goudsmit and J. L. Saunderson, Phys. Rev. {\bf 57}, 24 and {\bf 58}, 36 (1940);
K. Okei and T. Nakatsuka, Proceedings of the $17^{th}$ EGS User's Meeting.

\bibitem{attwood}
D. Attwood \etal, Nucl. Instrum. Methods {\bf B251}, 41 (2006)

\bibitem{shen}
G. Shen \etal, Phys. Rev. {\bf 20}, 1584 (1979)

\bibitem{gottschalk}
B. Gottschalk \etal, Nucl. Instrum. Meth. {\bf B74}, 467 (1993)

\bibitem{koch}
H.W. Koch and J.W. Motz, Rev. Mod. Phys. {\bf 31}, 920 (1959)

\bibitem{motz}
J.W. Motz, H. A. Olsen, and H.W. Koch, Rev. Mod. Phys. {\bf 41}, 581 (1969)

%\end{thebibliography}


\end{thebibliography} 


\end{document}
