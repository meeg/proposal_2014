\section{Proposed Measurements {\it Matt, Sarah, Raphael, Maurik, John, Stepan}}


The proposed experiment will search for a heavy photon (dark photon) in the mass range from 20 MeV to 1000 MeV in two settings of beam energy 2.2 GeV and 6.6 GeV. 
The proposed HPS experiment has the potential to discover ``true muonium'', a bound state of a
$\mu^+ \mu^-$ pair. 
The HPS experiment ultimately relies upon the precision measurement of two quantities: the invariant mass of the A$^\prime$ decay products and the position of the decay vertex. By placing a tracking and vertexing detector immediately downstream of the target inside an analyzing magnet, the complete kinematic information required for A$^\prime$ reconstruction can be obtained from a single system, whose proximity to the target naturally maximizes the acceptance of a relatively compact detector and provides excellent momentum and vertexing resolution. A finely segmented, fast electromagnetic calorimeter, just downstream of the tracker,  provides a powerful high rate trigger, identifies electrons, and augments  the electron energy measurement. Behind the ECal a muon system consist of four planes of scintillator hodoscopes sandwiched between iron absorbers will be positioned. The muon system will provide trigger for ($\mu^+\mu^-$) detection and will be used for muon identification. It will extend search for high mass A$^\prime$ in di-muon decay mode where electromagnetic backgrounds are much reduced. Very high rate data acquisition systems, for the tracker, Ecal and muon system, make it possible to trigger and transfer data at $10$�s of kHz, and run with negligible dead time.

\subsection{Experimental reach for heavy photon {\it Matt}}


\subsection{Search for true muonium {\it Sarah}}

The proposed HPS experiment has the potential to discover ``true muonium'', a bound state of a
$\mu^+ \mu^-$ pair, denoted here by $(\mu^+ \mu^-)$. 
We expect that HPS will discover the 1S, 2S, and 2P true muonium bound states with its proposed run plan. 
The detection of these states should demonstrate the capability of the HPS experiment 
to identify rare separated vertex decays, and will provide a natural calibration 
tool for improving searches for heavy photons. Details of the production and detection of true muonium using HPS detector can be found in \cite{HPS_PROP_UPD}. 
The $(\mu^+ \mu^-)$ atom is hydrogen-like, and so has a set of excited states characterized by a principal quantum number n. 
The binding energy of these states is E = $-1407$ eV/n$^2$. The $(\mu^+ \mu^-)$ ``atom'' can be produced by an electron beam incident on a target such 
as tungsten \cite{Holvik:1986ty,ArteagaRomero:2000yh}. 

With the existing proposal, HPS will search for true muonium
just as it does for heavy photons with separated vertices, requiring a vertex cut at about 1.5 cm to reject almost all
QED background events, then searching for a resonance at 2 m$_{\mu}$. An additional cut 
on the total energy of the $e^+e^-$ pair of $E_{e^-}+E_{e^+}> 0.8 \ E_{beam}$ will also be required
for triggering. 

Based on \cite{toAppear}, the total production yield for 1S, 2S, and 2P (including secondary production)
leaving a target of thickness $t_b$(or larger) and satisfying the above requirements is,
\begin{equation}
N_{(\mu^+ \mu^-)} = 600 \left( \frac{I}{450 \ nA} \right) \left( \frac{t}{3 \ months} \right)
\end{equation}
%
where a beam energy E$_{beam} = 6.6$ GeV, and the nominal conditions
of 450 nA beam current for 3 months ($\sim 7.8 \times 10^6$ s) on a single foil has been assumed.
The vertices near the cut of $1.5$ cm will be dominated by the 1S state, while 
a tail of vertices extending out beyond a few cm is dominated by 2S and 2P. 

Accounting for all the efficiencies associated with a separated vertex search, we would expect to see about 60--100 true muonium events 
(we caution that the acceptance parameterization here is uncertain at the 50\% level).
The HPS experiment should be able to identify enough events to claim a discovery, and in addition, should be able to measure the mass of true muonium.  There are certainly other properties of true muonium that would be interesting to measure.  A measurement of the lifetimes would be interesting, as the lifetimes are sensitive to physics that couples to leptonic currents.  With enough statistics, it should be possible to perform a measurement of the lifetimes of the 1S, 2S, and 2P states; work is ongoing to investigate this possibility.  

