%\section{Schedule and Cost Baseline}
\label{sec:schcost}

Cost estimates for engineering, designing, fabricating, assembling, testing, and installing the Heavy Photon Search detector 
are given below. The costs assumes considerable savings from the reuse of many parts of the Test Run, which was already assembled 
made from the donated silicon microstrip sensors from Fermilab, the use of some DAQ crates and equipment from SLAC,
and many contributions from JLab, including PbWO$_4$ calorimeter crystals, the chicane and analyzing magnets, magnet power supplies,
and beam diagnostic apparatus. Much of the calorimeter readout electronics utilizes designs which are already in place for the Hall
B 12 GeV upgrade, eliminating engineering and design expense. The SVT DAQ benefits from SLAC's development of an ATCA readout 
system, and incorporates many of its existing designs. Very significant cost savings come from utilizing the FADCs and data acquisition 
system being developed for the upgraded CLAS12 detector, which will be available free of charge to HPS. The Orsay group is 
contributing engineering and design efforts for the ECal and its vacuum chamber, affording additional savings. 

The costs are given in an accompanying WBS summary table, below, which itemizes the major items subsystem by subsystem, and 
indicates whether JLab (J) or SLAC (S) takes responsibility for construction. Engineering, design, and technician labor rates are fully loaded, including benefits and lab overheads, which differ between the two laboratories. Our 
DAQ and beamline cost estimates have been made by engineering groups at SLAC and JLab which are experienced in cost estimation 
and actively involved in many related projects. The SVT estimates came from physicists and engineers on the project, with 
experience in designing and fabricating silicon detector systems. The Ecal estimates come from physicists and engineers at 
JLab and Orsay who have constructed a similar system, the CLAS IC, in the recent past. 

The schedule for the overall project is included in a Project Summary table below. A brief description of the schedule for the 
different subsystems is also given. The overall schedule contingency is about 20\%, and depends critically on the assumption that 
funding is available by mid CY2013. The HPS construction project has been organized into a Work Breakdown Structure (WBS) for purposes of planning, 
managing and reporting project activities. Work elements are defined to be consistent with discrete increments of project work. 
Project Management efforts are distributed throughout the project, including conceptual design and R\&D. The HPS has 12 WBS
 Level-2 elements, see Table \ref{tb:wbs}. 

\begin{table}[htdp]
\caption{Project WBS structure.}
\begin{center}
\begin{tabular}{|c|c|}
\hline
WBS& NAME \\
\hline\hline
1.1 & Beamline \\
\hline
1.2 & SVT \\
\hline
1.3 & SVT DAQ \\
\hline
1.4 & ECAL \\
\hline
1.5 & Muon \\
\hline
1.6 & TDAQ \\
\hline
1.7 & Slow Controls \\
\hline
1.8 & Installation \& Commissioning \\
\hline
1.9 & Electron Running \\
\hline
1.10 & SLAC Travel Meetings \\
\hline
1.11 & SLAC Travel Running \\
\hline
1.12 & Project Management  \\
\hline
\hline
\end{tabular}
\end{center}
\label{tb:wbs}
\end{table}%

\subsection{Cost}

The costs include Labor and M\&S. The labor includes only engineering or technician manpower in professional centers at SLAC or JLAB. It does not include labor provided by physicists, which is the dominant contribution to the project. Labor rates have been applied following 
the official shop rates at SLAC and JLAB, which include already ~31\% or ~57\% fringe benefits, respectively. M\&S have been determined from a best estimation 
of the commercially available parts,  benefiting from our experience with the actual costs of the HPS Test Run. The overheads have been added to both labor 
and M\&S, being respectively 53\% and 7.65\% at SLAC, 49\% for both labor and M\&S at JLAB. SLAC travel includes 53\% overheads. Contingencies have been set at 10\% for catalogue items,  20-25\% for items similar to previous design, 30-50\%  if new design. Since the project is staged over three years, an annual inflation rate of 2.5\%  is included in 
the FY14 and FY15 costs.

Beamline expenses for HPS are held to a minimum by using the 18D36 magnet currently installed in Hall B as the analyzing magnet, the 
two existing JLab Frascati chicane magnets and the existing Test Run vacuum chamber with the SVT vacuum box.  Some overall engineering and design will be 
required, beam pipes fabricated, the Muon System vacuum chamber designed and built, a vacuum chamber built for the downstream 
Frascati magnet, and a photon dump and shielding inserted behind the second chicane magnet. Total beamline expenses are about \$324k.

Three out of the five planes of the SVT Test Run will be reused after modification of their supports which will provide improved 
mechanical stability and better cooling. Three new planes with double sensors and their supports will be designed and built from 
scratch. Fermilab will donate the needed silicon microstrip detectors, as it had for the HPS Test Run. The tracker/vertexer  will cost about \$639k.

The SVT DAQ requires small modifications to the  existing hybrid; new readout and flange board engineering design,prototyping, and 
production; 
APV25 and chip procurement; and  fabrication and testing. The SVT DAQ also requires designing and prototyping the Trigger Interrupt 
ACTA card and new firmware for the APV25 to provide event buffering to accommodate higher trigger rates. ATCA crates, and standard 
RCE cards are also required. The expenses are dominated by engineering development, and total \$607k.  

JLab will donate the PbWO$_4$ crystals used in the ECal. Orsay will donate engineering and design for a new enclosure for the crystals, 
but Jlab will need to fabricate the enclosure, the crystal support structure, the readout motherboard and connection board, and support 
fixtures. It will also make repairs to the existing motherboards and acquire new power supplies. The total expense will be 
roughly \$312k, including fabrication, assembly, and testing. 

Trigger and DAQ electronics for the ECAL are being developed for 
the CLAS upgrade, so relatively little engineering and technician time will be needed in preparation of the HPS Test Run except 
for providing special purpose firmware. Many components, including the 250 MHz FADC boards and crates will be provided at no cost 
since they can be borrowed from the CLAS upgrade. The system test expenses will also be borne by JLab Hall B. The total cost 
is \$184k. 

The Muon system costs are in purchasing  photomultipliers and scintillator, designing and fabricating  the absorbers and Muon System 
vacuum chamber, and building support stands and providing cables.The total is \$419k. 

The Slow Controls are needed to monitor the operations of the three sub-detectors. In addition, they will control and interlock the 
movements of the SVT with respect the beamline and provide beam protection interlocks. The total cost is \$221K, which is essentially 
the labor required to integrate the HPS with the existing Slow Control system in the Hall-B.

The offline computing resources will be provided by JLAB with the exception of the local storage at SLAC for analysis streams, ntuples, code releases and scratch areas.

Travel and lodging expenses for SLAC trips to JLab are also included in this proposal. During design and construction, 
there will be a small number of trips to solidify and review designs, and to work together to begin DAQ integration of the SLAC 
and JLab systems. Funds are reserved for collaboration meetings to be held during calendar 2013 and 2014 at JLab. 
Funds are also reserved  to staff installation, commissioning, and data taking runs. The total is \$172K.
%The total is \$190K (\$207K???).

{\bf The total cost for HPS is \$3.035 M. }

HPS is seeking funding from other sources for the Muon System and the Ecal.
William\&Mary will submit an MRI proposal to NSF for the Muon System, requesting $\sim \$200$k. IPN ORSAY (France) 
has submitted a proposal to a French funding agency for the ECal  Light Monitoring System (\$100k) and for a new, 
high performance  APD�s to improve crystal readout (\$500k) and other expenses related to ECal fabrication and test.
Note that the new APDs are not part of this proposal. If these requests are approved the corresponding funding will be subtracted 
from the total cost of the HPS. 

\subsection{Schedule}

Our goal is to be ready to install the HPS in Hall-B at JLAB  by September 2014, and proceed with commissioning on beam with the 
CEBAF early physics beam window opportunity in October 2014. The data taking will until summer 2015. Meeting this schedule
will require approval and funding as soon as possible, preferably by June 2013. Schedules for each of the major subsystems of the 
experiment are attached below, and summarized here. The total construction schedule extends over 16 months, assuming the funding 
available mid-2013. The schedule contingency is about 20\%. 

The conceptual design of the beamline will be done during 2013. Formal beamline engineering will start when funding is secured.  
A Beamline Engineering Design Review will be held in February 2014 to validate the concept before the start of major spending. 
Final Engineering and Construction will start in Spring 2014 and be completed well before the installation time in October 2014, 
providing substantial float. 

Using keep-alive funds, the Test Run SVT will be shipped back to SLAC by early February 2013 to 
rework the modules for the first three layers of the HPS and commission the motion control systems. The conceptual design of the 
Layers 1-2-3 and Layers 4-5-6 will start already in spring 2013 using keep alive funding. An Engineering Design Review of the SVT 
will be held in November 2013, after the funding has been released and before the final production. Preliminary engineering for 
the SVT DAQ is already underway, using the same keep alive funds. 

The SVT DAQ will formally begin work in  the second half 2013, 
after funding and an Engineering Review. The assembly and integration test at SLAC are expected in spring 2014 and the SVT will 
be ready for shipping on July 2014. The SVT will be ready for installation in mid-August 2014. SVT  installation in the analyzing 
magnet vacuum chamber will occur in September, depending on the schedule of the Hall-B.  The SVT schedule has 1 month of float 
between the shipping and the test at JLAB, which will be eventually used as contingency for  construction work at SLAC.

The Ecal work will start in the second half 2013 and run through June 2014. The ECAL will be ready for installation by June 2014.

 
The Muon System work will start in July 2013 with a design phase which will be validated in an Engineering Design Review in September 
2013. It will be ready for installation on August 2014, with one month of float with respect to  the installation date.

The schedule includes seven milestones to track the progress of each subsystem. They will monitor the subdetector readiness after 
testing at the respective assembly sites, and the readiness for installation at JLAB.  Also, ad-hoc Engineering Design Reviews 
will be conducted by the PM for each subsystem before major costs are incurred.

\begin{table}[htdp]
\caption{Project Milestones.}
\begin{center}
\begin{tabular}{|c|c|c|}
\hline
WBS & Milestones & Date\\
\hline\hline
1.3.2.8	&Flange Board ready	&25-Oct-13\\
\hline
1.3.1.11	&FE Board Ready&	25-Nov-13\\
\hline
1.3.4.6	&Hybrid Ready	&25-Nov-13\\
\hline
1.2.15	&Layer 1-3 Ready	&14-Mar-14 \\
\hline
1.3.5.6	&Flex Cable Ready	&17-Mar-14 \\
\hline
1.2.16	&Layer 4-6 Ready	&18-Apr-14 \\
\hline
1.3.3.15&	DAQ Ready	&25-Apr-14 \\
\hline
1.2.12&	SVT Ready to Ship	&16-Jun-14\\
\hline
1.2.14&	SVT Ready For Installation&	15-Aug-14\\
\hline
1.5.9	&Muon Ready for installation&	11-July-14\\
\hline
1.4.12&	ECAL Ready for the installation&	8-Aug-14\\
\hline
1.1.21&HPS Installed on Beamline&26-Sep-14 \\
\hline
1.10.7&	HPS ready for the beam&	27-Sep-14\\
\hline
\hline
\end{tabular}
\end{center}
\label{tb:milestones}
\end{table}%

\begin{table}[htdp]
\caption{Planned Review.}
\begin{center}
\begin{tabular}{|c|c|c|}
\hline
WBS&Engineering Reviews& Data\\
\hline
\hline
%1.1.2 &	Beamline  Review&	3-Feb-14\\
%\hline
1.2.2	&SVT Design Review	&14-Oct-13\\
\hline
1.5.1	&Muon Design Review&	4-Nov-13\\
\hline
1.10.1&	Installation Review	&25-Aug-14\\
\hline
\hline
\end{tabular}
\end{center}
\label{tb:reviews}
\end{table}%

\subsection{Manpower}

The manpower needed to design, fabricate, assemble, test, install, and commission the HPS is captured in the WBS tables. 
The HPS Collaboration has the personnel needed to realize this project.
 
Beamline design work will be done at JLab by Arne Freyberger, F-X Girod and Stepan Stepanyan and at SLAC by Ken Moffeit; 
engineering at SLAC by Marco Oriunno, Dieter Walz, and Clive Field; fabrication in the JLab shops; 
and installation by the Hall B crew. 

Engineering for the Ecal is being done by Philippe Rosier at Orsay in consultation with Marco Oriunno. 

Beam diagnostics and slow control will be supported by Nerses Gevorgyan (Yerevan) and Hovanes Egiyan.

 
The Tracker/Vertexer will be engineered and designed by Marco Oriunno, Tim Nelson and Per Hansson, with additional help from 
Vitaly Fadeyev, Alex Grillo, and Bill Cooper, all experienced with silicon detector systems. Others at SLAC and UCSC will 
help with test and assembly, including Matt Graham, Takashi Maruyama, John Jaros, and graduate students Sho 
Uemura and Omar Moreno.  Matt McCulloch will serve as the  technician at SLAC. 

The Ecal is being designed by the Orsay Group, especially Philippe Rosier, Emmanuel Rindel, Emmanuel Rauly, Raphael Dupre, 
and Michel Guidal, with participation by the Jlab group, especially Stepan Stepanyan, and F.-X. Girod. Others at JLab and 
in the collaboration will help in assembly and test of the ECal, especially the group from INFN Genova(Italy). 
 
The SVT DAQ is being done by Haller's group at SLAC, including Gunther Haller, Ryan Herbst, Tung Phan, and Raghuveer 
Ausoori. SVT Physicists Per Hansson, Alex Grillo, Vitaliy Fadeyev, and Tim Nelson will collaborate closely. Postdocs 
and students will help debug, test, and certify DAQ electronics. 

The Ecal Trigger/DAQ work is done in Sergey Boyarinov's group, which supports Hall B activities at JLAB, and with 
Chris Cuevas's group, which has designed the FADC250. R. Dupre and V. Kubarovsky will collaborate with this group in 
assembling and testing the electronics, programming the trigger, and integrating it with the Ecal hardware.

The HPS collaboration is nearly 60 strong, so has adequate manpower for overall installation, commissioning, and data taking.  
Simulation work is supported by Maurik Holtrop, Matt Graham, M.Ungaro, and Takashi Maruyama, along with help from students Sho 
Uemura and Omar Moreno, 
and Norman Graf  and Jeremy McCormick at SLAC. Data management and storage and computing infrastructure will be overseen 
by Sergey Boyarinov and Maurik Holtrop and Homer Neal, all very experienced professionals. Analysis and simulation studies 
have been initiated by Maurik Holtrop, Matt Graham, Sho Uemura, and Takashi Maruyama. Students are actively being engaged.
 
The HPS collaboration is managed by its three spokespersons, Maurik Holtrop, John Jaros, and Stepan Stepanyan and its 
Executive Committee, which consists of the spokespeople along with Takashi Maruyama, Matt Graham, Tim Nelson, and F-X Girod. 
Ten working groups supervise the progress of each sub-system. The Project Manager is Marco Oriunno.

\begin{table}[htdp]
\caption{Working groups.}
\begin{center}
\begin{tabular}{|c|c|c|}
\hline
HPS working Groups	& Chair (Deputy)\\
\hline\hline
Beamline	&K. Moffeit (FX Girod)\\
\hline
SVT	&T.Nelson (V.Fedayev)\\
\hline
ECAL	& R. Dupre (S.Stepanyan)\\
\hline
DAQ	 & S. Boiarinov (P.Hansson)\\
\hline
Trigger &	V. Kubarovsky (T.Maruyama)\\
\hline
Slow Control	& H. Egiyan (N. Gevorgyan)\\
\hline
Muon &	K.Griffioen (Y.Gershtein)\\
\hline
Software	& M.Holtrop (S. Uemura)\\
\hline
Analysis &	M. Graham (O. Moreno)\\
\hline
Project Management &	M. Oriunno (S. Stepanyan, J.Jaros)\\
\hline
\end{tabular}
\end{center}
\label{tb:groups}
\end{table}%

\begin{table}[htdp]
\caption{Total Labor (FTE).}
\begin{center}
\parbox{.45\linewidth}{
\begin{tabular}{c||cccc}
\multicolumn{5}{c}{SLAC}\\
 FTE	&FY13	&FY14	&FY15&	TOT\\
 \hline\hline
ME	&0.14	&0.67&	0.10&	0.91\\
MD&	0.11	&0.35&	0.00&	0.46\\
MT&	0.09	&0.63&	0.00	&0.72\\
EE &	0.21	&1.35&	0.03	&1.59\\
ET&	0.00	&0.02&	0.00	&0.03\\
\end{tabular}
%\end{center}
}
\parbox{.45\linewidth}{
%\begin{center}
\begin{tabular}{c||ccc}
\multicolumn{4}{c}{JLAB}\\
 FTE	&FY13	&FY14	&	TOT\\
 \hline\hline
ME	&0.05	&0.23&	0.28 \\
MD&	0.17	&0.48&	0.65\\
MT&	0.00	&0.31& 0.31\\
EE &	0.16	&1.7&1.86\\
ET&	0.03	&0.12&0.15\\
\end{tabular}
}
\end{center}
\label{tb:engin}
\end{table}%

\begin{table}[htdp]
\begin{center}
\caption{Summary of HPS Budget.}
\rotatebox{90}{
\begin{tabular}{c|c|c|c|c|c|c|c|c}
%WBS&&&&&&&
WBS&	Name&	Labor&	L Cont&	Material&	M cont&	Labor Tot&	Material Tot	&Total \\
\hline\hline
1	&HPS	&\$1,414 	&\$402 	&\$930 	&\$288	&\$1,816 	&\$1,219 	&\$3,036 \\
\hline
1.1	&Beamline (J)	&\$100 	&\$30 	&\$149 	&\$45 	&\$130 	&\$193 	&\$324 \\
1.2	&SVT (S)	&\$363 	&\$113 	&\$124 	&\$40 	&\$475 	&\$163 	&\$639 \\
1.3	&SVT DAQ (S)	&\$327 	&\$87 	&\$142 	&\$51 	&\$414 	&\$193 	&\$607 \\
1.4	 &ECAL (J)	&\$32 	&\$10 	&\$208 	&\$62 	&\$42 	&\$270 	&\$312 \\
1.5	&Muon (J)	&\$74 	&\$25 	&\$244 	&\$75 	&\$99 	&\$319 	&\$419 \\
1.6	&TDAQ (J)	&\$141 	&\$43 	&\$0 	&\$0 	&\$184 	&\$0 	&\$184 \\
1.7	&Slow Control (J)	&\$113 	&\$28 	&\$64 	&\$16 	&\$141 	&\$80 	&\$221 \\
1.8	&Installation \& Commissioning (S)	&\$59 	&\$18 	&\$0 	&\$0 	&\$76 	&\$0 	&\$76 \\
1.10	&SLAC Travel Meetings (S)	&\$64 	&\$9 	&\$0 	&\$0 	&\$73 	&\$0 	&\$73 \\
1.11	&SLAC Travel 	&\$78 	&\$21 	&\$0 	&\$0 	&\$99 	&\$0 	&\$99 \\
	&for Setup and Running (S)	&	& 	& 	& 	& 	&& \\
1.12	&Project management (S)	&\$63 	&\$19 	&\$0 	&\$0 	&\$82 	&\$0 	&\$82 \\
\end{tabular}
}
\end{center}
\label{tb:budget}
\end{table}%

%\begin{figure*}[h]
%\centering
%\vspace*{-5mm}
%\includegraphics[angle=90,width=0.34\textwidth]{cost_schedule/cost_summary.pdf} 
%\caption{Cost breakdown by sub-systems.}
%\label{fig:cost}
%\end{figure*}

%\begin{figure*}[h]
%\centering
%\vspace*{-5mm}
%\includegraphics[width=0.9\textwidth]{cost_schedule/cost_systems.jpg} 
%\caption{Cost breakdown by sub-systems.}
%\label{fig:cost}
%\end{figure*}

\begin{figure*}[h]
\centering
%\vspace*{-5mm}
\includegraphics[width=0.9\textwidth]{cost_schedule/spending.jpg} 
\caption{Spending profile (costs after overheads and contingency).}
\label{fig:spending}
\end{figure*}

\begin{figure*}[h]
\centering
%\vspace*{-5mm}
\includegraphics*[angle=90,width=0.75\textwidth]{cost_schedule/ScheduleHPSV90-1} 
\caption{HPS schedule.}
\label{fig:schedulea}
\end{figure*}
\begin{figure*}[h]
\centering
%\vspace*{-5mm}
\includegraphics*[angle=90,width=0.85\textwidth]{cost_schedule/ScheduleHPSV90-2} 
\caption{HPS schedule.}
\label{fig:scheduleb}
\end{figure*}

\begin{figure*}[h]
\centering
%\vspace*{-5mm}
\includegraphics*[angle=90,width=0.85\textwidth]{cost_schedule/ScheduleHPSV90-3} 
\caption{HPS schedule.}
\label{fig:schedulec}
\end{figure*}


%\clearpage
%\begin{figure*}[h]
%\centering
%\vspace*{-5mm}
%\includegraphics[angle=0,width=0\textwidth]{cost_schedule/p1} 
%\caption{HPS WBS.}
%\label{fig:schedulea}
%\end{figure*}

%\includepdf[pages=1-11,angle=90,scale=0.7]{cost_schedule/HPSV70p.pdf}
%\begin{rotate}{90}

